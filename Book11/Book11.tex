% !TEX root = ../Elements.tex
%%%%%%
% BOOK 11
%%%%%%
\pdfbookmark[0]{Book 11}{book11}
\pagestyle{empty}
\begin{center}
{\Huge ELEMENTS BOOK 11}\\
\spa\spa\spa
{\huge\it Elementary Stereometry}
\end{center}
\vspace{2cm}
\epsfysize=5in
\centerline{\epsffile{Book11/front.eps}}
\newpage

%%%%%%%
% Definitions
%%%%%%%
\pdfbookmark[1]{Definitions}{def11}
\pagestyle{fancy}
\cfoot{\gr{\thepage}}
\lhead{\large\gr{STOIQEIWN \ggn{11}.}}
\rhead{\large ELEMENTS BOOK 11}
\begin{Parallel}{}{} 
\ParallelLText{
\begin{center}
\large{\gr{<'Oroi}.}
\end{center}\vspace*{-7pt}

\ggn{1}.~\gr{Stere'on >esti t`o m\kern -.7pt ~hkoc ka`i pl'atoc ka`i b'ajoc >'eqon.}

\ggn{2}.~\gr{Stereo~u d`e p'erac >epif'aneia.}

\ggn{3}.~\gr{E>uje~ia pr`oc >ep'ipedon >orj'h >estin, <'otan pr`oc
p'asac t`ac <aptom'enac a\kern -.7pt >ut~hc e>uje'iac ka`i o>'usac >en t~w| [<upokeim'enw|] >epip'edw| >orj`ac poi~h| gwn'iac.}

\ggn{4}.~\gr{>Ep'ipedon pr`oc >ep'ipedon >orj'on >estin, <'otan a<i t~h|
koin~h| tom\kern -.7pt ~h| t~wn >epip'edwn pr`oc >orj`ac >ag'omenai e>uje~iai
>en <en`i t~wn >epip'edwn t~w| loip~w| >epip'edw| pr`oc
>orj`ac >~wsin.}

\ggn{5}.~\gr{E>uje'iac pr`oc >ep'ipedon kl'isic >est'in, <'otan >ap`o to~u
mete'wrou p'eratoc t~hc e>uje'iac >ep`i t`o >ep'ipedon k'ajetoc
>aqj~h|, ka`i >ap`o to~u genom'enou shme'iou >ep`i t`o >en t~w|
>epip'edw| p'erac t~hc e>uje'iac e>uje~ia >epizeuqj~h|, <h
perieqom'enh gwn'ia <up`o t~hc >aqje'ishc ka`i t~hc >efest'wshc.}

\ggn{6}.~\gr{>Epip'edou pr`oc >ep'ipedon kl'isic >est`in <h perieqom'enh
>oxe~ia gwn'ia <up`o t~wn pr`oc >orj`ac t~h| koin~h| tom\kern -.7pt ~h| >agom'enwn
pr`oc t~w| a\kern -.7pt >ut~w| shme'iw| >en <ekat'erw| t~wn >epip'edwn.}

\ggn{7}.~\gr{>Ep'ipedon pr`oc >ep'ipedon <omo'iwc kekl'isjai l'egetai
ka`i <'eteron pr`oc <'eteron, <'otan a<i e>irhm'enai t~wn
kl'isewn gwn'iai >'isai >all'hlaic >~wsin.}

\ggn{8}.~\gr{Par'allhla >ep'iped'a >esti t`a >as'umptwta.}

\ggn{9}.~\gr{<'Omoia stere`a sq'hmat'a >esti t`a <up`o <omo'iwn
>epip'edwn perieq'omena >'iswn t`o pl~hjoc.}

\ggn{10}.~\gr{>'Isa d`e ka`i <'omoia stere`a sq'hmat'a >esti t`a <up`o
<omo'iwn >epip'edwn perieq'omena >'iswn t~w| pl'hjei ka`i t~w|
meg'ejei.}

\ggn{11}.~\gr{Stere`a gwn'ia >est`in <h <up`o plei'onwn >`h d'uo gramm~wn
<aptom'enwn >all'hlwn ka`i m\kern -.7pt `h >en t~h| a\kern -.7pt >ut~h| >epifane'ia|
o>us~wn pr`oc p'asaic ta~ic gramma~ic kl'isic. >'allwc; stere`a gwn'ia
>est`in <h <up`o plei'onwn  >`h d'uo gwni~wn >epip'edwn
perieqom'enh m\kern -.7pt `h o>us~wn >en t~w| a\kern -.7pt >ut~w| >epip'edw|
pr`oc <en`i shme'iw| sunistam'enwn.}

\ggn{12}.~\gr{Puram'ic >esti sq~hma stere`on >epip'edoic periq'omenon
>ap`o <en`oc >epip'edou pr`oc <en`i shme'iw| sunest'wc.}

\ggn{13}.~\gr{Pr'isma >est`i sq~hma stere`on >epip'edoic perieq'omenon,
<~wn d'uo t`a >apenant'ion >'isa te ka`i <'omoi'a >esti ka`i
par'allhla, t`a d`e loip`a parallhl'ogramma.}

\ggn{14}.~\gr{Sfa~ir'a >estin, <'otan <hmikukl'iou meno'ushc t~hc diam'et\-rou
perieneqj`en t`o <hmik'uklion  e>ic t`o a\kern -.7pt >ut`o p'alin >apokatastaj~h|,
 <'ojen >'hrxato f'eresjai, t`o perilhfj`en sq~hma.}
 
\ggn{15}.~\gr{>'Axwn d`e t~hc sfa'irac >est`in <h m'enousa e>uje~ia, per`i
<`hn t`o <hmik'uklion str'efetai.}

\ggn{16}.~\gr{K'entron d`e t~hc sfa'irac >est`i t`o a\kern -.7pt >ut'o, <`o ka`i to~u
<hmikukl'iou.}

\ggn{17}.~\gr{Di'ametroc d`e t~hc sfa'irac >est`in e>uje~i'a tic di`a to~u
k'entrou >hgm'enh ka`i peratoum'enh >ef> <ek'atera t`a
m'erh <up`o t~hc >epifane'iac t~hc sfa'irac.}

\ggn{18}.~\gr{K~wn'oc >estin, <'otan >orjogwn'iou trig'wnou meno'ushc mi~ac
pleur~ac t~wn per`i t`hn >orj`hn gwn'ian perieneqj`en t`o tr'igwnon e>ic t`o 
a\kern -.7pt >ut`o p'alin >apokatastaj~h|, <'ojen >'hrxato f'eresjai, t`o perilhfj`en sq~hma. k>`an m`en <h m'enousa e>uje~ia >'ish >~h| t~h| loip~h| [t~h|] per`i
t`hn >orj`hn periferom'enh|, >orjog'wnioc >'estai <o k~wnoc, >e`an d`e >el'attwn, >amblug'wnioc, >e`an d`e me'izwn, >oxug'wnioc.}

\ggn{19}.~\gr{>'Axwn d`e to~u k'wnou >est`in <h m'enousa e>uje~ia, per`i
<`hn t`o tr'igwnon str'efetai.}

\ggn{20}.~\gr{B'asic d`e <o k'ukloc <o <up`o t~hc periferom'enhc e>uje'iac  graf'omenoc.}

\ggn{21}.~\gr{K'ulindr'oc >estin, <'otan >orjogwn'iou parallhlogr'am\-mou meno'ushc mi~ac pleur~ac t~wn per`i t`hn >orj`hn gwn'ian perieneqj`en t`o
parallhl'ogrammon e>ic t`o a\kern -.7pt >ut`o p'alin >apokatastaj~h|, <'ojen >'hrxato
f'eresjai,
t`o perilhfj`en sq~hma.}

\ggn{22}.~\gr{>'Axwn d`e to~u kul'indrou >est`in <h m'enousa e>uje~ia,
per`i <`hn t`o parallhl'ogrammon str'efetai.}

\ggn{23}.~\gr{B'aseic d`e o<i k'ukloi o<i <up`o t~wn >apenant'ion periagom'enwn
d'uo pleur~wn graf'omenoi.}

\ggn{24}.~\gr{<'Omoioi k~wnoi ka`i k'ulindro'i e>isin, <~wn o<'i te >'axonec ka`i
a<i di'ametroi t~wn b'asewn >an'alog'on e>isin.}

\ggn{25}.~\gr{K'uboc >est`i sq~hma stere`on <up`o <`ex tetrag'wnwn >'iswn
perieq'omenon.}

\ggn{26}.~\gr{>Okt'aedr'on >est`i sq~hma stere`on <up`o >okt`w trig'wnwn 
>'iswn ka`i >isople'urwn perieq'omenon.}

\ggn{27}.~\gr{E>ikos'aedr'on >esti sq~hma stere`on <up`o e>'ikosi trig'wnwn
>'iswn ka`i >isople'urwn perieq'omenon.}

\ggn{28}.~\gr{Dwdek'aedr'on >esti sq~hma stere`on <up`o d'wdeka pentag'wnwn
>'iswn ka`i >isople'urwn ka`i >isogwn'iwn perieq'omenon.}}

\ParallelRText{
\begin{center}
{\large Definitions}
\end{center}

1.~A solid is a (figure) having length and breadth and depth.

2.~The extremity of a solid (is) a surface.

3.~A straight-line is at right-angles to a plane when it makes right-angles with all of the
straight-lines joined to it which are also in the plane.

4.~A plane is at right-angles to a(nother) plane when (all of) the  straight-lines drawn in one of the planes, at right-angles to the common section
of the planes, are at right-angles to the remaining plane.

5.~The inclination of a straight-line to a plane is the angle contained
by the drawn and standing (straight-lines), when a perpendicular is
lead  to the plane from the end of the (standing) straight-line raised (out of the plane), and a straight-line
is (then) joined from the point (so) generated to the end of the (standing) straight-line (lying) in the plane.

6.~The inclination of a plane to a(nother) plane is the  acute angle
contained by the (straight-lines), (one)  in each of the planes, drawn at right-angles to the common segment
(of the planes), at the same point.

7.~A plane is said to be similarly inclined to a plane,
as another  to  another, when the aforementioned angles
of inclination are equal to one another.

8.~Parallel planes are those which do not meet (one another).

9.~Similar solid figures are those contained by equal numbers of similar
planes (which are similarly arranged).

10.~But equal and similar solid figures are those contained
by similar planes equal in number and in magnitude (which are similarly arranged).

11.~A solid angle is the inclination (constituted) by more than two lines joining one another (at the same point), and
not being in the same surface, to all of the lines. Otherwise,
a solid angle is that contained by more than two plane angles, not being
in the same plane, and constructed at one point.

12.~A pyramid is a solid figure, contained by planes, (which is) constructed from one plane to
one  point.

13.~A prism is a solid figure, contained by planes, of which the two opposite
(planes)
are equal, similar, and parallel, and the remaining (planes are) parallelograms.

14.~A sphere is the figure enclosed when, the diameter of a semicircle remaining (fixed), the semicircle is carried around,
 and  again established at the same (position) from which it began to be moved.
 
15.~And the axis of the sphere is the fixed straight-line about
 which the semicircle is turned.
 
16.~And the center of the sphere is the same as that of the semicircle.

17.~And the diameter of the sphere is any straight-line which is
 drawn through the center and terminated in both directions by
 the surface of the sphere.
 
18.~A cone is the figure enclosed when,  one of the sides of a right-angled triangle about the right-angle remaining (fixed),
  the triangle is carried around, and again established at the
 same (position) from which it began to be moved. And if the
 fixed straight-line is equal to the remaining (straight-line) about the
 right-angle, (which is) carried around, then the cone will be right-angled, and
 if less, obtuse-angled, and if greater, acute-angled.
 
19.~And the axis of the cone is the fixed straight-line about which
 the triangle is turned.
 
20.~And the base (of the cone is) the circle described by the (remaining) straight-line (about the right-angle which is)
 carried around (the axis).
 
21.~A cylinder is the figure enclosed when, one of the sides
 of a right-angled parallelogram about the right-angle remaining (fixed),
 the parallelogram is carried around, and again established at the same
 (position) from which it began to be moved.
 
22.~And the axis of the cylinder is the stationary straight-line about which
 the parallelogram is turned.
 
23.~And the bases (of the cylinder are) the circles described by the
 two opposite sides (which are) carried around.
 
24.~Similar cones and cylinders are those for which the axes and the diameters
 of the bases are proportional.
 
25.~A cube is a solid figure contained by six equal squares.

26.~An octahedron is a solid figure contained by eight equal and
 equilateral triangles.
 
 27.~An icosahedron is a solid figure contained by twenty equal and
 equilateral triangles.
 
28.~A dodecahedron is a solid figure contained by twelve equal,
 equilateral, and equiangular pentagons.}
 \end{Parallel}
 
%%%%
%11.1
%%%%
\pdfbookmark[1]{Proposition 11.1}{pdf11.1}
\begin{Parallel}{}{}
\ParallelLText{
\begin{center}
{\large \ggn{1}.}
\end{center}\vspace*{-7pt}

\gr{E>uje'iac gramm\kern -.7pt ~hc m'eroc m'en ti o>uk >'estin >en t~w| <upokeim'enw|
>epip'edw|, m'eroc d'e ti >en metewrot'erw|.}

\gr{E>i g`ar dunat'on, e>uje'iac gramm\kern -.7pt ~hc t~hc ABG m'eroc
m'en ti t`o AB >'estw >en t~w| <upokeim'enw| >epip'edw|,
m'eroc d'e ti t`o BG >en metewrot'erw|.}

\gr{>'Estai d'h tic t~h| AB sunveq`hc e>uje~ia >ep> e>uje'iac >en
t~w| <upokeim'enw| >epip'edw|. >'estw <h BD; d'uo >'ara e>ujei~wn
t~wn ABG, ABD koin`on tm\kern -.7pt ~hm'a >estin <h AB; <'oper >est`in
>ad'unaton, >epeid'hper >e`an k'entrw| t~w| B ka`i diast'hmati
t~w| AB k'uklon gr'aywmen, a<i di'ametroi >an'isouc >apol'hyontai
to~u k'uklou perifere'iac.}\\

\epsfysize=2.5in
\centerline{\epsffile{Book11/fig01g.eps}}

\gr{E>uje'iac >'ara gramm\kern -.7pt ~hc m'eroc m'en ti o>uk >'estin >en t~w| <upokeim'enw| >epip'edw|, t`o d`e >en metewrot'erw|; <'oper
>'edei de~ixai.}}

\ParallelRText{
\begin{center}
{\large Proposition 1}$^\dag$
\end{center}

Some part of a straight-line cannot be in a reference
plane, and some part in a more elevated (plane).

For, if possible, let some part, $AB$, of the straight-line $ABC$ be
in a reference plane, and some part, $BC$, in a more elevated (plane).

In the reference plane, there will be some straight-line continuous with, and straight-on to, $AB$.$^\ddag$ Let it be $BD$. Thus, $AB$ is a common
segment of the two (different) straight-lines $ABC$ and $ABD$. The
very thing is impossible, inasmuch as if we draw a circle with center $B$
and radius $AB$ then the diameters ($ABD$ and $ABC$) will cut off unequal circumferences
of the circle.

\epsfysize=2.5in
\centerline{\epsffile{Book11/fig01e.eps}}

Thus, some part of a straight-line cannot be in a reference
plane, and (some part) in a more elevated (plane). (Which is) the very thing it
was required to show.}
\end{Parallel}
{\footnotesize\noindent$^\dag$ The proofs of the first three propositions in this book are not at all rigorous. Hence,  these three propositions should properly be regarded as additional axioms.}\\
{\footnotesize\noindent$^\ddag$ This assumption essentially presupposes the validity of the proposition under discussion.} 

%%%%
%11.2
%%%%
\pdfbookmark[1]{Proposition 11.2}{pdf11.2}
\begin{Parallel}{}{}
\ParallelLText{
\begin{center}
{\large \ggn{2}.}
\end{center}\vspace*{-7pt}

\gr{>E`an d'uo e>uje~iai t'emnwsin >all'hlac, >en <en'i e>isin
>epip'edw|, ka`i p~an tr'igwnon >en <en'i >estin >epip'edw|.}\\

\epsfysize=2.in
\centerline{\epsffile{Book11/fig02g.eps}}

\gr{D'uo g`ar e>uje~iai a<i AB, GD temn'etwsan >all'hlac kat`a t`o E
shme~ion. l'egw, <'oti a<i AB, GD >en <en'i e>isin >epip'edw|,
ka`i  p~an tr'igwnon >en <en'i >estin >epip'edw|.}

\gr{E>il'hfjw g`ar >ep`i t~wn EG, EB tuq'onta shme~ia t`a Z, H,
ka`i >epeze'uqjwsan a<i GB, ZH, ka`i di'hqjwsan a<i ZJ, HK;
l'egw pr~wton, <'oti t`o EGB tr'igwnon >en <en'i >estin >epip'edw|.
e>i g'ar >esti to~u EGB trig'wnou m'eroc >'htoi t`o ZJG >`h t`o HBK
>en t~w| <upokeim'enw|
 [>epip'edw|], t`o d`e loip`on >en >'allw|,
>'estai ka`i mi~ac t~wn EG, EB e>ujei~wn m'eroc m'en ti >en t~w|
<upokeim'enw| >epip'edw|, t`o d`e >en allw|. e>i d`e to~u EGB
trig'wnou t`o ZGBH m'eroc >~h| >en t~w| <upokeim'enw|
>epip'edw|, t`o d`e loip`on >en >'allw|, >'estai ka`i >amfot'erwn t~wn
EG, EB e>ujei~wn m'eroc m'en ti >en t~w| <upokeim'enw|
>epip'edw|, t`o d`e >en >'allw; <'oper >'atopon >ede'iqjh. t`o >'ara
EGB tr'igwnon >en <en'i >estin >epip'edw|. >en <~w| d'e >esti
t`o EGB tr'igwnon, >en to'utw| ka`i <ekat'era t~wn EG,
EB, >en <~w| d`e <ekat'era t~wn EG, EB, >en to'utw| ka`i
a<i AB, GD. a<i AB, GD >'ara e>uje~iai >en <en'i e>isin
>epip'edw|, ka`i p~an tr'igwnon >en <en'i >estin >epip'edw|;
<'oper >'edei de~ixai.}}

\ParallelRText{
\begin{center}
{\large Proposition 2}
\end{center}

If two straight-lines cut one another then they are
in one plane, and every triangle (formed using segments of
both lines) is in one plane.

\epsfysize=2.in
\centerline{\epsffile{Book11/fig02e.eps}}

For let the two straight-lines $AB$ and $CD$ have cut one another at point
$E$. I say that $AB$ and $CD$ are in one plane, and that every
triangle (formed using segments of both lines) is in one plane.

For let the random points $F$ and $G$ be taken on $EC$ and
$EB$ (respectively). And let $CB$ and $FG$ be joined, and let
$FH$ and $GK$ be drawn across. I say, first of all, that triangle
$ECB$ is in one (reference) plane. For if part of triangle $ECB$, either $FHC$ or
$GBK$, is in the reference [plane], and the remainder in a different (plane)
then a part  of one the straight-lines $EC$ and $EB$
will also be in the reference plane, and (a part) in a different (plane).
And if the part $FCBG$ of triangle $ECB$ is in the reference plane, and
the remainder in a different (plane) then parts of both of the straight-lines
$EC$ and $EB$ will also be in  the reference plane, and (parts) in a
different (plane). The very thing was shown to be absurb [Prop.~11.1]. Thus, triangle $ECB$ is in
one plane. And in whichever (plane) triangle $ECB$ is (found),  in that (plane) $EC$ and $EB$ (will) each also (be found). And in whichever (plane) 
 $EC$
and $EB$ (are) each (found), in that (plane) $AB$ and $CD$ (will) also (be found)  [Prop.~11.1]. Thus, the straight-lines $AB$ and
$CD$ are in one plane, and every triangle (formed using segments of
both lines) is in one plane. (Which is) the very thing it was required to
show.}
\end{Parallel}

%%%%
%11.3
%%%%
\pdfbookmark[1]{Proposition 11.3}{pdf11.3}
\begin{Parallel}{}{}
\ParallelLText{
\begin{center}
{\large \ggn{3}.}
\end{center}\vspace*{-7pt}

\gr{>E`an d'uo >ep'ipeda temn~h| >'allhla, <h koin`h a\kern -.7pt >ut~wn tom\kern -.7pt `h e>uje~i'a >estin.}

\epsfysize=1.8in
\centerline{\epsffile{Book11/fig03g.eps}}

\gr{D'uo g`ar >ep'ipeda t`a AB, BG temn'etw >'allhla, koin`h d`e a\kern -.7pt >ut~wn
tom\kern -.7pt `h >'estw <h DB gramm\kern -.7pt 'h; l'egw, <'oti <h DB gramm\kern -.7pt `h e>uje~i'a
>estin.}

\gr{E>i g`ar m\kern -.7pt 'h, >epeze'uqjw >ap`o to~u D >ep`i t`o B >en
m`en t~w| AB >epip'edw| e>uje~ia <h DEB, >en d`e t~w|
BG >epip'edw| e>uje~ia <h DZB. >'estai d`h d'uo e>ujei~wn
t~wn DEB, DZB t`a a\kern -.7pt >ut`a p'erata, ka`i peri'exousi dhlad`h
qwr'ion; <'oper >'atopon. o>'uk >'ara a>i DEB, DZB e>uje~ia'i
e>isin. <omo'iwc d`h de'ixomen, <'oti o>ud`e >'allh tic >ap`o
to~u D >ep`i t`o B >epizeugnum'enh e>uje~ia >'estai pl`hn
t~hc DB koin~hc tom\kern -.7pt ~hc t~wn AB, BG >epip'edwn.}

\gr{>E`an >'ara d'uo >ep'ipeda t'emnh| >'allhla, <h koin`h a\kern -.7pt >ut~wn
tom\kern -.7pt `h e>uje~i'a >estin; <'oper >'edei de~ixai.}}

\ParallelRText{
\begin{center}
{\large Proposition 3}
\end{center}

If two planes cut one another then their common
section is a straight-line.

\epsfysize=1.8in
\centerline{\epsffile{Book11/fig03e.eps}}

For let the two planes $AB$ and $BC$ cut one another, and let their
common section be the line $DB$. I say that the line $DB$ is straight.

For, if not,  let the straight-line $DEB$ be joined from $D$ to $B$
in the plane $AB$, and the straight-line $DFB$ in the plane $BC$. 
So two straight-lines, $DEB$ and $DFB$, will have the same ends, and
they will clearly enclose an area. The very thing (is) absurd. Thus,
$DEB$ and $DFB$ are not straight-lines.  So, similarly, we can show
than no other straight-line  can be joined from $D$ to $B$  except $DB$, the common section of the planes $AB$ and $BC$.

Thus, if  two planes cut one another then their common
section is a straight-line. (Which is) the very thing it was required to
show.}
\end{Parallel}

%%%%
%11.4
%%%%
\pdfbookmark[1]{Proposition 11.4}{pdf11.4}
\begin{Parallel}{}{}
\ParallelLText{
\begin{center}
{\large \ggn{4}.}
\end{center}\vspace*{-7pt}

\gr{>E`an e>uje~ia d'uo e>uje'iaic  temno'usaic >all'hlac pr`oc >orj`ac >ep`i
t~hc koin~hc tom\kern -.7pt ~hc >epistaj~h|, ka`i t~w| di> a\kern -.7pt >ut~wn >epip'edw|
pr`oc >orj`ac >'estai.}

\gr{E>uje~ia g'ar tic <h EZ d'uo e>uje'iaic ta~ic AB, GD
temno'usaic >all'hlac kat`a t`o E shme~ion >ap`o to~u
E pr`oc >orj`ac >efest'atw; l'egw, <'oti <h EZ ka`i t~w|
di`a t~wn AB, GD >epip'edw| pr`oc >orj'ac >estin.}

\gr{>Apeil'hfjwsan g`ar a<i AE, EB, GE, ED >'isai >all'hlaic, ka`i
di'hqjw tic di`a to~u E, <wc >'etuqen, <h HEJ, ka`i >epeze'uqjwsan
a<i AD, GB, ka`i >'eti >ap`o tuq'ontoc to~u Z >epeze'uqjwsan
a<i ZA, ZH, ZD, ZG, ZJ, ZB.}\\~\\~\\

\epsfysize=2.in
\centerline{\epsffile{Book11/fig04g.eps}}

\gr{Ka`i >epe`i d'uo a<i AE, ED dus`i ta~ic
GE, EB >'isai e>is`i ka`i gwn'iac >'isac peri'eqousin, b'asic >'ara
<h AD b'asei t~h| GB >'ish >est'in, ka`i t`o AED tr'igwnon t~w| GEB
trig'wnw| >'ison >'estai; <'wste ka`i gwn'ia <h <up`o DAE gwn'ia|
t~h| <up`o EBG >'ish [>est'in]. >'esti d`e ka`i <h <up`o AEH gwn'ia
t~h| <up`o BEJ >'ish. d'uo d`h tr'igwn'a >esti t`a AHE, BEJ t`ac
d'uo gwn'iac dus`i gwn'iaic >'isac >'eqonta <ekat'eran <ekat'era| ka`i
m'ian pleur`an mi~a| pleur~a| >'ishn t`hn pr`oc ta~ic >'isaic gwn'iaic
t`hn AE t~h| EB; ka`i t`ac loip`ac >'ara pleur`ac ta~ic loipa~ic
pleura~ic >'isac <'exousin. >'ish >'ara <h m`en HE t~h| EJ, <h d`e
AH t~h| BJ. ka`i >epe`i >'ish >est`in <h AE t~h| EB, koin`h
d`e ka`i pr`oc >orj`ac <h ZE, b'asic >'ara <h ZA b'asei t~h| ZB
>estin >'ish. di`a t`a a\kern -.7pt >ut`a d`h ka`i <h ZG t~h| ZD >estin >'ish.
ka`i >epe`i >'ish >est`in <h AD t~h| GB, >'esti
d`e ka`i <h ZA t~h| ZB >'ish, d'uo d`h a<i ZA, AD dus`i ta~ic
ZB, BG >'isai e>is`in <ekat'era <ekat'era|; ka`i b'asic <h ZD
b'asei t~h| ZG >ede'iqjh >'ish; ka`i gwn'ia >'ara
<h <up`o ZAD gwn'ia| t~h| <up`o ZBG >'ish >est'in. ka`i
>epe`i p'alin >ede'iqjh <h AH t~h| BJ >'ish, >all`a m\kern -.7pt `hn ka`i
<h ZA t~h| ZB >'ish, d'uo d`h a<i ZA, AH dus`i ta~ic
ZB, BJ >'isai e>is'in. ka`i gwn'ia <h <up`o ZAH >ede'iqjh
>'ish t~h| <up`o ZBJ; b'asic >'ara <h ZH b'asei t~h| ZJ >estin >'ish. ka`i
>epe`i p'alin >'ish >ede'iqjh <h HE t~h| EJ, koin`h d`e <h EZ,
d'uo d`h a<i HE, EZ dus`i ta~ic JE, EZ >'isai e>is'in; ka`i
b'asic <h ZH b'asei t~h| ZJ >'ish; gwn'ia >'ara <h <up`o HEZ gwn'ia|
t~h| <up`o JEZ >'ish >est'in. >orj`h >'ara <ekat'era t~wn <up`o
HEZ, JEZ gwni~wn.  <h ZE >'ara pr`oc t`hn  HJ tuq'ontwc di`a
to~u E >aqje~isan >orj'h >estin. <omo'iwc d`h de'ixomen,
<'oti <h ZE ka`i pr`oc p'asac t`ac <aptom'enac a\kern -.7pt >ut~hc
e>uje'iac ka`i o>'usac >en t~w| <upokeim'enw| >epip'edw|
>orj`ac poi'hsei gwn'iac. e>uje~ia d`e pr`oc >ep'ipedon
>orj'h >estin, <'otan pr`oc p'asac t`ac <aptom'enac a\kern -.7pt >ut~hc
e>uje'iac ka`i o>'usac >en t~w| a\kern -.7pt >ut~w| >epip'edw| >orj`ac poi~h|
gwn'iac; <h ZE >'ara t~w| <upokeim'enw| >epip'edw| pr`oc >orj'ac
>estin. t`o d`e <upoke'imenon >ep'iped'on >esti t`o di`a t~wn
AB, GD e>ujei~wn. <h ZE >'ara pr`oc >orj'ac >esti t~w| di`a
t~wn AB, GD >epip'edw|.}

\gr{>E`an >'ara e>uje~ia d'uo e>uje'iaic  temno'usaic >all'hlac pr`oc >orj`ac >ep`i
t~hc koin~hc tom\kern -.7pt ~hc >epistaj~h|, ka`i t~w| di> a\kern -.7pt >ut~wn >epip'edw|
pr`oc >orj`ac >'estai; <'oper >'edei de~ixai.}}

\ParallelRText{
\begin{center}
{\large Proposition 4}
\end{center}

If a straight-line is set up at right-angles to two straight-lines cutting one another, at the common  point of section,  then it will also be
at right-angles to the plane (passing) through them (both).

For let some straight-line $EF$ have (been) set up  at right-angles to two
straight-lines, $AB$ and $CD$, cutting one another at point $E$, at $E$. I say that
$EF$ is also at right-angles to the plane (passing) through $AB$ and $CD$.

For let $AE$, $EB$, $CE$ and $ED$ be cut off from (the two straight-lines
so as to be) equal to one another. And let $GEH$ be drawn, at random,
through $E$ (in the plane passing through $AB$ and $CD$). And let $AD$ and $CB$ be joined. And, furthermore,
let $FA$, $FG$, $FD$, $FC$, $FH$, and $FB$ be joined
from the random (point) $F$ (on $EF$).

\epsfysize=2.in
\centerline{\epsffile{Book11/fig04e.eps}}

For since the two (straight-lines) $AE$ and $ED$ are equal to the two (straight-lines) $CE$ and $EB$, and they  enclose equal angles [Prop.~1.15], the base $AD$ is thus equal to the
base $CB$, and triangle $AED$ will be equal to triangle $CEB$ [Prop.~1.4]. Hence, the angle $DAE$ [is] equal to
the angle $EBC$. And the angle $AEG$ (is)  also equal to the angle
$BEH$ [Prop.~1.15]. So $AGE$ and $BEH$
are two triangles having two angles equal to two angles, respectively, and
one side equal to one side---(namely), those by the equal angles, $AE$ and $EB$. Thus, they will also have the remaining sides equal to the remaining
sides [Prop.~1.26]. Thus, $GE$ (is) equal to $EH$, 
and $AG$ to $BH$. And since $AE$ is equal to $EB$, and $FE$ is
common and at right-angles, the base $FA$ is thus equal to the base $FB$ [Prop.~1.4]. So, for the same (reasons), $FC$ is also
equal to $FD$. And since $AD$ is equal to $CB$, and $FA$ is also equal to
$FB$, the two (straight-lines) $FA$ and $AD$ are equal to the
two (straight-lines) $FB$ and $BC$, respectively. And the base $FD$
was shown (to be) equal to the base $FC$. Thus, the angle $FAD$ is also
equal to the angle $FBC$ [Prop.~1.8]. And, again,
since $AG$ was shown (to be) equal to $BH$, but $FA$ (is) also equal to
$FB$, the two (straight-lines) $FA$ and $AG$ are equal to the
two (straight-lines) $FB$ and $BH$ (respectively). And the angle $FAG$
was shown (to be) equal to the angle $FBH$. Thus, the base $FG$ is equal
to the base $FH$ [Prop.~1.4]. And, again, since
$GE$ was shown (to be) equal to $EH$, and $EF$ (is) common, the two
(straight-lines) $GE$ and $EF$ are equal to the two (straight-lines)
$HE$ and $EF$ (respectively). And the base $FG$
(is) equal to the base $FH$. Thus, the angle $GEF$ is equal to the
angle $HEF$ [Prop.~1.8]. Each of the
angles $GEF$ and $HEF$ (are) thus right-angles [Def.~1.10]. Thus, $FE$ is
at right-angles to $GH$, which was drawn at random through $E$ (in the reference plane passing though $AB$ and $AC$). So, similarly, we can show that
$FE$ will make right-angles with all straight-lines joined to it which are in the reference plane. And a straight-line is at right-angles
to a plane when it makes right-angles with all straight-lines joined to
it which are in the plane [Def.~11.3]. Thus,
$FE$ is at right-angles to the reference plane. And the reference plane
is that (passing) through the straight-lines $AB$ and $CD$. Thus,
$FE$ is at right-angles to the plane (passing) through $AB$ and
$CD$.

Thus, if a straight-line is set up at right-angles to two straight-lines cutting one another, at the common point of section, then it will also be
at right-angles to the plane (passing) through them (both). (Which is)
the very thing it was required to show.}
\end{Parallel}

%%%%
%11.5
%%%%
\pdfbookmark[1]{Proposition 11.5}{pdf11.5}
\begin{Parallel}{}{}
\ParallelLText{
\begin{center}
{\large \ggn{5}.}
\end{center}\vspace*{-7pt}

\gr{>E`an e>uje~ia tris`in e>uje'iaic <aptom'enaic >all'hlwn pr`oc >orj`ac
>ep`i t~hc koin~hc tom\kern -.7pt ~hc >epistaj~h|, a<i tre~ic e>uje~iai >en <en'i
e>isin >epip'edw|.}

\epsfysize=2.in
\centerline{\epsffile{Book11/fig05g.eps}}

\gr{E>uje~ia g'ar tic <h AB tris`in e>uje'iaic ta~ic BG, BD, BE pr`oc >orj`ac
>ep`i t~hc kat`a t`o B <af~hc >efest'atw; l'egw, <'oti a<i BG, BD, BE >en
<en'i e>isin >epip'edw|.}

\gr{m\kern -.7pt `h g'ar, >all> e>i dunat'on, >'estwsan a<i m`en BD, BE >en t~w| <upokeim'enw| >epip'edw|, <h d`e BG >en metewrot'erw|, ka`i >ekbebl'hsjw
t`o d`ia t~wn AB, BG >ep'ipedon; koin`hn d`h tom\kern -.7pt `hn poi'hsei >en t~w|
<upokeim'enw| >epip'edw| e>uje~ian. poie'itw t`hn BZ. >en <en`i >'ara
e>is`in >epip'edw| t~w| dihgm'enw| di`a t~wn AB, BG a<i tre~ic
e>uje~iai a<i AB, BG, BZ. ka`i >epe`i <h AB >orj'h >esti pr`oc
<ekat'eran t~wn BD, BE, ka`i t~w| di`a t~wn BD, BE
>'ara >epip'edw| >orj'h >estin <h AB. t`o d`e di`a t~wn BD, BE >ep'ipedon
t`o <upoke'imen'on >estin; <h AB >'ara >orj'h >esti pr`oc t`o <upoke'imenon
>ep'ipedon. <'wste ka`i pr`oc p'asac t`ac <aptom'enac a\kern -.7pt >ut~hc e>uje'iac
ka`i o>'usac >en t~w| <upokeim'enw| >epip'edw| >orj`ac poi'hsei gwn'iac
<h AB. <'aptetai d`e a\kern -.7pt >ut~hc <h BZ o~>usa >en t~w| <upokeim'enw|
>epip'edw|; <h >'ara <up`o ABZ gwn'ia >orj'h >estin. <up'okeitai
d`e ka`i <h <up`o ABG >orj'h; >'ish >'ara <h <up`o ABZ gwn'ia t~h|
<up`o ABG. ka'i e>isin >en <en`i >epip'edw|; <'oper
>est`in >ad'unaton. o>uk >'ara <h BG e>uje~ia >en metewrot'erw| >est`in
>epip'edw|; a<i tre~ic >'ara e>uje~iai a<i BG, BD, BE >en <en'i
e>isin >epip'edw|.}

\gr{>E`an >'ara  e>uje~ia tris`in e>uje'iaic <aptom'enaic >all'hlwn >ep`i t~hc
<af~hc pr`oc >orj`ac
>epistaj~h|, a<i tre~ic e>uje~iai >en <en'i
e>isin >epip'edw|; <'oper >'edei de~ixai.}}

\ParallelRText{
\begin{center}
{\large Proposition 5}
\end{center}

If a straight-line is set up at right-angles to three straight-lines cutting one another, at the common
point of section, then the three straight-lines are in one plane.

\epsfysize=2.in
\centerline{\epsffile{Book11/fig05e.eps}}

For let some straight-line $AB$ be set up at right-angles
to three straight-lines $BC$, $BD$, and $BE$, at the (common) point of
section $B$. I say that $BC$, $BD$, and $BE$ are in  one plane.

For (if) not, and if possible, let $BD$ and $BE$ be in the reference plane,
and $BC$ in a more elevated (plane).  And let the plane through $AB$ and
$BC$ be produced. So it will make a straight-line as a
common section with the reference plane [Def.~11.3]. 
Let it make $BF$. Thus, the three straight-lines $AB$, $BC$, and $BF$
are in one plane---(namely), that drawn through $AB$ and $BC$. And since
$AB$ is at right-angles to each of $BD$ and $BE$, $AB$ is thus also at right-angles to the plane (passing) through $BD$ and $BE$ [Prop.~11.4]. And the plane (passing)
through $BD$ and $BE$ is the reference plane. Thus, $AB$ is at
right-angles to the reference plane. Hence, $AB$ will also make right-angles
with all straight-lines joined to it which are also in the reference plane [Def.~11.3].
And $BF$, which is in the reference plane, is joined to  it. Thus, the angle
$ABF$ is a right-angle. And $ABC$ was also assumed to be a right-angle.
Thus, angle $ABF$ (is) equal to $ABC$. And they are in one
plane. The very thing is impossible. Thus, $BC$ is not in a more elevated
plane. Thus, the three straight-lines $BC$, $BD$, and $BE$ are in one plane.

Thus, if a straight-line is set up at right-angles to three straight-lines cutting
one another, at the (common) point of section, then the three straight-lines
are in one plane. (Which is) the very thing it was required to show.}
\end{Parallel}

%%%%
%11.6
%%%%
\pdfbookmark[1]{Proposition 11.6}{pdf11.6}
\begin{Parallel}{}{}
\ParallelLText{
\begin{center}
{\large \ggn{6}.}
\end{center}\vspace*{-7pt}

\gr{>E`an d'uo e>uje~iai t~w| a\kern -.7pt >ut~w| >epip'edw| pr`oc >orj`ac >~wsin,
par'allhloi >'esontai a<i e>uje~iai.}

\epsfysize=2.in
\centerline{\epsffile{Book11/fig06g.eps}}

\gr{D'uo g`ar e>uje~iai a<i AB, GD t~w| <upokeim'enw| >epip'edw| pr`oc
>orj`ac >'estwsan; l'egw, <'oti par'allhl'oc >estin <h AB t~h| GD.}

\gr{Sumball'etwsan g`ar t~w| <upokeim'enw| >epip'edw| kat`a
t`a B, D shme~ia, ka`i >epeze'uqjw <h BD e>uje~ia, ka`i >'hqjw
t~h| BD pr`oc >orj`ac >en t~w| <upokeim'enw| >epip'edw|
<h DE, ka`i ke'isjw t~h| AB >'ish <h DE, ka`i >epeze'uqjwsan a<i
BE, AE, AD.}

\gr{Ka`i >epe`i <h AB >orj'h >esti pr`oc t`o <upoke'imenon >ep'ipedon,
ka`i pr`oc p'asac [>'ara] t`ac <aptom'enac a\kern -.7pt >ut~hc e>uje'iac
ka`i o>'usac >en t~w| <upokeim'enw| >epip'edw| >orj`ac
poi'hsei gwn'iac. <'aptetai d`e t~hc AB <ekat'era t~wn BD, BE o>~usa
>en t~w| <upokeim'enw| >epip'edw|; >orj`h >'ara 
>est`in <ekat'era t~wn <up`o ABD, ABE gwni~wn. di`a t`a a\kern -.7pt >ut`a d`h ka`i
<ekat'era t~wn <up`o GDB, GDE >orj'h >estin. ka`i >epe`i >'ish >est`in
<h AB t~h| DE, koin`h d`e <h BD, d'uo d`h a<i AB, BD dus`i ta~ic
ED, DB >'isai e>is'in; ka`i gwn'iac >orj`ac peri'eqousin; b'asic
>'ara <h AD b'asei
t~h| BE >estin >'ish. ka`i >epe`i >'ish >est`in <h AB t~h| DE, >all`a ka`i <h
AD t~h| BE, d'uo d`h a<i AB, BE dus`i ta~ic ED, DA >'isai e>is'in; ka`i
b'asic a\kern -.7pt >ut~wn koin`h <h AE; gwn'ia >'ara <h <up`o ABE gwni'a| t~h| <up`o
EDA >estin >'ish. >orj`h d`e <h <up`o ABE; >orj`h >'ara
ka`i <h <up`o EDA; <h ED >'ara pr`oc t`hn DA >orj'h >estin.
>'esti d`e ka`i pr`oc <ekat'eran t~wn BD, DG >orj'h. <h ED >'ara tris`in
e>uje'iaic ta~ic BD, DA, DG pr`oc >orj`ac >ep`i t~hc <af~hc >ef'esthken;
a<i tre~ic >'ara e>uje~iai a<i BD,
DA, DG >en <en'i e>isin >epip'edw|. >en <~w| d`e a<i DB,
DA, >en to'utw| ka`i <h AB; p~an g`ar tr'igwnon >en <en'i
>estin >epip'edw|; a<i  >'ara AB, BD, DG e>uje~iai >en <en'i
e>isin >epip'edw|. ka'i >estin >orj`h <ekat'era t~wn <up`o ABD, BDG gwni~wn; par'allhloc >'ara >est`in <h AB t~h| GD.}

\gr{>E`an >'ara d'uo e>uje~iai t~w| a\kern -.7pt >ut~w| >epip'edw| pr`oc >orj`ac
>~wsin, par'allhloi >'esontai a<i e>uje~iai; <'oper >'edei de~ixai.}}

\ParallelRText{
\begin{center}
{\large Proposition 6}
\end{center}

If two straight-lines are at right-angles to the same
plane then the straight-lines will be parallel.$^\dag$

\epsfysize=2.in
\centerline{\epsffile{Book11/fig06e.eps}}

For let the two straight-lines $AB$ and $CD$ be at right-angles
to a reference plane. I say that $AB$ is parallel to $CD$.

For let them meet the reference plane at points $B$ and $D$ (respectively).
And let the straight-line $BD$ be joined. And let $DE$ 
be drawn at right-angles to $BD$ in the reference plane. And
let $DE$ be made equal to $AB$. And let $BE$, $AE$, and $AD$ be joined.

And since $AB$ is at right-angles to the reference plane, it will [thus] also
make right-angles with all straight-lines joined to it which are in the
reference plane [Def.~11.3]. And
$BD$ and $BE$, which are in the reference plane, are each  joined to $AB$.
Thus, each of the angles $ABD$ and $ABE$ are right-angles. So, for the
same (reasons), each of the angles $CDB$ and $CDE$ are also right-angles.
And since $AB$ is equal to $DE$, and $BD$ (is) common, the two
(straight-lines) $AB$ and $BD$ are equal to the two (straight-lines) 
$ED$ and $DB$ (respectively). And they contain right-angles. Thus, the
base $AD$ is equal to the base $BE$ [Prop.~1.4].
And since $AB$ is equal to $DE$, and $AD$ (is) also (equal) to $BE$,  the two
(straight-lines) $AB$ and $BE$ are thus equal to the two (straight-lines) $ED$ and $DA$ (respectively). And their base $AE$ (is) common. Thus, angle $ABE$ is equal to angle $EDA$ [Prop.~1.8]. 
And $ABE$ (is) a right-angle. Thus, $EDA$ (is) also a right-angle.
$ED$ is thus at right-angles to $DA$.  And it is also at right-angles to
each of $BD$ and $DC$. Thus, $ED$ is standing at right-angles
to the three straight-lines $BD$, $DA$, and $DC$ at the (common)
point of section. Thus, the three straight-lines $BD$, $DA$,
and $DC$ are in one plane [Prop.~11.5]. And in
which(ever) plane $DB$ and $DA$ (are found), in that (plane) $AB$ (will) also (be found).
For every triangle is in one plane [Prop.~11.2].
And each of the angles $ABD$ and $BDC$ is a right-angle. Thus, $AB$ is
parallel to $CD$ [Prop.~1.28].

Thus, if  two straight-lines are at right-angles to the same
plane then the straight-lines will be parallel. (Which is) the very thing it
was required to show.}
\end{Parallel}
{\footnotesize\noindent$^\dag$ In other words,
the two straight-lines lie in the same plane, and never meet when
produced in either direction.}

%%%%
%11.7
%%%%
\pdfbookmark[1]{Proposition 11.7}{pdf11.7}
\begin{Parallel}{}{}
\ParallelLText{
\begin{center}
{\large\ggn{7}.}
\end{center}\vspace*{-7pt}

\gr{>E`an >~wsi d'uo e>uje~iai par'allhloi, lhfj~h| d`e >ef> <ekat'erac a\kern -.7pt >ut~wn tuq'onta shme~ia, <h >ep`i t`a shme~ia >epizeugnum'enh e>uje~ia >en t~w|
a\kern -.7pt >ut~w| >epip'edw| >est`i ta~ic parall'hloic.}

\epsfysize=1.5in
\centerline{\epsffile{Book11/fig07g.eps}}

\gr{>'Estwsan d'uo e>uje~iai par'allhloi a<i AB, GD, ka`i e>il'hfjw >ef>
<ekat'erac a\kern -.7pt >ut~wn tunq'onta shme~ia t`a E, Z; l'egw, <'oti <h >ep`i
t`a E, Z shme~ia >epizeugnum'enh e>uje~ia >en t~w| a\kern -.7pt >ut~w| >epip'edw|
>est`i ta~ic parall'hloic.}

\gr{m\kern -.7pt `h g'ar, >all> e>i dunat'on, >'estw >en metewrot'erw| <wc <h EHZ, ka`i
di'hqjw di`a t~hc EHZ >ep'ipedon; tom\kern -.7pt `hn d`h poi'hsei >en t~w| <upokeim'enw| >epip'edw| e>uje~ian. poie'itw <wc t`hn EZ; d'uo >'ara
e>uje~iai a<i EHZ, EZ qwr'ion peri'exousin; <'oper >est`in >ad'unaton.
o>uk >'ara <h >ap`o to~u E >ep`i t`o Z >epizeugnum'enh e>uje~iai >en
metewrot'erw| >est`in >epip'edw|; >en t~w| di`a t~wn AB, GB >'ara
parall'hlwn >est`in >epip'edw| <h >ap`o to~u E >ep`i t`o Z >epizeugnum'enh
e>uje~ia.}

\gr{>E`an >'ara >~wsi d'uo e>uje~iai par'allhloi, lhfj~h| d`e >ef> <ekat'erac
a\kern -.7pt >ut~wn tuq'onta shme~ia, <h >ep`i t`a shme~ia >epizeugnum'enh
e>uje~ia >en t~w| a\kern -.7pt >ut~w| >epip'edw| >est`i ta~ic parall'hloic; <'oper
>'edei de~ixai.}}

\ParallelRText{
\begin{center}
{\large Proposition 7}
\end{center}

If there are two parallel straight-lines, and
random points are taken on each of them, then the straight-line joining the
two points is in the same plane as the parallel (straight-lines).

\epsfysize=1.5in
\centerline{\epsffile{Book11/fig07e.eps}}

Let $AB$ and $CD$ be two parallel straight-lines, and let the random
points $E$ and $F$ be taken on each of them (respectively). I say that the straight-line joining points $E$ and $F$ is in the same (reference) plane as the parallel
(straight-lines).

For (if)  not, and if possible, let it be in a more elevated (plane), such as
$EGF$. And let a plane be drawn through $EGF$. So it will
make a straight cutting in the reference plane [Prop.~11.3]. Let it make $EF$. Thus, two straight-lines (with the
same end-points), $EGF$ and $EF$,  will enclose an area. The very thing is
impossible. Thus, the straight-line joining $E$ to $F$ is not in a more
elevated plane. The straight-line joining $E$ to $F$ is thus in the plane
through the parallel (straight-lines) $AB$ and $CD$.

Thus, if there are two parallel straight-lines, and
random points are taken on each of them, then the straight-line joining the
two points is in the same plane as the parallel (straight-lines). (Which is)
the very thing it was required to show.}
\end{Parallel}

%%%%
%11.8
%%%%
\pdfbookmark[1]{Proposition 11.8}{pdf11.8}
\begin{Parallel}{}{}
\ParallelLText{
\begin{center}
{\large \ggn{8}.}
\end{center}\vspace*{-7pt}

\gr{>E`an >~wsi d'uo e>uje~iai par'allhloi, <h d`e <et'era a\kern -.7pt >ut~wn >epip'edw|
tin`i pr`oc >orj`ac >~h|, ka`i <h loip`h t~w| a\kern -.7pt >ut~w| >epip'edw| pr`oc >orj`ac
>'estai.}

\gr{>'Estwsan d'uo e>uje~iai par'allhloi a<i AB, GD, <h d`e <et'era a\kern -.7pt >ut~wn
<h AB t~w| <upokeim'enw| >epip'edw| pr`oc >orj`ac >'estw; l'egw, <'oti
ka`i <h loip`h <h GD t~w| a\kern -.7pt >ut~w| >epip'edw| pr`oc >orj`ac >'estai.}

\gr{Sumball'etwsan g`ar a<i AB, GD t~w| <upokeim'enw| >epip'edw| kat`a t`a
B, D shme~ia, ka`i >epez'euqjw <h BD; a<i AB, GD, BD >'ara >en <en'i
e>isin >epip'edw|. >'hqjw t~h| BA pr`oc >orj`ac >en t~w| <upokeim'enw|
>epip'edw| <h DE, ka`i ke'isjw t~h| AB >'ish <h DE, ka`i >epeze'uqjwsan a<i
BE, AE, AD.}\\

\epsfysize=2.5in
\centerline{\epsffile{Book11/fig08g.eps}}

\gr{Ka`i >epe`i <h AB <orj'h >esti pr`oc t`o <upoke'imenon
>ep'ipedon, ka`i pr`oc p'asac >'ara t`ac <aptom'enac a\kern -.7pt >ut~hc e>uje'iac
ka`i o>'usac >en t~w| <upokeim'enw| >epip'edw| pr`oc >orj'ac
>estin <h AB; >orj`h >'ara [>est`in] <ekat'era t~wn <up`o ABD, ABE
gwni~wn. ka`i >epe`i e>ic parall'hlouc t`ac AB, GD e>uje~ia >emp'eptwken
<h BD, a<i >'ara <up`o ABD, GDB gwn'iai dus`in >orja~ic
>'isai e>is'in. >orj`h d`e <h <up`o ABD; >orj`h >'ara ka`i <h <up`o
GDB; <h GD >'ara pr`oc t`hn BD >orj'h >estin. ka`i >epe`i >'ish
>est`in <h AB t~h| DE, koin`h d`e <h BD, d'uo d`h a<i AB, BD dus`i ta~ic
ED, DB >'isai e>is'in; ka`i gwn'ia <h <up`o ABD gwn'ia| t~h| <up`o EDB >'ish;
>orj`h g`ar <ekat'era; b'asic >'ara <h AD b'asei t~h| BE >'ish. ka`i
>epe`i >'ish >est`in <h m`en AB t~h| DE, <h d`e BE t~h| AD, d'uo d`h
a<i AB, BE dus`i ta~ic ED, DA >'isai e>is`in <ekat'era <ekat'era|. ka`i
b'asic a\kern -.7pt >ut~wn koin`h <h AE; gwn'ia >'ara <h <up`o ABE gwn'ia| t~h|
<up`o EDA >estin >'ish. >orj`h d`e <h <up`o ABE; >orj`h >'ara
ka`i <h <up`o EDA; <h ED >'ara pr`oc t`hn AD >orj'h >estin.
>'esti d`e ka`i pr`oc t`hn DB >orj'h; <h ED >'ara ka`i t`w| di`a
t~wn BD, DA >epip'edw| >orj'h
 >estin. ka`i pr`oc p'asac >'ara t`ac <aptom'enac
a\kern -.7pt >ut~hc e>uje'iac ka`i o>'usac >en t~w| di`a t~wn BDA >epip'edw|
>orj`ac poi'hsei gwn'iac <h ED. >en d`e t~w| di`a t~wn BDA >epip'edw|
>est`in <h DG, >epeid'hper >en t~w| di`a t~wn BDA >epip'edw|
>est`in a<i AB, BD, >en <~w| d`e a<i AB, BD, >en to'utw| >est`i
ka`i <h DG. <h ED >'ara t~h| DG pr`oc >orj'ac >estin; <'wste ka`i <h GD
t~h| DE pr`oc >orj'ac >estin. >'esti d`e ka`i <h GD t~h| BD
pr`oc >orj'ac. <h GD >'ara d'uo e>uje'iaic temno'usaic >all'hlac
ta~ic DE, DB >ap`o t~hc kat`a t`o D tom\kern -.7pt ~hc pr`oc >orj`ac >ef'esthken;
<'wste <h GD ka`i t~w| di`a t~wn DE, DB
>epip'edw| pr`oc >orj'ac >estin. t`o d`e di`a t~wn DE, DB >ep'ipedon
t`o <upoke'imen'on >estin; <h GD >'ara t~w| <upokeim'enw| >epip'edw|
pr`oc >orj'ac >estin.}

\gr{>E`an >'ara >~wsi d'uo e>uje~iai par'allhloi, <h d`e m'ia a\kern -.7pt >ut~wn >epip'edw|
tin`i pr`oc >orj`ac >~h|, ka`i <h loip`h t~w|
a\kern -.7pt >ut~w| >epip'edw| pr`oc >orj`ac >'estai; <'oper
>'edei de~ixai.}}

\ParallelRText{
\begin{center}
{\large Proposition 8}
\end{center}

If two  straight-lines are parallel, and one of them is at right-angles to some plane, then the remaining (one) will also
be at right-angles to the same plane.

Let $AB$ and $CD$ be two parallel straight-lines, and let one of them, $AB$,
be at right-angles to a reference plane. I say that the remaining (one), $CD$,
will also be at right-angles to the same plane.

For let $AB$ and $CD$ meet the reference plane at points $B$ and $D$
(respectively). And let $BD$ be joined. $AB$, $CD$, and $BD$
are thus in one plane [Prop.~11.7]. 
Let $DE$ be drawn at right-angles to $BD$ in the reference plane, and
let $DE$ be made equal to $AB$, and let $BE$, $AE$, and $AD$ be
joined.

\epsfysize=2.5in
\centerline{\epsffile{Book11/fig08e.eps}}

And since $AB$ is at right-angles to the reference plane, $AB$ is thus also
at right-angles to all of the straight-lines joined to it which are in the
reference plane [Def.~11.3].  Thus, the angles $ABD$ and $ABE$ [are] each right-angles. And since the straight-line $BD$ has
met the parallel (straight-lines) $AB$ and $CD$, the (sum of the) angles
$ABD$ and $CDB$ is thus equal to two right-angles [Prop.~1.29]. And $ABD$ (is) a right-angle. Thus, $CDB$ (is)
also a right-angle. $CD$ is thus at right-angles to $BD$. And since $AB$
is equal to $DE$, and $BD$ (is) common, the two (straight-lines) $AB$ and
$BD$ are equal to the two (straight-lines) $ED$ and $DB$ (respectively). And angle
$ABD$ (is) equal to angle $EDB$. For each (is) a right-angle.
Thus, the base $AD$ (is) equal to the base $BE$ [Prop.~1.4]. And since $AB$ is equal to $DE$, and $BE$ to $AD$, 
the two (sides) $AB$, $BE$ are equal to the two (sides) $ED$, $DA$,
respectively. And their base $AE$ is common. Thus, angle $ABE$ is equal to
angle $EDA$ [Prop.~1.8]. And $ABE$ (is) a
right-angle. $EDA$ (is) thus also a right-angle. Thus, $ED$ is
at right-angles to $AD$. And it is also at right-angles to $DB$. Thus, $ED$
is also at right-angles to the plane through $BD$ and $DA$ [Prop.~11.4]. And $ED$ will thus  make right-angles
with all of the straight-lines joined to it which are also in the plane through
$BDA$. And $DC$ is in the plane through $BDA$, inasmuch as
$AB$ and $BD$ are in the plane through $BDA$ [Prop.~11.2], and in which(ever plane)
$AB$ and $BD$ (are found), $DC$ is also (found). Thus, $ED$ is at right-angles to $DC$. Hence, $CD$ is also at right-angles to $DE$. And
$CD$ is also at right-angles to $BD$. Thus, $CD$ is standing  at right-angles
to two straight-lines, $DE$ and $DB$, which meet one another, at the 
(point) of section, $D$. Hence, $CD$ is also at
right-angles to the plane through $DE$ and $DB$ [Prop.~11.4]. And the plane through $DE$ and $DB$ is the reference
(plane). $CD$ is thus at right-angles to the reference plane.

Thus, if  two  straight-lines are parallel, and one of them is at right-angles to some plane, then the remaining (one) will also
be at right-angles to the same plane. (Which is) the very thing it was required to show.}
\end{Parallel}

%%%%
%11.9
%%%%
\pdfbookmark[1]{Proposition 11.9}{pdf11.9}
\begin{Parallel}{}{}
\ParallelLText{
\begin{center}
{\large \ggn{9}.}
\end{center}\vspace*{-7pt}

\gr{A<i t~h| a\kern -.7pt >ut~h| e>uje'ia| par'allhloi ka`i m\kern -.7pt `h o>~usai a\kern -.7pt >ut~h| >en
t~w| a\kern -.7pt >ut~w| >epip'edw| ka`i >all'hlaic e>is`i par'allhloi.}\\

\epsfysize=1.6in
\centerline{\epsffile{Book11/fig09g.eps}}

\gr{>'Estw g`ar <ekat'era t~wn AB, GD t~h| EZ par'allhloc m\kern -.7pt `h o>~usai
a\kern -.7pt >ut~h| >en t~w| a\kern -.7pt >ut~w| >epip'edw|; l'egw, <'oti par'allhl'oc >estin
<h AB t~h| GD.}

\gr{E>il'hfjw g`ar >ep`i t~hc EZ tuq`on shme~ion t`o H, ka`i >ap>
a\kern -.7pt >uto~u t~h| EZ >en m`en t~w| di`a t~wn EZ, AB >epip'edw| pr`oc
>orj`ac >'hqjw <h HJ, >en d`e t~w| di`a t~wn ZE, GD t~h| EZ p'alin
pr`oc >orj`ac >'hqjw <h HK.}

\gr{Ka`i >epe`i <h EZ pr`oc <ekat'eran
t~wn HJ, HK >orj'h >estin, <h EZ
>'ara ka`i
t~w| di`a t~wn HJ, HK >epip'edw| pr`oc >orj'ac >estin. ka'i >estin
<h EZ t~h| AB par'allhloc; ka`i <h AB >'ara t~w| di`a t~wn JHK >epip'edw|
pr`oc >orj'ac >estin. di`a t`a a\kern -.7pt >ut`a d`h ka`i <h GD t~w| di`a t~wn
JHK >epip'edw| pr`oc >orj'ac >estin; <ekat'era >'ara t~wn AB, GD t~w|
di`a t~wn JHK >epip'edw| pr`oc >orj'ac >estin. >e`an d`e d'uo e>uje~iai
t~w| a\kern -.7pt >ut~w| >epip'edw| pr`oc >orj`ac >~wsin, par'allhlo'i e>isin
a<i e>uje~iai; par'allhloc >'ara >est`in <h AB t~h| GD; <'oper
>'edei de~ixai.}}

\ParallelRText{
\begin{center}
{\large Proposition 9}
\end{center}

(Straight-lines) parallel to the same straight-line,
and which are not in the same plane as it, are also parallel to one another.

\epsfysize=1.6in
\centerline{\epsffile{Book11/fig09e.eps}}

For let $AB$ and $CD$ each be parallel to $EF$, not being in the
same plane as it. I say that $AB$ is parallel to $CD$.

For let some point $G$ be taken at random on $EF$. And from it
let $GH$ be drawn at right-angles to $EF$ in the plane through
$EF$ and $AB$. And let $GK$ be drawn, again at right-angles
to $EF$, in the plane through  $FE$ and $CD$.

And since $EF$ is at right-angles to each of $GH$ and $GK$, $EF$
is thus also at right-angles to the plane through $GH$ and $GK$ 
[Prop.~11.4]. And $EF$ is parallel to
$AB$. Thus, $AB$ is also at right-angles to the plane through $HGK$
[Prop.~11.8]. So, for the same (reasons), $CD$
is also at right-angles to the plane through $HGK$.  Thus, $AB$ and
$CD$ are each at right-angles to the plane through $HGK$.
And if two straight-lines are at right--angles to the same plane then the 
straight-lines are parallel [Prop.~11.6]. Thus, $AB$ is parallel to
$CD$. (Which is) the very thing it was required to show.}
\end{Parallel}

%%%%
%11.10
%%%%
\pdfbookmark[1]{Proposition 11.10}{pdf11.10}
\begin{Parallel}{}{}
\ParallelLText{
\begin{center}
{\large \ggn{10}.}
\end{center}\vspace*{-7pt}

\gr{>E`an d'uo e>uje~iai <apt'omenai >all'hlwn par`a d'uo e>uje'iac <aptom'enac
>all'hlwn >~wsi m\kern -.7pt `h >en t~w| a\kern -.7pt >ut~w| >epip'edw|, <'isac gwn'iac
peri'exousin.}

\gr{D'uo g`ar e>uje~iai a<i AB, BG <apt'omenai >all'hlwn par`a d'uo
e>uje'iac t`ac DE, EZ <aptom'enac >all'hlwn >'estwsan m\kern -.7pt `h >en
t~w| a\kern -.7pt >ut~w| >epip'edw|; l'egw, <'oti >'ish >est`in <h <up`o
ABG gwn'ia t~h| <up`o DEZ.}

\gr{>Apeil'hfjwsan g`ar a<i BA, BG, ED, EZ >'isai >all'hlaic, ka`i
>epeze'uqjwsan a<i AD, GZ, BE, AG, DZ.}

\gr{Ka`i >epe`i <h BA
t~h| ED >'ish >est`i ka`i par'allhloc, ka`i <h AD >'ara t~h| BE >'ish
>est`i ka`i par'allhloc.
 di`a t`a a\kern -.7pt >ut`a d`h ka`i <h GZ t~h| BE >'ish >est`i ka`i par'allhloc; <ekat'era
 >'ara t~wn AD, GZ t~h| BE >'ish >est`i ka`i par'allhloc. a<i d`e t~h|
 a\kern -.7pt >ut~h| e>uje'ia| par'allhloi ka`i m\kern -.7pt `h o>~usai a\kern -.7pt >ut~h| >en t~w| a\kern -.7pt >ut~w|
  >epip'edw| ka`i >all'hlaic e>is`i par'allhloi; par'allhloc >'ara >est`in
 <h AD t~h| GZ ka`i >'ish. ka`i >epizeugn'uousin a\kern -.7pt >ut`ac a<i
 AG, DZ; ka`i <h AG >'ara t~h| DZ >'ish >est`i ka`i par'allhloc.
 ka`i >epe`i d'uo a<i AB, BG dus`i ta~ic DE, EZ >'isai e>is'in,
 ka`i b'asic <h AG b'asei t~h| DZ >'ish, gwn'ia >'ara <h <up`o
 ABG gwn'ia| t~h| <up`o DEZ >estin >'ish.}\\~\\

\epsfysize=2.in
\centerline{\epsffile{Book11/fig10g.eps}}
 
 \gr{>E`an >'ara d'uo e>uje~iai <apt'omenai >all'hlwn par`a d'uo e>uje'iac <aptom'enac
>all'hlwn >~wsi m\kern -.7pt `h >en t~w| a\kern -.7pt >ut~w| >epip'edw|, <'isac gwn'iac
peri'exousin; <'oper >'edei de~ixai.}}

\ParallelRText{
\begin{center}
{\large Proposition 10}
\end{center}

If two straight-lines joined to one another
are (respectively) parallel to two straight-lines joined to one another, (but are) not in the same
plane, then they will contain equal angles.

For let the two straight-lines joined to one another, $AB$ and $BC$, be (respectively) parallel to the two straight-lines joined to one another, $DE$ and $EF$, (but) not in the same plane. I say that angle $ABC$ is equal to
(angle) $DEF$.

For let $BA$, $BC$, $ED$, and $EF$ be cut off (so as to be, respectively) equal
to one another. And let $AD$, $CF$, $BE$, $AC$, and $DF$
be joined.

And since $BA$ is equal and parallel to $ED$, $AD$ is thus also equal
and parallel to $BE$ [Prop.~1.33]. So, for the
same reasons, $CF$ is also equal and parallel to $BE$. Thus, $AD$ and
$CF$ are each equal and parallel to $BE$. And straight-lines parallel to
the same straight-line, and which are not in the same plane as it, are also parallel
to one another [Prop.~11.9]. Thus, $AD$
is parallel and equal to $CF$. And $AC$ and $DF$ join them. Thus, $AC$
is also equal and parallel to $DF$ [Prop.~1.33].
And since the two (straight-lines) $AB$ and $BC$ are equal to the
two (straight-lines) $DE$ and $EF$ (respectively),  and the base $AC$  (is) equal to the base $DF$, the angle $ABC$ is thus equal to the (angle) $DEF$   
[Prop.~1.8].

\epsfysize=2.in
\centerline{\epsffile{Book11/fig10e.eps}}

Thus, if two straight-lines joined to one another
are (respectively) parallel to two straight-lines joined to one another, (but are) not in the same
plane, then they will contain equal angles. (Which is) the very thing it
was required to show.}
\end{Parallel}

%%%%
%11.11
%%%%
\pdfbookmark[1]{Proposition 11.11}{pdf11.11}
\begin{Parallel}{}{}
\ParallelLText{
\begin{center}
{\large \ggn{11}.}
\end{center}\vspace*{-7pt}

\gr{>Ap`o to~u doj'entoc shme'iou mete'wrou >ep`i t`o doj`en >ep'ipedon
k'ajeton e>uje~ian gramm\kern -.7pt `hn >agage~in.}

\epsfysize=2.5in
\centerline{\epsffile{Book11/fig11g.eps}}

\gr{>'Estw t`o m`en doj`en shme~ion met'ewron t`o A, t`o d`e doj`en >ep'ipedon
t`o <upoke'imenon; de~i d`h >ap`o to~u A shme'iou >ep`i t`o <upoke'imenon
>ep'ipedon k'ajeton e>uje~ian gramm\kern -.7pt `hn >agage~in.}

\gr{Di'hqjw g'ar tic >en t~w| <upokeim'enw| >epip'edw| e>uje~ia,
<wc >'etuqen, <h BG, ka`i >'hqjw >ap`o to~u A shme'iou >ep`i t`hn BG
k'ajetoc <h AD. e>i m`en o>~un <h AD k'ajet'oc >esti ka`i >ep`i t`o
<upoke'imenon >ep'ipedon, gegon`oc >`an e>'ih
t`o >epitaqj'en. e>i d`e o>'u, >'hqjw >ap`o to~u D shme'iou t~h| BG
>en t~w| <upokeim'enw| >epip'edw| pr`oc >orj`ac <h DE, ka`i
>'hqjw >ap`o to~u A >ep`i t`hn DE k'ajetoc <h AZ, ka`i di`a to~u Z
shme'iou t~h| BG par'allhloc >'hqjw <h HJ.}

\gr{Ka`i >epe`i <h BG <ekat'era| t~wn DA, DE pr`oc >orj'ac >estin,
<h BG >'ara ka`i t~w| di`a t~wn EDA >epip'edw| pr`oc >orj'ac
>estin. ka'i >estin a\kern -.7pt >ut~h| par'allhloc <h HJ; >e`an d`e >~wsi
d'uo e>uje~iai par'allhloi, <h d`e m'ia a\kern -.7pt >ut~wn >epip'edw|
tin`i pr`oc >orj`ac >~h|, ka`i <h loip`h t~w| a\kern -.7pt >ut~w| >epip'edw|
pr`oc >orj`ac >'estai; ka`i <h HJ >'ara t~w| di`a t~wn ED, DA >epip'edw|
pr`oc >orj'ac >estin. ka`i pr`oc p'asac >'ara t`ac <aptom'enac a\kern -.7pt >ut~hc
e>uje'iac ka`i o>'usac >en t~w| di`a t~wn ED, DA >epip'edw| >orj'h
>estin <h HJ. <'aptetai d`e a\kern -.7pt >ut~hc <h AZ o>~usa >en t~w| di`a
t~wn ED, DA >epip'edw|; <h HJ >'ara >orj'h >esti pr`oc
t`hn ZA; <'wste ka`i <h ZA >orj'h >esti pr`oc t`hn JH. >'esti
d`e <h AZ ka`i pr`oc t`hn DE >orj'h; <h AZ >'ara pr`oc <ekat'eran
t~wn HJ, DE >orj'h >estin. >e`an d`e e>uje~ia dus`in e>uje'iaic temno'usaic
>all'hlac >ep`i t~hc tom\kern -.7pt ~hc pr`oc >orj`ac 
>epistaj~h|, ka`i t~w| di> a\kern -.7pt >ut~wn >epip'edw| pr`oc >orj`ac
>'estai; <h ZA >'ara t~w|
di`a t~wn ED, HJ >epip'edw| pr`oc >orj'ac >estin. t`o d`e di`a t~wn
ED, HJ >ep'iped'on >esti t`o <upoke'imenon; <h AZ >'ara t~w|
<upokeim'enw| >epip'edw| pr`oc >orj'ac >estin.}

\gr{>Ap`o to~u >'ara doj'entoc shme'iou mete'wrou to~u A >ep`i t`o <upoke'imenon >ep'ipedon k'ajetoc e>uje~ia gramm\kern -.7pt `h >~hktai <h AZ;
 <'oper >'edei poi~hsai.}}
 
\ParallelRText{
\begin{center}
{\large Proposition 11}
\end{center}

To draw a perpendicular straight-line from a
given raised point to a given plane.

\epsfysize=2.5in
\centerline{\epsffile{Book11/fig11e.eps}}

Let $A$ be the given raised point, and the given plane the reference (plane).
So, it is required to draw a perpendicular straight-line from point $A$
to the reference plane.

Let some random straight-line $BC$ be drawn across in the reference plane, and let the  (straight-line) $AD$ be
drawn from point $A$ perpendicular to $BC$ [Prop.~1.12].
If, therefore, $AD$ is also perpendicular to the reference plane then that which
was prescribed will have occurred. And, if not, let $DE$ be
drawn in the reference plane from point $D$  at right-angles to $BC$ [Prop.~1.11], and let the (straight-line) $AF$
be drawn from $A$ perpendicular to $DE$ [Prop.~1.12],
and let $GH$ be drawn through point $F$, parallel to $BC$ [Prop.~1.31].

And since $BC$ is at right-angles to each of $DA$ and $DE$, $BC$ is
thus also at right-angles to the plane through $EDA$ [Prop.~11.4]. And $GH$ is parallel to it. And if two 
straight-lines are parallel, and one of them is at right-angles to some plane, then the
remaining (straight-line) will also be at right-angles to the same plane
[Prop.~11.8]. Thus, $GH$ is also
at right-angles to the plane through $ED$ and $DA$. And $GH$ is thus
at right-angles to all of the straight-lines joined to it which are also
in the plane through $ED$ and $AD$ [Def.~11.3].
And $AF$, which is in the plane through $ED$ and $DA$, is joined to it.
Thus, $GH$ is at right-angles to $FA$. Hence, $FA$ is also
at right-angles to $HG$. And $AF$ is also at right-angles to $DE$.
Thus, $AF$ is at right-angles to each of $GH$ and $DE$. And if a straight-line is set up at right-angles to two straight-lines cutting one another, at the
point of section, then it will also be at right-angles to the plane through them
[Prop.~11.4]. Thus, $FA$ is at right-angles to the
plane through $ED$ and $GH$. And the plane through $ED$ and
$GH$ is the reference (plane). Thus, $AF$ is at right-angles to
the reference plane.

Thus, the  straight-line $AF$ has been drawn from the given raised point 
$A$ perpendicular to the reference plane. (Which is) the very thing it was required to do.}
\end{Parallel}

%%%%
%11.12
%%%%
\pdfbookmark[1]{Proposition 11.12}{pdf11.12}
\begin{Parallel}{}{}
\ParallelLText{
\begin{center}
{\large \ggn{12}.}
\end{center}\vspace*{-7pt}

\gr{T~w| doj'enti >epip'edw| >ap`o to~u pr`oc a\kern -.7pt >ut~w| doj'entoc shme'iou
pr`oc >orj`ac e>uje~ian gramm\kern -.7pt `hn >anast~hsai.}

\epsfysize=2.75in
\centerline{\epsffile{Book11/fig12g.eps}}

\gr{>'Estw t`o m`en doj`en >ep'ipedon t`o <upoke'imenon, t`o d`e pr`oc
a\kern -.7pt >ut~w| shme~ion t`o A; de~i d`h >ap`o to~u A shme'iou
t~w| <upokeim'enw| >epip'edw| pr`oc >orj`ac e>uje~ian gramm\kern -.7pt `hn
>anast~hsai.}

\gr{Neno'hsjw ti shme~ion met'ewron t`o B, ka`i >ap`o to~u B >ep`i
t`o <upoke'imenon >ep'ipedon k'ajetoc >'hqjw <h BG, ka`i di`a
to~u A shme'iou t~h| BG par'allhloc >'hqjw <h AD.}

\gr{>Epe`i o~>un d'uo e>uje~iai par'allhlo'i e>isin a<i AD, GB, <h d`e m'ia
a\kern -.7pt >ut~wn <h BG t~w| <upokeim'enw| >epip'edw| pr`oc >orj'ac
>estin, ka`i <h loip`h >'ara <h AD t~w| <upokeim'enw| >epip'edw|
pr`oc >orj`ac >estin.}

\gr{T~w| >'ara doj'enti >epip'edw| >ap`o to~u pr`oc a\kern -.7pt >ut~w| shme'iou
to~u A pr`oc >orj`ac >an'estatai <h AD; <'oper >'edei poi~hsai.
}}

\ParallelRText{
\begin{center}
{\large Proposition 12}
\end{center}

To set up a straight-line at right-angles to a given plane from a given point in it.

\epsfysize=2.75in
\centerline{\epsffile{Book11/fig12e.eps}}

Let the given plane be the reference (plane), and $A$ a point in it. So, it is required to set up a straight-line at right-angles to the reference plane at point $A$.

Let some raised point $B$ be assumed, and let the perpendicular
(straight-line) $BC$ be drawn from $B$ to the reference plane
[Prop.~11.11]. And let $AD$ be drawn
from point $A$ parallel to $BC$ [Prop.~1.31].

Therefore, since $AD$ and $CB$ are two parallel straight-lines, and one of
them, $BC$, is at right-angles to the reference plane, the remaining (one) $AD$
is thus also at right-angles to the reference plane [Prop.~11.8].

Thus, $AD$ has been set up at right-angles to the given plane, from the
 point in it, $A$. (Which is) the very thing it was required to do.}
\end{Parallel}

%%%%
%11.13
%%%%
\pdfbookmark[1]{Proposition 11.13}{pdf11.13}
\begin{Parallel}{}{}
\ParallelLText{
\begin{center}
{\large \ggn{13}.}
\end{center}\vspace*{-7pt}

\gr{>Ap`o to~u a\kern -.7pt >uto~u shme'iou t~w| a\kern -.7pt >ut~w| >epip'edw| d'uo e>uje~iai
pr`oc >orj`ac o>uk >anast'hsontai >ep`i t`a a\kern -.7pt >ut`a m'erh.}

\epsfysize=2.5in
\centerline{\epsffile{Book11/fig13g.eps}}

\gr{E>i g`ar dunat'on, >ap`o to~u a\kern -.7pt >uto~u shme'iou to~u A t~w| <upokeim'enw|
>epip'edw| d'uo e>uje~iai a<i AB, BG pr`oc >orj`ac >anest'atwsan >ep`i t`a
a\kern -.7pt >ut`a m'erh, ka`i di'hqjw t`o di`a t~wn BA, AG >ep`ipedon; tom\kern -.7pt `hn
d`h poi'hsei di`a to~u A >en t~w| <upokeim'enw| >epip'edw| e>uje~ian. poie'itw t`hn DAE; a<i >'ara AB, AG, DAE e>uje~iai >en <eni e>isin
>epip'edw|. ka`i >epe`i <h GA t~w| <upokeim'enw| >epip'edw| pr`oc >orj'ac
>estin, ka`i pr`oc p'asac >'ara t`ac <aptom'enac a\kern -.7pt >ut~hc e>uje'iac ka`i 
o>'usac >en t~w| <upokeim'enw| >epip'edw| >orj`ac poi'hsei gwn'iac.
<'aptetai d`e a\kern -.7pt >ut~hc <h DAE o>~usa >en t~w| <upokeim'enw| >epip'edw|;
<h >'ara <up`o GAE gwn'ia >orj'h >estin. di`a t`a a\kern -.7pt >ut`a d`h ka`i <h <up`o 
BAE >orj'h >estin; >'ish >'ara <h <up`o GAE t~h| <up`o BAE  ka'i e>isin >en <en`i
>epip'edw|; <'oper >est`in >ad'unaton.}

\gr{O>uk >'ara >ap`o to~u a\kern -.7pt >uto~u shme'iou t~w| a\kern -.7pt >ut~w| >epip'edw| d'uo e>uje~iai
pr`oc >orj`ac >anastaj'hsontai >ep`i t`a a\kern -.7pt >ut`a m'erh; <'oper >'edei de~ixai.}}

\ParallelRText{
\begin{center}
{\large Proposition 13}
\end{center}

Two (different) straight-lines cannot be set up at the same point at right-angles to the same plane, on the same side.

\epsfysize=2.5in
\centerline{\epsffile{Book11/fig13e.eps}}

For, if possible, let the two straight-lines $AB$ and $AC$ be
set up at the same point $A$ at right-angles to the reference plane, on the same
side. And let the plane through $BA$ and $AC$ be drawn. So
it will make a straight cutting (passing) through (point) $A$ in the reference plane [Prop.~11.3]. Let it have made $DAE$. Thus, $AB$,
$AC$, and $DAE$ are straight-lines in one plane. And
since $CA$ is at right-angles to the reference plane, it will thus
also make right-angles with all of the straight-lines  joined to it which are
also in the reference plane [Def.~11.3]. 
And $DAE$, which is in the reference plane, is joined to it. Thus, angle
$CAE$ is a right-angle. So, for the same (reasons), $BAE$ is also
a right-angle. Thus, $CAE$ (is) equal to  $BAE$. And they are in one
plane. The very thing is impossible.

Thus, two (different) straight-lines cannot be set up at the same point at right-angles to the same plane, on the same side. (Which is) the very thing it was required to show.\\~\\}
\end{Parallel}

%%%%
%11.14
%%%%
\pdfbookmark[1]{Proposition 11.14}{pdf11.14}
\begin{Parallel}{}{}
\ParallelLText{
\begin{center}
{\large \ggn{14}.}
\end{center}\vspace*{-7pt}

\gr{Pr`oc <`a >ep'ipeda <h a\kern -.7pt >ut`h e>uje~ia >orj'h >estin,
par'allhla >'estai t`a >ep'ipeda.}

\gr{E>uje~ia g'ar tic <h AB pr`oc <ek'ateron t~wn GD, EZ
>epip'edwn pr`oc >orj`ac >'estw; l'egw, <'oti par'allhl'a >esti t`a
>ep'ipeda.}

\epsfysize=2.in
\centerline{\epsffile{Book11/fig14g.eps}}

\gr{E>i g`ar m\kern -.7pt 'h, >ekball'omena sumpeso~untai. sumpipt'et\-wsan; poi'hsousi
d`h koin`hn tom\kern -.7pt `hn e>uje~ian. poie'itwsan t`hn HJ, ka`i e>il'hfjw
>ep`i t~hc HJ tuq`on shme~ion t`o K, ka`i >epeze'uqjwsan a<i AK,
BK.}

\gr{Ka`i >epe`i <h AB >orj'h >esti pr`oc t`o EZ >ep'ipedon,
ka`i pr`oc t`hn BK >'ara e>uje~ian o>~usan >en t~w| EZ >ekblhj'enti
>epip'edw| >orj'h >estin <h AB; <h >'ara <up`o ABK
gwn'ia >orj'h >estin. di`a t`a a\kern -.7pt >ut`a d`h ka`i <h <up`o
BAK >orj'h >estin. trig'wnou d`h to~u ABK a<i d'uo gwn'iai
a<i <up`o ABK, BAK dus`in >orja~ic e>isin >'isai;
<'oper >est`in >ad'unaton. o>uk >'ara t`a GD, EZ >ep'ipeda
>ekball'omena sumpeso~untai; par'allhla >'ara >est`i t`a GD, EZ
>ep'ipeda.}

\gr{Pr`oc <`a >ep'ipeda >'ara <h a\kern -.7pt >ut`h e>uje~ia >orj'h >estin,
par'allhl'a >esti t`a >ep'ipeda; <'oper >'edei de~ixai.}}

\ParallelRText{
\begin{center}
{\large Proposition 14}
\end{center}

Planes to which the same straight-line
is at right-angles will be  parallel planes.

For let some straight-line $AB$ be at right-angles to each of the planes
$CD$ and $EF$. I say that the planes are parallel.\\

\epsfysize=2.in
\centerline{\epsffile{Book11/fig14e.eps}}

For, if not, being produced, they will meet. Let them have
met. So they will make a straight-line as a common section [Prop.~11.3]. Let them have made $GH$. And let some
random point $K$ be taken on $GH$. And let $AK$ and
$BK$ be joined.

And since $AB$ is at right-angles to the plane $EF$, $AB$ is thus also
at right-angles to  $BK$, which is a straight-line in the produced plane $EF$ [Def.~11.3]. 
Thus, angle $ABK$ is a right-angle.
So, for the same (reasons), $BAK$ is also a right-angle. So the (sum of the)
two angles $ABK$ and $BAK$ in the triangle $ABK$ is equal to
two right-angles. The very thing is impossible [Prop.~1.17]. Thus, planes $CD$ and $EF$, being produced, will not meet.
Planes $CD$ and $EF$ are thus parallel [Def.~11.8].

Thus, planes to which the same straight-line
is at right-angles are  parallel planes. (Which is) the very thing it was required to show.}
\end{Parallel}

%%%%
%11.15
%%%%
\pdfbookmark[1]{Proposition 11.15}{pdf11.15}
\begin{Parallel}{}{}
\ParallelLText{
\begin{center}
{\large \ggn{15}.}
\end{center}\vspace*{-7pt}

\gr{>E`an d'uo e>uje~iai <apt'omenai >all'hlwn par`a d'uo e>uje'iac <aptom'enac
>all'hlwn >~wsi m\kern -.7pt `h >en t~w| a\kern -.7pt >ut~w| >epip'edw| o>~usai, par'allhl'a
>esti t`a di> a\kern -.7pt >ut~wn >ep'ipeda.}

\gr{D'uo g`ar e>uje~iai <apt'omenai >all'hlwn a<i AB, BG par`a d'uo e>uje'iac
<aptom'enac >all'hlwn t`ac DE, EZ >'estwsan m\kern -.7pt `h >en t~w| a\kern -.7pt >ut~w|
>epip'edw| o>~usai; l'egw, <'oti >ekball'omena t`a di`a t~wn
AB, BG, DE, EZ >ep'ipeda o>u sumpese~itai >all'hloic.}

\gr{>'Hqjw g`ar >ap`o to~u B shme'iou >ep`i t`o di`a t~wn DE, EZ
>ep'ipedon k'ajetoc <h BH ka`i sumball'etw t~w| >epip'edw|
kat`a t`o H shme~ion, ka`i di`a to~u H t~h| m`en ED par'allhloc
>'hqjw <h HJ, t~h| d`e EZ <h HK.}\\~\\

\epsfysize=2.2in
\centerline{\epsffile{Book11/fig15g.eps}}

\gr{Ka`i >epe`i <h BH >orj'h >esti
pr`oc t`o di`a t~wn DE, EZ >ep'ipedon, ka`i pr`oc p'asac
>'ara t`ac <aptom'enac a\kern -.7pt >ut~hc e>uje'iac ka`i o>'usac >en t~w| di`a t~wn
DE, EZ >epip'edw| >orj`ac poi'hsei gwn'iac. <'aptetai d`e a\kern -.7pt >ut~hc
<ekat'era t~wn HJ, HK o>~usa >en t~w| di`a t~wn DE, EZ >epip'edw|;
>orj`h >'ara >est`in <ekat'era t~wn <up`o BHJ, BHK gwni~wn. ka`i
>epe`i par'allhl'oc >estin <h BA t~h| HJ, a<i >'ara <up`o HBA, BHJ
gwn'iai dus`in >orja~ic >'isai e>is'in.
>orj`h d`e <h <up`o BHJ; >orj`h >'ara ka`i <h <up`o HBA; <h HB
>'ara t~h| BA pr`oc >orj'ac >estin. di`a t`a a\kern -.7pt >ut`a d`h <h HB
 ka`i t~h| BG >esti pr`oc >orj'ac. >epe`i o>~un e>uje~ia <h HB
dus`in e>uje'iaic ta~ic BA, BG temno'usaic >all'hlac pr`oc >orj`ac
>ef'esthken, <h HB >'ara ka`i t~w| di`a t~wn BA, BG >epip'edw|
pr`oc >orj'ac >estin. [di`a t`a a\kern -.7pt >ut`a d`h <h BH ka`i t~w| di`a t~wn
HJ, HK >epip'edw| pr`oc >orj'ac >estin. t`o d`e di`a t~wn HJ, HK
>ep'iped'on >esti t`o di`a t~wn DE, EZ; <h BH >'ara t~w| di`a t~wn
DE, EZ >epip'edw| >est`i pr`oc >orj'ac. >ede'iqjh
d`e <h HB ka`i t~w| di`a t~wn AB, BG >epip'edw| pr`oc
>orj'ac]. pr`oc <`a d`e >ep'ipeda <h a\kern -.7pt >ut`h e>uje~ia >orj'h
>estin, par'allhl'a >esti t`a >ep'ipeda; par'allhlon >'ara >est`i t`o di`a
t~wn AB, BG >ep'ipedon t~w| di`a t~wn DE, EZ.}

\gr{>E`an >'ara d'uo e>uje~iai <apt'omenai >all'hlwn par`a d'uo e>uje'iac <aptom'enac
>all'hlwn >~wsi m\kern -.7pt `h >en t~w| a\kern -.7pt >ut~w| >epip'edw|, par'allhl'a
>esti t`a di> a\kern -.7pt >ut~wn >ep'ipeda; <'oper >'edei de~ixai.}}

\ParallelRText{
\begin{center}
{\large Proposition 15}
\end{center}

If two straight-lines joined to one another
are parallel (respectively) to two straight-lines joined to one another, which
are not in the same plane,
then the planes through them are parallel (to one another).

For let the two straight-lines joined to one another, $AB$ and $BC$, be parallel to the two straight-lines joined to one another, $DE$ and $EF$ (respectively), not being in the same plane. I say that the planes
through $AB$, $BC$ and $DE$, $EF$ will not meet one another (when) produced.

For let $BG$ be drawn from point $B$ perpendicular to the
plane through $DE$ and $EF$ [Prop.~11.11], and
let it meet the plane at point $G$. And let $GH$ be drawn through
$G$ parallel to $ED$, and $GK$ (parallel) to $EF$ [Prop.~1.31].

\epsfysize=2.2in
\centerline{\epsffile{Book11/fig15e.eps}}

And since $BG$ is at right-angles to the plane through $DE$ and $EF$,
it will thus also make right-angles with all of the straight-lines joined to it, which
are also in the plane through $DE$ and $EF$ [Def.~11.3]. And each of $GH$ and
$GK$, which are in the plane through $DE$ and $EF$,  are joined to it. 
Thus, each of the angles $BGH$ and $BGK$ are right-angles. And since
$BA$ is parallel to $GH$ [Prop.~11.9], the (sum of the) angles $GBA$ and $BGH$
is equal to two right-angles [Prop.~1.29]. 
And $BGH$ (is) a right-angle. $GBA$ (is) thus also a right-angle.
Thus, $GB$ is at right-angles to $BA$. So, for the same (reasons), $GB$
is also at right-angles to $BC$. Therefore, since the straight-line $GB$
has been set up at right-angles to two straight-lines, $BA$ and $BC$, cutting
one another, $GB$ is thus at right-angles to the plane through $BA$ and
$BC$ [Prop.~11.4]. [So, for the same (reasons), $BG$ is also at right-angles to the plane through $GH$ and $GK$. And the plane
through $GH$ and $GK$ is the (plane) through $DE$ and $EF$. And it was also shown that $GB$ is at right-angles to the plane through $AB$ and
$BC$.] And planes to which the same straight-line is at right-angles are
parallel planes [Prop.~11.14]. Thus, the plane
through $AB$ and $BC$ is parallel to the (plane) through $DE$ and $EF$.

Thus, if two straight-lines joined to one another
are parallel (respectively) to two straight-lines joined to one another, which
are not in the same plane,
then the planes through them are parallel (to one another). (Which is) the very thing it was required to show.}
\end{Parallel}

%%%%
%11.16
%%%%
\pdfbookmark[1]{Proposition 11.16}{pdf11.16}
\begin{Parallel}{}{}
\ParallelLText{
\begin{center}
{\large \ggn{16}.}
\end{center}\vspace*{-7pt}

\gr{>E`an d'uo >ep'ipeda par'allhla <up`o >epip'edou tin`oc t'emnhtai, a<i
koina`i a\kern -.7pt >ut~wn toma`i par'allhlo'i e>isin.}

\gr{D'uo g`ar >ep'ipeda par'allhla t`a AB, GD <up`o >epip'edou to~u EZHJ
temn'esjw, koina`i d`e a\kern -.7pt >ut~wn toma`i >'estwsan a<i EZ, HJ;
l'egw, <'oti par'allhl'oc >estin <h EZ t~h| HJ.}

\gr{E>i g`ar m\kern -.7pt 'h, >ekball'omenai a<i EZ, HJ >'htoi >ep`i t`a Z, J m'erh >`h
>ep`i t`a E, H sumpeso~untai. >ekbebl'hsjwsan <wc >ep`i t`a
Z, J m'erh ka`i sumpipt'etwsan pr'oteron kat`a t`o K. ka`i >epe`i
<h EZK >en t~w| AB >estin >epip'edw|, ka`i p'anta >'ara t`a >ep`i 
t~hc EZK shme~ia >en t~w| AB >estin >epip'edw|. <`en d`e t~wn 
>ep`i t~hc EZK e>uje'iac shme'iwn >est`i t`o K; t`o K >'ara >en t~w|
AB >estin >epip'edw|. di`a  t`a a\kern -.7pt >ut`a d`h t`o K ka`i >en t~w| GD 
>estin >epip'edw|; t`a AB, GD >'ara >ep'ipeda >ekball'omena sumpeso~untai.
o>u sump'iptousi d`e di`a t`o par'allhla <upoke~isjai; o>uk >'ara a<i 
EZ, HJ e>uje~iai >ekball'omenai  >ep`i t`a Z, J m'erh sumpeso~untai.
<omo'iwc d`h de'ixomen, <'oti a<i EZ, HJ e>uje~iai  o>ud'e >ep`i 
t`a E, H m'erh >ekball'omenai sumpeso~untai. a<i d`e >ep`i m\kern -.7pt hd'etera 
t`a m'erh sump'iptousai par'allhlo'i e>isin. par'allhloc >'ara 
>est`in  <h EZ t~h| HJ.}\\

\epsfysize=2.in
\centerline{\epsffile{Book11/fig16g.eps}}

\gr{>E`an >'ara d'uo >ep'ipeda par'allhla <up`o >epip'edou tin`oc t'emnhtai, a<i
koina`i a\kern -.7pt >ut~wn toma`i par'allhlo'i e>isin; <'oper >'edei de~ixai.}}

\ParallelRText{
\begin{center}
{\large Proposition 16}
\end{center}

If two parallel planes are cut by some plane
then their common sections are parallel.

For let the two parallel planes $AB$ and $CD$ be cut by the
plane $EFGH$. And let $EF$ and $GH$ be their common sections.
I say that $EF$ is parallel to $GH$.

For, if not, being produced, $EF$ and $GH$ will  meet either in the direction of $F$, $H$, 
 or of $E$, $G$. Let them  be produced, as  in the  direction of $F$, $H$, and let them, first of all, have met at $K$. And since $EFK$ is in the plane $AB$, all
 of the points on $EFK$ are thus also in the plane $AB$ [Prop.~11.1]. And $K$ is one of the points on $EFK$. Thus, $K$ is in the
 plane $AB$. So, for the same (reasons), $K$ is also in the plane $CD$. 
 Thus, the planes $AB$ and $CD$, being produced, will meet. But they do
 not meet, on account of being (initially) assumed (to be mutually) parallel. Thus,
 the straight-lines $EF$ and $GH$, being produced in the
direction of $F$, $H$, will not meet. So, similarly, we can show that the straight-lines
 $EF$ and $GH$, being produced in the direction of $E$, $G$, will not meet
 either. And (straight-lines in one plane which), being produced,  do not meet in either direction
 are parallel [Def.~1.23]. $EF$ is thus parallel to $GH$.

\epsfysize=2.in
\centerline{\epsffile{Book11/fig16e.eps}}
 
 Thus, if two parallel planes are cut by some plane
then their common sections are parallel. (Which is) the very thing it was
required to show.}
\end{Parallel}

%%%%
%11.17
%%%%
\pdfbookmark[1]{Proposition 11.17}{pdf11.17}
\begin{Parallel}{}{}
\ParallelLText{
\begin{center}
{\large \ggn{17}.}
\end{center}\vspace*{-7pt}

\gr{>E`an d'uo e>uje~iai <up`o parall'hlwn >epipedwn t'emnwntai, e>ic
to`uc a\kern -.7pt >uto`uc l'ogouc tm\kern -.7pt hj'hsontai.}

\gr{D'uo g`ar e>uje~iai a<i AB, GD <up`o parall'hlwn >epip'edwn t~wn HJ, KL,
MN temn'esjwsan kat`a t`a A, E, B, G, Z, D shme~ia; l'egw, <'oti >est`in
<wc <h AE e>uje~ia pr`oc t`hn  EB, o<'utwc <h GZ pr`oc t`hn ZD.}

\gr{>Epeze'uqjwsan g`ar a<i AG, BD, AD, ka`i sumball'etw <h AD t~w| KL
>epip'edw| kat`a t`o X shme~ion, ka`i >epeze'uqjwsan a<i EX, XZ.}

\gr{Ka`i >epe`i d'uo >ep'ipeda par'allhla t`a KL, MN <up`o >epip'edou
to~u EBDX t'emnetai, a<i koina`i a\kern -.7pt >ut~wn toma`i a<i EX, BD
par'allhlo'i e>isin. di`a t`a a\kern -.7pt >ut`a d`h >epe`i d'uo >ep'ipeda 
par'allhla t`a HJ, KL <up`o >epip'edou to~u AXZG t'emnetai, a<i 
koina`i a\kern -.7pt >ut~wn toma`i a<i AG, XZ par'allhlo'i e>isin. ka`i >epe`i
trig'wnou to~u ABD par`a m'ian t~wn pleur~wn t`hn BD e>uje~ia
>~hktai <h EX, >an'alogon >'ara >est`in <wc <h AE pr`oc EB, o<'utwc
<h AX pr`oc XD. p'alin >epe`i trig'wnou to~u ADG par`a m'ian 
t~wn pleur~wn t`hn AG e>uje~ia >~hktai <h XZ, >an'alog'on 
>estin <wc <h AX pr`oc XD, o<'utwc <h GZ pr`oc ZD. >ede'iqjh
d`e ka`i <wc <h AX pr`oc XD, o<'utwc <h AE pr`oc EB; ka`i <wc
>'ara <h AE pr`oc EB, o<'utwc <h GZ pr`oc ZD.}

\epsfysize=3in
\centerline{\epsffile{Book11/fig17g.eps}}

\gr{>E`an >'ara d'uo e>uje~iai <up`o parall'hlwn >epip'edwn t'emnwn\-tai,
e>ic to`uc a\kern -.7pt >uto`uc l'ogouc tm\kern -.7pt hj'hsontai; <'oper >'edei deixai.}}

\ParallelRText{

\begin{center}
{\large Proposition 17}
\end{center}

If two straight-lines are cut by parallel planes then
they will be cut in the same ratios.

For let the two straight-lines $AB$ and $CD$ be cut by the parallel planes
$GH$, $KL$, and $MN$ at the points $A$, $E$, $B$,  and $C$, $F$, $D$ (respectively). I say that as the straight-line $AE$ is to $EB$, so $CF$ (is)
to $FD$.

For let $AC$, $BD$, and $AD$ be joined, and let $AD$ meet the
plane $KL$ at point $O$, and let $EO$ and $OF$ be joined.

And since two parallel planes $KL$ and $MN$ are cut by the
plane $EBDO$, their common sections $EO$ and $BD$ are parallel
[Prop.~11.16]. So, for the same (reasons), 
since two parallel planes $GH$ and $KL$ are cut by the plane
$AOFC$, their common sections $AC$ and $OF$ are 
parallel [Prop.~11.16]. And since the straight-line $EO$ has been drawn parallel to
one of the sides $BD$ of triangle $ABD$, thus, proportionally, as 
$AE$ is to $EB$, so $AO$ (is) to $OD$ [Prop.~6.2]. 
Again, since the straight-line $OF$ has been drawn parallel to one of the sides $AC$
of triangle $ADC$, proportionally, as $AO$ is to $OD$, so $CF$ (is) to $FD$ [Prop.~6.2]. And it was also shown that as $AO$ (is) to $OD$, so $AE$ (is) to $EB$.
And thus as $AE$ (is) to $EB$, so $CF$ (is) to $FD$ [Prop.~5.11].\\

\epsfysize=3in
\centerline{\epsffile{Book11/fig17e.eps}}

Thus, if two straight-lines are cut by parallel planes then
they will be cut in the same ratios. (Which is) the very thing it was required to
show.}
\end{Parallel}

%%%%
%11.18
%%%%
\pdfbookmark[1]{Proposition 11.18}{pdf11.18}
\begin{Parallel}{}{}
\ParallelLText{ 
\begin{center}
{\large \ggn{18}.}
\end{center}\vspace*{-7pt}

\gr{>E`an e>uje~ia >epip'edw| tin`i pr`oc >orj`ac >~h|,
ka`i p'anta t`a di> a\kern -.7pt >ut~hc >ep'ipeda t~w| a\kern -.7pt >ut~w| >epip'edw|
pr`oc >orj`ac >'estai.}

\gr{E>uje~ia g'ar tic <h AB t~w| <upokeim'enw| >epip'edw|
pr`oc >orj`ac >'estw; l'egw, <'oti ka`i p'anta t`a di`a t~hc AB >ep'ipeda
t~w| <upokeim'enw| >epip'edw| pr`oc >orj'ac >estin.}

\gr{>Ekbebl'hsjw g`ar di`a t~hc AB >ep'ipedon t`o DE,
ka`i >'estw koin`h tom\kern -.7pt `h to~u DE >epip'edou ka`i to~u <upokeim'enou
<h GE, ka`i e>il'hfjw >ep`i t~hc GE tuq`on shme~ion t`o Z, ka`i >ap`o
to~u Z t~h| GE pr`oc >orj`ac >'hqjw >en t~w| DE >epip'edw| <h ZH.}

\gr{Ka`i >epe`i <h AB pr`oc t`o <upoke'imenon >ep'ipedon >orj'h
>estin, ka`i pr`oc p'asac >'ara t`ac <aptom'enac a\kern -.7pt >ut~hc e>uje'iac
ka`i o>'usac >en t~w| <upokeim'enw| >epip'edw| >orj'h
>estin <h AB; <'wste ka`i pr`oc t`hn GE >orj'h >estin; <h >'ara <up`o
ABZ gwn'ia >orj'h >estin. >'esti d`e ka`i <h <up`o HZB >orj`h;
par'allhloc  >'ara >est`in <h AB t~h| ZH. <h d`e AB t~w| <upokeim'enw|
>epip'edw| pr`oc >orj'ac >estin; ka`i <h ZH >'ara t~w| <upokeim'enw|
>epip'edw| pr`oc >orj'ac >estin. ka`i
 >ep'ipedon pr`oc >ep'ipedon >orj'on >estin, <'otan a<i t~h| koin~h|
 tom\kern -.7pt ~h| t~wn >epip'edwn pr`oc >orj`ac >ag'omenai e>uje~iai >en <en`i
 t~wn >epip'edwn t~w| loip~w| >epip'edw| pr`oc >orj`ac >~wsin.
 ka`i t~h| koin~h| tom\kern -.7pt ~h| t~wn >epip'edwn t~h| GE >en <en`i t~wn
 >epip'edwn t~w| DE pr`oc >orj`ac >aqje~isa <h ZH >ede'iqjh
 t~w| <upokeim'enw| >epip'edw| pr`oc >orj'ac; t`o >'ara DE >ep'ipedon
 >orj'on >esti pr`oc t`o <upoke'imenon. <omo'iwc d`h deiqj'hsetai
 ka`i p'anta t`a di`a t~hc AB >ep'ipeda >orj`a tugqanonta pr`oc t`o <upoke'imenon >ep'ipedon.}\\~\\~\\~\\

\epsfysize=2.in
\centerline{\epsffile{Book11/fig18g.eps}}

\gr{>E`an >'ara e>uje~ia >epip'edw| tin`i pr`oc >orj`ac >~h|, ka`i p'anta t`a
 di> a\kern -.7pt >ut~hc >ep'ipeda t~w| a\kern -.7pt >ut~w| >epip'edw| pr`oc >orj`ac >'estai;
 <'oper >'edei de~ixai.}}
 
\ParallelRText{
\begin{center}
{\large Proposition 18}
\end{center}

If a straight-line is at right-angles to some plane then all of the planes (passing) through it will also be at right-angles to the same
plane.

For let some straight-line $AB$ be at right-angles to a reference plane.
I say that all of the planes (passing) through $AB$ are also at right-angles to 
the reference plane.

For let the plane $DE$ be produced through $AB$. And let $CE$
be the common section of the plane $DE$ and the reference (plane).
And let some random point $F$ be taken on $CE$. And let
$FG$ be drawn from $F$, at right-angles to $CE$, in the plane $DE$
[Prop.~1.11].

And since $AB$ is at right-angles to the reference plane, $AB$ is thus also
at right-angles to all of the straight-lines joined to it which are also in the reference
plane [Def.~11.3]. Hence, it is also at right-angles to
$CE$. Thus, angle $ABF$ is a right-angle. And $GFB$ is also a right-angle.
Thus, $AB$ is parallel to $FG$ [Prop.~1.28].
And $AB$ is at right-angles to the reference plane. Thus, $FG$ is also
at right-angles to the reference plane [Prop.~11.8].
And a plane is at right-angles to a(nother) plane when the straight-lines drawn at
right-angles to the common section
of the planes,  (and lying) in one of the planes,
are at right-angles to the remaining plane [Def.~11.4].
And $FG$, (which was) drawn  at right-angles to the common
section of the planes, $CE$, in one of the planes, $DE$,  was shown  to be at right-angles to the reference plane. Thus, plane $DE$ is at right-angles
to the reference (plane). So, similarly, it can be shown that   all of the planes
(passing) at random through $AB$ (are) at right-angles to the
reference plane.

\epsfysize=2.in
\centerline{\epsffile{Book11/fig18e.eps}}

Thus, if  a straight-line is at right-angles to some plane then all of the planes (passing) through it will also be at right-angles to the same
plane. (Which is) the very thing it was required to show.}
\end{Parallel}

%%%%
%11.19
%%%%
\pdfbookmark[1]{Proposition 11.19}{pdf11.19}
\begin{Parallel}{}{}
\ParallelLText{
\begin{center}
{\large \ggn{19}.}
\end{center}\vspace*{-7pt}

\gr{>E`an d'uo >ep'ipeda t'emnonta >'allhla >epip'edw| tin`i pr`oc >orj`ac >~h|,
ka`i <h koin`h a\kern -.7pt >ut~wn tom\kern -.7pt `h t~w| a\kern -.7pt >ut~w| >epip'edw| pr`oc >orj`ac
>'estai.}

\epsfysize=2.2in
\centerline{\epsffile{Book11/fig19g.eps}}

\gr{D'uo g`ar >ep'ipeda t`a AB, BG t~w| <upokeim'enw| >epip'edw| pr`oc
>orj`ac >'estw, koin`h d`e a\kern -.7pt >ut~wn tom\kern -.7pt `h >'estw <h BD;
l'egw, <'oti <h BD t~w| <upokeim'enw| >epip'edw| pr`oc >orj'ac >estin.}

\gr{m\kern -.7pt `h g'ar, ka`i >'hqjwsan >ap`o to~u D shme'iou >en m`en t~w| AB >epip'edw| t~h| AD e>uje'ia| pr`oc >orj`ac <h DE, >en d`e t~w| BG >epip'edw| t~h| GD pr`oc >orj`ac <h DZ.}

\gr{Ka`i >epe`i t`o AB >ep'ipedon
>orj'on >esti pr`oc t`o <upoke'imenon, ka`i t~h| koin~h| a\kern -.7pt >ut~wn
tom\kern -.7pt ~h| t~h| AD pr`oc >orj`ac >en t~w| AB >epip'edw| >~hktai <h DE, <h
DE >'ara >orj'h >esti pr`oc t`o <upoke'imenon >ep'ipedon. <omo'iwc
d`h de'ixomen, <'oti ka`i <h DZ >orj'h >esti pr`oc t`o <upoke'imenon
>ep'ipedon. >ap`o to~u a\kern -.7pt >uto~u >'ara shme'iou to~u D t~w| <upokeim'enw|
>epip'edw| d'uo e>uje~ia pr`oc >orj`ac >anestam'enai e>is`in >ep`i
t`a a\kern -.7pt >ut`a m'erh; <'oper >est`in >ad'unaton. o>uk >'ara t~w| <upokeim'enw|
>epip'edw| >ap`o to~u D shme'iou >anastaj'hsetai pr`oc >orj`ac pl`hn
t~hc DB koin~hc tom\kern -.7pt ~hc t~wn AB, BG >epip'edwn.}

\gr{>E`an >'ara d'uo >ep'ipeda t'emnonta >'allhla >epip'edw| tin`i pr`oc >orj`ac >~h|,
ka`i <h koin`h a\kern -.7pt >ut~wn tom\kern -.7pt `h t~w| a\kern -.7pt >ut~w| >epip'edw| pr`oc >orj`ac
>'estai; <'oper >'edei de~ixai.}}

\ParallelRText{
\begin{center}
{\large Proposition 19}
\end{center}

If two planes cutting one another are at right-angles to some plane then their common section will also be at right-angles to the
same plane.

\epsfysize=2.2in
\centerline{\epsffile{Book11/fig19e.eps}}

For let the two planes $AB$ and $BC$ be at right-angles to a reference
plane, and let their common section be $BD$. I say that $BD$ is at right-angles to the reference plane.

For (if) not, let $DE$ also be drawn from point $D$,  in the plane
$AB$, at right-angles to the straight-line $AD$, and $DF$, in the plane $BC$, at right-angles to $CD$.

And since the plane $AB$ is at right-angles to the reference (plane),
and $DE$ has been drawn at right-angles to their common section $AD$,
in the plane $AB$, $DE$ is thus at right-angles to the reference 
plane [Def.~11.4]. So, similarly, we can show that
$DF$ is also at right-angles to the reference plane. Thus, two (different) straight-lines
are set up, at the same point $D$, at right-angles to the reference plane,
on the same side. The very thing is impossible [Prop.~11.13]. Thus, no (other straight-line) except the common section
$DB$ of the planes $AB$ and $BC$ can be set up at point $D$,
at right-angles to the reference plane.

Thus, if two planes cutting one another are at right-angles to some plane then their common section will also be at right-angles to the
same plane. (Which is) the very thing it was required to show.}
\end{Parallel}

%%%%
%11.20
%%%%
\pdfbookmark[1]{Proposition 11.20}{pdf11.20}
\begin{Parallel}{}{}
\ParallelLText{
\begin{center}
{\large\ggn{20}.}
\end{center}\vspace*{-7pt}

\gr{>E`an stere`a gwn'ia <up`o tri~wn gwni~wn >epip'edwn peri'eqh\-tai, d'uo
<opoiaio~un t~hc loip~hc me'izon'ec e>isi p'anth| metalamban'omenai.}

\epsfysize=1.5in
\centerline{\epsffile{Book11/fig20g.eps}}

\gr{Stere`a g`ar gwn'ia <h pr`oc t~w| A <up`o tri~wn gwni~wn >epip'edwn t~wn
<up`o BAG, GAD, DAB perieq'esjw; l'egw, <'oti t~wn <up`o BAG, GAD, DAB gwni~wn d'uo <opoiaio~un t~hc loip~hc me'izon'ec e>isi p'anth|
metalamban'omenai.}

\gr{E>i m`en o>~un a<i <up`o BAG, GAD, DAB gwn'iai >'isai >all'hlaic
e>is'in, faner'on, <'oti d'uo <opoiaio~un t~hc loip~hc me'izon'ec
e>isin. e>i d`e o>'u, >'estw me'izwn <h <up`o BAG, ka`i sunest'atw
pr`oc t~h| AB e>uje'ia| ka`i t~w| pr`oc a\kern -.7pt >ut~h| shme'iw| t~w| A t~h|
<up`o DAB gwn'ia| >en t~w| di`a t~wn BAG >epip'edw| >'ish <h <up`o BAE,
ka`i ke'isjw t~h| AD >'ish <h AE, ka`i di`a to~u E shme'iou diaqje~isa
<h BEG temn'etw t`ac AB, AG e>uje'iac kat`a t`a B, G shme~ia,
ka`i >epeze'uqjwsan a<i DB, DG.}

\gr{Ka`i >epe`i >'ish >est`in <h DA t~h|
AE, koin`h d`e <h AB, d'uo dus`in >'isai; ka`i gwn'ia <h <up`o
DAB gwn'ia| t~h| <up`o BAE >'ish; b'asic >'ara <h DB b'asei t~h| BE
>estin >'ish. ka`i >epe`i d'uo a<i BD, DG t~hc BG me'izon'ec e>isin,
<~wn <h DB t~h| BE >ede'iqjh >'ish, loip`h >'ara <h DG loip~hc
t~hc EG me'izwn >est'in. ka`i >epe`i >'ish >est`in <h DA t~h| AE,
koin`h d`e <h AG, ka`i b'asic <h DG b'asewc t~hc EG me'izwn
>est'in, gwn'ia >'ara <h <up`o DAG gwn'aic t~hc <up`o EAG me'izwn
>est'in.
 >ede'iqjh d`e ka`i <h <up`o DAB t~h| <up`o BAE >'ish;
a<i >'ara <up`o DAB, DAG t~hc <up`o BAG me'izon'ec e>isin.
<omo'iwc d`h de'ixomen, <'oti ka`i a<i loipa`i s'unduo lamban'omenai
t~hc loip~hc me'izon'ec e>isin.}

\gr{>E`an >'ara stere`a gwn'ia <up`o tri~wn gwni~wn >epip'edwn peri'eqhtai,
d'uo <opoiaio~un t~hc loip~hc me'izon'ec e>isi p'anth| metalamban'omenai;
<'oper >'edei de~ixai.}}

\ParallelRText{
\begin{center}
{\large Proposition 20}
\end{center}

If a solid angle is contained by three plane angles
then (the sum of) any two (angles) is greater than the remaining (one), (the angles) being taken up in any (possible way).

\epsfysize=1.5in
\centerline{\epsffile{Book11/fig20e.eps}}

For let the solid angle $A$ be contained by the three plane angles $BAC$, $CAD$, and $DAB$. I say that (the sum of) any two of the angles $BAC$, $CAD$, and $DAB$ is greater than the remaining (one),
(the angles) being taken up in any (possible way).

For if the angles $BAC$, $CAD$, and $DAB$ are equal to one another
then (it is) clear that (the sum of) any two is greater than
the remaining (one). But, if not, let $BAC$ be greater (than $CAD$ or $DAB$). And let (angle) $BAE$, equal to the angle $DAB$,  be constructed in the plane through $BAC$, on the straight-line $AB$,
at the point $A$ on it. And let $AE$ be made equal to $AD$. And $BEC$
being drawn across through point $E$, let it cut the straight-lines $AB$
and $AC$ at points $B$ and $C$ (respectively). And let
$DB$ and $DC$ be joined.

And since $DA$ is equal to $AE$, and $AB$ (is) common, the
two (straight-lines $AD$ and $AB$ are) equal to the two (straight-lines
$EA$ and $AB$, respectively). And angle $DAB$ (is) equal
to angle $BAE$. Thus, the base $DB$ is equal to the base $BE$ [Prop.~1.4]. And since the
(sum of the) two (straight-lines) $BD$ and $DC$ is greater than $BC$
[Prop.~1.20], of which $DB$ was shown
(to be) equal to $BE$, the  remainder $DC$ is thus greater than the
remainder $EC$. And since $DA$ is equal to $AE$, but
$AC$ (is) common, and the base $DC$ is  greater than the base
$EC$, the angle $DAC$ is thus greater than the angle $EAC$
[Prop.~1.25]. And $DAB$
was also shown (to be) equal to $BAE$. Thus, 
(the sum of) $DAB$ and $DAC$ is greater than $BAC$. So,
similarly, we can also show that the remaining  (angles), being taken in pairs, are greater than the remaining (one).

Thus,  if a solid angle is contained by three plane angles
then (the sum of) any two (angles) is greater than the remaining (one), (the angles) being taken up in any (possible way). (Which is) the very thing it
was required to show.}
\end{Parallel}

%%%%
%11.21
%%%%
\pdfbookmark[1]{Proposition 11.21}{pdf11.21}
\begin{Parallel}{}{}
\ParallelLText{
\begin{center}
{\large \ggn{21}.}
\end{center}\vspace*{-7pt}

\gr{<'Apasa stere`a gwn'ia <up`o >elass'onwn [>`h] tess'arwn >orj~wn
gwni~wn >epip'edwn peri'eqetai.}

\epsfysize=1.5in
\centerline{\epsffile{Book11/fig21g.eps}}

\gr{>'Estw stere`a gwn'ia <h pr`oc t~w| A perieqom'enh <up`o
>epip'edwn gwni~wn t~wn <up`o BAG, GAD, DAB; l'egw, <'oti
a<i <up`o BAG, GAD, DAB tess'arwn >orj~wn >el'asson'ec e>isin.}

\gr{E>il'hfjw g`ar >ef> <ek'asthc t~wn AB, AG, AD tuq'onta shme~ia
t`a B, G, D, ka`i >epeze'uqjwsan a<i BG, GD, DB. ka`i >epe`i stere`a gwn'ia
<h pr`oc t~w| B <up`o tri~wn gwni~wn >epip'edwn peri'eqetai t~wn
<up`o GBA, ABD, GBD, d'uo <opoiaio~un t~hc loip~hc me'izon'ec
e>isin; a<i >'ara <up`o GBA, ABD t~hc <up`o GBD me'izon'ec
e>isin.
di`a t`a a\kern -.7pt >ut`a d`h ka`i a<i m`en <up`o BGA, AGD t~hc <up`o BGD
me'izon'ec e>isin, a<i  d`e <up`o GDA, ADB t~hc <up`o GDB me'izon'ec
e>isin; a<i <`ex >'ara gwn'iai a<i <up`o GBA, ABD, BGA, AGD, GDA, ADB tri~wn t~wn <up`o GBD, BGA, GDB me'izon'ec e>isin. >all`a a<i
tre~ic a<i <up`o GBD, BDG, BGD dus`in >orja~ic >'isai e>is'in;
a<i <`ex >'ara a<i <up`o GBA, ABD, BGA, AGD, GDA, ADB d'uo
>orj~wn me'izon'ec e>isin. ka`i >epe`i <ek'astou t~wn ABG, AGD, ADB
trig'wnwn a<i tre~ic gwn'iai dus`in >orja~ic >'isai e>is'in, a<i >'ara
t~wn tri~wn trig'wnwn >enn'ea gwn'iai a<i <up`o GBA, AGB, BAG, AGD,
GDA, GAD, ADB, DBA, BAD 	<`ex >orja~ic >'isai
e>is'in, <~wn a<i <up`o ABG, BGA, AGD, GDA, ADB,
DBA <`ex gwn'iai d'uo >orj~wn e>isi me'izonec; loipa`i >'ara
a<i <up`o BAG, GAD, DAB tre~ic [gwn'iai] peri'eqousai t`hn stere`an
gwn'ian tess'arwn >orj~wn >el'asson'ec e>isin.}

\gr{<'Apasa >'ara stere`a gwn'ia <up`o >elass'onwn [>'h] tess'arwn >orj~wn
gwni~wn >epip'edwn peri'eqetai; <'oper >'edei de~ixai.}}

\ParallelRText{
\begin{center}
{\large Proposition 21}
\end{center}

Any solid angle is contained by plane angles (whose sum is) less [than]
four  right-angles.$^\dag$

\epsfysize=1.5in
\centerline{\epsffile{Book11/fig21e.eps}}

Let the solid angle $A$ be contained by the plane angles $BAC$,
$CAD$, and $DAB$. I say that (the sum of) $BAC$, $CAD$, and $DAB$
is less than four right-angles.

For let the random points $B$, $C$, and $D$ be taken on each of (the straight-lines)
 $AB$, $AC$, and $AD$ (respectively). And let $BC$, $CD$, and $DB$ be joined.  And since the solid angle at $B$ is contained
by the three plane angles $CBA$, $ABD$, and $CBD$, (the sum of) any two
is greater than the remaining (one) [Prop.~11.20].
Thus, (the sum of) $CBA$ and $ABD$ is greater than $CBD$. So, for
the same (reasons), (the sum of) $BCA$ and $ACD$ is also greater than
$BCD$, and (the sum of) $CDA$ and $ADB$ is greater than $CDB$.
Thus, the (sum of the) six angles $CBA$, $ABD$, $BCA$, $ACD$, $CDA$,
and $ADB$ is greater than the (sum of the) three (angles) $CBD$, $BCD$,
and $CDB$. But, the (sum of the) three (angles) $CBD$, $BDC$, and
$BCD$ is equal to two right-angles [Prop.~1.32].
Thus, the (sum of the) six angles $CBA$, $ABD$, $BCA$, $ACD$, $CDA$,
and $ADB$ is greater than two right-angles.
And since the (sum of the) three angles of each of the triangles
$ABC$, $ACD$, and $ADB$ is equal to two right-angles, the (sum of the)
nine angles $CBA$, $ACB$, $BAC$, $ACD$, $CDA$, $CAD$, $ADB$,
$DBA$, and $BAD$ of the three triangles is equal to six right-angles,
of which the (sum of the) six angles $ABC$, $BCA$, $ACD$, $CDA$,
$ADB$, and $DBA$ is greater than two right-angles. Thus, the (sum of the)
remaining three [angles] $BAC$, $CAD$, and $DAB$, containing the
solid angle, is less than four right-angles.

Thus, any solid angle is contained by plane angles (whose sum is) less [than]
four  right-angles. (Which is) the very thing it was required to show.}
\end{Parallel}
{\footnotesize\noindent$^\dag$ This proposition is only proved for the case of a solid angle contained by three plane angles. However, the generalization to
a solid angle contained by more than three plane angles is straightforward.}

%%%%
%11.22
%%%%
\pdfbookmark[1]{Proposition 11.22}{pdf11.22}
\begin{Parallel}{}{}
\ParallelLText{
\begin{center}
{\large \ggn{22}.}
\end{center}\vspace*{-7pt}

\gr{>E`an <~wsi tre~ic gwn'iai >ep'ipedoi, <~wn a<i d'uo t~hc loip~hc me'izon'ec e>isi p'anth| metalamban'omenai, peri'eqwsi d`e a\kern -.7pt >ut`ac
>'isai e>uje~iai, dunat'on >estin >ek t~wn >epizeugnuous~wn t`ac
>'isac e>uje'iac tr'igwnon sust'hsasjai.}\\~\\

\epsfysize=.9in
\centerline{\epsffile{Book11/fig22g.eps}}

\gr{>'Estwsan tre~ic gwn'iai >ep'ipedoi a<i <up`o ABG, DEZ, HJK, <~wn
a<i d'uo t~hc loip~hc me'izon'ec e>isi p'anth| metalamban'omenai, a<i
m`en <up`o ABG, DEZ t~hc <up`o HJK, a<i d`e <up`o DEZ, HJK
t~hc <up`o ABG, ka`i >'eti a<i <up`o HJK, ABG t~hc <up`o DEZ,
ka`i >'estwsan >'isai a<i AB, BG, DE, EZ, HJ, JK e>uje~iai,
ka`i >epeze'uqjwsan a<i AG, DZ, HK; l'egw, <'oti dunat'on >estin
>ek t~wn >'iswn ta~ic AG, DZ, HK tr'igwnon sust'hsasjai, tout'estin
<'oti t~wn AG, DZ, HK d'uo <opoiaio~un t~hc loip~hc me'izon'ec
e>isin.}

\gr{E>i m`en o>~un a<i <up`o ABG, DEZ, HJK gwn'iai >'isai >all'hlaic
e>is'in, faner'on, <'oti ka`i t~wn AG, DZ, HK >'iswn ginom'enwn dunat'on
>estin >ek t~wn >'iswn ta~ic AG, DZ, HK tr'igwnon sust'hsasjai.
e>i d`e o>'u,  >'estwsan >'anisoi, ka`i sunest'atw pr`oc t~h| JK
e>uje'ia| ka`i t~w| pr`oc a\kern -.7pt >ut~h| shme'iw| t~w| J t~h| <up`o ABG
gwn'ia| >'ish <h <up`o KJL; ka`i ke'isjw mi~a| t~wn AB, BG, DE, EZ,
HJ, JK >'ish <h JL, ka`i >epeze'uqjwsan a<i KL, HL. ka`i >epe`i
d'uo a<i AB, BG dus`i ta~ic KJ, JL
>'isai e>is'in, ka`i gwn'ia <h pr`oc t~w| B gwn'ia| t~h| <up`o KJL >'ish,
b'asic >'ara <h AG b'asei t~h| KL >'ish. ka`i >epe`i a<i <up`o ABG,
HJK t~hc <up`o DEZ me'izon'ec e>isin, >'ish d`e <h <up`o ABG
t~h| <up`o KJL, <h >'ara <up`o HJL t~hc <up`o DEZ me'izwn >est'in. ka`i
>epe`i d'uo a<i HJ, JL d'uo ta~ic DE, EZ >'isai e>is'in, ka`i gwn'ia
<h <up`o HJL gwn'iac t~hc <up`o DEZ me'izwn, b'asic >'ara
<h HL b'asewc t~hc DZ me'izwn >est'in. >all`a a<i HK, KL
t~hc HL me'izon'ec e>isin. poll~w| >'ara a<i HK, KL t~hc
DZ me'izon'ec e>isin. >'ish d`e <h KL t~h| AG; a<i AG, HK >'ara
t~hc loip~hc t~hc DZ me'izon'ec e>isin. <omo'iwc d`h de'ixomen, 
<'oti ka`i a<i m`en AG, DZ t~hc HK me'izon'ec e>isin, ka`i >'eti
a<i DZ, HK t~hc AG me'izon'ec e>isin. dunat`on >'ara >est`in >ek
t~wn >'iswn ta~ic AG, DZ, HK tr'igwnon sust'hsasjai; <'oper >'edei
de~ixai.}}

\ParallelRText{
\begin{center}
{\large Proposition 22}
\end{center}

If there are three plane angles, of which (the sum of any)
two is greater than the remaining (one), (the angles) being taken up in
any (possible way), and if equal straight-lines contain them, then it is
possible to construct a triangle from (the straight-lines created by) joining the
(ends of the) equal straight-lines.

\epsfysize=0.9in
\centerline{\epsffile{Book11/fig22e.eps}}

Let $ABC$, $DEF$, and $GHK$ be three plane angles, of which the sum of any)
two is greater than the remaining (one), (the angles) being taken up in
any (possible way)---(that is), $ABC$ and $DEF$ (greater) than $GHK$,
$DEF$ and $GHK$ (greater) than $ABC$, and, further, $GHK$
and $ABC$ (greater) than $DEF$. And let $AB$, $BC$, $DE$, $EF$,
$GH$, and $HK$ be equal straight-lines. And let $AC$, $DF$, and
$GK$ be joined. I say that that it is possible to construct a
triangle out of (straight-lines) equal to $AC$, $DF$, and $GK$---that is to say, that (the sum of) any two of $AC$, $DF$, and $GK$ is greater than
the remaining (one).

Now, if the angles $ABC$, $DEF$, and $GHK$ are equal to one another then
(it is)  clear that, (with) $AC$, $DF$, and $GK$ also becoming equal, it is
possible to construct a triangle from (straight-lines) equal to
$AC$, $DF$, and $GK$. And if not, let them be unequal, and let $KHL$,
equal to angle $ABC$, be constructed on the straight-line
$HK$, at the point $H$ on it. And let $HL$ be made equal to one
of $AB$, $BC$, $DE$, $EF$, $GH$, and $HK$. And let $KL$
and $GL$ be joined. And since the two (straight-lines) $AB$ and
$BC$ are equal to the two (straight-lines) $KH$ and $HL$ (respectively), 
and the angle at $B$ (is) equal to $KHL$, the base $AC$ is thus equal to the
base $KL$ [Prop.~1.4]. And since (the sum of)
$ABC$ and $GHK$ is greater than $DEF$, and $ABC$
equal to $KHL$,  $GHL$ is thus greater than $DEF$. And since the
two (straight-lines) $GH$ and $HL$ are equal to the two (straight-lines)
$DE$ and $EF$ (respectively), and angle $GHL$ (is) greater than $DEF$, the base $GL$ is thus greater than the base $DF$ [Prop.~1.24]. 
But, (the sum of) $GK$ and $KL$ is greater than $GL$ [Prop.~1.20]. Thus, (the sum of) $GK$ and $KL$
is much greater than $DF$.  And $KL$ (is) equal to $AC$. Thus, (the sum of)
$AC$ and $GK$ is greater than the remaining (straight-line) $DF$.
So, similarly, we can show that (the sum of) $AC$ and $DF$
is greater than $GK$, and, further, that (the sum of) $DF$ and $GK$
is greater than $AC$. Thus, it is possible to construct a triangle from
(straight-lines) equal to $AC$, $DF$, and $GK$. (Which is) the very
thing it was required to show.}
\end{Parallel}

%%%%
%11.23
%%%%
\pdfbookmark[1]{Proposition 11.23}{pdf11.23}
\begin{Parallel}{}{}
\ParallelLText{
\begin{center}
{\large\ggn{23}.}
\end{center}\vspace*{-7pt}

\gr{>Ek tri~wn gwni~wn >epip'edwn, <~wn a<i d'uo t~hc loip~hc
me'izon'ec e>isi p'anth| metalamban'omenai, stere`an gwn'ian sust'hsasjai;
de~i d`h t`ac tre~ic tess'arwn >orj~wn >el'ass\-onac e>~inai.}\\~\\

\epsfysize=1.1in
\centerline{\epsffile{Book11/fig23g.eps}}

\gr{>'Estwsan a<i doje~isai tre~ic gwn'iai >ep'ipedoi a<i <up`o ABG, DEZ,
HJK, <~wn a<i d'uo t~hc loip~hc me'izonec >'estwsan p'anth| metalamban'omenai, >'eti d`e a<i tre~ic tess'arwn >orj~wn >el'assonec;
de~i
 d`h >ek t~wn >'iswn ta~ic <up`o ABG, DEZ,
HJK stere`an gwn'ian sust'hsasjai.}

\gr{>Apeil'hfjwsan >'isai a<i AB, BG, DE, EZ, HJ, JK, ka`i
>epeze'uqjwsan a<i AG, DZ, HK; dunat`on >'ara >est`in >ek t~wn
>'iswn ta~ic AG, DZ, HK tr'igwnon sust'hsasjai. sunest'atw t`o LMN,
<'wste >'ishn e>~inai t`hn m`en AG t~h| LM, t`hn d`e DZ t~h| MN,
ka`i >'eti t`hn HK t~h| NL, ka`i perigegr'afjw per`i t`o LMN
tr'igwnon k'ukloc <o LMN, ka`i e>il'hfjw a\kern -.7pt >uto~u t`o k'entron
ka`i >'estw t`o X, ka`i >epeze'uqjwsan a<i LX, MX, NX;}\\~\\~\\

\epsfysize=2.in
\centerline{\epsffile{Book11/fig23ag.eps}}

\gr{ L'egw,
<'oti <h AB me'izwn >est`i t~hc LX. e>i g`ar m\kern -.7pt 'h, >'htoi >'ish
>est`in <h AB t~h| LX >`h >el'attwn. >'estw pr'oteron >'ish. ka`i
>epe`i >'ish >est`in <h AB t~h| LX, >all`a <h m`en AB t~h| BG >estin
>'ish, <h d`e XL t~h| XM, d'uo d`h a<i AB, BG d'uo ta~ic LX, XM
>'isai e>is`in <ekat'era <ekat'era|; ka`i b'asic <h AG b'asei t~h| LM <up'okeitai >'ish; gwn'ia >'ara <h <up`o ABG gwn'ia| t~h| <up`o
LXM >estin >'ish. di`a t`a a\kern -.7pt >ut`a d`h ka`i <h m`en <up`o DEZ
t~h| <up`o MXN >estin >'ish, ka`i >'eti <h <up`o HJK t~h| <up`o
NXL; a<i >'ara tre~ic a<i <up`o ABG, DEZ, HJK gwn'iai tris`i
ta~ic <up`o LXM, MXN, NXL e>isin >'isai. >all`a a<i tre~ic a<i
<up`o LXM, MXN, NXL t'ettarsin >orja~ic e>isin >'isai;
ka`i a<i tre~ic >'ara a<i <up`o ABG, DEZ, HJK t'ettarsin >orja~ic
>'isai e>is'in. <up'okeintai d`e ka`i tess'arwn >orj~wn >el'assonec;
<'oper >'atopon. o>uk >'ara <h AB t~h| LX >'ish >est'in. l'egw d'h,
<'oti o<ud`e >el'attwn >est`in <h AB t~hc LX. e>i g`ar dunat'on,
>'estw; ka`i ke'isjw t~h| m`en AB >'ish <h XO, t~h| d`e BG >'ish
<h XP, ka`i >epeze'uqjw <h OP. ka`i >epe`i >'ish >est`in <h AB
t~h| BG, >'ish >est`i ka`i <h XO t~h| XP; <'wste ka`i loip`h <h LO
t~h| PM >estin >'ish. par'allhloc >'ara >est`in <h LM t~h| OP, ka`i
>isog'wnion t`o LMX t~w| OPX; >'estin >'ara <wc <h XL
pr`oc LM, o<'utwc <h XO pr`oc OP; >enall`ax <wc <h LX pr`oc
XO, o<'utwc <h LM pr`oc OP. me'izwn d`e <h LX t~hc XO; me'izwn
>'ara ka`i <h LM t~hc OP. >all`a <h LM ke~itai t~h| AG >'ish; ka`i
<h AG >'ara t~hc OP me'izwn >est'in. >epe`i o>~un d'uo a<i AB, BG
dus`i ta~ic OX, XP >'isai e>is'in, ka`i b'asic <h AG b'asewc t~hc OP
me'izwn >est'in, gwn'ia >'ara <h <up`o ABG gwn'iac t~hc <up`o
OXP me~izwn >est'in. <omo'iwc d`h de'ixomen, <'oti ka`i <h m`en
<up`o DEZ t~hc <up`o MXN me'izwn >est'in,\, <h d`e <up`o
HJK t~hc <up`o NXL. a<i >'ara tre~ic gwn'iai a<i <up`o ABG, DEZ, HJK
tri~wn t~wn <up`o LXM, MXN, NXL me'izon'ec e>isin. >all`a a<i <up`o
ABG, DEZ, HJK tess'arwn >orj~wn >el'assonec <up'okeintai; poll~w|
>'ara a<i <up`o LXM, MXN, NXL tess'arwn >orj~wn >el'asson'ec e>isin.
>all`a ka`i >'isai; <'oper >est`in >'atopon. o>uk >'ara <h AB >el'asswn
>est`i t~hc LX. >ede'iqjh d'e, <'oti o>ud`e >'ish; me'izwn >'ara <h AB
t~hc LX.}

\gr{>Anest'atw d`h >ap`o to~u X shme'iou t~w| to~u LMN k'uklou
>epip'edw| pr`oc >orj`ac <h XR, ka`i <~w| me~iz'on >esti t`o >ap`o t~hc
AB tetr'agwnon to~u >ap`o t~hc LX, >eke'inw| >'ison >'estw t`o
>ap`o t~hc XR, ka`i >epeze'uqjwsan a<i RL, RM, RN.}

\gr{Ka`i >epe`i <h RX >orj`h >esti pr`oc t`o to~u LMN k'uklou
>ep'ipedon, ka`i pr`oc <ek'asthn >'ara t~wn LX, MX, NX
>orj'h >estin <h RX.
 ka`i >epe`i >'ish
>est`in <h LX t~h| XM, koin`h d`e ka`i pr`oc >orj`ac <h XR, b'asic >'ara
<h RL b'asei t`h| RM >estin >'ish. di`a t`a a\kern -.7pt >ut`a d`h ka`i <h RN
<ekat'era| t~wn RL, RM >estin >'ish; a<i tre~ic >'ara a<i RL,
RM, RN >'isai >all'hlaic e>is'in. ka`i >epe`i <~w| me~iz'on
>esti t`o >ap`o t~hc AB to~u >ap`o t~hc LX, >eke'inw|
>'ison <up'okeitai t`o >ap`o t~hc XR, t`o >'ara >ap`o t~hc AB >'ison
>est`i to~ic >ap`o t~wn LX, XR.  to~ic d`e >ap`o t~wn LX, XR
>'ison >est`i t`o >ap`o t~hc LR; >orj`h g`ar <h <up`o LXR;
t`o >'ara >ap`o t~hc AB >'ison >est`i t~w| >ap`o t~hc RL;
>'ish >'ara <h AB t~h| RL. >all`a t~h| m`en AB >'ish >est`in <ek'asth
t~wn BG, DE, EZ, HJ, JK, t~h| d`e RL >'ish <ekat'era t~wn
RM, RN;  <ek'asth >'ara t~wn AB, BG, DE, EZ, HJ, JK 
<ek'asth| t~wn RL, RM, RN >'ish >est'in. ka`i >epe`i d'uo
a<i LR, RM dus`i ta~ic AB, BG >'isai e>is'in, ka`i b'asic <h LM
b'asei t~h| AG <up'okeitai >'ish, gwn'ia >'ara <h <up`o LRM
gwn'ia| t~h| <up`o ABG >estin >'ish. di`a t`a a\kern -.7pt >ut`a d`h ka`i <h m`en <up`o
MRN t~h| <up`o DEZ >estin >'ish, <h d`e <up`o LRN t~h| <up`o
HJK.}

\gr{>Ek tri~wn >'ara gwni~wn >epip'edwn t~wn <up`o LRM, MRN, LRN,
a<'i e>isin >'isai tris`i ta~ic doje'isaic ta~ic <up`o ABG, DEZ, HJK,
stere`a gwn'ia sun'estatai <h pr`oc t~w| R perieqom'enh <up`o t~wn
LRM, MRN, LRN gwni~wn; <'oper >'edei poi~hsai.}}

\ParallelRText{
\begin{center}
{\large Proposition 23}
\end{center}

To construct a solid angle from three (given)
plane angles,  (the sum of) two of which is greater than the remaining (one, the
angles) being taken up in any (possible way). So, it is necessary for the (sum of the) three (angles) to be less than four right-angles [Prop.~11.21].

\epsfysize=1.1in
\centerline{\epsffile{Book11/fig23e.eps}}

Let $ABC$, $DEF$, and $GHK$ be the three given plane angles, of which let
(the sum of) two be greater than the remaining (one, the angles) being taken
up in any (possible way), and, further, (let) the (sum of the) three (be) less than
four right-angles. So, it is necessary to construct a solid angle from
(plane angles) equal to $ABC$, $DEF$, and $GHK$.

Let $AB$, $BC$, $DE$, $EF$, $GH$, and $HK$ be cut off (so as to be)
equal (to one another). And let $AC$, $DF$, and $GK$ be joined.
It is, thus, possible to construct a triangle from (straight-lines) equal to
$AC$, $DF$, and $GK$ [Prop.~11.22]. Let
(such a triangle), $LMN$, have be constructed, such that $AC$ is equal to
$LM$, $DF$ to $MN$, and, further, $GK$ to $NL$. And let the circle
$LMN$ be circumscribed about triangle $LMN$ [Prop.~4.5]. And let its center be found, and let it be (at) $O$.
And let $LO$, $MO$, and $NO$ be joined.

\epsfysize=2.in
\centerline{\epsffile{Book11/fig23ae.eps}}

 I say that $AB$ is greater
than $LO$. For, if not, $AB$ is either equal to, or less than, $LO$.
Let it, first of all, be equal. And since $AB$ is equal to $LO$, but $AB$
is equal to $BC$, and $OL$ to $OM$, so the two (straight-lines)
$AB$ and $BC$ are equal to the two (straight-lines) $LO$ and $OM$, respectively. And the base $AC$ was assumed (to be) equal to the base
$LM$. Thus, angle $ABC$ is equal to angle $LOM$ [Prop.~1.8]. So, for the same (reasons), $DEF$ is also equal to $MON$,
and, further, $GHK$ to $NOL$. Thus, the three angles $ABC$, $DEF$,
and $GHK$ are equal to the three angles $LOM$, $MON$, and $NOL$,
respectively. But, the (sum of the) three angles $LOM$, $MON$,
and $NOL$ is equal to four right-angles. 
Thus, the (sum of the) three angles $ABC$, $DEF$, and $GHK$ is
also equal to four right-angles.
And it was also assumed (to be)
less than four right-angles. The very thing (is) absurd. Thus, $AB$
is not equal to $LO$. So, I say that $AB$ is not less than $LO$ either.
For, if possible, let it be (less). And let $OP$ be made equal to $AB$, and
$OQ$ equal to $BC$, and let $PQ$ be joined. And since $AB$ is equal
to $BC$, $OP$ is also equal to $OQ$. Hence, the remainder $LP$ is also
equal to (the remainder) $QM$. $LM$ is thus parallel to $PQ$ [Prop.~6.2], and (triangle) $LMO$ (is) equiangular with (triangle) $PQO$
[Prop.~1.29]. 
Thus, as $OL$ is to $LM$, so $OP$ (is) to $PQ$ [Prop.~6.4]. Alternately, as $LO$ (is) to $OP$, so $LM$ (is) to $PQ$
[Prop.~5.16]. And $LO$ (is) greater than $OP$. 
Thus, $LM$ (is) also greater than $PQ$ [Prop.~5.14]. 
But $LM$ was made equal to $AC$. Thus, $AC$ is also greater than $PQ$. Therefore, since the two
(straight-lines) $AB$ and $BC$ are equal
to the two (straight-lines) $PO$ and $OQ$ (respectively), and the base $AC$ is greater
than the base $PQ$, the angle $ABC$ is thus greater than the
angle $POQ$ [Prop.~1.25]. So, similarly,
we can show that $DEF$ is also greater than $MON$, and $GHK$ than
$NOL$. Thus, the (sum of the) three angles $ABC$, $DEF$, and
$GHK$ is greater than the (sum of the) three angles $LOM$, $MON$, and
$NOL$. But, (the sum of) $ABC$, $DEF$, and $GHK$ was assumed (to be)
less than four right-angles. Thus, (the sum of) $LOM$, $MON$, and
$NOL$ is much less than four right-angles. But, (it is) also equal (to four right-angles). The very thing is absurd. Thus, $AB$ is not less than $LO$.
And it was shown (to be) not equal either. Thus, $AB$ (is) greater  than
$LO$.

So let $OR$ be set up at point $O$ at right-angles to the plane
of circle $LMN$ [Prop.~11.12]. And let the (square) on $OR$ be equal to that (area) by which the square on $AB$ is greater than the
(square) on $LO$ [Prop.~11.23~lem.]. And let $RL$, $RM$, and $RN$ be joined.

And since $RO$ is at right-angles to the plane of circle $LMN$, $RO$
is thus also at right-angles to each of $LO$, $MO$, and $NO$. And since
$LO$ is equal to $OM$, and $OR$ is common and at right-angles,
the base $RL$ is thus equal to the base $RM$ [Prop.~1.4]. So, for the same (reasons), $RN$ is also equal to
each of $RL$ and $RM$. Thus, the three (straight-lines) $RL$, $RM$,
and $RN$ are equal to one another. And since the (square) on $OR$
was assumed to be equal to that (area) by which the (square) on $AB$
is greater than the (square) on $LO$, the (square) on $AB$ is thus equal
to the (sum of the squares) on $LO$ and $OR$. And the (square) on 
$LR$ is equal to the (sum of the squares) on $LO$ and $OR$. For $LOR$ (is) a right-angle  [Prop.~1.47]. Thus,
the (square) on $AB$ is equal to the (square) on $RL$. Thus, $AB$ (is)
equal to $RL$. But, each of $BC$, $DE$, $EF$, $GH$, and $HK$
is equal to $AB$, and each of $RM$ and $RN$ equal to $RL$. Thus,
each of $AB$, $BC$, $DE$, $EF$, $GH$, and $HK$ is equal to each of
$RL$, $RM$, and $RN$. And since the two (straight-lines)
$LR$ and $RM$ are equal to the two (straight-lines) $AB$ and $BC$ (respectively), and the base $LM$ was assumed (to be) equal to the
base $AC$, the angle $LRM$ is thus equal to the angle $ABC$
[Prop.~1.8]. So, for the same (reasons), $MRN$
is also equal to $DEF$, and $LRN$ to $GHK$.

Thus, the solid angle $R$, contained by the angles $LRM$, $MRN$, and $LRN$,  has been constructed out of the three plane
angles $LRM$, $MRN$, and $LRN$, which are equal to the three
given (plane angles) $ABC$, $DEF$, and $GHK$ (respectively). 
(Which is) the very thing it was required to do.}
\end{Parallel}

\begin{Parallel}{}{}
\ParallelLText{

\epsfysize=1.2in
\centerline{\epsffile{Book11/fig23bg.eps}}

\begin{center}
{\large \gr{L~hmma}.}
\end{center}\vspace*{-7pt}

\gr{<`On d`e tr'opon, <~w| me~iz'on >esti t`o >ap`o t~hc AB to~u >ap`o
t~hc LX, >eke'inw| >'ison labe~in >'esti t`o >ap`o t~hc XR, de'ixomen o<'utwc. >ekke'isjwsan a<i AB, LX e>uje~iai, ka`i >'estw me'izwn
<h AB, ka`i gegr'afjw >ep> a\kern -.7pt >ut~hc <hmik'uklion t`o ABG, ka`i
e>ic t`o ABG <hmik'uklion >enhrm'osjw t~h| LX e>uje'ia| m\kern -.7pt `h me'izoni
o>'ush| t~hc AB diam'etrou >'ish <h AG, ka`i >epeze'uqjw <h GB.
>epe`i o>~un >en <hmikukl'iw| t~w| AGB gwn'ia >est`in <h <up`o
AGB, >orj`h >'ara >est`in <h <up`o AGB. t`o >'ara >ap`o t~hc
AB >'ison >est`i to~ic >ap`o t~wn AG, GB. <'wste t`o >ap`o t~hc
AB to~u >ap`o t~hc AG me~iz'on >esti t~w| >ap`o t~hc GB. >'ish
d`e <h AG t~h| LX. t`o >'ara >ap`o t~hc AB to~u >ap`o t~hc
LX me~iz'on >esti t~w| >ap`o t~hc GB. >e`an o>~un t~h| BG
>'ishn t`hn XR >apol'abwmen, >'estai t`o >ap`o t~hc AB to~u
>ap`o t~hc LX me~izon t~w| >ap`o t~hc XR; <'oper pro'ekeito
poi~hsai.}}

\ParallelRText{

\epsfysize=1.2in
\centerline{\epsffile{Book11/fig23be.eps}}

\begin{center}
{\large Lemma}
\end{center}\vspace*{-7pt}

And we can demonstrate, thusly,  in which manner to take the (square) on $OR$ equal to that (area) by which
the (square) on $AB$ is greater than the (square) on $LO$. Let the straight-lines $AB$ and $LO$ be set out, and let $AB$ be greater, and let the semicircle $ABC$ be drawn around it.  And
let $AC$, equal to the  straight-line $LO$, which is not greater than the diameter $AB$,  be inserted into the semicircle $ABC$ [Prop.~4.1]. And let $CB$ be joined. Therefore, since
the angle  $ACB$ is in the semicircle $ACB$, $ACB$ is thus a right-angle
[Prop.~3.31]. Thus, the (square) on $AB$
is  equal to the (sum of the) squares on $AC$ and $CB$ [Prop.~1.47]. Hence, the (square) on $AB$ is greater than the
(square) on $AC$ by the (square) on $CB$. And $AC$ (is) equal to $LO$.
Thus, the (square) on $AB$ is greater than the
(square) on $LO$ by the (square) on $CB$. Therefore, if we take
$OR$ equal to $BC$ then the (square) on $AB$ will be greater than the
(square) on $LO$ by the (square) on $OR$. (Which is) the very thing
it was prescribed to do.}
\end{Parallel}

%%%%
%11.24
%%%%
\pdfbookmark[1]{Proposition 11.24}{pdf11.24}
\begin{Parallel}{}{}
\ParallelLText{
\begin{center}
{\large \ggn{24}.}
\end{center}\vspace*{-7pt}

\gr{>E`an stere`on <up`o parall'hlwn >epip'edwn peri'eqhtai, t`a >apenant'ion
a\kern -.7pt >uto~u >ep'ipeda >'isa te ka`i parallhl'ogramm'a >estin.}

\epsfysize=1.5in
\centerline{\epsffile{Book11/fig24g.eps}}

\gr{Stere`on g`ar t`o GDJH <up`o parall'hlwn >epip'edwn perieq'esjw
t~wn AG, HZ, AJ, DZ, BZ, AE; l'egw, <'oti t`a >apenant'ion a\kern -.7pt >uto~u
>ep'ipeda >'isa te ka`i parallhl'ogramm'a >estin.}

\gr{>Epe`i g`ar d'uo >ep'ipeda par'allhla t`a BH, GE <up`o >epip'edou
to~u AG t'emnetai, a<i koina`i a\kern -.7pt >ut~wn toma`i par'allhlo'i e>isin. par'allhloc
>'ara >est`in <h AB t~h| DG. p'alin, <epe`i d'uo >ep'ipeda par'allhla t`a BZ,
AE <up`o >epip'edou to~u AG t'emnetai, a<i koina`i a\kern -.7pt >ut~wn toma`i par'allhlo'i e>isin. par'allhloc >'ara >est`in <h BG t~h| AD. >ede'iqjh d`e
ka`i <h AB t~h| DG par'allhloc; parallhl'ogrammon >'ara >est`i t`o AG.
<omo'iwc d`h de'ixomen, <'oti ka`i <'ekaston t~wn DZ, ZH, HB, BZ, AE
parallhl'ogramm'on >estin.}

\gr{>Epeze'uqjwsan a<i AJ, DZ. ka`i >epe`i par'allhl'oc >estin <h m`en AB
t~h| DG, <h d`e BJ t~h| GZ, d'uo d`h a<i AB, BJ <apt'omenai >all'hlwn
par`a d'uo e>uje'iac t`ac DG, GZ <aptom'enac >all'hlwn e>is`in o>uk
>en t~w| a\kern -.7pt >ut~w| >epip'edw|; >'isac >'ara gwn'iac peri'exousin; >'ish
>'ara <h <up`o ABJ gwn'ia t~h| <up`o DGZ. ka`i >epe`i d'uo a<i AB, BJ
dus`i ta~ic DG, GZ >'isai e>is'in, ka`i gwn'ia <h <up`o ABJ
gwn'ia| t~h| <up`o DGZ >estin >'ish, b'asic >'ara <h AJ b'asei
t~h| DZ >estin >'ish, ka`i t`o ABJ tr'igwnon t~w| DGZ trig'wnw| >'ison
>est'in. ka'i >esti to~u m`en ABJ dipl'asion t`o BH parallhl'ogrammon,
to~u d`e DGZ dipl'asion t`o GE parallhl'ogrammon; >'ison >'ara
t`o BH parallhl'ogrammon t~w| GE parallhlogr'ammw|; <omo'iwc
d`h de'ixomen, <'oti ka`i t`o m`en AG t~w| HZ >estin >'ison, t`o d`e
AE t~w| BZ.}

\gr{>E`an >'ara stere`on <up`o parall'hlwn >epip'edwn peri'eq\-htai, t`a >apenant'ion
a\kern -.7pt >uto~u >ep'ipeda >'isa te ka`i parallhl'ogramm'a >estin; <'oper >'edei
de~ixai.}}

\ParallelRText{
\begin{center}
{\large Proposition 24}
\end{center}

If a solid (figure) is contained by (six) parallel planes then its opposite planes
are both equal and parallelogrammic.

\epsfysize=1.5in
\centerline{\epsffile{Book11/fig24e.eps}}

For let the solid (figure) $CDHG$ be contained by the parallel planes
$AC$, $GF$, and $AH$, $DF$, and $BF$, $AE$. I say that its opposite
planes are both equal and parallelogrammic.

For since the two parallel planes $BG$ and $CE$ are cut by the plane
$AC$, their common sections are parallel [Prop.~11.16]. Thus, $AB$ is parallel to $DC$.  Again, since
the two parallel planes $BF$ and $AE$ are cut by the plane
$AC$, their common sections are parallel [Prop.~11.16]. Thus,  $BC$ is parallel to $AD$. And
$AB$ was also shown (to be) parallel to $DC$. Thus, $AC$ is a parallelogram. So, similarly, we can also show that $DF$, $FG$, $GB$,
$BF$, and $AE$ are each parallelograms.

Let $AH$ and $DF$ be joined. And since $AB$ is parallel to
$DC$, and $BH$ to $CF$, so the two (straight-lines) joining one another,
$AB$ and $BH$,
are parallel to the two straight-lines joining one another, $DC$ and
$CF$ (respectively), not (being) in the same plane. Thus, they
will contain equal angles [Prop.~11.10]. Thus,
angle $ABH$ (is) equal to (angle) $DCF$. And since the two (straight-lines)
$AB$ and $BH$ are equal to the two (straight-lines) $DC$ and $CF$ (respectively) [Prop.~1.34], and angle $ABH$ is equal to angle $DCF$, the base
$AH$ is thus equal to the base $DF$, and triangle $ABH$ is equal to
triangle $DCF$ [Prop.~1.4]. And parallelogram
$BG$ is double (triangle) $ABH$, and parallelogram $CE$
double (triangle) $DCF$ [Prop.~1.34]. Thus,
parallelogram $BG$ (is) equal to parallelogram $CE$. So, similarly,
we can show that $AC$ is also equal to $GF$, and $AE$ to $BF$.

Thus, if a solid (figure) is contained by (six) parallel planes then its opposite planes
are both equal and parallelogrammic. (Which is) the very thing it was required to
show.}
\end{Parallel}

%%%%
%11.25
%%%%
\pdfbookmark[1]{Proposition 11.25}{pdf11.25}
\begin{Parallel}{}{}
\ParallelLText{
\begin{center}
{\large \ggn{25}.}
\end{center}\vspace*{-7pt}

\gr{>E`an stere`on parallhlep'ipedon >epip'edw| tm\kern -.7pt hj~h| parall'hlw| >'onti
to~ic >apenant'ion >epip'edoic, >'estai <wc <h b'asic pr`oc t`hn b'asin,
o<'utwc t`o stere`on pr`oc t`o stere'on.}\\

\epsfysize=1.6in
\centerline{\epsffile{Book11/fig25g.eps}}

\gr{Stere`on g`ar parallhlep'ipedon t`o ABGD >epip'edw| t~w| ZH tetm\kern -.7pt 'hsjw
parall'hlw| >'onti to~ic >apenant'ion >epip'edoic to~ic RA, DJ; l'egw,
<'oti >est`in <wc <h AEZF b'asic pr`oc t`hn EJGZ b'asin, o<'utwc
t`o ABZU stere`on pr`oc t`o EHGD stere'on.}

\gr{>Ekbebl'hsjw g`ar <h AJ >ef> <ek'atera t`a m'erh, ka`i ke'isjwsan t~h|
m`en AE >'isai <osaidhpoto~un a<i AK, KL, t~h| d`e EJ >'isai <osaidhpoto~un a<i JM, MN, ka`i sumpeplhr'wsjw t`a LO, KF, JQ, MS parallhl'ogramma ka`i t`a LP, KR, DM, MT stere'a.}

\gr{Ka`i >epe`i >'isai e>is`in a<i LK, KA, AE e>uje~iai >all'hlaic,
 >'isa  >est`i
ka`i t`a m`en LO, KF, AZ parallhl'ogramma >all'hloic, t`a d`e KX, KB,
AH >all'hloic ka`i >'eti t`a LY, KP, AR >all'hloic; >apenant'ion g'ar.
di`a t`a a\kern -.7pt >ut`a d`h ka`i t`a m`en EG, JQ, MS parallhl'ogramma >'isa
e>is`in >all'hloic, t`a d`e JH, JI, IN >'isa e>is`in >all'hloic, ka`i >'eti
t`a DJ, MW, NT; tr'ia >'ara >ep'ipeda t~wn LP, KR, AU stere~wn
tris`in >epip'edoic >est`in >'isa. >all`a t`a tr'ia tris`i to~ic >apenant'ion
>est`in >'isa; t`a >'ara tr'ia stere`a t`a LP, KR, AU >'isa >all'hloic
>est'in. di`a t`a a\kern -.7pt >ut`a d`h ka`i t`a tr'ia stere`a t`a ED, DM, MT >'isa
>all'hloic >est'in; <osaplas'iwn >'ara >est`in <h LZ b'asic t~hc AZ b'asewc,
tosautapl'asi'on >esti ka`i t`o LU stere`on to~u AU stereo~u. di`a t`a
a\kern -.7pt >ut`a d`h <osaplas'iwn >est`in <h NZ b'asic t~hc ZJ b'asewc,
tosautapl'asi'on >esti ka`i t`o NU stere`on to~u JU stereo~u. ka`i
e>i >'ish >est`in <h LZ b'asic t~h| NZ b'asei, >'ison >est`i ka`i t`o LU
stere`on t~w| NU stere~w|, ka`i e>i <uper'eqei <h LZ b'asic t~hc NZ
b'asewc, <uper'eqei ka`i t`o LU stere`on to~u NU stereo~u, ka`i e>i
>elle'ipei, >elle'ipei. tess'arwn d`h >'ontwn megej~wn, d'uo m`en b'asewn
t~wn AZ, ZJ, d'uo d`e stere~wn t~wn AU, UJ,
e>'ilhptai >is'akic pollapl'asia t~hc m`en AZ b'asewc ka`i to~u AU stereo~u
<'h te LZ b'asic ka`i t`o LU stere'on, t~hc d`e JZ b'asewc ka`i to~u JU
stereo~u <'h te NZ b'asic ka`i t`o NU
stere'on, ka`i d'edeiktai, <'oti e<i <uper'eqei <h LZ b'asic t~hc ZN
b'asewc, <uper'eqei ka`i t`o LU stere`on to~u NU
 [stereo~u],  ka`i e>i >'ish, >'ison,
ka`i e>i >elle'ipei, >elle'ipei. >'estin >'ara <wc <h AZ b'asic pr`oc
t`hn ZJ b'asin, o<'utwc t`o AU stere`on pr`oc t`o UJ stere'on; <'oper
>'edei de~ixai.}}

\ParallelRText{
\begin{center}
{\large Proposition 25}
\end{center}

If a parallelipiped solid is cut by a plane which is 
parallel to the opposite planes (of the parallelipiped) then as the base (is) to
the base, so the solid will be to the solid.\\

\epsfysize=1.6in
\centerline{\epsffile{Book11/fig25e.eps}}

For let the parallelipiped solid $ABCD$ be cut by the
plane $FG$ which is parallel to the opposite planes $RA$ and $DH$.
I say that as the base $AEFV$ (is) to the base $EHCF$, so the solid
$ABFU$ (is) to the solid $EGCD$.

For let $AH$ be produced in each direction. And let any number whatsoever (of lengths), $AK$ and $KL$, be made equal to $AE$,
and any number whatsoever (of lengths), $HM$ and $MN$, equal to $EH$.
And let the parallelograms $LP$, $KV$, $HW$, and $MS$ be completed, and the solids $LQ$, $KR$, $DM$, and $MT$.

And since the straight-lines $LK$, $KA$, and $AE$ are equal to one another,
the parallelograms $LP$, $KV$, and $AF$ are also equal to
one another, and $KO$, $KB$, and $AG$ (are equal) to one another, and,
further, $LX$, $KQ$, and $AR$ (are equal) to one another. For (they
are) opposite [Prop.~11.24]. So, for the
same (reasons), the parallelograms $EC$, $HW$, and $MS$
are also equal to one another, and $HG$, $HI$, and $IN$ are equal to
one another, and, further, $DH$, $MY$, and $NT$ (are equal to one another).
Thus, three planes of (one of) the solids $LQ$, $KR$, and $AU$ are equal to
the (corresponding) three planes (of the others). But, the three planes
(in one of the soilds) are equal to the three opposite planes [Prop.~11.24]. Thus, the three solids $LQ$,
$KR$, and $AU$ are equal to one another [Def.~11.10]. So, for the same (reasons), the
three solids $ED$, $DM$, and $MT$ are also equal to one another.
Thus, as many multiples as the base $LF$ is of the base $AF$, so many
multiples is the solid $LU$ also of the the solid $AU$. So, for the same (reasons), 
as many multiples as the base $NF$ is of the base $FH$, so many
multiples is the solid $NU$ also of the solid $HU$. And if the base
$LF$ is equal to the base $NF$ then the solid $LU$ is also
equal to the solid $NU$.$^\dag$ And if the base $LF$  exceeds the base $NF$
then the solid $LU$ also exceeds the solid $NU$. And if ($LF$) is less than ($NF$) then ($LU$) is (also) less than ($NU$). So, there are four magnitudes, the
two bases $AF$ and $FH$, and the two solids $AU$ and $UH$, and
equal multiples be taken of the base $AF$ and the solid $AU$--- (namely), the base $LF$ and the solid $LU$---and of the base
$HF$ and the solid $HU$---(namely), the base $NF$ and the solid $NU$.
And it has been shown that if the base $LF$ exceeds the base $FN$ then the
solid $LU$ also exceeds the [solid] $NU$, and if ($LF$ is) equal (to $FN$) then
($LU$ is) equal (to $NU$), and if ($LF$ is) less than ($FN$) then 
($LU$ is) less than ($NU$). Thus, as the base $AF$ is to the base
$FH$, so the solid $AU$ (is) to the solid $UH$ [Def.~5.5]. (Which is) the very thing it was required to show.}
\end{Parallel}
{\footnotesize\noindent$^\dag$ Here, Euclid assumes that $LF\gtreqqless NF$ implies $LU \gtreqqless NU$. This is easily demonstrated.}

%%%%
%11.26
%%%%
\pdfbookmark[1]{Proposition 11.26}{pdf11.26}
\begin{Parallel}{}{}
\ParallelLText{
\begin{center}
{\large \ggn{26}.}
\end{center}\vspace*{-7pt}

\gr{Pr`oc t~h| doje'ish| e>uje'ia| ka`i t~w| pr`oc a\kern -.7pt >ut~h| shme'iw| t~h| doje'ish|
stere~a| gwn'ia| >'ishn stere`an gwn'ian sust'hsasjai.}

\gr{>'Estw <h m`en doje~isa e>uje~ia <h AB, t`o d`e pr`oc a\kern -.7pt >ut~h| doj`en
shme~ion t`o A, <h d`e doje~isa stere`a gwn'ia <h pr`oc t~w| D perieqom'enh <up`o t~wn <up`o EDG, EDZ, ZDG gwni~wn >epip'edwn; de~i d`h pr`oc
t~h| AB e>uje'ia| ka`i t~w| pr`oc a\kern -.7pt >ut~h| shme'iw| t~w| A t~h| pr`oc t~w|
D stere~a| gwn'ia| >'ishn stere`an gwn'ian sust'hsasjai.}

\gr{E>il'hfjw g`ar >ep`i t~hc DZ tuq`on shme~ion t`o Z, ka`i >'hqjw >ap`o
to~u Z >ep`i t`o di`a t~wn ED, DG >ep'ipedon k'ajetoc <h ZH, ka`i sumball'etw t~w| >epip'edw| kat`a t`o H, ka`i >epeze'uqjw <h DH, ka`i
sunest'atw pr`oc t~h| AB e>uje'ia| ka`i t~w| pr`oc a\kern -.7pt >ut~h| shme'iw|
t~w| A t~h| m`en <up`o EDG gwn'ia| >'ish <h <up`o BAL, t~h| d`e
<up`o EDH >'ish <h <up`o BAK, ka`i ke'isjw t~h| DH >'ish  <h AK, ka`i
>anest'atw >ap`o to~u K shme'iou t~w| di`a t~wn BAL >epip'edw| pr`oc
>orj`ac <h KJ, ka`i ke'isjw >'ish t~h| HZ <h KJ, ka`i >epeze'uqjw <h JA;
l'egw, <'oti <h pr`oc t~w| A stere`a gwn'ia perieqom'enh <up`o t~wn
BAL, BAJ, JAL gwni~wn >'ish >est`i t~h| pr`oc t~w| D stere~a|
gwn'ia| t~h| perieqom'enh| <up`o t~wn EDG, EDZ, ZDG gwni~wn.}

\epsfysize=1.8in
\centerline{\epsffile{Book11/fig26g.eps}}

\gr{>Apeil'hfjwsan g`ar >'isai a<i AB, DE, ka`i >epeze'uqj\-wsan a<i JB, KB, ZE, HE. ka`i >epe`i <h ZH >orj'h >esti pr`oc t`o <upoke'imenon >ep'ipedon,
ka`i pr`oc p'asac >'ara t`ac <aptom'enac a\kern -.7pt >ut~hc e>uje'iac ka`i o>'usac
>en t~w| <upokeim'enw| >epip'edw| >orj`ac poi'hsei gwn'iac;
>orj`h >'ara  >est`in <ekat'era t~wn <up`o ZHD, ZHE gwni~wn. di`a
t`a a\kern -.7pt >ut`a d`h ka`i <ekat'era t~wn <up`o JKA, JKB gwni~wn >orj'h >estin.
ka`i >epe`i d'uo a<i KA, AB d'uo ta~ic HD, DE >'isai e>is`in <ekat'era <ekat'era|, ka`i gwn'iac >'isac peri'eqousin, b'asic >'ara <h KB b'asei t~h|
HE >'ish >est'in. >'esti d`e ka`i <h KJ t~h| HZ >'ish; ka`i gwn'iac
>orj`ac peri'eqousin; >'ish >'ara ka`i <h JB t~h| ZE. p'alin >epe`i d'uo
a<i AK, KJ dus`i ta~ic DH, HZ >'isai e>is'in, ka`i gwn'iac >orj`ac
peri'eqousin, b'asic >'ara <h AJ b'asei t~h| ZD >'ish >est'in. >'esti d`e
ka`i <h AB t~h| DE >'ish; d'uo d`h a<i JA, AB d'uo ta~ic DZ, DE
>'isai e>is'in. ka`i b'asic <h JB b'asei t~h| ZE >'ish; gwn'ia
>'ara <h <up`o BAJ gwn'ia| t~h| <up`o EDZ >estin >'ish. di`a t`a a\kern -.7pt >ut`a
d`h ka`i <h <up`o JAL t~h| <up`o ZDG >estin >'ish. >'esti d`e ka`i <h <up`o
BAL t~h| <up`o EDG >'ish.}

\gr{Pr`oc >'ara t~h| doje'ish| e>uje'ia| t~h| AB ka`i t~w| pr`oc 	a\kern -.7pt >ut~h|
shme'iw| t~w| A t~h| doje'ish| stere~a| gwn'ia| t~h| pr`oc
t~w| D >'ish sun'estatai; <'oper >'edei poi~hsai.}}

\ParallelRText{
\begin{center}
{\large Proposition 26}
\end{center}

To construct a solid angle equal to a given solid
angle on a given straight-line, and at a given point on it.

Let $AB$ be the given straight-line, and $A$ the given point on it, and
$D$ the given solid angle, contained by the plane angles $EDC$, $EDF$, and
$FDC$. So, it is necessary to construct a solid angle equal to the solid angle $D$ on the straight-line $AB$, and at the point $A$ 
on it.

For let some random point $F$ be taken on $DF$, and let
$FG$ be drawn from $F$ perpendicular to the plane through $ED$ and
$DC$  [Prop.~11.11],  and let it meet the plane at $G$, and let $DG$ be joined.
And let $BAL$, equal to the angle $EDC$,  and $BAK$, equal to $EDG$, be constructed 
on the straight-line $AB$ at the point $A$ on it [Prop.~1.23]. And let $AK$ be made equal to $DG$.  And let $KH$
be set up at the point $K$ at right-angles to the plane through $BAL$
[Prop.~11.12]. And let $KH$ be made equal to
$GF$. And let $HA$ be joined. I say that the solid angle at $A$,
contained by the (plane) angles $BAL$, $BAH$, and $HAL$, is equal
to the solid angle at $D$, contained by the (plane) angles $EDC$,
$EDF$, and $FDC$.

\epsfysize=1.8in
\centerline{\epsffile{Book11/fig26e.eps}}

For let $AB$ and $DE$ be cut off (so as to be) equal, and let
$HB$, $KB$, $FE$, and $GE$ be joined. And since $FG$
is at right-angles to the reference plane ($EDC$), it will also   make right-angles
with all of the straight-lines joined to it which are also in the reference
plane [Def.~11.3]. Thus, the angles $FGD$ and
$FGE$ are right-angles.  So, for the same (reasons), the angles $HKA$
and $HKB$ are also right-angles. And since the two (straight-lines)
$KA$ and $AB$ are equal to the two (straight-lines) $GD$ and $DE$,
respectively, and they contain equal angles, the base $KB$ is thus equal
to the base $GE$ [Prop.~1.4]. And $KH$ is also equal to $GF$. And they contain right-angles (with the respective bases). 
Thus, $HB$ (is) also equal to $FE$ [Prop.~1.4]. 
Again, since the two (straight-lines) $AK$ and $KH$ are equal to
the two (straight-lines) $DG$ and $GF$ (respectively), and they contain
right-angles, the base $AH$ is thus equal to the base $FD$ [Prop.~1.4].  And $AB$ (is) also equal to $DE$. So, the
two (straight-lines) $HA$ and $AB$ are equal to the two (straight-lines)
$DF$ and $DE$ (respectively).  And the base $HB$ (is) equal to the
base $FE$. Thus, the angle $BAH$ is equal to the angle $EDF$
[Prop.~1.8]. So, for the same (reasons),  $HAL$
is also equal to $FDC$. And $BAL$ is also equal to $EDC$.

Thus, (a solid angle) has been constructed, equal to the given solid angle at
$D$, on the given straight-line $AB$, at the given point $A$ on it.
(Which is) the very thing it was required to do.}
\end{Parallel}

%%%%
%11.27
%%%%
\pdfbookmark[1]{Proposition 11.27}{pdf11.27}
\begin{Parallel}{}{}
\ParallelLText{
\begin{center}
{\large \ggn{27}.}
\end{center}\vspace*{-7pt}

\gr{>Ap`o t~hc doje'ishc e>uje'iac t~w| doj'enti stere~w| parallhlepip'edw|
<'omoi'on te ka`i <omo'iwc ke'imenon stere`on parallhlep'ipedon
>anagr'ayai.}

\gr{>'Estw <h m`en doje~isa e>uje~ia <h AB, t`o d`e doj`en stere`on
parallhlep'ipedon t`o GD; de~i d`h >ap`o t~hc doje'ishc e>uje'iac
t~hc AB t~w| doj'enti stere~w| parallhlepip'edw| t~w| GD
<'omoi'on te ka`i <omo'iwc ke'imenon stere`on parallhlep'ipedon
>anagr'ayai.}

\gr{Sunest'atw g`ar pr`oc t~h|  AB e>uje'ia| ka`i t~w| pr`oc a\kern -.7pt >ut~h|
shme'iw| t~w| A t~h| pr`oc t~w| G stere~a| gwn'ia| >'ish
<h perieqom'enh <up`o t~wn BAJ, JAK, KAB, <'wste >'ishn e>~inai t`hn
m`en <up`o BAJ gwn'ian t~h| <up`o EGZ, t`hn d`e <up`o BAK t~h|
<up`o EGH, t`hn d`e <up`o KAJ t~h| <up`o HGZ; ka`i gegon'etw <wc m`en
<h EG pr`oc t`hn GH, o<'utwc <h BA pr`oc t`hn AK, <wc d`e
<h HG pr`oc t`hn GZ,  o<'utwc <h KA pr`oc t`hn AJ. ka`i di> >'isou
>'ara >est`in <wc <h EG pr`oc t`hn GZ, o<'utwc <h BA pr`oc
t`hn AJ. ka`i sumpeplhr'wsjw t`o JB parallhl'ogrammon ka`i t`o AL
stere'on.}

\epsfysize=1.5in
\centerline{\epsffile{Book11/fig27g.eps}}

\gr{Ka`i >epe'i >estin <wc <h EG pr`oc t`hn GH, o<'utwc <h BA
pr`oc t`hn AK, ka`i per`i >'isac gwn'iac t`ac <up`o EGH, BAK
a<i pleura`i >an'alog'on e>isin, <'omoion >'ara >est`i t`o HE
parallhl'ogrammon t~w| KB parallhlogr'ammw|. di`a t`a a\kern -.7pt >ut`a
d`h ka`i t`o m`en KJ parallhl'ogrammon t~w| HZ parallhlogr'ammw|
<'omoi'on >esti ka`i >'eti t`o ZE t~w| JB; tr'ia >'ara parallhl'ogramma to~u
GD stereo~u tris`i parallhlogr'ammoic to~u AL stereo~u <'omoi'a >estin.
>all`a t`a m`en tr'ia tris`i to~ic >apenant'ion >'isa t'e >esti
ka`i <'omoia, t`a d`e tr'ia tris`i to~ic >apenant'ion
>'isa t'e >esti ka`i <'omoia; <'olon >'ara t`o GD stere`on <'olw| t~w|
AL stere~w| <'omoi'on >estin.}

\gr{>Ap`o t~hc doje'ishc >'ara e>uje'iac t~hc AB t~w| doj'enti
stere~w| parallhlepip'edw| t~w| GD <'omoi'on te ka`i <omo'iwc
ke'imenon >anag'egraptai t`o AL; <'oper >'edei poi~hsai.}}

\ParallelRText{
\begin{center}
{\large Proposition 27}
\end{center}

To describe a parallelepiped solid similar,  and
similarly laid out, to a given parallelepiped solid on a given straight-line.

Let the given straight-line be $AB$, and the given parallelepiped solid
$CD$. So, it is necessary to describe a parallelepiped solid similar, and
similarly laid out, to the given parallelepiped solid $CD$ on the given straight-line $AB$.

For, let a (solid angle) contained by the (plane angles)
$BAH$, $HAK$, and $KAB$ be constructed, equal to solid angle at
$C$, on the straight-line $AB$ at the point $A$ on it [Prop.~11.26], such that angle $BAH$ is equal to $ECF$, and $BAK$ to $ECG$,
and $KAH$ to $GCF$. And let it be contrived that as $EC$
(is) to $CG$, so $BA$ (is) to $AK$, and as $GC$ (is) to $CF$, so
$KA$ (is) to $AH$ [Prop.~6.12]. And thus, via equality, as $EC$ is to $CF$, so
$BA$ (is) to $AH$ [Prop.~5.22]. And let the parallelogram $HB$ be completed, and the solid $AL$.

\epsfysize=1.5in
\centerline{\epsffile{Book11/fig27e.eps}}

And since as $EC$ is to $CG$, so $BA$ (is) to $AK$, and the sides about the equal angles $ECG$ and $BAK$ are (thus) proportional, the parallelogram
$GE$ is thus similar to the parallelogram $KB$. So, for the same (reasons),
the parallelogram $KH$ is also similar to the parallelogram $GF$, and,
further, $FE$ (is similar) to $HB$. Thus, three of the parallelograms
of solid $CD$ are similar to three of the parallelograms of solid $AL$.
But, the (former) three are equal and similar to the three opposite, and
the (latter) three are equal and similar to the three opposite. Thus, the
whole solid $CD$ is similar to the whole solid $AL$ [Def.~11.9].

Thus, $AL$, similar, and similarly laid out, to the given parallelepiped solid
$CD$, has been described on the given straight-lines $AB$. (Which is) the very thing it was required to do.}
\end{Parallel}

%%%%
%11.28
%%%%
\pdfbookmark[1]{Proposition 11.28}{pdf11.28}
\begin{Parallel}{}{}
\ParallelLText{
\begin{center}
{\large \ggn{28}.}
\end{center}\vspace*{-7pt}

\gr{>E`an stere`on parallhlep'ipedon >epip'edw| tm\kern -.7pt hj~h| kat`a t`ac diagwn'iouc
t~wn >apenant'ion >epip'edwn, d'iqa tm\kern -.7pt hj'hsetai t`o stere`on <up`o to~u
>epip'edou.}

\epsfysize=1.8in
\centerline{\epsffile{Book11/fig28g.eps}}

\gr{Stere`on g`ar parallhlep'ipedon t`o AB >epip'edw| t~w| GDEZ tetm\kern -.7pt 'hsjw
kat`a t`ac diagwn'iouc t~wn >apenant'ion >epip'edwn t`ac GZ, DE;
l'egw, <'oti d'iqa tm\kern -.7pt hj'hsetai t`o AB stere`on <up`o to~u GDEZ
>epip'edou.}

\gr{>Epe`i g`ar >'ison >est`i t`o m`en GHZ tr'igwnon t~w| GZB trig'wnw|, t`o d`e ADE t~w| DEJ, >'esti d`e ka`i t`o m`en GA parallhl'ogrammon t~w| EB
>'ison; >apenant'ion g'ar; t`o d`e HE t~w| GJ, ka`i t`o pr'isma >'ara t`o perieq'omenon <up`o d'uo m`en trig'wnwn t~wn GHZ, ADE, tri~wn d`e parallhlogr'ammwn t~wn HE, AG, GE >'ison >est`i t~w| pr'ismati t~w|
perieqom'enw| <up`o d'uo m`en trig'wnwn t~wn GZB, DEJ, tri~wn
d`e parallhlogr'ammwn t~wn GJ, BE, GE; <up`o g`ar >'iswn >epip'edwn
peri'eqontai t~w| te pl'hjei ka`i t~w| meg'ejei. <'wste
<'olon t`o AB stere`on d'iqa t'etm\kern -.7pt htai <up`o to~u GDEZ >epip'edou;
<'oper >'edei de~ixai.}}
 
\ParallelRText{
\begin{center}
{\large Proposition 28}
\end{center}

If a parallelepiped solid is cut by a plane (passing) through the diagonals of  (a pair of) opposite planes then the solid will be cut in half by the
plane.

\epsfysize=1.8in
\centerline{\epsffile{Book11/fig28e.eps}}

For let the parallelepiped solid $AB$ be cut by the plane
$CDEF$ (passing) through the diagonals of the opposite planes $CF$ and $DE$.$^\dag$
I say that the solid $AB$ will be cut in half by the plane $CDEF$.

For since triangle $CGF$ is equal to triangle $CFB$, and $ADE$ (is equal)
to $DEH$ [Prop.~1.34], and parallelogram
$CA$ is also equal to $EB$---for (they are) opposite [Prop.~11.24]---and $GE$ (equal) to $CH$,  thus the prism contained
by the two triangles $CGF$ and $ADE$, and the three parallelograms $GE$, $AC$, and $CE$, is also equal to the prism contained by the two
triangles $CFB$ and $DEH$, and the three parallelograms $CH$, $BE$, and
$CE$. For they are contained by planes (which are) equal in number and in magnitude
[Def.~11.10].$^\ddag$ Thus, the whole of solid $AB$
is cut in half by the plane $CDEF$. (Which is) the very thing it was required to show.}
\end{Parallel}
{\footnotesize\noindent$^\dag$ Here, it is assumed that the two diagonals lie in the same plane. The proof is easily supplied.\\
\noindent$^\ddag$ However, strictly speaking, the prisms are not similarly arranged, being mirror images of one another.}

%%%%
%11.29
%%%%
\pdfbookmark[1]{Proposition 11.29}{pdf11.29}
\begin{Parallel}{}{}
\ParallelLText{
\begin{center}
{\large \ggn{29}.}
\end{center}\vspace*{-7pt}

\gr{T`a >ep`i t~hc a\kern -.7pt >ut~hc b'asewc >'onta stere`a parallhlep'ipeda ka`i <up`o
t`o a\kern -.7pt >ut`o <'uyoc, <~wn a<i >efest~wsai >ep`i t~wn a\kern -.7pt >ut~wn e>isin
e>ujei~wn, >'isa >all'hloic >est'in.}\\

\epsfysize=1.7in
\centerline{\epsffile{Book11/fig29g.eps}}

\gr{>'Estw >ep`i t~hc a\kern -.7pt >ut~hc b'asewc t~hc AB stere`a parallhlep'ipeda t`a GM,
GN <up`o t`o a\kern -.7pt >ut`o <'uyoc, <~wn a<i >efest~wsai a<i AH, AZ, LM,
LN, GD, GE, BJ, BK >ep`i t~wn a\kern -.7pt >ut~wn e>ujei~wn >'estwsan
t~wn ZN, DK; l'egw, <'oti >'ison >est`i t`o GM stere`on t~w| GN stere~w|.}

\gr{>Epe`i g`ar parallhl'ogramm'on >estin <ek'ateron t~wn GJ, GK,
>'ish >est`in <h GB <ekat'era| t~wn DJ, EK; <'wste ka`i <h DJ
t~h| EK >estin >'ish. koin`h >afh|r'hsjw <h EJ; loip`h
>'ara <h DE loip~h| t~h| JK >estin >'ish. <'wste ka`i t`o m`en
DGE tr'igwnon t~w| JBK trig'wnw| >'ison >est'in, t`o d`e DH
parallhl'ogrammon t~w| JN parallhlogr'ammw|. di`a t`a a\kern -.7pt >ut`a
d`h ka`i t`o AZH tr'igwnon t~w| MLN trig'wnw| >'ison >est'in.
>'esti d`e ka`i t`o m`en GZ parallhl'ogrammon t~w| BM parallhlogr'ammw|
>'ison, t`o d`e GH t~w| BN; >apenant'ion g'ar; ka`i t`o pr'isma
>'ara t`o perieq'omenon <up`o d'uo m`en trig'wnwn t~wn AZH, DGE,
tri~wn d`e parallhlogr'ammwn t~wn AD, DH, GH >'ison >est`i
t~w| pr'ismati t~w| perieqom'enw| <up`o d'uo m`en trig'wnwn t~wn
MLN, JBK, tri~wn d`e parallhlogr'ammwn t~wn BM, JN,
BN. koin`on proske'isjw t`o stere`on, o<~u b'asic m`en t`o AB
parallhl'ogrammon, >apenant'ion d`e t`o HEJM;
<'olon >'ara t`o GM stere`on parallhlep'ipedon <'olw| t~w| GN
stere~w| parallhlepip'edw| >'ison >est'in.}

\gr{T`a >'ara >ep`i t~hc a\kern -.7pt >ut~hc b'asewc >'onta stere`a parallhlep'ipeda
ka`i <up`o t`o a\kern -.7pt >ut`o <'uyoc, <~wn a<i >efest~wsai
>ep`i t~wn a\kern -.7pt >ut~wn e>isin e>ujei~wn, >'isa >all'hloic >est'in; <'oper
>'edei de~ixai.}}

\ParallelRText{
\begin{center}
{\large Proposition 29}
\end{center}

Parallelepiped solids which are on the same base,
and (have) the same height, and in which the (ends of the straight-lines) standing up are on the same straight-lines, 
are equal to one another.

\epsfysize=1.7in
\centerline{\epsffile{Book11/fig29e.eps}}

For let the parallelepiped solids $CM$ and $CN$ be on the same base $AB$,
and (have) the same height, and let the (ends of the straight-lines)
standing up in them, $AG$, $AF$, $LM$, $LN$, $CD$, $CE$, $BH$, and $BK$,
be  on the same straight-lines, $FN$ and $DK$. I say that  solid $CM$
is equal to solid $CN$.

For since $CH$ and $CK$ are each parallelograms, $CB$ is equal to each 
of $DH$ and $EK$ [Prop.~1.34]. Hence,
$DH$ is also equal to $EK$. Let $EH$ be subtracted from both.
Thus, the remainder $DE$ is equal to the remainder $HK$.
Hence, triangle $DCE$ is also equal to triangle $HBK$ [Props.~1.4, 1.8], and parallelogram $DG$ to parallelogram $HN$
[Prop.~1.36]. So, for the same (reasons), traingle
$AFG$ is also equal to triangle $MLN$. And parallelogram $CF$ is
also equal to parallelogram $BM$, and $CG$ to $BN$ [Prop.~11.24]. For they are opposite. Thus, the prism contained by the
two triangles $AFG$ and $DCE$, and the three parallelograms $AD$,
$DG$, and $CG$, is equal to the prism contained by the two triangles
$MLN$ and $HBK$, and the three parallelograms $BM$, $HN$, and
$BN$. Let the solid whose base (is) parallelogram $AB$, and
(whose) opposite (face is) $GEHM$, be added to both (prisms).
Thus, the whole parallelepiped solid $CM$ is equal to the whole parallelepiped solid $CN$.

Thus, parallelepiped solids which are on the same base,
and (have) the same height, and in which the (ends of the straight-lines) standing up (are) on the same straight-lines, 
are equal to one another. (Which is) the very thing it was required to show.}
\end{Parallel}

%%%%
%11.30
%%%%
\pdfbookmark[1]{Proposition 11.30}{pdf11.30}
\begin{Parallel}{}{}
\ParallelLText{
\begin{center}
{\large \ggn{30}.}
\end{center}\vspace*{-7pt}

\gr{T`a >ep`i t~hc a\kern -.7pt >ut~hc b'asewc >'onta stere`a parallhlep'ipeda ka`i <up`o
t`o a\kern -.7pt >ut`o <'uyoc, <~wn a<i >efest~wsai o>uk e>is`in >ep`i t~wn
a\kern -.7pt >ut~wn e>ujei~wn, >'isa >all'hloic >est'in.}\\

\epsfysize=2.2in
\centerline{\epsffile{Book11/fig30g.eps}}

\gr{>'Estw >ep`i t~hc a\kern -.7pt >ut~hc b'asewc t~hc AB stere`a parallhlep'ipeda
t`a GM, GN <up`o t`o a\kern -.7pt >ut`o <'uyoc, <~wn a<i >efest~wsai a<i AZ,
AH, LM, LN, GD, GE, BJ, BK m\kern -.7pt `h >'estwsan >ep`i t~wn a\kern -.7pt >ut~wn
e>ujei~wn; l'egw, <'oti >'ison >est`i t`o GM stere`on t~w| GN stere~w|.}

\gr{>Ekbebl'hsjwsan g`ar a<i NK, DJ ka`i sumpipt'etwsan >all'hlaic kat`a
t`o R, ka`i >'eti >ekbebl'hsjwsan a<i ZM, HE >ep`i t`a O, P, ka`i
>epeze'uqjwsan a<i AX, LO, GP, BR. >'ison d'h >esti t`o GM stere'on,
o<~u b'asic m`en t`o AGBL parallhl'ogrammon, >apenant'ion d`e t`o
ZDJM, t~w| GO stere~w|, o<~u b'asic m`en t`o AGBL parallhl'ogrammon,
>apenant'ion d`e t`o XPRO; >ep'i te g`ar t~hc a\kern -.7pt >ut~hc b'ase'wc e>isi
t~hc AGBL ka`i <up`o t`o a\kern -.7pt >ut`o <'uyoc, <~wn a<i >efest~wsai a<i
AZ, AX, LM, LO, GD, GP, BJ, BR >ep`i t~wn a\kern -.7pt >ut~wn e>isin
e>ujei~wn t~wn ZO, DR. >all`a t`o GO stere'on, o<~u b'asic
m'en >esti
t`o AGBL parallhl'ogrammon, >apenant'ion d`e t`o 
XPRO, >'ison >est`i t~w| GN stere~w|, o<~u b'asic m`en t`o
AGBL parallhl'ogrammon, >apenant'ion d`e t`o
HEKN; >ep'i
te g`ar p'alin t~hc a\kern -.7pt >ut~hc b'ase'wc e>isi t~hc AGBL ka`i
<up`o t`o a\kern -.7pt >ut`o <'uyoc, <~wn a<i >efest~wsai a<i AH, AX, GE, GP,
LN, LO, BK, BR >ep`i t~wn a\kern -.7pt >ut~wn e>isin e>ujei~wn t~wn HP, NR. <'wste
ka`i t`o GM stere`on >'ison >est`i t~w| GN stere~w|.}

\gr{T`a >'ara >ep`i t~hc a\kern -.7pt >ut~hc b'asewc  stere`a parallhlep'ipeda ka`i <up`o
t`o a\kern -.7pt >ut`o <'uyoc, <~wn a<i >efest~wsai o>uk e>is`in >ep`i t~wn
a\kern -.7pt >ut~wn e>ujei~wn, >'isa >all'hloic >est'in; <'oper >'edei de~ixai.}}

\ParallelRText{
\begin{center}
{\large Proposition 30}
\end{center}

Parallelepiped solids which are on the same base,
and (have) the same height, and in which the (ends of the straight-lines)
standing up are not on the same straight-lines, are equal to one another.

\epsfysize=2.2in
\centerline{\epsffile{Book11/fig30e.eps}}

Let the parallelepiped solids $CM$ and $CN$ be on the same base, $AB$,
and (have) the same height, and let the (ends of the straight-lines)
standing up in them, $AF$, $AG$, $LM$, $LN$, $CD$, $CE$, $BH$,
and $BK$, not be on the same straight-lines. I say that the solid $CM$
is equal to the solid $CN$.

For let $NK$ and $DH$ be produced, and let them have joined
one another at $R$. And, further, let $FM$ and $GE$ be produced
to $P$ and $Q$ (respectively). And let $AO$, $LP$, $CQ$, and
$BR$ be joined. So, solid $CM$, whose base (is)
parallelogram $ACBL$, and opposite (face) $FDHM$, is equal to solid
$CP$, whose base (is) parallelogram $ACBL$, and opposite
(face) $OQRP$. For they are on the same base, $ACBL$, and (have) the
same height, and the (ends of the straight-lines) standing up in them,
$AF$, $AO$, $LM$, $LP$, $CD$, $CQ$, $BH$, and $BR$, are on the
same straight-lines, $FP$ and $DR$ [Prop.~11.29]. 
But,  solid $CP$, whose base is parallelogram
$ACBL$, and opposite (face) $OQRP$, is equal to solid
$CN$, whose base (is) parallelogram $ACBL$, and opposite (face)
$GEKN$. For, again, they are on the same base, $ACBL$,
and (have) the same height, and the (ends of the straight-lines) standing up
in them, $AG$, $AO$, $CE$, $CQ$, $LN$, $LP$, $BK$, and $BR$,
are on the same straight-lines, $GQ$ and $NR$ [Prop.~11.29]. Hence, solid $CM$ is also equal to solid $CN$.

Thus, parallelepiped solids (which are) on the same base,
and (have) the same height, and in which the (ends of the straight-lines)
standing up are not on the same straight-lines, are equal to one another.
(Which is) the very thing it was required to show.}
\end{Parallel}

%%%%
%11.31
%%%%
\pdfbookmark[1]{Proposition 11.31}{pdf11.31}
\begin{Parallel}{}{}
\ParallelLText{
\begin{center}
{\large \ggn{31}.}
\end{center}\vspace*{-7pt}

\gr{T`a >ep`i >'iswn b'asewn >'onta stere`a parallhlep'ipeda ka`i <up`o
t`o a\kern -.7pt >ut`o <'uyoc >'isa >all'hloic >est'in.}

\gr{>'Estw >ep`i >'iswn b'asewn t~wn AB, GD stere`a parallhlep'ipeda
t`a AE, GZ <up`o t`o a\kern -.7pt >ut`o <'uyoc. l'egw, <'oti >'ison >est`i t`o AE
stere`on t~w| GZ stere~w|.}

\gr{>'Estwsan d`h pr'oteron a<i >efesthku~iai a<i JK, BE, AH, LM, OP,
DZ, GX, RS pr`oc >orj`ac ta~ic AB, GD b'asesin, ka`i >ekbebl'hsjw
>ep> e>uje'iac t~h| GR e>uje~ia <h RT, ka`i sunest'atw pr`oc t~h|
RT e>uje'ia| ka`i t~w| pr`oc a\kern -.7pt >ut~h| shme'iw| t~w| R t~h| <up`o ALB
gwn'ia| >'ish <h <up`o TRU, ka`i ke'isjw t~h| m`en AL >'ish <h RT,
t~h| d`e LB >'ish <h RU, ka`i sumpeplhr'wsjw <'h te RQ b'asic
ka`i t`o YU stere'on.}

\epsfysize=4in
\centerline{\epsffile{Book11/fig31g.eps}}

\gr{Ka`i >epe`i d'uo a<i TR, RU dus`i ta~ic AL,
LB >'isai e>is'in, ka`i gwn'iac >'isac peri'eqousin, >'ison >'ara
ka`i <'omoion t`o RQ parallhl'ogrammon t~w| JL parallhlogr'ammw|.
ka`i >epe`i p'alin >'ish m`en <h AL t~h| RT, <h d`e LM t~h| RS, ka`i
gwn'iac >orj`ac peri'eqousin, >'ison >'ara ka`i <'omoi'on >esti
t`o RY parallhl'ogrammon t~w| AM parallhlogr'ammw|. di`a t`a
a\kern -.7pt >ut`a d`h ka`i t`o LE t~w| SU >'ison t'e >esti ka`i <'omoion;
tr'ia >'ara parallhl'ogramma to~u AE stereo~u tris`i parallhlogr'ammoic
to~u YU stereo~u >'isa t'e >esti ka`i <'omoia. >all`a t`a m`en tr'ia
tris`i to~ic >apenant'ion >'isa t'e >esti ka`i <'omoia, t`a d`e tr'ia tris`i
to~ic >apenant'ion; <'olon >'ara t`o AE stere`on parallhlep'ipedon
<'olw| t~w| YU stere~w| parallhlepip'edw| >'ison >est'in. di'hqjwsan
a<i DR, QU ka`i sumpipt'etwsan >all'hlaic kat`a t`o W,
ka`i di`a to~u T t~h| DW par'allhloc >'hqjw <h aT\sampi,
ka`i >ekbebl'hsjw <h OD kat`a t`o a, ka`i sumpeplhr'wsjw t`a WY, RI stere'a. >'ison d'h >esti t`o YW stere'on, o<~u b'asic m'en >esti t`o
RY parallhl'ogrammon, >apenant'ion
d`e t`o W\qoppa, t~w| YU stere~w|, o<~u b'asic m`en t`o RY parallhl'ogrammon, >apenant'ion d`e t`o UF; >ep'i te g`ar t~hc a\kern -.7pt >ut~hc
b'ase'wc e>isi t~hc RY ka`i <up`o t`o a\kern -.7pt >ut`o <'uyoc, <~wn a<i
>efest~wsai a<i RW, RU, T\sampi, TQ, S\stigma, S$\tilde{\mbox{o}}$, Y\qoppa,
YF >ep`i t~wn a\kern -.7pt >ut~wn e>isin e>ujei~wn t~wn WQ, \stigma F. >all`a
t`o YU stere`on t~w| AE >estin >'ison; ka`i t`o YW >'ara stere`on t~w|
AE stere~w| >estin >'ison. ka`i >epe`i >'ison >est`i t`o RUQT
parallhl'ogrammon t~w| WT parallhlogr'ammw|; >ep'i te g`ar t~hc
a\kern -.7pt >ut~hc b'ase'wc e>isi t~hc RT ka`i >en ta~ic a\kern -.7pt >uta~ic
parall'hloic ta~ic RT, WQ; >all`a t`o RUQT t~w| GD >estin >'ison,
>epe`i ka`i t~w| AB, ka`i t`o WT >'ara parallhl'ogrammon t~w| GD
>estin >'ison. >'allo d`e t`o DT; >'estin >'ara <wc <h GD b'asic
pr`oc t`hn DT, o<'utwc <h WT pr`oc t`hn DT. ka`i >epe`i
stere`on parallhlep'ipedon t`o GI >epip'edw| t~w| RZ  t'etm\kern -.7pt htai parall'hlw|
>'onti to~ic >apenant'ion >epip'edoic, >'estin <wc <h GD b'asic
pr`oc t`hn DT b'asin, o<'utwc t`o GZ stere`on pr`oc t`o RI
stere'on. di`a t`a a\kern -.7pt >ut`a d'h, >epe`i stere`on parallhlep'ipedon t`o WI
>epip'edw| t~w| RY t'etm\kern -.7pt htai parall'hlw| >'onti to~ic >apenant'ion
>epip'edoic, >'estin <wc <h WT b'asic pr`oc t`hn TL b'asin, o<'utwc
t`o WY
stere`on pr`oc t`o RI. >all> <wc <h GD
b'asic pr`oc t`hn DT, o<'utwc <h WT pr`oc t`hn DT; ka`i
<wc >'ara t`o GZ stere`on pr`oc t`o RI stere'on, o<'utwc t`o WY
stere`on pr`oc t`o RI. <ek'ateron >'ara t~wn GZ, WY stere~wn pr`oc t`o RI
t`on a\kern -.7pt >ut`on >'eqei l'ogon; >'ison >'ara >est`i t`o GZ stere`on
t~w| WY stere~w|. >all`a t`o WY t~w| AE >ede'iqjh >'ison; ka`i t`o AE
>'ara t~w| GZ >estin >'ison.}\\

\epsfysize=1.8in
\centerline{\epsffile{Book11/fig31ag.eps}}

\gr{m\kern -.7pt `h >'estwsan d`h a<i >efesthku~iai a<i AH, JK, BE, LM, GX, OP,
DZ, RS pr`oc >orj`ac ta~ic AB, GD b'asesin; l'egw p'alin, <'oti
<'ison t`o AE stere`on t~w| GZ stere~w|. >'hqjwsan g`ar >ap`o t~wn K,
E, H, M, P, Z, X, S shme'iwn >ep`i t`o <upoke'imenon >ep'ipedon k'ajetoi
a<i KN, ET, HU, MF, PQ, ZY, XW, SI, ka`i sumball'etwsan t~w| >epip'edw|
kat`a t`a N, T, U, F, Q, Y, W, I shme~ia, ka`i >epeze'uqjwsan a<i NT, NU,
UF, TF, QY, QW, WI, IY. >'ison d'h >esti t`o KF stere`on t~w| PI
stere~w|; >ep'i te g`ar >'iswn b'ase'wn e>isi t~wn KM, PS ka`i
<up`o t`o a\kern -.7pt >ut`o <'uyoc, <~wn a<i >efest~wsai pr`oc >orj'ac
e>isi ta~ic b'asesin. >all`a t`o m`en KF stere`on t~w| AE stere~w|
>estin >'ison, t`o d`e PI t~w| GZ; >ep'i te g`ar t~hc a\kern -.7pt >ut~hc b'ase'wc
e>isi ka`i <up`o t`o a\kern -.7pt >ut`o <'uyoc, <~wn a<i >efest~wsai o>'uk e>isin
>ep`i t~wn a\kern -.7pt >ut~wn e>ujei~wn. ka`i t`o AE >'ara stere`on t~w|
GZ stere~w| >estin >'ison.}

\gr{T`a >'ara >ep`i >'iswn b'asewn >'onta stere`a parallhlep'ipeda ka`i
<up`o t`o a\kern -.7pt >ut`o <'uyoc >'isa >all'hloic
>est'in; <'oper >'edei de~ixai.}}

\ParallelRText{
\begin{center}
{\large Proposition 31}
\end{center}

Parallelepiped solids which are on equal bases,
and (have) the same height, are equal to one another.

Let the parallelepiped solids $AE$ and $CF$ be on the equal bases $AB$
and $CD$ (respectively), and (have) the same height. I say that solid
$AE$ is equal to solid $CF$.

 So, let the
(straight-lines) standing up, $HK$, $BE$, $AG$, $LM$, $PQ$, $DF$,
$CO$, and $RS$, first of all, be at right-angles to the bases $AB$
and $CD$. And let $RT$ be produced in a straight-line with
$CR$. And let  (angle) $TRU$, equal to angle $ALB$, be
constructed on the straight-line $RT$, at the point $R$ on it [Prop.~1.23]. And let
$RT$ be made equal to $AL$, and $RU$ to $LB$. And let the base
$RW$, and the solid $XU$,  be completed.

\epsfysize=4in
\centerline{\epsffile{Book11/fig31e.eps}}

And since the two (straight-lines) $TR$ and $RU$ are
equal to the two (straight-lines) $AL$ and $LB$ (respectively), and
they contain equal angles, parallelogram $RW$ is thus  equal
and similar to parallelogram $HL$ [Prop.~6.14]. 
And, again, since $AL$ is equal to $RT$, and $LM$ to $RS$, and they contain right-angles, parallelogram $RX$ is thus equal and similar to
parallelogram $AM$ [Prop.~6.14].  So, for the
same (reasons), $LE$ is also equal and similar to $SU$. Thus, three
parallelograms of solid $AE$ are equal and similar to three parallelograms of
 solid $XU$. But, the three (faces of the former solid)
are equal and similar to the three opposite (faces), and the three (faces
of the
latter solid) to the three opposite (faces) [Prop.~11.24]. Thus, the whole
parallelepiped solid $AE$ is equal to the whole parallelepiped solid
$XU$ [Def.~11.10]. Let $DR$ and
$WU$ be drawn across, and let them have met one another at $Y$.
And let $aTb$ be drawn through $T$ parallel to $DY$. And let
$PD$ be produced to $a$. And let the solids $YX$ and $RI$ be 
completed. So,  solid $XY$, whose base is parallelogram $RX$,
and opposite (face) $Yc$,  is equal to  solid $XU$, whose base
(is) parallelogram $RX$, and opposite (face) $UV$. For they are
on the same base $RX$, and (have) the same height,  and the (ends of the straight-lines) standing up in them, $RY$, $RU$, $Tb$, $TW$, $Se$, $Sd$, $Xc$ and $XV$, are on the same straight-lines, $YW$ and $eV$ 
[Prop.~11.29].  But, solid $XU$ is equal to
$AE$. Thus, solid $XY$ is also equal to solid $AE$. And since
parallelogram $RUWT$ is equal to parallelogram $YT$. For they are
on the same base $RT$, and between the same parallels $RT$ and $YW$
[Prop.~1.35]. But, $RUWT$ is equal to
$CD$, since (it is) also (equal) to $AB$. Parallelogram $YT$ is thus also
equal to $CD$. And $DT$ is another (parallelogram). Thus, as base $CD$
is to $DT$, so $YT$ (is) to $DT$ [Prop.~5.7]. 
And since the parallelepiped solid $CI$ has been cut by the plane
$RF$, which is parallel to the opposite planes (of $CI$), as  base $CD$
is to base $DT$, so solid $CF$ (is) to solid $RI$ [Prop.~11.25]. So, for the same (reasons), since the parallelepiped solid
$YI$ has been cut by the plane $RX$, which is parallel to the opposite
planes (of $YI$), as base $YT$ is to base $TD$, so solid $YX$ (is)
 to solid $RI$ [Prop.~11.25]. But, as base $CD$ (is) to $DT$, so $YT$ (is) to $DT$. And, thus, as solid $CF$ (is) to solid $RI$,
so solid $YX$ (is) to solid $RI$. Thus,  solids $CF$ and
$YX$ each have the same ratio to $RI$ [Prop.~5.11]. Thus, solid $CF$
is equal to solid $YX$ [Prop.~5.9]. But,
$YX$ was show (to be) equal to $AE$. Thus, $AE$ is also equal to
$CF$.

\epsfysize=1.8in
\centerline{\epsffile{Book11/fig31ae.eps}}

And so let the (straight-lines) standing up, $AG$, $HK$, $BE$,
$LM$, $CO$, $PQ$, $DF$, and $RS$, not be at right-angles to
the bases $AB$ and $CD$. Again, I say that solid $AE$ (is) equal to
solid  $CF$. For let $KN$, $ET$, $GU$, $MV$, $QW$, $FX$, $OY$, and
$SI$ be drawn from points $K$, $E$, $G$, $M$, $Q$, $F$, $O$, and
$S$ (respectively) perpendicular to the reference plane ({\rm  i.e.}, the plane of the bases $AB$ and $CD$), and let them have met
the plane at points $N$, $T$, $U$, $V$, $W$, $X$, $Y$, and $I$ (respectively). And let $NT$, $NU$, $UV$, $TV$, $WX$, $WY$, $YI$, and
$IX$ be joined. So solid $KV$ is equal to solid $QI$. For they are on
the equal bases $KM$ and $QS$, and (have) the same height, and the (straight-lines) standing up in them are at right-angles to their bases (see first part of proposition).  But, solid $KV$ is equal to
solid $AE$, and $QI$ to $CF$.
For they are on the same base, and (have)
the same height, and the (straight-lines) standing up in them are not on
the same straight-lines [Prop.~11.30]. Thus,
solid $AE$ is also equal to solid $CF$.

Thus, parallelepiped solids which are on equal bases,
and (have) the same height, are equal to one another. (Which is) the very thing
it was required to show.}
\end{Parallel}

%%%%
%11.32
%%%%
\pdfbookmark[1]{Proposition 11.32}{pdf11.32}
\begin{Parallel}{}{}
\ParallelLText{
\begin{center}
{\large \ggn{32}.}
\end{center}\vspace*{-7pt}

\gr{T`a <up`o t`o a\kern -.7pt >ut`o <'uyoc >'onta stere`a parallhlep'ipeda pr`oc >'allhl'a
>estin <wc a<i b'aseic.}

\epsfysize=1.in
\centerline{\epsffile{Book11/fig32g.eps}}

\gr{>'Estw <up`o t`o a\kern -.7pt >ut`o <'uyoc stere`a parallhlep'ipeda t`a AB, GD;
l'egw, <'oti t`a AB, GD stere`a parallhlep'ipeda pr`oc >'allhl'a >estin
<wc a<i b'aseic, tout'estin <'oti >est`in <wc <h AE b'asic pr`oc t`hn
GZ b'asin, o<'utwc t`o AB stere`on pr`oc t`o GD stere'on.}

\gr{Parabebl'hsjw g`ar par`a t`hn ZH t~w| AE >'ison t`o ZJ, ka`i >ap`o b'asewc m`en t~hc ZJ, <'uyouc d`e to~u a\kern -.7pt >uto~u t~w| GD stere`on parallhlep'ipedon
sumpeplhr'wsjw t`o HK. >'ison d'h >esti t`o AB stere`on t~w| HK
stere~w|; >ep'i te g`ar >'iswn b'ase'wn e>isi t~wn AE, ZJ ka`i <up`o t`o
a\kern -.7pt >uto <'uyoc. ka`i >epe`i stere`on parallhlep'ipedon t`o GK >epip'edw|
t~w| DH t'etm\kern -.7pt htai parall'hlw| >'onti to~ic >apenant'ion >epip'edoic,
>'estin >'ara <wc <h GZ b'asic pr`oc t`hn ZJ b'asin, o<'utwc t`o GD
stere`on pr`oc t`o DJ stere'on. >'ish d`e <h m`en ZJ b'asic t~h| AE
b'asei, t`o d`e HK stere`on t~w| AB stere~w|; >'estin >'ara ka`i <wc <h
AE b'asic pr`oc t`hn GZ b'asin, o<'utwc t`o AB stere`on pr`oc t`o GD stere'on.}

\gr{T`a >'ara <up`o t`o a\kern -.7pt >ut`o <'uyoc >'onta stere`a parallhlep'ipeda pr`oc
>'allhl'a >estin <wc a<i b'aseic; <'oper >'edei de~ixai.}}

\ParallelRText{
\begin{center}
{\large Proposition 32}
\end{center}

Parallelepiped solids which (have) the same height
are to one another as their bases.

\epsfysize=1.in
\centerline{\epsffile{Book11/fig32e.eps}}

Let $AB$ and $CD$ be parallelepiped solids (having) the same height.
I say that the parallelepiped solids $AB$ and $CD$
are to one another as their bases. That is to say, as base $AE$ is to base
$CF$, so solid $AB$ (is) to solid $CD$.

For let $FH$, equal to $AE$, be applied to $FG$ (in the angle
$FGH$ equal to angle $LCG$) [Prop.~1.45]. And let the parallelepiped solid $GK$, (having) the same height as $CD$, 
be completed on the base $FH$. So solid $AB$ is equal to solid
$GK$. For they are on the equal bases $AE$ and $FH$, and
(have) the same height [Prop.~11.31]. 
And since the parallelepiped solid $CK$ has been cut by the plane
$DG$, which is parallel to the opposite planes (of $CK$), thus as
the base $CF$ is to the base $FH$, so the solid $CD$ (is) to the
solid $DH$ [Prop.~11.25]. And  base
$FH$ (is) equal to base $AE$, and solid $GK$ to solid $AB$. And thus
as base $AE$ is to base $CF$, so solid $AB$ (is) to solid $CD$.

Thus, parallelepiped solids which (have) the same height
are to one another as their bases. (Which is) the very thing it was required to
show.}
\end{Parallel}

%%%%
%11.33
%%%%
\pdfbookmark[1]{Proposition 11.33}{pdf11.33}
\begin{Parallel}{}{}
\ParallelLText{
\begin{center}
{\large\ggn{33}.}
\end{center}\vspace*{-7pt}

\gr{T`a <'omoia stere`a parallhlep'ipeda pr`oc >'allhla >en triplas'ioni
l'ogw| e>is`i t~wn <omol'ogwn pleur~wn.}

\gr{>'Estw <'omoia stere`a parallhlep'ipeda t`a AB, GD, <om'ologoc d`e
>'estw <h AE t~h| GZ; l'egw, <'oti t`o AB stere`on pr`oc t`o GD stere`on
triplas'iona l'ogon >'eqei, >'hper <h AE pr`oc t`hn GZ.}

\gr{>Ekbebl'hsjwsan g`ar >ep> e>uje'iac ta~ic AE, HE, JE a<i EK, EL, EM,
ka`i ke'isjw t~h| m`en GZ >'ish <h EK, t~h| d`e ZN >'ish <h EL, ka`i
>'eti t~h| ZR >'ish <h EM, ka`i sumpeplhr'wsjw t`o KL parallhl'ogrammon
ka`i t`o KO stere'on.}\\

\epsfysize=1.9in
\centerline{\epsffile{Book11/fig33g.eps}}

\gr{Ka`i >epe`i d'uo a<i KE, EL dus`i ta~ic GZ, ZN >'isai e>is'in, >all`a
ka`i gwn'ia <h <up`o KEL gwn'ia| t~h| <up`o GZN >estin >'ish, >epeid'hper
ka`i <h <up`o AEH t~h| <up`o GZN >estin >'ish di`a t`hn <omoi'othta
t~wn AB, GD stere~wn, >'ison >'ara >est`i [ka`i <'omoion] t`o KL
parallhl'ogrammon t~w| GN parallhlogr'ammw|. di`a t`a a\kern -.7pt >ut`a d`h
ka`i t`o m`en KM parallhl'ogrammon >'ison >est`i ka`i <'omoion
t~w| GR [parallhlogr'ammw|] ka`i >'eti t`o EO t~w| DZ; tr'ia >'ara parallhl'ogramma to~u KO stereo~u tris`i parallhlogr'ammoic to~u
GD stereo~u >'isa >est`i ka`i <'omoia. >all`a  t`a m`en tr'ia tris`i to~ic
>apenant'ion >'isa >est`i ka`i <'omoia, t`a d`e tr'ia tris`i to~ic
>apenant'ion >'isa >est`i ka`i <'omoia; <'olon >'ara t`o KO stere`on
<'olw| t~w| GD stere~w| >'ison >est`i ka`i <'omoion. sumpeplhr'wsjw
t`o HK parallhl'ogrammon, ka`i >ap`o b'asewn m`en t~wn HK, KL
parallhl'ogrammwn, <'uyouc d`e to~u a\kern -.7pt >uto~u t~w| AB stere`a sumpeplhr'wsjw t`a EX, LP.  ka`i >epe`i di`a t`hn <omoi'othta t~wn
AB, GD stere~wn >estin <wc <h AE pr`oc t`hn GZ, o<'utwc <h EH
pr`oc t`hn ZN, ka`i  <h EJ pr`oc t`hn ZR, >'ish d`e <h m`en GZ t~h|
EK, <h d`e ZN t~h| EL, <h d`e ZR t~h| EM, >'estin >'ara <wc <h
AE pr`oc t`hn EK, o<'utwc <h HE pr`oc t`hn EL ka`i <h JE pr`oc
t`hn EM. >all> <wc m`en <h AE pr`oc t`hn EK, o<'utwc t`o AH
[parallhl'ogrammon] pr`oc t`o HK parallhl'ogrammon, <wc d`e <h HE
pr`oc t`hn EL, o<'utwc t`o HK pr`oc t`o KL,
<wc d`e <h JE pr`oc EM, o<'utwc t`o PE pr`oc t`o KM; ka`i <wc
>'ara t`o AH parallhl'ogrammon pr`oc t`o HK, o<'utwc t`o HK pr`oc t`o
KL ka`i t`o PE pr`oc t`o KM. >all> <wc m`en t`o AH pr`oc t`o HK,
o<'utwc t`o AB stere`on pr`oc t`o EX stere'on, <wc d`e t`o HK
pr`oc t`o KL, o<'utwc t`o XE stere`on pr`oc t`o PL stere'on, <wc d`e
t`o PE pr`oc t`o KM, o<'utwc t`o PL stere`on pr`oc t`o KO stere'on;
ka`i <wc >'ara t`o AB stere`on pr`oc t`o EX, o<'utwc t`o EX pr`oc t`o 
PL ka`i t`o PL pr`oc t`o KO. >e`an d`e t'essara meg'ejh kat`a t`o
suneq`ec >an'alogon >~h|, t`o pr~wton pr`oc t`o t'etarton triplas'iona l'ogon
>'eqei >'hper pr`oc t`o de'uteron; t`o AB >'ara stere`on pr`oc t`o KO
triplas'iona l'ogon >'eqei >'hper t`o AB pr`oc t`o EX. >all> <wc t`o
AB pr`oc t`o EX, o<'utwc t`o AH parallhl'ogrammon pr`oc t`o HK
ka`i <h AE e>uje~ia pr`oc t`hn EK; <'wste ka`i t`o AB stere`on pr`oc
t`o KO  triplas'iona l'ogon >'eqei >'hper <h AE pr`oc t`hn EK. >'ison
d`e t`o [m`en] KO stere`on t~w| GD stere~w|, <h d`e EK e>uje~ia
t~h| GZ; ka`i t`o AB >'ara stere`on pr`oc t`o GD stere`on triplas'iona
l'ogon >'eqei >'hper <h <om'ologoc a\kern -.7pt >uto~u pleur`a <h AE pr`oc
t`hn <om'ologon pleur`an t`hn GZ.}

\epsfysize=2.5in
\centerline{\epsffile{Book11/fig33ag.eps}}

\gr{T`a >'ara <'omoia stere`a parallhlep'ipeda >en triplas'ioni
l'ogw| >est`i t~wn <omol'ogwn pleur~wn; <'oper >'edei
de~ixai.}}

\ParallelRText{

\begin{center}
{\large Proposition 33}
\end{center}

Similar parallelepiped solids are to one another
as the cubed ratio of their corresponding sides.

Let $AB$ and $CD$ be similar parallelepiped solids, and let $AE$
correspond to $CF$. I say that solid $AB$ has to solid $CD$ the
cubed ratio that $AE$ (has) to $CF$.

For let $EK$, $EL$, and $EM$ be produced in a straight-line
with $AE$, $GE$, and $HE$ (respectively). And let
$EK$ be made equal to $CF$, and $EL$ equal to $FN$, and, further,
$EM$ equal to $FR$. And let the parallelogram $KL$ be
completed, and the solid $KP$.\\

\epsfysize=1.9in
\centerline{\epsffile{Book11/fig33e.eps}}

And since the two (straight-lines) $KE$ and $EL$ are equal to the
two (straight-lines) $CF$ and $FN$, but angle $KEL$ is also
equal to angle $CFN$, inasmuch as $AEG$ is also equal to $CFN$,
on account of the similarity of the solids $AB$ and $CD$, parallelogram
$KL$ is thus equal [and similar] to parallelogram $CN$. So,
for the same (reasons), parallelogram $KM$ is also  equal and similar to
[parallelogram] $CR$, and, further, $EP$ to $DF$. Thus,
three parallelograms of solid $KP$ are equal and similar to three
parallelograms of solid $CD$. But the three (former parallelograms) are equal and
similar to the three opposite (parallelograms), and the three (latter parallelograms)
are equal and similar to the three opposite (parallelograms) 
[Prop.~11.24]. Thus, the whole of solid $KP$
is equal and similar to the whole of solid $CD$ [Def.~11.10]. Let parallelogram $GK$ be completed. And
let the the solids $EO$ and $LQ$, with bases the parallelograms
$GK$ and $KL$ (respectively), and with the same height as $AB$, be completed. And since, on account of the similarity of solids $AB$ and $CD$, 
as $AE$ is to $CF$, so $EG$ (is) to $FN$,  and $EH$  to $FR$ [Defs.~6.1, 11.9],
and $CF$ (is) equal to $EK$, and $FN$ to $EL$, and $FR$ to $EM$, 
thus as $AE$ is to $EK$, so $GE$ (is) to $EL$, and $HE$ to $EM$. But,
as $AE$ (is) to $EK$, so [parallelogram] $AG$ (is) to parallelogram
$GK$, and as $GE$ (is) to $EL$, so $GK$ (is) to $KL$,
and as $HE$ (is) to $EM$, so $QE$ (is) to $KM$ [Prop.~6.1]. And thus as parallelogram $AG$ (is) to $GK$, so $GK$
(is) to $KL$, and $QE$ (is) to $KM$. But, as $AG$
(is) to $GK$, so solid $AB$ (is) to solid $EO$, and as $GK$ (is) to $KL$,
so solid $OE$ (is) to  solid $QL$, and as $QE$ (is) to $KM$, so solid
$QL$ (is) to solid $KP$ [Prop.~11.32].
And, thus, as solid $AB$ is to $EO$, so $EO$ (is) to $QL$, and
$QL$ to $KP$.
And if four magnitudes are continuously proportional then the first has to
the fourth the cubed ratio that (it has) to the second [Def.~5.10].  Thus, solid $AB$ has to $KP$ the 
cubed ratio
which $AB$ (has) to $EO$. 
But, as $AB$ (is) to $EO$, so parallelogram $AG$
(is) to $GK$, and the straight-line $AE$ to $EK$ [Prop.~6.1]. Hence, 
solid $AB$ also has to $KP$ the cubed ratio  that $AE$ (has) to
$EK$. And solid $KP$ (is) equal to solid $CD$, and straight-line $EK$ to
$CF$.  Thus, solid $AB$ also has to solid $CD$ the cubed ratio
which its corresponding side $AE$ (has) to the corresponding side
$CF$.\\~\\

\epsfysize=2.5in
\centerline{\epsffile{Book11/fig33ae.eps}}

Thus, similar parallelepiped solids are to one another
as the cubed ratio of  their corresponding sides. (Which is) the very thing it
was required to show.}
\end{Parallel}

\begin{Parallel}{}{}
\ParallelLText{
\begin{center}
{\large \gr{P'orisma}.}
\end{center}\vspace*{-7pt}

\gr{>Ek d`h to'utou faner'on, <'oti >e`an t'essarec e>uje~iai
>an'alogon >~wsin, >'estai <wc <h pr'wth pr`oc t`hn tet'arthn, o<'utw
t`o >ap`o t~hc pr'wthc stere`on parallhlep'ipedon pr`oc t`o >ap`o
t~hc deut'erac t`o <'omoion ka`i <omo'iwc >anagraf'omenon, >epe'iper
ka`i <h pr'wth pr`oc t`hn tet'arthn triplas'iona  l'ogon >'eqei
>'hper pr`oc t`hn deut'eran.}}

\ParallelRText{
\begin{center}
{\large Corollary}
\end{center}
So, (it is) clear, from this,  that if four straight-lines are (continuously)
proportional then as the first is to the fourth, so the parallelepiped
solid on the first will be to the similar, and similarly described, parallelepiped solid on the second, since the first also has to the fourth the
cubed ratio  that (it has) to the second.}
\end{Parallel}

%%%%
%11.34
%%%%
\pdfbookmark[1]{Proposition 11.34}{pdf11.34}
\begin{Parallel}{}{}
\ParallelLText{
\begin{center}
{\large \ggn{34}.}
\end{center}\vspace*{-7pt}

\gr{T~wn >'iswn stere~wn parallhlepip'edwn >antipep'onjasin a<i b'aseic
to~ic <'uyesin; ka`i <~wn stere~wn parallhlepip'edwn >antipep'onjasin
a<i b'aseic to~ic <'uyesin, <'isa >est`in >eke~ina.}

\gr{>'Estw >'isa stere`a parallhlep'ipeda t`a AB, GD; l'egw, <'oti
t~wn AB, GD stere~wn parallhlepip'edwn >antipep'onjasin a<i b'aseic
to~ic <'uyesin, ka'i >estin <wc <h EJ b'asic pr`oc t`hn NP b'asin, o<'utwc
t`o to~u GD stereo~u <'uyoc pr`oc t`o to~u AB stereo~u <'uyoc.}

\gr{>'Estwsan g`ar pr'oteron a<i >efesthku~iai a<i AH, EZ, LB, JK,
GM, NX, OD, PR pr`oc >orj`ac ta~ic b'asesin a\kern -.7pt >ut~wn; l'egw,
<'oti >est`in <wc <h EJ b'asic pr`oc t`hn NP b'asin, o<'utwc <h
GM pr`oc t`hn AH.}

\gr{E>i m`en o>~un >'ish >est`in <h EJ b'asin t~h| NP b'asei,
>'esti d`e ka`i t`o AB stere`on t~w| GD stere~w| >'ison, >'estai
ka`i <h GM t~h| AH >'ish. t`a g`ar <up`o t`o a\kern -.7pt >ut`o <'uyoc stere`a
parallhlep'ipeda pr`oc >'allhl'a >estin <wc a<i b'aseic. ka`i >'estai <wc <h EJ b'asic pr`oc t`hn NP, o<'utwc <h GM
pr`oc t`hn AH, ka`i faner'on, <'oti t~wn AB, GD stere~wn parallhlepip'edwn
>antipep'onjasin a<i b'aseic to~ic <'uyesin.}

\epsfysize=1.9in
\centerline{\epsffile{Book11/fig34g.eps}}

\gr{m\kern -.7pt `h >'estw d`h >'ish <h EJ b'asic t~h| NP b'asei, >all> >'estw me'izwn
<h EJ. >'esti d`e ka`i t`o AB stere`on t~w| GD stere~w| >'ison; me'izwn >'ara
>est`i ka`i <h GM t~hc AH. ke'isjw o>~un t~h| AH >'ish
<h GT, ka`i sumpeplhr'wsjw >ap`o b'asewc m`en t~hc NP, <'uyouc
d`e to~u GT, stere`on parallhlep'ipedon t`o FG. ka`i >epe`i >'ison
>est`i t`o AB stere`on t~w| GD stere~w|, >'exwjen d`e t`o GF, t`a d`e >'isa
pr`oc t`o a\kern -.7pt >ut`o t`on a\kern -.7pt >ut`on >'eqei l'ogon, >'estin >'ara <wc t`o 
AB stere`on pr`oc t`o
GF stere'on, o<'utwc t`o GD stere`on pr`oc t`o GF stere'on.
>all> <wc m`en t`o AB stere`on pr`oc
 t`o GF stere'on, o<'utwc
<h EJ b'asic pr`oc t`hn NP b'asin; >iso"uy~h g`ar t`a AB, GF stere'a;
<wc d`e t`o GD stere`on pr`oc t`o GF stere'on, o<'utwc <h MP b'asic
pr`oc t`hn TP b'asin ka`i <h GM pr`oc t`hn GT; ka`i <wc >'ara
<h EJ b'asic pr`oc t`hn NP b'asin, o<'utwc <h MG pr`oc t`hn GT.
>'ish d`e <h GT t~h| AH; ka`i <wc >'ara <h EJ b'asic pr`oc t`hn NP
b'asin, o<'utwc <h MG pr`oc t`hn AH. t~wn AB, GD >'ara stere~wn
parallhlepip'edwn >antipep'onjasin a<i b'aseic to~ic
<'uyesin.}

\gr{P'alin d`h t~wn AB, GD stere~wn parallhlepip'edwn >antipeponj'etwsan
a<i b'aseic to~ic <'uyesin, ka`i >'estw <wc <h EJ b'asic pr`oc t`hn NP
b'asin, o<'utwc t`o to~u GD stereo~u <'uyoc pr`oc t`o to~u AB stereo~u
<'uyoc; l'egw, <'oti >'ison >est`i t`o AB stere`on t~w| GD stere~w|.}

\gr{>'Estwsan [g`ar] p'alin a<i >efesthku~iai pr`oc >orj`ac ta~ic b'asesin.
ka`i e>i m`en >'ish >est`in <h EJ b'asic t~h| NP b'asei, ka'i >estin <wc <h EJ
b'asic pr`oc t`hn NP b'asin, o<'utwc t`o to~u GD stereo~u <'uyoc pr`oc
t`o to~u AB stereo~u <'uyoc, >'ison >'ara >est`i ka`i t`o to~u GD stereo~u
<'uyoc t~w| to~u AB stereo~u <'uyei. t`a d`e >ep`i >'iswn b'asewn stere'a
parallhlep'ipeda ka`i <up`o t`o a\kern -.7pt >ut`o <'uyoc >'isa >all'hloic >est'in;
>'ison >'ara >est`i t`o AB stere`on t~w|
GD stere~w|.}

\gr{m\kern -.7pt `h >'estw d`h <h EJ b'asic t~h| NP [b'asei] >'ish, >all> >'estw me'izwn
<h EJ; me~izon >'ara >est`i ka`i t`o to~u GD stereo~u <'uyoc to~u
to~u AB stereo~u <'uyouc, tout'estin <h GM t~hc AH. ke'isjw t~h|
AH >'ish p'alin <h GT, ka`i sumpeplhr'wsjw <omo'iwc t`o GF stere'on.
>epe'i >estin <wc <h EJ b'asic pr`oc t`hn NP b'asin, o<'utwc
<h MG pr`oc t`hn AH, >'ish d`e <h AH t~h| GT, >'estin >'ara <wc <h EJ
b'asic pr`oc t`hn NP b'asin, o<'utwc <h GM pr`oc t`hn GT. >all>
<wc m`en <h EJ [b'asic] pr`oc t`hn NP b'asin, o<'utwc t`o AB
stere`on pr`oc t`o GF stere'on; >iso"uy~h g'ar >esti t`a AB, GF stere'a;
<wc d`e <h GM pr`oc t`hn GT, o<'utwc <'h te MP b'asic pr`oc t`hn PT
b'asin ka`i t`o GD stere`on pr`oc t`o GF stere'on. ka`i <wc >'ara t`o AB
stere`on pr`oc t`o GF stere'on, o<'utwc t`o GD stere`on pr`oc t`o GF
stere'on; <ek'ateron >'ara t~wn AB, GD pr`oc t`o GF t`on a\kern -.7pt >ut`on
>'eqei l'ogon. >'ison >'ara >est`i t`o AB stere`on t~w| GD stere~w|.}

\epsfysize=3in
\centerline{\epsffile{Book11/fig34ag.eps}}

\gr{m\kern -.7pt `h >'estwsan d`h a<i >efesthku~iai a<i ZE, BL, HA, KJ, XN, DO, MG, RP
pr`oc >orj`ac ta~ic b'asesin a\kern -.7pt >ut~wn, ka`i >'hqjwsan >ap`o t~wn Z,
H, B, K, X, M, R, D shme'iwn >ep`i t`a di`a t~wn EJ, NP >ep'ipeda
k'ajetoi ka`i sumball'etwsan to~ic >epip'edoic kat`a t`a S, T, U, F,
Q, Y, W, c, ka`i sumpeplhr'wsjw t`a ZF, XW stere'a; l'egw, <'oti ka`i
o<'utwc >'iswn >'ontwn t~wn AB, GD stere~wn >antipep'onjasin
a<i b'aseic to~ic <'uyesin, ka'i >estin <wc <h EJ b<asin pr`oc t`hn NP
b'asin, o<'utwc t`o to~u GD stereo~u <'uyoc pr`oc t`o to~u AB stereo~u
<'uyoc.}

\gr{>Epe`i >'ison >est`i t`o AB stere`on t~w| GD stere~w|, >all`a t`o m`en AB t~w| BT >estin >'ison; >ep'i te g`ar t~hc a\kern -.7pt >ut~hc b'ase'wc e>isi t~hc
ZK ka`i <up`o t`o a\kern -.7pt >ut`o <'uyoc;
t`o d`e GD stere`on t~w| DY >estin >'ison; >ep'i te g`ar p'alin t~hc
a\kern -.7pt >ut~hc b'ase'wc e>isi t~hc RX ka`i <up`o t`o a\kern -.7pt >ut`o
<'uyoc;
ka`i t`o BT >'ara stere`on t~w| DY stere~w| >'ison >est'in.
>'estin >'ara <wc <h ZK b'asic pr`oc t`hn XR b'asin, o<'utwc t`o to~u
DY stereo~u <'uyoc pr`oc t`o to~u BT stereo~u <'uyoc. >'ish
d`e <h m`en ZK b'asic t~h| EJ b'asei, <h d`e XR b'asic t~h| NP b'asei; >'estin
>'ara <wc <h EJ b'asic pr`oc t`hn NP b'asin, o<'utwc t`o to~u DY stereo~u
<'uyoc pr`oc t`o to~u BT stereo~u <'uyoc. t`a d> a\kern -.7pt >ut`a <'uyh >est`i t~wn
DY, BT stere~wn ka`i t~wn DG, BA; >'estin >'ara <wc <h EJ b'asic pr`oc
t`hn NP b'asin, o<'utwc t`o to~u DG stereo~u <'uyoc pr`oc t`o to~u
AB stereo~u <'uyoc. t~wn AB, GD >'ara stere~wn parallhlepip'edwn
>antipep'onjasin a<i b'aseic to~ic <'uyesin.}

\gr{P'alin d`h t~wn AB, GD stere~wn parallhlepip'edwn >antipeponj'etwsan
a<i b'aseic to~ic <'uyesin, ka`i >'estw <wc <h EJ b'asic pr`oc t`hn
NP b'asin, o<'utwc t`o to~u GD stereo~u <'uyoc pr`oc t`o to~u AB
stereo~u <'uyoc; l'egw, <'oti >'ison >est`i t`o AB stere`on t~w| GD
stere~w|.}

\gr{T~wn g`ar a\kern -.7pt >ut~wn kataskeuasj'entwn, >epe'i >estin <wc <h EJ b'asic
pr`oc t`hn NP b'asin, o<'utwc t`o to~u GD stereo~u <'uyoc pr`oc
t`o to~u AB stereo~u <'uyoc, >'ish d`e <h m`en EJ b'asic t~h| ZK b'asei,
<h d`e NP t~h| XR, >'estin >'ara <wc <h ZK b'asic pr`oc t`hn XR b'asin,
o<'utwc t`o to~u GD stereo~u <'uyoc pr`oc t`o to~u AB stereo~u
<'uyoc. t`a d> a\kern -.7pt >ut`a <'uyh >est`i t~wn AB, GD stere~wn ka`i t~wn
BT, DY; >'estin >'ara <wc <h ZK b'asic pr`oc t`hn XR b'asin, o<'utwc
t`o to~u DY stereo~u <'uyoc pr`oc t`o to~u BT stereo~u <'uyoc. t~wn
BT, DY >'ara stere~wn parallhlepip'edwn >antipep'onjasin a<i b'aseic
to~ic <'uyesin; >'ison
>'ara >est`i t`o BT stere`on t~w| DY stere~w|. >all`a t`o m`en
BT t~w| BA >'ison >est'in; >ep'i te g`ar t~hc a\kern -.7pt >ut~hc b'asewc [e>isi]
t~hc ZK ka`i <up`o t`o a\kern -.7pt >ut`o <'uyoc. t`o d`e DY stere`on t~w|
DG stere~w| >'ison >est'in. ka`i t`o AB >'ara stere`on t~w| GD stere~w|
>estin >'ison; <'oper >'edei de~ixai.}}

\ParallelRText{
\begin{center}
{\large Proposition 34}$^\dag$
\end{center}

The  bases of equal parallelepiped solids are reciprocally proportional to their heights. And those parallelepiped solids
whose bases are reciprocally proportional to their heights are equal.

Let $AB$ and $CD$ be equal parallelepiped solids. I say that the
bases of the parallelepiped solids $AB$ and $CD$ are reciprocally
proportional to their heights, and (so) as base $EH$ is to base $NQ$, so the
height of solid $CD$ (is) to the height of solid $AB$.

For, first of all, let the (straight-lines) standing up, $AG$, $EF$, $LB$, $HK$, $CM$,
$NO$, $PD$, and $QR$,  be at right-angles to their bases. I say that
as base $EH$ is to base $NQ$, so $CM$ (is) to $AG$.

Therefore, if base $EH$ is equal to base $NQ$, and solid $AB$ is also
equal to solid $CD$, $CM$ will also be equal to $AG$. For
parallelepiped solids of the same height are to one another as their bases [Prop.~11.32]. And as base $EH$ (is) to $NQ$,
so $CM$ will be to $AG$. And (so it is) clear that the bases of the parallelepiped
solids $AB$ and $CD$ are reciprocally proportional to their heights.

\epsfysize=1.9in
\centerline{\epsffile{Book11/fig34e.eps}}

So let base $EH$ not be equal to base $NQ$, but let $EH$ be greater. And
solid $AB$ is also equal to solid $CD$. Thus, $CM$ is also greater than $AG$.  Therefore, let $CT$ be made equal to $AG$. And let the parallelepiped
solid $VC$ be completed on the base $NQ$, with height $CT$.
And since solid $AB$ is equal to solid $CD$, and $CV$ (is)
extrinsic (to them), and equal (magnitudes) have the same ratio to the
same (magnitude) [Prop.~5.7], thus as solid
$AB$ is to solid $CV$, so solid $CD$ (is) to solid $CV$. But, as
solid $AB$ (is) to solid $CV$, so base $EH$ (is) to base $NQ$. For the solids $AB$ and $CV$ (are) of equal
height [Prop.~11.32]. And as solid $CD$ (is) to solid $CV$, so base $MQ$ (is) to 
base $TQ$ [Prop.~11.25], and $CM$ to $CT$
[Prop.~6.1].
And, thus, as base $EH$ is to base $NQ$, so $MC$ (is) to $AG$.
And $CT$ (is) equal to $AG$. And
thus as base $EH$ (is) to base $NQ$, so $MC$ (is) to $AG$. Thus, the
bases of the parallelepiped solids $AB$ and $CD$ are reciprocally proportional to their
heights.

So, again, let the bases of the parallelepipid solids $AB$ and $CD$
be reciprocally proportional to their heights, and let base $EH$
be to base $NQ$, as the height of solid $CD$ (is) to the
height of solid $AB$. I say that solid $AB$ is equal to solid $CD$.
\mbox{[}For] let the (straight-lines) standing up again be at right-angles to the
bases. And if base $EH$ is equal to base $NQ$, and as base $EH$
is to base $NQ$, so the height of solid $CD$ (is) to the
height of solid $AB$, the height of solid $CD$ is thus also
equal to the height of solid $AB$. And parallelepiped solids on equal
bases, and also with the same height,  are equal to one another [Prop.~11.31]. Thus, solid $AB$ is equal to solid $CD$.

So, let base $EH$ not be equal to [base] $NQ$, but let $EH$ be greater.
Thus, the height of solid $CD$ is also greater than the height of
solid $AB$, that is to say $CM$ (greater) than $AG$. Let $CT$ again
be made equal to $AG$, and let the solid $CV$ be similarly
completed.  Since as base $EH$ is to base $NQ$, so $MC$ (is) to
$AG$, and $AG$ (is) equal to $CT$, thus as base $EH$ (is) to
base $NQ$, so $CM$ (is) to $CT$.  But, as [base] $EH$ (is) to base
$NQ$, so solid $AB$ (is) to solid $CV$. For solids $AB$ and $CV$ are of equal heights [Prop.~11.32].
And as $CM$ (is) to $CT$, so (is) base $MQ$ to base $QT$ [Prop.~6.1], and solid $CD$ to solid $CV$ [Prop.~11.25]. And thus as solid $AB$ (is) to solid
$CV$, so solid $CD$ (is) to solid $CV$. Thus, $AB$ and $CD$ each
have the same ratio to $CV$. Thus, solid $AB$ is equal to solid
$CD$ [Prop.~5.9].\\~\\~\\~\\~\\

\epsfysize=3in
\centerline{\epsffile{Book11/fig34ae.eps}}

So, let the (straight-lines) standing up, $FE$, $BL$, $GA$, $KH$, $ON$,
$DP$, $MC$, and $RQ$, not be at right-angles to their bases.
And let perpendiculars be drawn to the planes through $EH$ and
$NQ$ from points $F$, $G$, $B$, $K$, $O$, $M$, $R$, and $D$, 
and let them have joined the planes at (points) $S$, $T$, $U$, $V$, $W$,
$X$, $Y$, and $a$ (respectively). And let the solids $FV$ and $OY$
be completed. In this case,  also, I say that the solids $AB$ and
$CD$ being equal, their bases are reciprocally proportional to their
heights, and (so) as base $EH$ is to base $NQ$, so the height of
solid $CD$ (is) to the height of solid $AB$.

Since solid $AB$ is equal to solid $CD$, but $AB$ is equal
to $BT$. For they are on the same base $FK$, and (have) the
same height [Props.~11.29, 11.30]. And solid $CD$ is equal is equal to $DX$. For, again,  they are on the same base $RO$, and (have) the same height [Props.~11.29, 11.30]. Solid
$BT$ is thus also equal to solid $DX$. Thus, as base $FK$ (is) to base $OR$,
so the height of solid $DX$ (is) to the height of solid $BT$ (see first part of 
proposition). And base
$FK$ (is) equal to base $EH$, and base $OR$ to $NQ$. Thus, as base
$EH$ is to base $NQ$, so the height of solid $DX$ (is) to the height
of solid $BT$. And solids $DX$, $BT$ are the same
height as (solids) $DC$,  $BA$ (respectively). Thus, as base $EH$
is to base $NQ$, so the height of solid $DC$ (is) to the height of solid
$AB$.  Thus, the bases of the parallelepiped solids $AB$ and $CD$
are reciprocally proportional to their heights.

So, again, let the bases of the parallelepiped solids $AB$ and $CD$ be
reciprocally proportional to their heights, and (so) let base $EH$ be to base $NQ$,
as the height of solid $CD$ (is) to the height of solid $AB$. I say
that solid $AB$ is equal to solid $CD$.

For, with the same construction (as before), since as base $EH$
is to base $NQ$, so the height of solid $CD$ (is) to the height
of solid $AB$, and base $EH$ (is) equal to base $FK$, and
$NQ$ to $OR$, thus as base $FK$ is to base $OR$, so the height
of solid $CD$ (is) to the height of solid $AB$. And solids $AB$, $CD$ are the same height as (solids) $BT$,  $DX$ (respectively). 
Thus, as base $FK$ is to base $OR$, so the height of solid $DX$ (is)
to the height of solid $BT$. Thus, the bases of the parallelepiped solids $BT$ and $DX$
are reciprocally proportional to their heights. Thus, solid $BT$ is equal to
solid $DX$ (see first part of  proposition). But, $BT$ is equal to $BA$. For [they are] on the same
base $FK$, and (have) the same height [Props.~11.29, 11.30]. And solid $DX$ is equal
to solid $DC$ [Props.~11.29, 11.30]. Thus, solid $AB$ is also equal to solid $CD$. (Which is) the very thing it was required to show.}
\end{Parallel}
{\footnotesize\noindent$^\dag$ This proposition assumes that (a) if
two parallelepipeds are equal, and have equal bases, then their heights are equal,
and (b) if the bases of two equal parallelepipeds are unequal, then that solid which
has the lesser base has the greater height.}

%%%%
%11.35
%%%%
\pdfbookmark[1]{Proposition 11.35}{pdf11.35}
\begin{Parallel}{}{}
\ParallelLText{
\begin{center}
{\large \ggn{35}.}
\end{center}\vspace*{-7pt}

\gr{>E`an >~wsi d'uo gwn'iai >ep'ipedoi >'isai, >ep`i d`e t~wn
koruf~wn a\kern -.7pt >ut~wn met'ewroi e>uje~iai >epistaj~wsin
>'isac gwn'iac peri'eqousai met`a t~wn >ex >arq~hc
e>ujei~wn <ekat'eran <ekat'era|, >ep`i d`e t~wn mete'wrwn
lhfj~h| tuq'onta shme~ia, ka`i >ap> a\kern -.7pt >ut~wn >ep`i t`a >ep'ipeda,
>en o<~ic e>isin a<i >ex >arq~hc gwn'iai, k'ajetoi >aqj~wsin,
>ap`o d`e t~wn genom'enwn shme'iwn >en to~ic >epip'edoic
>ep`i t`ac >ex >arq~hc gwn'iac >epizeuqj~wsin e>uje~iai,
>'isac gwn'iac peri'exousi met`a t~wn mete'wrwn.}\\

\epsfysize=2.4in
\centerline{\epsffile{Book11/fig35g.eps}}

\gr{>'Estwsan  d'uo gwn'iai e>uj'ugrammoi >'isai a<i <up`o BAG,
EDZ, >ap`o d`e t~wn A, D shme'iwn met'ewroi e>uje~iai >efest'at\-wsan
a<i AH, DM >'isac gwn'iac peri'eqousin met`a t~wn >ex >arq~hc e>ujei~wn
<ekat'eran <ekat'era|, t`hn m`en <up`o MDE t~h| <up`o HAB, t`hn d`e
<up`o MDZ t~h| <up`o HAG, ka`i e>il'hfjw >ep`i t~wn AH, DM tuq'onta
shme~ia t`a H, M, ka`i >'hqjwsan >ap`o t~wn H, M shme'iwn
>ep`i t`a di`a t~wn BAG, EDZ >ep'ipeda k'ajetoi a<i HL, MN, ka`i
sumball'etwsan to~ic >epip'edoic kat`a t`a L, N,
ka`i >epeze'uqjwsan a<i LA, ND; l'egw, <'oti >'ish >est`in <h <up`o
HAL gwn'ia t~h| <up`o MDN gwn'ia|.}

\gr{Ke'isjw t~h| DM >'ish <h AJ, ka`i >'hqjw di`a to~u J shme'iou t~h|
HL par'allhloc <h JK. <h d`e HL k'ajet'oc >estin >ep`i t`o di`a
t~wn BAG >ep'ipedon; ka`i <h JK >'ara k'ajet'oc >estin >ep`i
t`o di`a t~wn BAG >ep'ipedon. >'hqjwsan >ap`o t~wn K, N
shme'iwn >ep`i t`ac AG, DZ, AB, DE e>uje'iac k'ajetoi a<i KG, NZ,
KB, NE, ka`i >epeze'uqjwsan a<i JG, GB, MZ, ZE. >epe`i t`o >ap`o
t~hc JA >'ison >est`i to~ic >ap`o t~wn JK, KA, t~w| d`e >ap`o
t~hc KA >'isa >est`i t`a >ap`o t~wn KG, GA, ka`i t`o >ap`o
t~hc JA >'ara
 >'ison >est`i to~ic >ap`o t~wn  JK, KG, GA.
to~ic d`e >ap`o t~wn JK, KG >'ison >est`i t`o >ap`o t~hc JG;
t`o >'ara >ap`o t~hc JA >'ison >est`i to~ic >ap`o t~wn
JG, GA. >orj`h >'ara >est`in <h <up`o JGA gwn'ia. di`a t`a a\kern -.7pt >ut`a
d`h ka`i <h <up`o DZM gwn'ia >orj'h >estin. >'ish >'ara >est`in
<h <up`o AGJ gwn'ia t~h| <up`o DZM. >'esti d`e ka`i <h
<up`o JAG t~h| <up`o MDZ >'ish. d'uo d`h tr'igwn'a >esti
t`a MDZ, JAG d'uo gwn'iac dus`i gwn'iaic >'isac >'eqonta
<ekat'eran <ekat'era| ka`i m'ian pleur`an mi~a| pleur~a|
>'ishn t`hn <upote'inousan <up`o m'ian t~wn >'iswn gwni~wn
t`hn JA t~h| MD; ka`i t`ac loip`ac >'ara pleur`ac ta~ic loipa~ic
pleura~ic >'isac <'exei <ekat'eran <ekar'era|. >'ish >'ara >est`in
<h AG t~h| DZ. <omo'iwc d`h de'ixomen, <'oti ka`i
<h AB t~h| DE >estin >'ish. >epe`i o>~un >'ish >est`in <h
m`en AG t~h| DZ, <h d`e AB t~h| DE, d'uo d`h a<i GA, AB dus`i
ta~ic
ZD, DE >'isai e>is'in. >all`a ka`i gwn'ia <h <up`o
GAB gwn'ia| t~h| <up`o ZDE >estin >'ish; b'asic >'ara <h BG b'asei
t~h| EZ  >'ish >est`i ka`i t`o tr'igwnon t~w| trig'wnw| ka`i a<i loipa`i
gwn'iai ta~ic loipa~ic gwn'iaic;
>'ish >'ara <h <up`o AGB gwn'ia t~h| <up`o DZE. >'esti d`e
ka`i >orj`h <h <up`o AGK >orj~h| t~h| <up`o DZN >'ish; ka`i
loip`h >'ara <h <up`o BGK loip~h| t~h| <up`o EZN >estin
>'ish. di`a t`a a\kern -.7pt >ut`a d`h ka`i <h <up`o GBK t~h| <up`o ZEN
>estin >'ish.
d'uo d`h tr'igwn'a >esti t`a BGK, EZN
[t`ac] d'uo gwn'iac dus`i gwn'iaic >'isac >'eqonta <ekat'eran
<ekat'era| ka`i m'ian pleur`an mi~a| pleur~a| >'ishn t`hn pr`oc
ta~ic >'isaic gwn'iaic t`hn BG t~h| EZ; ka`i t`ac loip`ac
>'ara pleur`ac ta~ic loipa~ic pleura~ic >'isac <'exousin.
>'ish >'ara >est`in <h GK t~h| ZN. >'esti d`e ka`i <h AG t~h|
DZ >'ish; d'uo d`h a<i AG, GK dus`i ta~ic DZ, ZN >'isai
e>is'in; ka`i >orj`ac gwn'iac peri'eqousin. b'asic >'ara <h AK b'asei
t~h| DN >'ish >est'in. ka`i >epe`i >'ish >est`in <h AJ t~h|
DM, >'ison >est`i ka`i t`o >ap`o t~hc AJ t~w| >ap`o
t~hc DM. >all`a t~w| m`en >ap`o t~hc AJ
 >'isa >est`i t`a
>ap`o t~wn AK, KJ; >orj`h g`ar <h <up`o AKJ; t~w| d`e
>ap`o t~hc DM >'isa t`a >ap`o t~wn DN, NM; >orj`h g`ar
<h <up`o DNM; t`a >'ara >ap`o t~wn AK, KJ >'isa
>est`i to~ic >ap`o t~wn DN, NM, <~wn t`o >ap`o t~hc AK
>'ison >est`i t~w|  >ap`o t~hc DN; loip`on >'ara t`o >ap`o
t~hc KJ >'ison >est`i t~w| >ap`o t~hc NM; >'ish >'ara <h JK
t~h| MN. ka`i >epe`i d'uo a<i JA, AK dus`i ta~ic MD, DN
>'isai e>is`in <ekat'era <ekat'era|, ka`i b'asic <h JK b'asei
t~h| MN >ede'iqjh >'ish, gwn'ia >'ara <h <up`o JAK gwn'ia|
t~h| <up`o MDN >estin >'ish.}

\gr{>E`an >'ara >~wsi d'uo gwn'iai >ep'ipedoi >'isai ka`i t`a
<ex~hc t~hc prot'asewc [<'oper >'edei de~ixai].}}

\ParallelRText{
\begin{center}
{\large Proposition 35}
\end{center}

If there are two equal plane angles, and
raised straight-lines are stood on the apexes of them, containing equal angles respectively
with the original straight-lines (forming the angles), and random points are taken on the raised
(straight-lines), and perpendiculars are drawn from them to the planes
in which the original angles are, and straight-lines are joined  from the
points created in the planes to the (vertices of the) original angles, then they will enclose
equal angles with the raised (straight-lines).

\epsfysize=2.4in
\centerline{\epsffile{Book11/fig35e.eps}}

Let $BAC$ and $EDF$ be two equal rectilinear angles. And let the raised
straight-lines $AG$ and $DM$ be stood  on points $A$ and $D$,
containing equal angles respectively with the original straight-lines. (That
is) $MDE$ (equal) to $GAB$, and $MDF$ (to) $GAC$. And let the random
points $G$ and $M$ be taken on $AG$ and $DM$ (respectively). 
And let the  $GL$ and $MN$ be drawn  from 
points $G$ and $M$ perpendicular to the planes through $BAC$ and $EDF$ (respectively).
And let them have joined the planes at points $L$ and $N$ (respectively).
And let $LA$ and $ND$ be joined. I say that angle $GAL$ is
equal to angle $MDN$.

Let $AH$ be made equal to $DM$. And let $HK$ be drawn through
point $H$ parallel to $GL$. And $GL$  is perpendicular to the plane
through $BAC$. Thus, $HK$  is also perpendicular to the plane through
$BAC$ [Prop.~11.8]. And let  $KC$, $NF$, $KB$, and $NE$ 
be drawn from points $K$ and $N$ perpendicular to the straight-lines $AC$, $DF$,
$AB$, and $DE$. And let $HC$, $CB$, $MF$, and $FE$ 
be joined. Since the (square) on $HA$ is equal to the (sum of the squares)
on $HK$ and $KA$ [Prop.~1.47], 
and the (sum of the squares) on $KC$ and $CA$ is equal to the (square) on $KA$   [Prop.~1.47], thus the  (square) on $HA$ is equal to the (sum
of the squares) on $HK$, $KC$, and $CA$. And the (square) on $HC$ is equal to the (sum of
the squares) on $HK$ and $KC$  [Prop.~1.47].
Thus, the (square) on $HA$ is equal to the (sum of the squares) on $HC$ and
$CA$. Thus, angle $HCA$ is a right-angle [Prop.~1.48]. So, for the same (reasons), angle $DFM$ is also
a right-angle. Thus, angle $ACH$ is equal to (angle) $DFM$. And $HAC$
is also equal to $MDF$.  So, $MDF$ and $HAC$
are two triangles
having two angles equal to two angles, respectively, and one side
equal to one side---(namely), that subtending one of the equal angles ---(that is), $HA$ (equal) to $MD$. 
Thus, they will also have the remaining sides equal to the remaining sides, respectively [Prop.~1.26]. Thus, $AC$ is equal to $DF$. So, similarly, we can show that $AB$ is also equal to $DE$. 
Therefore, since $AC$ is equal to $DF$, and $AB$ to $DE$, so the two
(straight-lines) $CA$ and $AB$ are equal to the two (straight-lines)
$FD$ and $DE$ (respectively). But, angle
$CAB$ is also equal to angle $FDE$. Thus, base $BC$ is equal to base
$EF$, and triangle ($ACB$) to triangle ($DFE$), and the remaining angles
to the remaining angles (respectively) [Prop.~1.4].
Thus, angle  $ACB$ (is) equal to $DFE$. And the right-angle $ACK$
is also equal to the right-angle $DFN$. Thus, the remainder
$BCK$ is equal to the remainder $EFN$. So, for the same (reasons), 
$CBK$ is also equal to $FEN$. So, $BCK$ and $EFN$
are two triangles having two angles equal to two angles, respectively, 
and one side equal to one side---(namely), that by the equal angles---(that is), $BC$ (equal) to $EF$. Thus, they will also have the remaining
sides equal to the remaining sides (respectively) [Prop.~1.26]. Thus, $CK$ is equal to $FN$. And $AC$ (is) also
equal to $DF$. So, the two (straight-lines) $AC$ and $CK$
are equal to the two (straight-lines) $DF$ and $FN$ (respectively). 
And they enclose right-angles. Thus, base $AK$ is equal to base $DN$
[Prop.~1.4]. 
And since $AH$ is equal to $DM$, the (square) on $AH$ is also equal
to the (square) on $DM$. But, the the (sum of the squares) on $AK$ and $KH$ is equal to the (square) on $AH$. For angle $AKH$
(is) a right-angle [Prop.~1.47]. And the (sum of the 
squares) on $DN$ and
$NM$ (is) equal to the square
on $DM$. For angle $DNM$ (is) a right-angle [Prop.~1.47]. Thus, the (sum of the squares) on $AK$ and
$KH$ is equal to the (sum of the squares) on $DN$ and $NM$, of
which the (square) on $AK$ is equal to the (square) on $DN$. Thus, the
remaining (square) on $KH$ is equal to the (square) on $NM$. Thus,
$HK$ (is) equal to $MN$. And since the two (straight-lines) $HA$ and
$AK$ are equal to the two (straight-lines) $MD$ and $DN$, respectively,
and base $HK$ was shown (to be) equal to base $MN$,  angle
$HAK$ is thus equal to angle $MDN$ [Prop.~1.8].

Thus, if there are two equal plane angles, and so on of the proposition. [(Which is) the very thing it was required to show].}
\end{Parallel}

\begin{Parallel}{}{}
\ParallelLText{
\begin{center}
{\large \gr{P'orisma}.}
\end{center}\vspace*{-7pt}

\gr{>Ek d`h to'utou faner'on, <'oti, <e`an >~wsi d'uo gwn'iai >ep'ipedoi
>'isai, >epistaj~wsi d`e >ep> a\kern -.7pt >ut~wn met'ewroi e>uje~iai >'isai
>'isac gwn'iac peri'eqousai met`a t~wn >ex >arq~hc e>ujei~wn
<ekat'eran <ekat'era|, a<i >ap> a\kern -.7pt >ut~wn k'ajetoi
>ag'omenai >ep`i t`a >ep'ipeda, >en o<~ic e>isin a<i >ex >arq~hc
gwn'iai, >'isai >all'hlaic e>is'in. <'oper >'edei de~ixai.}}

\ParallelRText{
\begin{center}
{\large Corollary}
\end{center}

So,  it is clear, from this, that if there are two equal plane angles,
and equal raised straight-lines are stood on them (at their apexes), containing equal
angles respectively with the original straight-lines (forming the angles), then the perpendiculars
drawn from (the raised ends of) them to the planes in which the original angles lie are equal
to one another. (Which is) the very thing it was required to show.}
\end{Parallel}

%%%%
%11.36
%%%%
\pdfbookmark[1]{Proposition 11.36}{pdf11.36}
\begin{Parallel}{}{}
\ParallelLText{
\begin{center}
{\large \ggn{36}.}
\end{center}\vspace*{-7pt}

\gr{>E`an tre~ic e>uje~iai >an'alogon >~wsin, t`o >ek t~wn tri~wn
stere`on parallhlep'ipedon >'ison >est`i t~w| >ap`o t~hc m'eshc
stere~w| parallhlepip'edw| >isople'urw| m'en, >isogwn'iw| d`e t~w|
proeirhm'enw|.}\\~\\

\epsfysize=1.8in
\centerline{\epsffile{Book11/fig36g.eps}}

\gr{>'Estwsan tre~ic e>uje~iai >an'alogon a<i A, B, G, <wc <h A pr`oc t`hn
B, o<'utwc <h B pr`oc t`hn G; l'egw, <'oti t`o >ek t~wn A, B, G
stere`on >'ison >est`i t~w| >ap`o t~hc B stere~w| >isople'urw| m'en,
>isogwn'iw| d`e t~w| proeirhm'enw|.}

\gr{>Ekke'isjw stere`a gwn'ia <h pr`oc t~w| E perieqom'enh <up`o t~wn
<up`o DEH, HEZ, ZED, ka`i ke'isjw t~h| m`en B >'ish <ek'asth t~wn
DE, HE, EZ, ka`i sumpeplhr'wsjw t`o EK stere`on parallhlep'ipedon,
t~h| d`e A >'ish <h LM, ka`i sunest'atw pr`oc t~h| LM e>uje'ia|
ka`i t~w| pr`oc a\kern -.7pt >ut~h| shme'iw| t~w| L t~h| pr`oc t~w| E stere~a|
gwn'ia| >'ish stere`a gwn'ia <h pereiqom'enh <up`o t~wn NLX,
XLM, MLN, ka`i ke'isjw t~h| m`en B >'ish <h LX, t~h| d`e G
>'ish <h LN. ka`i >epe'i >estin <wc <h A pr`oc t`hn B, o<'utwc
<h B pr`oc t`hn G, >'ish d`e <h m`en A t~h| LM, <h d`e B <ekat'era| t~wn
LX, ED, <h d`e G t~h| LN, >'estin >'ara <wc <h LM pr`oc t`hn EZ,
o<'utwc <h DE pr`oc t`hn LN. ka`i per`i >'isac gwn'iac t`ac <up`o
NLM, DEZ a<i pleura`i >antipep'onjasin; >'ison >'ara >est`i t`o MN
parallhl'ogrammon t~w| DZ parallhlogram'ammw|. ka`i >epe`i d'uo
gwn'iai >ep'ipedoi e>uj'ugrammoi >'isai e>is`in a<i <up`o DEZ, NLM,
ka`i >ep> a\kern -.7pt >ut~wn met'ewroi e>uje~iai >efest~asin  a<i LX, EH
>'isai te >all'hlaic ka`i >'isac gwn'iac peri'eqousai met`a t~wn >ex
>arq~hc e>ujei~wn <ekat'eran <ekat'era|, a<i >'ara >ap`o t~wn
H, X shme'iwn k'ajetoi >ag'omenai >ep`i t`a di`a t~wn NLM, DEZ
>ep'ipeda >'isai >all'hlaic e>is'in; <'wste t`a LJ, EK stere`a <up`o
t`o a\kern -.7pt >ut`o <'uyoc >est'in. t`a d`e >ep`i >'iswn b'asewn stere`a parallhlep'ipeda ka`i <up`o t`o a\kern -.7pt >ut`o <'uyoc >'isa >all'hloic >est'in; >'ison >'ara >est`i t`o
JL stere`on t~w| EK stere~w|. ka'i >esti t`o m`en LJ t`o >ek t~wn A, B, G
stere'on, t`o d`e EK t`o >ap`o t~hc B stere'on; t`o >'ara >ek t~wn A, B, G
stere`on parallhlep'ipedon >'ison >est`i t~w| >ap`o t~hc B stere~w|
>isople'urw| m'en, >isogwn'iw| d`e t~w| proeirhm'enw|; <'oper
>'edei de~ixai.}}

\ParallelRText{
\begin{center}
{\large Proposition 36}
\end{center}

If three straight-lines are (continuously)
proportional then the parallelepiped solid (formed) from the three (straight-lines) is equal
to the equilateral parallelepiped solid on the middle (straight-line which is)
equiangular to the aforementioned (parallelepiped solid).

\epsfysize=1.8in
\centerline{\epsffile{Book11/fig36e.eps}}

Let $A$, $B$, and $C$ be three (continuously) proportional straight-lines,
(such that) as $A$ (is) to $B$, so $B$ (is) to $C$. I say that
the (parallelepiped) solid (formed) from $A$, $B$, and $C$ is
equal to the equilateral solid on $B$ (which is) equiangular with the
aforementioned (solid).

Let the solid angle at $E$, contained by $DEG$, $GEF$, and
$FED$, be set out. And let $DE$, $GE$, and $EF$ each be made equal
to $B$. And let the parallelepiped solid $EK$ be completed.
And (let) $LM$ (be made) equal to $A$. And let the solid angle
contained by $NLO$, $OLM$, and $MLN$ be constructed
on the straight-line $LM$, and at the point $L$ on it, (so as to be)
equal to the solid angle $E$ [Prop.~11.23]. 
And let $LO$ be made equal to $B$, and $LN$ equal  to $C$. And
since as $A$ (is) to $B$, so $B$ (is) to $C$, and $A$ (is) equal to
$LM$, and $B$ to each of $LO$ and $ED$, and $C$ to $LN$,
thus as $LM$ (is) to $EF$, so $DE$ (is) to $LN$. And (so) the sides
around the equal angles $NLM$ and $DEF$ are reciprocally
proportional. Thus, parallelogram $MN$ is equal to parallelogram
$DF$ [Prop.~6.14]. And since the
two plane rectilinear angles $DEF$ and $NLM$ are equal, and the raised straight-lines
stood on them (at their apexes), $LO$ and $EG$, are equal to one another,
and contain equal angles respectively with the original straight-lines (forming the angles),  the perpendiculars drawn from points $G$ and $O$
to the planes through $NLM$ and $DEF$ (respectively) are thus
equal to one another [Prop.~11.35~corr.]. 
Thus, the solids $LH$ and $EK$ (have) the same height. And
parallelepiped solids on equal bases, and with the same height, are
equal to one another [Prop.~11.31]. Thus,
solid $HL$ is equal to solid $EK$. And $LH$ is the solid (formed)
from $A$, $B$, and $C$, and $EK$ the solid on $B$. Thus,
the parallelepiped solid (formed) from $A$, $B$, and $C$ is equal
to the equilateral solid on $B$ (which is) equiangular with the
aforementioned (solid). (Which is) the very thing it was required to show.}
\end{Parallel}

%%%%
%11.37
%%%%
\pdfbookmark[1]{Proposition 11.37}{pdf11.37}
\begin{Parallel}{}{}
\ParallelLText{
\begin{center}
{\large \ggn{37}.}
\end{center}\vspace*{-7pt}

\gr{>E`an t'essarec e>uje~iai >an'alogon >~wsin, ka`i t`a >ap> a\kern -.7pt >ut~wn
stere`a parallhlep'ipeda <'omoi'a te ka`i <omo'iwc >anagraf'omena >an'alogon
>'estai; ka`i >e`an t`a >ap> a\kern -.7pt >ut~wn stere`a parallhlep'ipeda <'omoi'a
te ka`i <omo'iwc >anagraf'omena >an'alogon >~h|, ka`i a\kern -.7pt >uta`i a<i e>uje~iai
>an'alogon >'esontai.}\\

\epsfysize=1.9in
\centerline{\epsffile{Book11/fig37g.eps}}

\gr{>'Estwsan t'essarec e>uje~iai >an'alogon a<i AB, GD, EZ, HJ, <wc <h AB
pr`oc t`hn GD, o<'utwc <h EZ pr`oc t`hn HJ, ka`i >anagegr'afjwsan
>ap`o t~wn AB, GD, EZ, HJ <'omoi'a te ka`i <omo'iwc ke'imena stere`a
parallhlep'ipeda t`a KA, LG, ME, NH; l'egw, <'oti >est`in <wc t`o KA
pr`oc t`o LG, o<'utwc t`o ME pr`oc t`o NH.}

\gr{>Epe`i g`ar <'omoi'on >esti t`o KA stere`on parallhlep'ipedon t~w|
LG, t`o KA >'ara pr`oc t`o LG triplas'iona l'ogon >'eqei >'hper
<h AB pr`oc t`hn GD. di`a t`a a\kern -.7pt >ut`a d`h ka`i t`o ME pr`oc t`o NH triplas'iona
l'ogon >'eqei >'hper <h EZ pr`oc t`hn HJ. ka'i >estin <wc <h AB pr`oc
t`hn GD, o<'utwc <h EZ pr`oc t`hn HJ. ka`i <wc
>'ara t`o AK pr`oc t`o LG, o<'utwc t`o ME pr`oc t`o NH.}

\gr{>All`a d`h >'estw <wc t`o AK stere`on pr`oc t`o LG
stere'on, o<'utwc t`o ME stere`on pr`oc t`o NH; l'egw, <'oti
>est`in <wc <h AB e>uje~ia pr`oc t`hn GD, o<'utwc <h EZ pr`oc t`hn HJ.}

\gr{>Epe`i g`ar p'alin t`o KA pr`oc t`o LG triplas'iona l'ogon >'eqei
>'hper <h AB pr`oc t`hn GD, >'eqei d`e ka`i t`o ME pr`oc t`o NH
triplas'iona l'ogon >'hper <h EZ pr`oc t`hn HJ, ka'i >estin <wc t`o
KA pr`oc t`o LG, o<'utwc t`o ME pr`oc t`o NH, ka`i <wc >'ara
<h AB pr`oc t`hn GD, o<'utwc <h EZ pr`oc t`hn HJ.}

\gr{>E`an >'ara t'essarec e>uje~iai >an'alogon >~wsi ka`i t`a <ex~hc t~hc
prot'asewc; <'oper >'edei de~ixai.}}

\ParallelRText{
\begin{center}
{\large Proposition 37}$^\dag$
\end{center}

If four straight-lines are  proportional then the  similar, and similarly described, parallelepiped solids on them will
also be proportional. And if the similar, and similarly
described, parallelepiped solids on them are proportional
then the straight-lines themselves will be  proportional.\\

\epsfysize=1.9in
\centerline{\epsffile{Book11/fig37e.eps}}

Let $AB$, $CD$, $EF$, and $GH$, be four proportional
straight-lines, (such that) as $AB$ (is) to $CD$, so $EF$ (is) to $GH$. 
And let the similar, and similarly laid out, parallelepiped solids $KA$,
$LC$, $ME$ and $NG$ be described on $AB$, $CD$,
$EF$, and $GH$ (respectively). I say that as $KA$ is to $LC$, so
$ME$ (is) to $NG$.

For since the parallelepiped solid $KA$ is similar to $LC$, $KA$
thus has to $LC$ the cubed ratio that $AB$ (has) to $CD$ [Prop.~11.33]. So, for the same (reasons), $ME$ also has to
$NG$ the cubed ratio that $EF$ (has) to $GH$ [Prop.~11.33]. And since as $AB$
is to $CD$, so $EF$ (is) to $GH$, thus, also,  as $AK$ (is) to
$LC$, so $ME$ (is) to $NG$.

And so let solid $AK$ be to solid $LC$, as solid $ME$ (is) to $NG$.
I say that as straight-line $AB$ is to $CD$, so $EF$ (is) to $GH$.

For, again, since $KA$ has to $LC$ the cubed ratio that
$AB$ (has) to $CD$ [Prop.~11.33], and $ME$ also has to $NG$ the cubed ratio
that $EF$ (has) to $GH$ [Prop.~11.33], and as $KA$ is to $LC$, so $ME$
(is) to $NG$, thus, also, as $AB$ (is) to $CD$, so $EF$ (is) to $GH$.

Thus, if four straight-lines are proportional, and so on of
the proposition. (Which is) the very thing it was required to show.}
\end{Parallel}
{\footnotesize\noindent$^\dag$ This proposition assumes that if two
ratios are equal then the cube of the former is also equal to the cube of the
latter, and {\rm  vice versa}.}

%%%%
%11.38
%%%%
\pdfbookmark[1]{Proposition 11.38}{pdf11.38}
\begin{Parallel}{}{}
\ParallelLText{
\begin{center}
{\large \ggn{38}.}
\end{center}\vspace*{-7pt}

\gr{>E`an k'ubou t~wn >apenant'ion >epip'edwn a<i pleura`i d'iqa tm\kern -.7pt hj~wsin,
di`a d`e t~wn tom~wn >ep'ipeda >ekblhj~h|, <h koin`h tom\kern -.7pt `h
t~wn >epip'edwn ka`i <h to~u k'ubou di'ametroc d'iqa t'emnousin
>all'hlac.}

\epsfysize=2.3in
\centerline{\epsffile{Book11/fig38g.eps}}

\gr{K'ubou g`ar to~u AZ t~wn >apenant'ion >epip'edwn t~wn GZ, AJ
a<i pleura`i d'iqa tetm\kern -.7pt 'hsjwsan kat`a t`a K, L, M, N, X, P, O, R
shme~ia, di`a d`e t~wn tom~wn >ep'ipeda >ekbebl'hsjw t`a KN, XR, koin`h
d`e tom\kern -.7pt `h t~wn >epip'edwn >'estw <h US, to~u d`e AZ k'ubou diag'wnioc
<h DH. l'egw, <'oti >'ish >est`in <h m`en UT t~h| TS, <h d`e DT t~h| TH.}

\gr{>Epeze'uqjwsan g`ar a<i DU, UE, BS, SH. ka`i >epe`i par'allhl'oc >estin
<h DX t~h| OE, a<i >enall`ax gwn'iai a<i <up`o DXU, UOE >'isai
>all'hlaic e>is'in. ka`i >epe`i >'ish >est`in <h m`en DX t~h| OE, <h
d`e XU t~h| UO, ka`i gwn'iac >'isac peri'eqousin, b'asic >'ara <h DU
t~h| UE >estin >'ish, ka`i t`o DXU tr'igwnon t~w| OUE trig'wnw|
>est`in >'ison ka`i a<i loipa`i gwn'iai ta~ic loipa~ic
gwn'iaic >'isai; >'ish >'ara <h <up`o XUD gwn'ia t~h| <up`o OUE
gwn'ia|. di`a d`h to~uto e>uje~i'a >estin <h DUE. di`a t`a a\kern -.7pt >ut`a
d`h ka`i BSH e>uje~i'a >estin, ka`i >'ish <h BS t~h| SH. ka`i >epe`i
<h GA t~h| DB >'ish >est`i ka`i par'allhloc, >all`a <h GA ka`i t~h|
EH >'ish t'e >esti ka`i par'allhloc, ka`i <h DB >'ara t~h| EH >'ish t'e >esti ka`i
par'allhloc. ka`i >epizeugn'uousin a\kern -.7pt >ut`ac e>uje~iai a<i DE, BH; par'allhloc
>'ara >est`in <h DE t~h| BH. >'ish >'ara <h m`en <up`o EDT gwn'ia
t~h| <up`o BHT; >enall`ax g'ar; <h d`e <up`o DTU t~h| <up`o HTS.
d'uo d`h tr'igwn'a >esti t`a DTU, HTS t`ac d'uo gwn'iac ta~ic dus`i
gwn'iaic >'isac >'eqonta ka`i m'ian pleur`an mi~a| pleur~a| >'ishn t`hn
<upote'inousan <up`o m'ian t~wn >'iswn gwni~wn t`hn DU t~h| HS;
<hm'iseiai g'ar e>isi t~wn DE, BH; ka`i t`ac loip`ac pleur`ac ta~ic
loipa~ic pleura~ic >'isac <'exei. >'ish >'ara <h m`en DT t~h| TH, <h d`e UT t~h| TS.}

\gr{>E`an >'ara k'ubou t~wn >apenant'ion >epip'edwn a<i pleura`i d'iqa tm\kern -.7pt hj~wsin,
di`a d`e t~wn tom~wn >ep'ipeda >ekblhj~h|, <h koin`h tom\kern -.7pt `h
t~wn >epip'edwn ka`i <h to~u k'ubou di'ametroc d'iqa t'emnousin
>all'hlac; <'oper >'edei de~ixai.}}

\ParallelRText{
\begin{center}
{\large Proposition 38}
\end{center}

If the sides of the opposite planes of a cube are cut in half, and planes are produced through the pieces, then the common section of the (latter) planes and the diameter of the cube cut one another in half.

\epsfysize=2.3in
\centerline{\epsffile{Book11/fig38e.eps}}

For let the opposite planes $CF$ and $AH$ of the cube $AF$ be
cut in half at the points $K$, $L$, $M$, $N$, $O$, $Q$, $P$, and $R$. 
And let the planes $KN$ and $OR$ be produced through the
pieces. And let $US$ be the common section of the planes, and
$DG$ the diameter of cube $AF$. I say that $UT$ is equal to $TS$, and
$DT$ to $TG$.

For let $DU$, $UE$, $BS$, and $SG$ be joined. And since
$DO$ is parallel to $PE$, the alternate angles $DOU$ and $UPE$
are equal to one another [Prop.~1.29].
And since $DO$ is equal to $PE$, and $OU$ to $UP$, and they contain
equal angles, base $DU$ is thus equal to base $UE$, and triangle
$DOU$ is equal to triangle $PUE$, and the remaining angles (are) equal to
the remaining angles [Prop.~1.4]. Thus,
angle $OUD$ (is) equal to angle $PUE$. So, for  this (reason),
$DUE$ is a straight-line [Prop.~1.14].
So, for the same (reason), $BSG$ is also a straight-line, and $BS$ equal to
$SG$. And since $CA$ is equal and parallel to $DB$, but
$CA$ is also equal and parallel to $EG$, $DB$ is thus also equal and parallel 
to $EG$ [Prop.~11.9]. And the straight-lines
$DE$ and $BG$ join them. $DE$ is thus parallel to $BG$ [Prop.~1.33]. Thus, angle $EDT$ (is) equal to $BGT$.
For (they are) alternate [Prop.~1.29]. And (angle)
$DTU$ (is equal) to $GTS$ [Prop.~1.15]. So, $DTU$ and $GTS$ are two triangles having
two angles equal to two angles, and one side equal to one side---(namely),
that subtended by one of the equal angles---(that is), $DU$ (equal) to
$GS$. For they are halves of $DE$ and $BG$ (respectively).
(Thus), they will also have the remaining sides equal to the remaining
sides [Prop.~1.26]. Thus, $DT$ (is) equal
to $TG$, and $UT$ to $TS$.

Thus, if the sides of the opposite planes of a cube are cut in half, and planes are produced through the pieces, then the common section of the (latter) planes and the diameter of the cube cut one another in half. (Which is) the very thing it
was required to show.}
\end{Parallel}

%%%%
%11.39
%%%%
\pdfbookmark[1]{Proposition 11.39}{pdf11.39}
\begin{Parallel}{}{}
\ParallelLText{
\begin{center}
{\large \ggn{39}.}
\end{center}\vspace*{-7pt}

\gr{>E`an >~h| d'uo pr'ismata >iso"uy~h, ka`i t`o m`en >'eq~h| b'asin parallhl'ogrammon, t`o d`e tr'igwnon, dipl'asion d`e >~h| t`o parallhl'ogrammon to~u trig'wnou, >'isa >'estai t`a pr'ismata.}

\epsfysize=1.2in
\centerline{\epsffile{Book11/fig39g.eps}}

\gr{>'Estw d'uo pr'ismata >iso"uy~h t`a ABGDEZ, HJKLMN, ka`i t`o
m`en >eq'etw b'asin t`o AZ parallhl'ogrammon, t`o d`e t`o HJK tr'igwnon,
dipl'asion d`e >'estw t`o AZ parallhl'ogrammon to~u HJK trig'wnou;
l'egw, <'oti >'ison >est`i t`o ABGDEZ pr'isma t~w| HJKLMN
pr'ismati.}

\gr{Sumpeplhr'wsjw g`ar t`a AX, HO stere'a. >epe`i dipl'asi'on >esti t`o AZ
parallhl'ogrammon to~u HJK trig'wnou, >'esti d`e ka`i t`o JK parallhl'ogrammon dipl'asion to~u HJK trig'wnou, >'ison >'ara >est`i
t`o AZ parallhl'ogrammon t~w| JK parallhlogr'ammw|. t`a d`e >ep`i
>'iswn b'asewn >'onta stere`a parallhlep'ipeda ka`i
<up`o t`o a\kern -.7pt >ut`o <'uyoc >'isa >all'hloic >est'in; >'ison >'ara >est`i
t`o AX stere`on t~w| HO stere~w|. ka'i >esti to~u m`en AX stereo~u
<'hmisu t`o ABGDEZ pr'isma, to~u d`e HO stereo~u <'hmisu t`o
HJKLMN pr'isma; >'ison >'ara >est`i t`o ABGDEZ pr'isma t~w|
HJKLMN pr'ismati.}

\gr{>E`an >'ara >~h| d'uo pr'ismata >iso"uy~h, ka`i t`o m`en >'eq~h| b'asin parallhl'ogrammon, t`o d`e tr'igwnon, dipl'asion d`e >~h| t`o parallhl'ogrammon to~u trig'wnou, >'isa >'est`i t`a pr'ismata; <'oper
>'edei de~ixai.}}

\ParallelRText{
\begin{center}
{\large Proposition 39}
\end{center}

If there are two equal height prisms,
and one has a parallelogram, and the other a triangle,  (as a) base, 
and the parallelogram is double the triangle, then the prisms will be equal.

\epsfysize=1.2in
\centerline{\epsffile{Book11/fig39e.eps}}

Let $ABCDEF$ and $GHKLMN$ be two equal height prisms, and
let the former have the parallelogram $AF$, and the latter the
triangle $GHK$, as a base. And let parallelogram $AF$ be twice
triangle $GHK$. I say that prism $ABCDEF$ is equal to prism
$GHKLMN$.

For let the solids $AO$ and $GP$ be completed. 
Since parallelogram $AF$ is double triangle $GHK$, and parallelogram
$HK$ is also double triangle $GHK$ [Prop.~1.34],
parallelogram $AF$ is thus equal to parallelogram $HK$. And parallelepiped
solids which are on equal bases, and (have) the same height, are equal
to one another [Prop.~11.31]. Thus, solid
$AO$ is equal to solid $GP$.  And prism $ABCDEF$ is half of solid
$AO$, and prism $GHKLMN$ half of solid $GP$ [Prop.~11.28]. Prism $ABCDEF$ is thus equal to prism $GHKLMN$.

Thus, if there are two equal height prisms,
and one has a parallelogram, and the other a triangle,  (as a) base, 
and the parallelogram is double the triangle, then the prisms are equal.
(Which is) the very thing it was required to show.}
\end{Parallel}