% !TEX root = ../Elements.tex
%Greek and English text last checked 30/03/2013
%%%%%%
% BOOK 7
%%%%%%
\pdfbookmark[0]{Book 7}{book7}
\pagestyle{empty}
\begin{center}
{\Huge ELEMENTS BOOK 7}\\
\spa\spa\spa
{\huge\it Elementary Number Theory\symbolfootnote[2]{The propositions contained in Books 7--9 are generally attributed to the
school of Pythagoras.}}
\end{center}
%\vspace{2cm}
%\epsfysize=5in
%\centerline{\epsffile{Book07/front.eps}}
\newpage

%%%%%%%
% Definitions
%%%%%%%
\pdfbookmark[1]{Definitions}{def7}
\pagestyle{fancy}
\cfoot{\gr{\thepage}}
\lhead{\large\gr{STOIQEIWN \ggn{7}.}}
\rhead{\large ELEMENTS BOOK 7}
\begin{Parallel}{}{} 
\ParallelLText{
\begin{center}
\large{\gr{<'Oroi}.}
\end{center}\vspace*{-7pt}

\ggn{1}.~\gr{Mon'ac >estin, kaj> <`hn <'ekaston t~wn >'ontwn <`en l'egetai.}

\ggn{2}.~\gr{>Arijm`oc d`e t`o >ek mon'adwn sugke'imenon  pl~hjoc.}

\ggn{3}.~\gr{M'eroc >est`in >arijm`oc >arijmo~u <o >el'asswn to~u me'izonoc, <'otan katametr~h| t`on me'izona.}

\ggn{4}.~\gr{M'erh d'e, <'otan m\kern -.7pt `h katametr~h|.}

\ggn{5}.~\gr{Pollapl'asioc d`e <o me'izwn to~u >el'assonoc, <'otan katame\-tr~htai <up`o to~u >el'assonoc.}

\ggn{6}.~\gr{>'Artioc >arijm'oc >estin <o d'iqa diairo'umenoc.}

\ggn{7}.~\gr{Periss`oc d`e <o m\kern -.7pt `h diairo'umenoc d'iqa >`h [<o] mon'adi
diaf'erwn >art'iou >arijmo~u.}

\ggn{8}.~\gr{>Arti'akic >'artioc >arijm'oc  >estin <o <up`o >art'iou
>arijmo~u metro'umenoc kat`a >'artion >arijm'on.}

\ggn{9}.~\gr{>'Arti'akic d`e periss'oc >estin <o <up`o >art'iou >arijmo~u metro'umenoc kat`a periss`on >arijm'on.}

\ggn{10}.~\gr{Periss'akic d`e periss`oc >arijm'oc >estin <o <up`o perisso~u >arijmo~u metro'umenoc kat`a periss`on >arijm'on.}

\ggn{11}.~\gr{Pr~wtoc >arijm'oc >estin <o mon'adi m'onh| metro'umenoc.}

\ggn{12}.~\gr{Pr~wtoi pr`oc >all'hlouc >arijmo'i e>isin o<i mon'adi m'onh| 
metro'umenoi koin~w| m'etrw|.}

\ggn{13}.~\gr{S'unjetoc >arijm'oc >estin <o >arijm~w| tini metro'umenoc.}

\ggn{14}.~\gr{S'unjetoi d`e pr`oc >all'hlouc >arijmo'i e>isin o<i >arijm~w| tini metro'umenoi koin~w| m'etrw|.}

\ggn{15}.~\gr{>Arijm`oc >arijm`on pollaplasi'azein l'egetai, <'otan, <'osai
e>is`in >en a\kern -.7pt >ut~w| mon'adec, tosaut'akic suntej~h| <o pollaplasiaz'omenoc,
ka`i g'enhta'i tic.}

\ggn{16}.~\gr{<'Otan d`e d'uo >arijmo`i pollaplasi'asantec >all'hlouc poi~ws'i tina, <o gen'omenoc >ep'ipedoc kale~itai, pleura`i d`e a\kern -.7pt >uto~u o<i
pollaplasi'asantec >all'hlouc >arijmo'i.}

\ggn{17}.~\gr{<'Otan d`e tre~ic >arijmo`i pollaplasi'asantec >all'hlouc poi~ws'i tina, <o gen'omenoc stere'oc >estin, pleura`i d`e a\kern -.7pt >uto~u o<i
pollaplasi'asantec >all'hlouc >arijmo'i.}

\ggn{18}.~\gr{Tetr'agwnoc >arijm'oc >estin <o >is'akic >'isoc >`h [<o]
<up`o d'uo >'iswn >arijm~wn perieq'omenoc.}

\ggn{19}.~\gr{K'uboc d`e <o >is'akic >'isoc >is'akic >`h [<o] <up`o
tri~wn >'iswn >arijm~wn perieq'omenoc.}

\ggn{20}.~\gr{>Arijmo`i >an'alog'on e>isin, <'otan <o pr~wtoc
to~u deut'erou ka`i <o tr'itoc to~u tet'artou >is'akic >~h|
pollapl'asioc >`h t`o a\kern -.7pt >ut`o m'eroc >`h t`a a\kern -.7pt >ut`a m'erh
>~wsin.}

\ggn{21}.~\gr{<'Omoioi >ep'ipedoi ka`i stereo`i >arijmo'i e>isin o<i an'alogon >'eqontec t`ac pleur'ac.}

\ggn{22}.~\gr{T'eleioc >arijm'oc >estin <o to~ic <eauto~u
m'eresin >'isoc >'wn.}}

\ParallelRText{
\begin{center}
{\large Definitions}
\end{center}

1.~A unit is (that) according to which each  existing (thing)  is said (to be) one.

2.~And a number (is) a multitude composed of units.$^\dag$

3.~A number is part of a(nother) number, the lesser of the greater, when
it measures the greater.$^\ddag$

4.~But (the lesser is) parts (of the greater) when it does not measure it.$^\S$

5.~And the greater (number is) a multiple of the lesser when it is measured by
the lesser.

6.~An even number is one (which can be) divided in half.

7.~And an odd number is one (which can)not (be) divided in half, 
or which  differs from an even number by a unit.

8.~An even-times-even number is one (which is) measured by an
even number according to an even number.$^\P$

9.~And an even-times-odd number is one (which is) measured by an
even number according to an odd number.$^\ast$

10.~And an odd-times-odd number is one (which is) measured by an
odd number according to an odd number.$^\$$

11.~A prime$^\flat$  number is one (which is) measured by a unit alone.

12.~Numbers prime to one another are those (which are) measured by a unit
alone as a common measure.

13.~A composite number is one (which is) measured by some number.

14.~And numbers composite to one another are those (which are) measured
by some number as a common measure.

15.~A number is said to multiply a(nother) number when the (number being) multiplied is added (to itself) as many times as there are units
 in the former (number), and (thereby) some (other number) is produced.
 
16.~And when two numbers multiplying one another make some
(other number then) the (number so) created  is called plane, and its sides (are) the numbers
which multiply one another.

17.~And when three numbers multiplying one another make some (other number
then) the (number so) created is (called) solid, and its sides (are) the numbers which multiply
one another.

18.~A square number is an equal times an equal, or (a plane number)
contained by two equal numbers.

19.~And a cube (number) is an equal times an equal times an equal,
or (a solid number) contained by three equal numbers.

20.~Numbers are proportional when the first is the same multiple, or
the same part, or the same parts,
of the second that the third (is) of the fourth.

21.~Similar plane and solid numbers are those having proportional
sides.

22.~A perfect number is that which is equal to its own parts.$^{\dag\dag}$}
\end{Parallel}

{\footnotesize\noindent$^\dag$ In other words, a
``number'' is a positive integer greater than unity.\\[0.5ex]
$^\ddag$ In other words, a number $a$ is part of
another number $b$ if there exists some number $n$ such
that $n\,a = b$.\\[0.5ex]
$^\S$  In other words, a number $a$ is parts of another number
$b$ (where $a<b$) if there exist distinct numbers, $m$ and $n$, such that $n\,a=m\,b$.\\[0.5ex]
$^\P$ In other words.
an even-times-even number is the product of two even numbers.\\[0.5ex]
$^\ast$ In other words,
an even-times-odd number is the product of an even and an odd number.\\[0.5ex]
$^\$$ In other words,
an odd-times-odd number is the product of two odd numbers.\\[0.5ex]
$^\flat$ Literally, ``first''.\\[0.5ex]
$^{\dag\dag}$  In other words, a perfect number is equal to the sum of its own factors.}

%%%%%%
% Prop 7.1
%%%%%%
\pdfbookmark[1]{Proposition 7.1}{pdf7.1}
\begin{Parallel}{}{} 
\ParallelLText{
\begin{center}
{\large \ggn{1}.}
\end{center}\vspace*{-7pt}

\gr{D'uo >arijm~wn >an'iswn >ekkeim'enwn, >anjufairoum'enou d`e >ae`i
to~u >el'assonoc >ap`o to~u me'izonoc, >e`an <o leip'omenoc
m\kern -.7pt hd'epote katametr~h| t`on pr`o <eauto~u, <'ewc o<~u leifj~h|
mon'ac, o<i >ex >arq~hc >arijmo`i pr~wtoi pr`oc >all`hlouc
>'esontai.}\\

\epsfysize=1.8in
\centerline{\epsffile{Book07/fig01g.eps}}

\gr{D'uo g`ar [>an'iswn] >arijm~wn t~wn AB, GD >anjufairoum'enou >ae`i to~u >el'assonoc >ap`o to~u me'izonoc <o leip'omenoc m\kern -.7pt hd'epote katametre'itw
t`on pr`o <eauto~u, <'ewc o<~u leifj~h| mon'ac; l'egw, <'oti o<i
AB, GD pr~wtoi pr`oc >all'hlouc e>is'in, tout'estin <'oti
to`uc AB, GD mon`ac m'onh metre~i.}

\gr{E>i g`ar m\kern -.7pt 'h e>isin o<i AB, GD pr~wtoi pr`oc >all'hlouc, metr'hsei tic
a\kern -.7pt >uto`uc >arijm'oc. metre'itw, ka`i >'estw <o E; ka`i <o m`en GD t`on 
BZ metr~wn leip'etw <eauto~u >el'assona t`on ZA, <o d`e AZ t`on
DH metr~wn leip'etw <eauto~u >el'assona t`on HG, <o d`e HG
t`on ZJ metr~wn leip'etw mon'ada t`hn JA.}

\gr{>Epe`i o>~un <o E t`on GD metre~i, <o d`e GD t`on BZ metre~i, ka`i
<o E >'ara t`on BZ metre~i; metre~i d`e ka`i <'olon t`on BA; ka`i
loip`on >'ara t`on AZ metr'hsei. <o d`e AZ t`on DH metre~i; ka`i <o
E >'ara t`on DH metre~i; metre~i d`e ka`i <'olon t`on DG;
ka`i loip`on >'ara t`on GH metr'hsei. <o d`e GH t`on ZJ metre~i;
ka`i <o E >'ara t`on ZJ metre~i; metre~i d`e ka`i <'olon t`on ZA;
ka`i loip`hn >'ara t`hn AJ mon'ada metr'hsei >arijm`oc
>'wn; <'oper >est`in >ad'unaton. o>uk >'ara to`uc AB, GD
>arijmo`uc metr'hsei tic >arijm'oc; o<i AB, GD >'ara pr~wtoi
pr`oc >all'hlouc e>is'in; <'oper >'edei de~ixai.}}

\ParallelRText{
\begin{center}
{\large Proposition 1}
\end{center}

Two unequal numbers (being) laid down, and
the lesser being continually subtracted, in turn,  from the greater, if the remainder never measures the (number) preceding it, until   a unit remains, 
(then) the original numbers will be prime to one another.

\epsfysize=1.8in
\centerline{\epsffile{Book07/fig01e.eps}}

For  two [unequal] numbers, $AB$ and $CD$, the lesser being continually
subtracted, in turn, from the greater, let the remainder never measure the (number) preceding it, until a unit remains. I say that $AB$ and
$CD$ are prime to one another---that is to say, that a unit alone measures
(both) $AB$ and $CD$.

For if $AB$ and $CD$ are not prime to one another, (then) some number will measure
 them. Let (some number) measure  them, and let it be $E$. And let $CD$ measuring $BF$ leave $FA$
less than itself, and let $AF$ measuring $DG$ leave $GC$ less than itself, and 
let $GC$ measuring $FH$ leave a unit, $HA$.

In fact, since $E$ measures $CD$, and $CD$ measures $BF$, $E$ thus also measures
$BF$.$^\dag$ 
And ($E$) also measures the whole of $BA$. Thus, ($E$) will also measure the
remainder $AF$.$^\ddag$  And $AF$ measures $DG$. Thus, $E$ also
measures $DG$. And ($E$) also measures the whole of $DC$. Thus, ($E$) will
also measure the remainder $CG$. And $CG$ measures $FH$. Thus, $E$ also measures $FH$. And ($E$) also measures the whole of $FA$. Thus, ($E$) will also measure the remaining unit $AH$, (despite) being a number. The very thing is impossible.
Thus, some number does not measure (both) the numbers $AB$ and $CD$.
Thus, $AB$ and $CD$ are prime to one another. (Which is) the very thing it
was required to show.}
\end{Parallel}
{\footnotesize\noindent$^\dag$ Here, use is made of the unstated common notion that if
$a$ measures $b$, and $b$ measures $c$, then $a$ also measures $c$,
where all symbols denote numbers.\\[0.5ex]
$^\ddag$ Here, use is made of the unstated common notion
that if $a$ measures $b$, and $a$ measures part of $b$, then $a$
also measures the remainder of $b$, where all
symbols denote numbers.}

%%%%%%
% Prop 7.2
%%%%%%
\pdfbookmark[1]{Proposition 7.2}{pdf7.2}
\begin{Parallel}{}{} 
\ParallelLText{
\begin{center}
{\large \ggn{2}.}
\end{center}\vspace*{-7pt}

\gr{D'uo  >arijm~wn doj'entwn m\kern -.7pt `h pr'wtwn pr`oc >all'hlouc t`o m'egiston
a\kern -.7pt >ut~wn koin`on m'etron e<ure~in.}

\epsfysize=1.8in
\centerline{\epsffile{Book07/fig02g.eps}}

\gr{>'Estwsan o<i doj'entec d'uo >arijmo`i m\kern -.7pt `h pr~wtoi pr`oc >all'hlouc
o<i AB, GD. de~i d`h t~wn AB, GD t`o m'egiston koin`on m'etron
e<ure~in.}

\gr{E>i m`en o>~un <o GD t`on AB metre~i, metre~i d`e ka`i <eaut'on,
<o GD >'ara t~wn GD, AB koin`on m'etron >est'in. ka`i faner'on, <'oti
ka`i m'egiston; o>ude`ic g`ar me'izwn to~u GD t`on GD metr'hsei.}

\gr{E>i d`e o>u metre~i <o GD t`on AB, t~wn AB, GD >anjufairoum'enou
>ae`i to~u >el'assonoc >ap`o to~u me'izonoc leifj'hseta'i tic
>arijm'oc, <`oc metr'hsei t`on pr`o <eauto~u. mon`ac m`en g`ar o>u
leifj'hsetai; e>i d`e m\kern -.7pt 'h, >'esontai o<i AB, GD pr~wtoi pr`oc >all'hlouc;
<'oper o>uq <up'okeitai. leifj'hseta'i tic >'ara >arijm`oc, <`oc metr'hsei
t`on pr`o <eauto~u. ka`i <o m`en GD t`on BE metr~wn leip'etw <eauto~u
>el'assona t`on EA, <o d`e EA t`on DZ metr~wn leip'etw <eauto~u
>el'assona t`on ZG, <o d`e GZ t`on AE metre'itw. >epe`i o>~un <o
GZ t`on AE metre~i, <o d`e AE t`on DZ metre~i, ka`i <o GZ >'ara
t`on DZ metr'hsei. metre~i d`e ka`i <eaut'on; ka`i <'olon >'ara t`on
GD metr'hsei.
 <o d`e GD t`on BE metre~i; ka`i <o GZ >'ara 
 t`on BE metre~i; metre~i d`e ka`i t`on EA; ka`i <'olon
 >'ara t`on BA metr'hsei; metre~i d`e ka`i t`on GD; <o GZ
 >'ara
 to`uc
AB, GD metre~i. <o GZ >'ara t~wn AB, GD koin`on m'etron >est'in. l'egw
d'h, <'oti ka`i m'egiston. e>i g`ar m\kern -.7pt 'h >estin <o GZ t~wn AB, GD
m'egiston koin`on m'etron, metr'hsei tic to`uc AB, GD >arijmo`uc
>arijm`oc me'izwn >`wn to~u GZ. metre'itw, ka`i >'estw <o H. ka`i
>epe`i <o H t`on GD metre~i, <o d`e GD t`on BE metre~i, ka`i <o H
>'ara t`on BE metre~i; metre~i d`e ka`i <'olon t`on BA; ka`i
loip`on >'ara t`on AE metr'hsei. <o d`e AE t`on DZ metre~i; ka`i
<o H >'ara t`on DZ metr'hsei; metre~i d`e ka`i <'olon t`on DG; ka`i
loip`on >'ara t`on GZ metr'hsei <o me'izwn t`on >el'assona;
<'oper >est`in >ad'unaton; o>uk >'ara to`uc AB, GD >arijmo`uc
>arijm'oc tic metr'hsei me'izwn >`wn to~u GZ; <o GZ >'ara t~wn
AB, GD m'egist'on >esti koin`on m'etron [<'oper >'edei de~ixai].}\\~\\~\\~\\~\\~\\

\begin{center}
{\large \gr{P'orisma}.}
\end{center}\vspace*{-7pt}

\gr{>Ek d`h to'utou faner'on, <'oti >e`an >arijm`oc d'uo >arijmo`uc
metr~h|, ka`i t`o m'egiston a\kern -.7pt >ut~wn koin`on m'etron metr'hsei;
<'oper >'edei de~ixai.}}

\ParallelRText{
\begin{center}
{\large Proposition 2}
\end{center}

To find the greatest common measure of two given numbers (which are) not prime to one
another.

\epsfysize=1.8in
\centerline{\epsffile{Book07/fig02e.eps}}

Let $AB$ and $CD$ be the two given numbers (which are) not prime to one another.
So it is required to find the greatest common measure of $AB$ and $CD$.

In fact, if $CD$ measures $AB$,
$CD$ is thus a common measure of $CD$ and $AB$, (since $CD$) also measures itself. And  (it is) manifest that
(it is) also the greatest (common measure). For nothing greater than $CD$ can
measure $CD$.

But if $CD$ does not measure $AB$, (then) some number will
remain from $AB$ and $CD$,  the lesser being continually subtracted, in turn, from the greater,  which will measure the (number) preceding it. For a unit
will not be left. But if not, $AB$ and $CD$ will be prime to one another [Prop.~7.1]. The very opposite thing was assumed.
Thus, some number will remain which will measure the (number) preceding
it. And let $CD$ measuring $BE$ leave $EA$ less than itself, and let $EA$ measuring
$DF$ leave $FC$ less than itself, and let $CF$ measure $AE$. Therefore, since
$CF$ measures $AE$, and $AE$ measures $DF$, $CF$ will thus also measure
$DF$. And it also measures itself. Thus, it will also measure the whole of
$CD$. And $CD$ measures $BE$. Thus, $CF$ also measures $BE$. And it
also
measures $EA$. Thus, it
will also measure the whole of $BA$. And it also measures $CD$. Thus,
$CF$ measures (both) $AB$ and $CD$. Thus, $CF$ is a common measure of
$AB$ and $CD$. So I say that (it is) also the greatest (common measure).
For if $CF$ is not the greatest common measure of $AB$ and $CD$, (then) some number
which is greater than $CF$ will measure the numbers $AB$ and $CD$.
Let it (so) measure ($AB$ and $CD$), and let it be $G$. And since $G$ measures
$CD$, and $CD$ measures $BE$, $G$ thus also measures $BE$. And it also measures
the whole of $BA$. Thus, it will also measure the remainder $AE$. And 
$AE$ measures $DF$. Thus, $G$ 
will also measure $DF$. And it also
measures the whole of $DC$. Thus, it will also measure the remainder
$CF$, the greater (measuring) the lesser. The very thing is impossible.
Thus, some number which is greater than $CF$ cannot measure the numbers
$AB$ and $CD$. Thus, $CF$ is the greatest common measure of $AB$ and $CD$.
[(Which is) the very thing it was required to show].\\

\begin{center}
{\large Corollary}
\end{center}\vspace*{-7pt}

So  it is manifest, from this, that if a number measures two numbers,  (then)
it will also measure their greatest common measure. (Which is) the
very thing it was required to show.}
\end{Parallel}

%%%%%%
% Prop 7.3
%%%%%%
\pdfbookmark[1]{Proposition 7.3}{pdf7.3}
\begin{Parallel}{}{} 
\ParallelLText{
\begin{center}
{\large \ggn{3}.}
\end{center}\vspace*{-7pt}

\gr{Tri~wn >arijm~wn doj'entwn m\kern -.7pt `h pr'wtwn pr`oc >all'hlouc t`o m'egiston a\kern -.7pt >ut~wn koin`on m'etron e<ure~in.}

\epsfysize=2in
\centerline{\epsffile{Book07/fig03g.eps}}

\gr{>'Estwsan o<i doj'entec tre~ic >arijmo`i m\kern -.7pt `h pr~wtoi pr`oc >all'hlouc o<i A, B, G; de~i d`h t~wn A, B, G t`o m'egiston koin`on m'etron e<ure~in.}

\gr{E>il'hfjw g`ar d'uo t~wn A, B t`o m'egiston koin`on m'etron <o D;
<o d`h D t`on G >'htoi metre~i >`h o>u metre~i. metre'itw pr'oteron;
metre~i d'e ka`i to`uc A, B; <o D >'ara to`uc A, B, G metre~i; <o D >'ara
t~wn A, B, G koin`on m'etron >est'in. l'egw d'h, <'oti ka`i m'egiston.
e>i g`ar m\kern -.7pt 'h >estin <o D t~wn A, B, G m'egiston koin`on m'etron,
metr'hsei tic to`uc A, B, G >arijmo`uc >arijm`oc me'izwn >`wn to~u
D. metre'itw, ka`i >'estw <o E. >epe`i o>~un <o E to`uc A, B, G metre~i,
ka`i to`uc A, B  >'ara metr'hsei; ka`i t`o t~wn A, B >'ara m'egiston koin`on
m'etron metr'hsei. t`o d`e t~wn A, B m'egiston koin`on m'etron
>est`in <o D; <o E >'ara t`on D metre~i <o me'izwn t`on >el'assona; <'oper
>est`in >ad'unaton. o>uk >'ara to`uc A, B, G >arijmo`uc
>arijm'oc tic metr'hsei me'izwn >`wn to~u D; <o D >'ara t~wn A, B, G
m'egist'on >esti koin`on m'etron.}

\gr{m\kern -.7pt `h metre'itw d`h <o D t`on G; l'egw pr~wton, <'oti o<i G, D o>'uk
e>isi pr~wtoi pr`oc >all'hlouc. >epe`i g`ar o<i A, B, G o>'uk e>isi
pr~wtoi pr`oc >all'hlouc, metr'hsei tic a\kern -.7pt >uto`uc >arijm'oc. <o
d`h to`uc A, B, G metr~wn ka`i to`uc A, B metr'hsei, ka`i t`o t~wn
A, B m'egiston koin`on m'etron t`on D metr'hsei; metre~i d`e ka`i t`on G;
to`uc D, G >'ara >arijmo`uc >arijm'oc tic metr'hsei; o<i D, G >'ara o>'uk
e>isi pr~wtoi pr`oc >all'hlouc. e>il'hfjw o>~un a\kern -.7pt >ut~wn t`o m'egiston
koin`on m'etron <o E. ka`i >epe`i <o E t`on D metre~i, <o d`e D to`uc
A, B metre~i, ka`i <o E >'ara to`uc A, B metre~i; metre~i d`e ka`i t`on G;
<o E >'ara to`uc A, B, G metre~i. <o E >'ara t~wn A, B, G koin'on
>esti m'etron. l'egw d'h, <'oti ka`i m'egiston. e>i g`ar m\kern -.7pt 'h >estin <o E
t~wn A, B, G t`o m'egiston koin`on m'etron, metr'hsei tic to`uc A, B, G
>arijmo`uc >arijm`oc me'izwn >`wn to~u E. metre'itw, ka`i >'estw <o Z.
ka`i >epe`i <o Z to`uc A, B, G metre~i, ka`i to`uc A, B metre~i;
ka`i t`o t~wn A, B >'ara m'egiston koin`on m'etron metr'hsei. t`o d`e t~wn
A, B m'egiston koin`on m'etron >est`in <o D; <o Z >'ara t`on D metre~i;
metre~i d`e ka`i t`on G; <o Z >'ara to`uc D, G metre~i; ka`i t`o t~wn
D, G >'ara m'egiston koin`on m'etron metr'hsei. t`o d`e t~wn D, G m'egiston
koin`on m'etron >est`in <o E; <o Z >'ara t`on E metre~i <o me'izwn
t`on >el'assona; <'oper >est`in >ad'unaton. o>uk >'ara to`uc
A, B, G
>arijmo`uc >arijm'oc tic metr'hsei me'izwn >`wn to~u E; <o E >'ara
t~wn A, B, G m'egist'on >esti koin`on m'etron; <'oper >'edei de~ixai.}}

\ParallelRText{
\begin{center}
{\large Proposition 3}
\end{center}

To find the greatest common measure of three given
numbers (which are) not prime to one another.

\epsfysize=2in
\centerline{\epsffile{Book07/fig03e.eps}}

Let $A$, $B$,  and $C$ be the three given numbers (which are) not prime to one another. So it is
required to find the greatest common measure of $A$, $B$,  and $C$.

For let the greatest common measure, $D$, of the two (numbers) $A$ and $B$ be taken [Prop.~7.2]. So $D$ either measures, or
does not measure, $C$. First of all, let it measure ($C$). And it also measures
$A$ and $B$. Thus, $D$ measures $A$, $B$, and $C$. Thus, $D$ is a common measure
of $A$, $B$,  and $C$. So I say that (it is) also the greatest (common measure).
For if $D$ is not the greatest common measure of $A$, $B$,  and $C$,  (then) some number
greater than $D$ will measure the numbers $A$, $B$,  and $C$. Let it (so) measure ($A$, $B$, and $C$),
and let it be $E$. Therefore, since $E$ measures $A$, $B$,  and $C$, it will thus
also measure $A$ and $B$. Thus, it will also measure the greatest common
measure of $A$ and $B$ [Prop.~7.2~corr.].
And $D$ is the greatest common measure of $A$ and $B$. Thus, $E$ measures
$D$, the greater (measuring) the lesser. The very thing is impossible.
Thus, some number which is greater than $D$ cannot measure the numbers $A$, $B$,  and $C$. Thus, $D$ is the greatest common measure of $A$, $B$, and $C$.

So let $D$ not measure $C$. I say, first of all, that $C$ and $D$ are not prime
to one another. For since $A$, $B$,  $C$ are not prime to one another,
some number will measure them. So the (number) measuring $A$, $B$, and
$C$ will also measure $A$ and $B$, and it will also measure the greatest common
measure, $D$,  of $A$ and $B$ [Prop.~7.2~corr.].
And it also measures $C$. Thus, some number will measure the numbers $D$ and
$C$. Thus, $D$ and $C$ are not prime to one another. Therefore, let their
greatest common measure, $E$, be taken  [Prop.~7.2]. And since $E$ measures $D$, and $D$
measures $A$ and $B$, $E$ thus also measures $A$ and $B$. And it also measures $C$.
Thus, $E$ measures $A$, $B$,  and $C$. Thus, $E$ is a common measure of
$A$, $B$,  and $C$. So I say that (it is) also the greatest (common measure).
For if $E$ is not the greatest common measure of $A$, $B$, and $C$, (then)
some number greater than $E$ will measure the numbers $A$, $B$,  and $C$.
Let it (so) measure ($A$, $B$, and $C$), and let it be $F$. And since $F$
measures $A$, $B$,  and $C$, it also measures $A$ and $B$. 
 Thus, it will also measure the greatest common measure of $A$ and $B$ [Prop.~7.2~corr.]. And $D$ is the greatest common measure
of $A$ and $B$. Thus,  $F$ measures $D$. And it also measures $C$. Thus, $F$
measures $D$ and $C$. Thus, it will also measure the greatest common
measure of $D$ and $C$ [Prop.~7.2~corr.]. And
$E$ is the greatest common measure of $D$ and $C$. Thus, $F$ measures $E$, the
greater (measuring) the lesser. The very thing is impossible. Thus,
some number which is greater than $E$ does not measure the numbers $A$, $B$, and $C$.
Thus, $E$ is the greatest common measure of $A$, $B$, and $C$. (Which is)
the very thing it was required to show.}
\end{Parallel}

%%%%%%
% Prop 7.4
%%%%%%
\pdfbookmark[1]{Proposition 7.4}{pdf7.4}
\begin{Parallel}{}{} 
\ParallelLText{
\begin{center}
{\large \ggn{4}.}
\end{center}\vspace*{-7pt}

\gr{<'Apac >arijm`oc pant`oc >arijmo~u <o >el'asswn to~u me'izonoc 
>'htoi m'eroc >est`in >`h m'erh.}

\gr{>'Estwsan d'uo >arijmo`i o<i A, BG, ka`i >'estw >el'asswn <o BG;
l'egw, <'oti <o BG to~u A >'htoi m'eroc >est`in >`h m'erh.}

\gr{O<i A, BG g`ar >'htoi pr~wtoi pr`oc >all'hlouc e>is`in >`h o>'u. >'estwsan
pr'oteron o<i A, BG pr~wtoi pr`oc >all'hlouc. diairej'entoc d`h to~u BG
e>ic t`ac >en a\kern -.7pt >ut~w| mon'adac >'estai <ek'asth mon`ac t~wn >en t~w|
BG m'eroc ti to~u A; <'wste m'erh >est`in <o BG to~u A.}

\epsfysize=2in
\centerline{\epsffile{Book07/fig04g.eps}}

\gr{m\kern -.7pt `h >'estwsan d`h o<i  A, BG pr~wtoi pr`oc >all'hlouc; <o d`h BG t`on
A >'htoi metre~i >`h o>u metre~i. e>i m`en o>~un <o BG t`on A
metre~i, m'eroc >est`in <o BG to~u A. e>i d`e o>'u, e>il'hfjw
t~wn A, BG m'egiston koin`on m'etron <o D, ka`i dih|r'hsjw
<o BG e>ic to`uc t~w| D >'isouc to`uc BE, EZ, ZG. ka`i >epe`i 
<o D t`on A metre~i, m'eroc >est`in <o D to~u A; >'isoc d`e <o 
D <ek'astw| t~wn BE, EZ, ZG; ka`i <'ekastoc >'ara t~wn BE, EZ, ZG
to~u A m'eroc >est'in; <'wste m'erh >est`in <o BG to~u A.}

\gr{<'Apac >'ara >arijm`oc pant`oc >arijmo~u <o >el'asswn
to~u me'izonoc >'htoi m'eroc >est`in >`h m'erh; <'oper >'edei de~ixai.}}

\ParallelRText{
\begin{center}
{\large Proposition 4}
\end{center}

Any number is either part or parts of  any (other) number,
the lesser of the greater.

Let $A$ and $BC$ be two numbers, and let $BC$ be the lesser. I say that $BC$
is either part or parts of $A$.

For $A$ and $BC$ are either prime to one another, or not. Let $A$
and $BC$, first of all, be prime to one another. So separating $BC$ into its
constituent units, each of the units in $BC$ will be some part of $A$. Hence, $BC$ is parts of $A$.\\

\epsfysize=2in
\centerline{\epsffile{Book07/fig04e.eps}}

So let  $A$ and $BC$  be not prime to one another. So $BC$ either measures, or
does not measure, $A$. Therefore, if $BC$ measures $A$,  (then) $BC$ is part of
$A$. And if not, let the greatest common measure, $D$, of $A$ and
$BC$ be taken [Prop.~7.2], and
let $BC$ be divided into $BE$, $EF$, and $FC$, equal to $D$. And since
$D$ measures $A$, $D$ is a part of $A$. And $D$ is equal to each of $BE$, $EF$, and
$FC$. Thus, $BE$, $EF$, and $FC$ are also each part of $A$. Hence, $BC$ is parts of $A$.

Thus, any number is either part or parts of any (other) number, the
lesser of the greater. (Which is) the very thing it was required to show.}
\end{Parallel}

%%%%%%
% Prop 7.5
%%%%%%
\pdfbookmark[1]{Proposition 7.5}{pdf7.5}
\begin{Parallel}{}{} 
\ParallelLText{
\begin{center}
{\large \ggn{5}.}
\end{center}\vspace*{-7pt}

\gr{>E`an >arijm`oc >arijmo~u m'eroc >~h|, ka`i <'eteroc <et'erou t`o a\kern -.7pt >ut`o m'eroc >~h|, ka`i sunamf'oteroc sunamfot'erou t`o a\kern -.7pt >ut`o m'eroc
>'estai, <'oper <o e<~ic to~u <en'oc.}\\

\epsfysize=2in
\centerline{\epsffile{Book07/fig05g.eps}}

\gr{>Arijm`oc g`ar <o A [>arijmo~u] to~u BG m'eroc >'estw, ka`i <'eteroc
<o D <et'erou to~u EZ t`o a\kern -.7pt >ut`o m'eroc, <'oper <o A to~u BG; l'egw,
<'oti ka`i sunamf'oteroc <o A, D sunamfot'erou to~u BG, EZ t`o a\kern -.7pt >ut`o
m'eroc >est'in, <'oper <o A to~u BG.}

\gr{>Epe`i g'ar, <`o m'eroc >est`in <o A to~u BG, t`o a\kern -.7pt >ut`o m'eroc >est`i ka`i <o
D to~u EZ, <'osoi >'ara e>is`in >en t~w| BG >arijmo`i >'isoi t~w| A, toso~uto'i e>isi ka`i >en t~w| EZ >arijmo`i >'isoi t~w| D. di~h|r'hsjw <o m`en
BG e>ic to`uc t~w| A >'isouc to`uc BH, HG, <o d`e EZ e>ic to`uc t~w| D
>'isouc to`uc EJ, JZ; >'estai d`h >'ison t`o pl~hjoc  t~wn BH, HG t~w| pl'hjei t~wn EJ, JZ. ka`i >epe`i >'isoc
>est`in <o m`en BH t~w| A, <o d`e EJ t~w| D, ka`i o<i BH, EJ >'ara to~ic
A, D >'isoi. di`a t`a a\kern -.7pt >ut`a d`h ka`i o<i HG, JZ to~ic A, D. <'osoi >'ara
[e>is`in] >en t~w| BG >arijmo`i >'isoi t~w| A, toso~uto'i e>isi ka`i >en
to~ic BG, EZ >'isoi to~ic A, D. <osaplas'iwn >'ara >est`in <o BG
to~u A, tosautaplas'iwn >est`i ka`i sunamf'oteroc <o BG, EZ sunamfot'erou
to~u A, D. <`o >'ara m'eroc >est`in <o A to~u BG,
t`o a\kern -.7pt >ut`o m'eroc >est`i ka`i sunamf'oteroc <o A, D sunamfot'erou to~u
BG, EZ; <'oper >'edei de~ixai.}}

\ParallelRText{
\begin{center}
{\large Proposition 5}$^\dag$
\end{center}

If a number is part of a number, and another
(number) is the same part of another, (then) the sum (of the leading numbers) will also be the
same part of the sum (of the following numbers) that one (number) is of another.

\epsfysize=2in
\centerline{\epsffile{Book07/fig05e.eps}}

For let a number $A$ be part of a [number] $BC$, and another (number)
$D$ (be) the same part of another (number) $EF$ that $A$ (is) of $BC$. I say that
the sum  $A$, $D$ is also the same part of the sum $BC$, $EF$ that
$A$ (is) of $BC$.

For since which(ever) part $A$ is of $BC$, $D$ is the same part of $EF$, thus as many
numbers as are in $BC$ equal to $A$, so many numbers are also in 
$EF$ equal to $D$. Let $BC$ be divided into $BG$ and $GC$, equal to $A$, and
$EF$ into $EH$ and $HF$, equal to $D$. So the multitude of (divisions) $BG$,
$GC$ will be equal to the multitude of (divisions) $EH$, $HF$. And
since $BG$ is equal to $A$, and $EH$ to $D$, thus $BG$, $EH$ (is) also equal to $A$, $D$.
So, for the same (reasons), $GC$, $HF$ (is) also (equal) to $A$, $D$. Thus,
as many numbers as [are] in $BC$ equal to $A$, so many are
also in $BC$, $EF$ equal to $A$, $D$. Thus, as many times as $BC$ is (divisible)
by $A$, so many times is the sum $BC$, $EF$ also (divisible)  by the sum $A$, $D$.
Thus, which(ever) part $A$ is of $BC$, the sum $A$, $D$ is also the same
part of the sum $BC$, $EF$. (Which is) the very thing it was required to show.}
\end{Parallel}
{\footnotesize\noindent$^\dag$ In modern notation, this
proposition states that if $a = (1/n)\,b$ and $c = (1/n)\,d$ then $(a+c) = (1/n)\,(b+d)$,
where all symbols denote numbers.}

%%%%%%
% Prop 7.6
%%%%%%
\pdfbookmark[1]{Proposition 7.6}{pdf7.6}
\begin{Parallel}{}{} 
\ParallelLText{
\begin{center}
{\large \ggn{6}.}
\end{center}\vspace*{-7pt}

\gr{>E`an >arijm`oc >arijmo~u m'erh >~h|, ka`i <'eteroc <et'erou t`a a\kern -.7pt >ut`a
m'erh >~h|, ka`i sunamf'oteroc sunamfot'erou t`a a\kern -.7pt >ut`a m'erh >'estai,
<'oper <o e<~ic to~u <en'oc.}\\

\epsfysize=2in
\centerline{\epsffile{Book07/fig06g.eps}}

\gr{>Arijm`oc g`ar <o AB >arijmo~u to~u G m'erh >'estw, ka`i <'eteroc <o
DE <et'erou to~u Z t`a a\kern -.7pt >ut`a m'erh, <'aper <o AB to~u G; l'egw, <'oti
ka`i sunamf'oteroc <o AB, DE sunamfot'erou to~u G, Z t`a a\kern -.7pt >ut`a m'erh >est'in,
<'aper <o AB to~u G.}

\gr{>Epe`i g'ar, <`a m'erh >est`in <o AB to~u G, t`a a\kern -.7pt >ut`a m'erh ka`i <o
DE to~u Z, <'osa >'ara >est`in >en t~w| AB m'erh to~u G, tosa\kern -.7pt ~ut'a
>esti ka`i >en t~w| DE m'erh to~u Z. dih|r'hsjw <o m`en AB e>ic
t`a to~u G m'erh t`a AH, HB, <o d`e DE e>ic t`a to~u Z m'erh
t`a DJ, JE; >'estai d`h >'ison t`o pl~hjoc t~wn AH, HB t~w| pl'hjei
t~wn DJ, JE. ka`i >epe'i, <`o m'eroc >est`in <o AH to~u G, t`o a\kern -.7pt >ut`o
m'eroc >est`i ka`i <o DJ to~u Z, <`o >'ara m'eroc >est`in <o AH
to~u G, t`o a\kern -.7pt >ut`o m'eroc >est`i ka`i sunamf'oteroc <o AH, DJ
sunamfot'erou to~u G, Z. di`a t`a a\kern -.7pt >ut`a d`h ka`i <`o m'eroc >est`in
<o HB to~u G, t`o a\kern -.7pt >ut`o m'eroc >est`i ka`i sunamf'oteroc <o HB, JE
sunamfot'erou to~u G, Z. <`a >'ara m'erh >est`in <o AB to~u G, t`a
a\kern -.7pt >ut`a m'erh >est`i ka`i sunamf'oteroc <o AB, DE sunamfot'erou
to~u G, Z; <'oper >'edei de~ixai.}}

\ParallelRText{
\begin{center}
{\large Proposition 6}$^\dag$
\end{center}

If a number is parts of a number, and another
(number) is the same parts of another, (then) the sum (of the leading numbers)
will also be the same parts of the sum (of the following numbers) that one (number) is of another.

\epsfysize=2in
\centerline{\epsffile{Book07/fig06e.eps}}

For let a number $AB$ be parts of a number $C$, and another
(number) $DE$ (be)  the same parts of  another (number) $F$ that $AB$ (is) of $C$.
I say that the sum $AB$, $DE$ is also the same parts of the sum
$C$, $F$ that $AB$ (is) of $C$.

For since which(ever) parts $AB$ is of $C$, $DE$ (is) also the same parts  of $F$,
thus as many parts of $C$ as are in $AB$, so many 
 parts of $F$ are also in $DE$. Let $AB$ be divided into the parts of
$C$, $AG$ and $GB$, and $DE$ into the parts of $F$, $DH$ and $HE$. So the
multitude of (divisions) $AG$, $GB$ will be equal to the multitude
of (divisions) $DH$, $HE$. And since which(ever) part $AG$ is of $C$, $DH$ is
also the same part of $F$, thus which(ever) part $AG$ is of $C$, the sum $AG$, $DH$
is  also the same part of the sum $C$, $F$ [Prop.~7.5].  And so, for the same (reasons), 
which(ever) part $GB$ is of $C$, the sum $GB$, $HE$ is also
the same part of the sum $C$, $F$. Thus, which(ever) parts $AB$ is of $C$, the
sum $AB$, $DE$ is also the same parts of the sum $C$, $F$. (Which is)
the very thing it was required to show.}
\end{Parallel}
{\footnotesize\noindent$^\dag$ In modern notation,
this proposition states that if $a = (m/n)\,b$ and $c=(m/n)\,d$
then $(a+c) = (m/n)\,(b+d)$, where all symbols denote numbers.}

%%%%%%
% Prop 7.7
%%%%%%
\pdfbookmark[1]{Proposition 7.7}{pdf7.7}
\begin{Parallel}{}{} 
\ParallelLText{
\begin{center}
{\large\ggn{7}.}
\end{center}\vspace*{-7pt}

\gr{>E`an >arijm`oc >arijmo~u m'eroc >~h|, <'oper >afaireje`ic >afairej'entoc,
ka`i <o loip`oc to~u loipo~u t`o a\kern -.7pt >ut`o m'eroc >'estai, <'oper <o <'oloc
to~u <'olou.}

\epsfysize=0.75in
\centerline{\epsffile{Book07/fig07g.eps}}

\gr{>Arijm`oc g`ar <o AB >arijmo~u to~u GD m'eroc >'estw, <'oper
>afaireje`ic <o AE >afairej'entoc to~u GZ; l'egw, <'oti
ka`i loip`oc <o EB loipo~u to~u ZD t`o a\kern -.7pt >ut`o m'eroc >est'in, <'oper
<'oloc <o AB <'olou to~u GD.}

\gr{<`O g`ar m'eroc >est`in <o AE to~u GZ, t`o a\kern -.7pt >ut`o m'eroc >'estw
ka`i <o EB to~u GH. ka`i >epe'i, <`o m'eroc
>est`in <o AE to~u GZ, t`o a\kern -.7pt >ut`o m'eroc >est`i ka`i <o EB to~u GH,
<`o >'ara m'eroc >est`in <o AE to~u GZ, t`o a\kern -.7pt >ut`o m'eroc >est`i
ka`i <o AB to~u HZ. <`o d`e m'eroc >est`in <o AE to~u GZ, t`o a\kern -.7pt >ut`o
m'eroc <up'okeitai ka`i <o AB to~u GD; <`o >'ara m'eroc >est`i ka`i
<o AB to~u HZ, t`o a\kern -.7pt >ut`o m'eroc >est`i ka`i to~u GD; >'isoc
>'ara >est`in <o HZ t~w| GD. koin`oc >afh|r'hsjw <o GZ; loip`oc
>'ara <o HG loip~w| t~w| ZD >estin >'isoc. ka`i >epe'i, <`o m'eroc
>est`in <o AE to~u GZ, t`o a\kern -.7pt >ut`o m'eroc [>est`i] ka`i <o EB
to~u HG, >'isoc d`e <o HG t~w| ZD,  <`o >'ara m'eroc >est`in
<o AE to~u GZ, t`o a\kern -.7pt >ut`o m'eroc >est`i ka`i <o EB to~u
ZD. >all`a <`o m'eroc >est`in <o AE to~u GZ, t`o a\kern -.7pt >ut`o m'eroc
>est`i ka`i <o AB to~u GD; ka`i loip`oc >'ara <o EB loipo~u
to~u ZD t`o a\kern -.7pt >ut`o m'eroc >est'in, <'oper <'oloc <o AB
<'olou to~u GD; <'oper >'edei de~ixai.}}

\ParallelRText{
\begin{center}
{\large Proposition 7}$^\dag$
\end{center}

If a number is that part of a number that a (part) taken away (is) of a (part) taken away, (then) the remainder will also
be the same part of the remainder that the whole (is) of the whole.

\epsfysize=0.75in
\centerline{\epsffile{Book07/fig07e.eps}}

For let a number $AB$ be that part of a number $CD$  that a (part) taken away $AE$ (is) of a part taken away $CF$. I say that the remainder $EB$ is also the
same part of the remainder $FD$ that the whole $AB$ (is) of the whole $CD$.

For which(ever) part $AE$ is of $CF$, let $EB$ also be the same part of $CG$.
And since which(ever) part $AE$ is of $CF$, $EB$ is also the same part
of $CG$, thus which(ever) part $AE$ is of $CF$, $AB$ is also the same part of
$GF$ [Prop.~7.5]. And which(ever) part $AE$ is of $CF$, $AB$ is also assumed (to be)
the same part of $CD$. Thus, also, which(ever) part $AB$ is of $GF$, ($AB$) is also
the same part of $CD$. Thus, $GF$ is equal to $CD$. Let $CF$ be subtracted from
both. Thus, the remainder $GC$ is equal to the remainder $FD$. And since
which(ever) part $AE$ is of $CF$, $EB$ [is] also the same part of $GC$, and $GC$ (is)
equal to $FD$, thus which(ever) part $AE$ is of $CF$, $EB$ is also the same part of
$FD$.
 But, 
which(ever) part $AE$ is of $CF$, $AB$ is also the same part of $CD$. Thus, the
remainder $EB$ is also the same part of the remainder $FD$ that
the whole $AB$ (is) of the whole $CD$. (Which is) the very thing it
was required to show.}
\end{Parallel}
{\footnotesize \noindent$^\dag$ In modern notation, this
proposition states that if $a=(1/n)\,b$ and $c=(1/n)\,d$ then $(a-c)=
(1/n)\,(b-d)$, where all symbols denote numbers.}

%%%%%%
% Prop 7.8
%%%%%%
\pdfbookmark[1]{Proposition 7.8}{pdf7.8}
\begin{Parallel}{}{} 
\ParallelLText{
\begin{center}
{\large \ggn{8}.}
\end{center}\vspace*{-7pt}

\gr{>E`an >arijm`oc >arijmo~u m'erh >~h|, <'aper >afaireje`ic
>afairej'entoc, ka`i <o loip`oc to~u loipo~u
t`a a\kern -.7pt >ut`a m'erh >'estai, <'aper <o <'oloc to~u <'olou.}\\

\epsfysize=1.4in
\centerline{\epsffile{Book07/fig08g.eps}}

\gr{>Arijm`oc g`ar <o AB >arijmo~u to~u GD m'erh >'estw,
<'aper >afaireje`ic <o AE >afairej'entoc to~u GZ; l'egw, <'oti ka`i loip`oc
<o EB loipo~u to~u ZD t`a a\kern -.7pt >ut`a m'erh >est'in, <'aper
<'oloc <o AB <'olou to~u GD.}

\gr{Ke'isjw g`ar t~w| AB >'isoc <o HJ, <`a >'ara m'erh >est`in <o HJ
to~u GD, t`a a\kern -.7pt >ut`a m'erh >est`i ka`i <o AE to~u GZ. dih|r'hsjw
<o m`en HJ e>ic t`a to~u GD m'erh t`a HK, KJ, <o d`e AE e>ic t`a
to~u GZ m'erh t`a AL, LE; >'estai d`h >'ison t`o pl~hjoc
t~wn HK, KJ t~w| pl'hjei t~wn AL, LE. ka`i >epe'i, <`o m'eroc
>est`in <o HK to~u GD, t`o a\kern -.7pt >ut`o m'eroc >est`i ka`i <o AL to~u
GZ, me'izwn d`e <o GD to~u GZ, me'izwn >'ara ka`i <o HK
to~u AL. ke'isjw t~w| AL >'isoc <o HM. <`o >'ara m'eroc
>est`in <o HK to~u GD, t`o a\kern -.7pt >ut`o m'eroc >est`i ka`i <o HM
to~u GZ; ka`i loip`oc >'ara <o MK loipo~u to~u ZD t`o a\kern -.7pt >ut`o 
m'eroc >est'in, <'oper <'oloc <o HK <'olou to~u GD. p'alin
>epe'i, <`o m'eroc >est`in <o KJ to~u GD, t`o a\kern -.7pt >ut`o m'eroc
>est`i ka`i <o EL to~u GZ, me'izwn d`e <o GD to~u GZ, me'izwn
>'ara ka`i <o JK to~u EL. ke'isjw t~w| EL >'isoc <o KN.
<`o >'ara m'eroc >est`in <o KJ to~u  GD, t`o a\kern -.7pt >ut`o
m'eroc >est`i ka`i <o KN to~u GZ; ka`i loip`oc >'ara <o NJ
loipo~u to~u ZD t`o a\kern -.7pt >ut`o m'eroc >est'in, <'oper <'oloc <o KJ
<'olou to~u GD. >ede'iqjh d`e ka`i loip`oc <o MK loipo~u
to~u ZD t`o a\kern -.7pt >ut`o m'eroc >'wn, <'oper <'oloc
<o HK <'olou to~u GD; ka`i sunamf'oteroc >'ara <o MK, NJ
to~u DZ t`a a\kern -.7pt >ut`a m'erh >est'in, <'aper <'oloc <o JH <'olou to~u
GD. >'isoc d`e sunamf'oteroc m`en <o MK, NJ
t~w| EB, <o d`e JH t~w| BA; ka`i loip`oc >'ara <o EB
loipo~u to~u ZD t`a a\kern -.7pt >ut`a m'erh >est'in, <'aper <'oloc
<o AB <'olou to~u GD; <'oper >'edei de~ixai.}}

\ParallelRText{
\begin{center}
{\large Proposition 8}$^\dag$
\end{center}

If  a number is those parts of a number that a
(part) taken away (is) of a (part) taken away, (then) the remainder will also be
the same parts of the remainder that the whole (is) of the whole.

\epsfysize=1.4in
\centerline{\epsffile{Book07/fig08e.eps}}

For let a number $AB$ be those parts of a number $CD$ that a
(part) taken away $AE$ (is) of a (part) taken away $CF$. I say that the remainder
$EB$ is also the same parts of the remainder $FD$ that the whole $AB$ (is) of the
whole $CD$.

For let $GH$ be laid down equal to $AB$. Thus, which(ever) parts $GH$ is of $CD$,
$AE$ is also the same parts of $CF$. Let $GH$ be divided into the parts
of $CD$, $GK$ and $KH$, and $AE$ into the parts of $CF$, $AL$ and $LE$. So the
multitude of (divisions) $GK$, $KH$ will be equal to the multitude of (divisions)
$AL$, $LE$. And since which(ever) part $GK$ is of $CD$, $AL$ is also the same part 
of $CF$, and $CD$ (is) greater than $CF$,  $GK$ (is) thus also greater than $AL$.
Let $GM$ be made equal to $AL$. Thus, which(ever) part $GK$ is of $CD$, $GM$ is also the same part of $CF$. Thus, the remainder $MK$ is also the same part of the
remainder $FD$ that  the whole $GK$ (is) of the whole $CD$ [Prop.~7.5]. Again, since
which(ever) part $KH$ is of $CD$, $EL$ is also the same part of $CF$, and
$CD$ (is) greater than $CF$, $HK$ (is) thus also greater than $EL$. Let $KN$ be
made equal to $EL$. Thus, which(ever) part $KH$ (is) of $CD$, $KN$ is also the
same part of $CF$. Thus, the remainder $NH$ is also the same part 
of the remainder $FD$ that the whole $KH$ (is) of the whole $CD$  [Prop.~7.5]. And the remainder $MK$
was also shown to be the same part of the remainder $FD$ that the whole
$GK$ (is) of the whole $CD$. Thus, the sum $MK$, $NH$ is the same parts of
$DF$ that the whole $HG$ (is) of the whole $CD$. And the sum $MK$, $NH$
(is) equal to $EB$, and $HG$ to $BA$. Thus, the remainder $EB$ is also
the same parts of the remainder $FD$ that the whole $AB$ (is)
of the whole $CD$. (Which is) the very thing it was required to show.}
\end{Parallel}
{\footnotesize\noindent$^\dag$ In modern notation, this
proposition states that if $a=(m/n)\,b$ and $c=(m/n)\,d$ then $(a-c)=
(m/n)\,(b-d)$, where all symbols denote numbers.}

%%%%%%
% Prop 7.9
%%%%%%
\pdfbookmark[1]{Proposition 7.9}{pdf7.9}
\begin{Parallel}{}{} 
\ParallelLText{
\begin{center}
{\large \ggn{9}.}
\end{center}\vspace*{-7pt}

\gr{>E`an >arijm`oc >arijmo~u m'eroc >~h|, ka`i <'eteroc <et'erou t`o a\kern -.7pt >ut`o m'eroc >~h|, ka`i >enall'ax, <`o m'eroc >est`in >`h m'erh <o pr~wtoc
to~u tr'itou, t`o a\kern -.7pt >ut`o m'eroc >'estai >`h t`a a\kern -.7pt >ut`a m'erh ka`i <o de'uteroc
to~u tet'artou.}

\epsfysize=2in
\centerline{\epsffile{Book07/fig09g.eps}}

\gr{>Arijm`oc g`ar <o A >arijmo~u to~u BG m'eroc >'estw, ka`i <'eteroc <o D
<et'erou to~u EZ t`o a\kern -.7pt >ut`o m'eroc, <'oper <o A to~u BG; l'egw, <'oti
ka`i >enall'ax, <`o m'eroc >est`in <o A to~u D >`h m'erh, t`o a\kern -.7pt >ut`o
m'eroc >est`i ka`i <o BG to~u EZ >`h m'erh.}

\gr{>Epe`i g`ar <`o m'eroc >est`in <o A to~u BG, t`o a\kern -.7pt >ut`o m'eroc >est`i
ka`i <o D to~u EZ, <'osoi >'ara e>is`in >en t~w| BG >arijmo`i >'isoi
t~w| A, toso~uto'i e>isi ka`i >en t~w| EZ >'isoi t~w| D. dih|r'hsjw <o m`en
BG e>ic to`uc t~w| A >'isouc to`uc BH, HG, <o d`e EZ e>ic to`uc t~w|
D >'isouc to`uc EJ, JZ; >'estai d`h >'ison t`o pl~hjoc t~wn BH, HG t~w|
pl'hjei t~wn EJ, JZ.}

\gr{Ka`i >epe`i >'isoi e>is`in o<i BH, HG >arijmo`i >all'hloic, e>is`i d`e
ka`i o<i EJ, JZ >arijmo`i >'isoi >all'hloic,  ka'i >estin >'ison t`o pl~hjoc
t~wn BH, HG t~w| pl'hjei t~wn EJ, JZ, <`o >'ara m'eroc >est`in <o BH
to~u EJ >`h m'erh, t`o a\kern -.7pt >ut`o m'eroc >est`i ka`i <o HG to~u JZ >`h
t`a a\kern -.7pt >ut`a m'erh; <'wste ka`i <`o m'eroc >est`in <o BH to~u EJ
>`h m'erh, t`o a\kern -.7pt >ut`o m'eroc >est`i ka`i sunamf'oteroc <o BG
sunamfot'erou to~u EZ >`h t`a a\kern -.7pt >ut`a m'erh. >'isoc d`e <o m`en BH
t~w| A, <o d`e EJ t~w| D; <`o >'ara m'eroc >est`in
<o A to~u D >`h m'erh, t`o a\kern -.7pt >ut`o m'eroc >est`i ka`i <o BG
to~u EZ >`h t`a a\kern -.7pt >ut`a m'erh; <'oper >'edei de~ixai.}}

\ParallelRText{
\begin{center}
{\large Proposition 9}$^\dag$
\end{center}

If a number is part of a number, and another (number) is the same part of another, also, alternately, which(ever)
part, or parts,  the first (number)  is of the third, the second (number) will also be the same
part, or the same parts, of the fourth.

\epsfysize=2in
\centerline{\epsffile{Book07/fig09e.eps}}

For let a number $A$ be part of a number $BC$, and another (number) $D$ (be) the same
part of another $EF$ that $A$ (is) of $BC$. I say that, also, alternately, which(ever) part, or
parts, $A$ is of $D$, $BC$ is also the same part, or parts, of $EF$.

For since which(ever) part $A$ is of $BC$, $D$ is also the same part of $EF$, thus
as many numbers as are in $BC$ equal to $A$, so many are also in $EF$ equal to
$D$. Let $BC$ be divided into $BG$ and $GC$, equal to $A$, and $EF$ into
$EH$ and $HF$, equal to $D$. So the multitude of (divisions) $BG$, $GC$ will be
equal to the multitude of (divisions) $EH$, $HF$.

And since the numbers $BG$ and $GC$ are equal to one another, and
the numbers $EH$ and $HF$ are also equal to one another, and the
multitude of (divisions) $BG$, $GC$ is equal to the multitude of (divisions)
$EH$, $HC$, thus which(ever) part, or parts, $BG$ is of $EH$, $GC$ is also
the same part, or the same parts, of $HF$. And hence, which(ever)
part, or parts, $BG$ is of $EH$, the sum $BC$ is also the same
part, or the same parts, of the sum $EF$ [Props.~7.5, 7.6]. And $BG$ (is) equal to $A$, and $EH$ to $D$.
 Thus, which(ever) part, or parts, $A$ is of $D$, $BC$ is also the same part, or
 the same parts, of $EF$. (Which is) the very thing it was required to show.}
\end{Parallel}
{\footnotesize\noindent$^\dag$ In modern notation, this
proposition states that if $a=(1/n)\,b$ and $c=(1/n)\,d$ then if $a=(k/l)\,c$
then $b = (k/l)\,d$, where all symbols denote numbers.}

%%%%%%
% Prop 7.10
%%%%%%
\pdfbookmark[1]{Proposition 7.10}{pdf7.10}
\begin{Parallel}{}{} 
\ParallelLText{
\begin{center}
{\large \ggn{10}.}
\end{center}\vspace*{-7pt}

\gr{>E`an >arijm`oc >arijmo~u m'erh >~h|, ka`i <'eteroc <et'erou t`a a\kern -.7pt >ut`a
m'erh >~h|, ka`i >enall'ax, <`a m'erh >est`in <o pr~wtoc to~u tr'itou
>`h m'eroc, t`a a\kern -.7pt >ut`a m'erh >'estai ka`i <o de'uteroc to~u tet'artou
>`h t`o a\kern -.7pt >ut`o m'eroc.}

\gr{>Arijm`oc g`ar <o AB >arijmo~u to~u G m'erh >'estw, ka`i <'eteroc
<o DE <et'erou to~u Z t`a a\kern -.7pt >ut`a m'erh; l'egw, <'oti
ka`i >enall'ax, <`a m'erh >est`in <o AB to~u DE >`h m'eroc, t`a a\kern -.7pt >ut`a m'erh
>est`i ka`i <o G to~u Z >`h t`o a\kern -.7pt >ut`o m'eroc.}

\epsfysize=2in
\centerline{\epsffile{Book07/fig10g.eps}}

\gr{>Epe`i g'ar, <`a m'erh >est`in <o AB to~u G, t`a a\kern -.7pt >ut`a m'erh >est`i
ka`i <o DE to~u Z, <'osa >'ara >est`in >en t~w| AB  m'erh to~u G, tosa\kern -.7pt ~uta
ka`i >en t~w| DE m'erh to~u Z. dih|r'hsjw <o m`en AB e>ic t`a to~u G m'erh
t`a AH, HB, <o d`e DE e>ic t`a to~u Z m'erh t`a DJ, JE; >'estai  d`h >'ison
t`o pl~hjoc t~wn AH, HB t~w| pl'hjei t~wn
DJ, JE. ka`i >epe'i, <`o m'eroc >est`in <o AH to~u G, t`o a\kern -.7pt >ut`o m'eroc
>est`i ka`i <o DJ to~u Z, ka`i >enall'ax, <`o m'eroc >est`in <o AH to~u DJ
>`h m'erh, t`o a\kern -.7pt >ut`o m'eroc >est`i ka`i <o G to~u Z >`h t`a a\kern -.7pt >ut`a m'erh.
di`a t`a a\kern -.7pt >ut`a d`h ka'i, <`o m'eroc >est`in <o HB to~u JE >`h m'erh, t`o
a\kern -.7pt >ut`o m'eroc >est`i ka`i <o G to~u Z >`h t`a a\kern -.7pt >ut`a m'erh; <'wste
ka'i [<`o m'eroc >est`in <o AH to~u DJ >`h m'erh, t`o a\kern -.7pt >ut`o m'eroc
>est`i ka`i <o HB to~u JE >`h t`a a\kern -.7pt >ut`a m'erh; ka`i <`o >'ara m'eroc 
>est`in <o AH to~u DJ >`h m'erh, t`o a\kern -.7pt >ut`o m'eroc 
 >est`i ka`i <o AB to~u DE >`h t`a a\kern -.7pt >ut`a m'erh; >all> <`o m'eroc >est`in <o AH
to~u DJ >`h m'erh, t`o a\kern -.7pt >ut`o m'eroc >ede'iqjh ka`i <o G to~u Z >`h t`a a\kern -.7pt >ut`a m'erh, ka`i] <`a [>'ara] m'erh >est`in <o AB to~u DE >`h
m'eroc, t`a a\kern -.7pt >ut`a m'erh >est`i ka`i <o G to~u Z >`h t`o a\kern -.7pt >ut`o m'eroc;
<'oper >'edei de~ixai.}}

\ParallelRText{
\begin{center}
{\large Proposition 10}$^\dag$
\end{center}

If a number is parts of a number, and another (number) is the same parts of another, also, alternately, which(ever) parts, or part, 
the first (number) is of the third, the second will also be the same
parts, or the same part, of the fourth.

For let a number $AB$ be parts of a number $C$, and another (number)
$DE$ (be) the same parts of another $F$. I say that, also, alternately, which(ever)
parts, or part, $AB$ is of $DE$, $C$ is also the same parts, or the same part,
of $F$.

\epsfysize=2in
\centerline{\epsffile{Book07/fig10e.eps}}

For since which(ever) parts $AB$ is of $C$, $DE$ is also the same parts of $F$,
thus as many parts of $C$ as are in $AB$, so many parts of $F$ (are)
also in $DE$. Let $AB$ be divided into the parts of $C$, $AG$ and $GB$, and
$DE$ into the parts of $F$, $DH$ and $HE$.  So the multitude of
(divisions) $AG$, $GB$ will be equal to the multitude of (divisions) $DH$, $HE$.
And since which(ever) part $AG$ is of $C$, $DH$ is also the same part of $F$,
also, alternately, which(ever) part, or parts, $AG$ is
of $DH$, $C$ is also the same part, or the same parts, of $F$  [Prop.~7.9]. And so, for the same (reasons), 
which(ever) part, or parts, $GB$ is of $HE$, $C$ is also the same
part, or the same parts, of $F$ [Prop.~7.9]. And so [which(ever) part, or parts, $AG$ is of
$DH$, $GB$ is also the same part, or the same parts, of $HE$. And thus,
which(ever) part, or parts, $AG$ is of $DH$, $AB$ is also the same part, or the
same parts, of $DE$ [Props.~7.5, 7.6]. But, which(ever) part, or parts, $AG$ is of $DH$, $C$ was also
shown (to be) the same part, or the same parts, of $F$. And, thus] which(ever)
parts, or part, $AB$ is of $DE$, $C$ is also the  same parts, or the same part, of $F$. (Which
is) the very thing it was required to show.}
\end{Parallel}
{\footnotesize\noindent$^\dag$ In modern notation, this
proposition states that if $a=(m/n)\,b$ and $c=(m/n)\,d$ then if $a=(k/l)\,c$
then $b = (k/l)\,d$, where all symbols denote numbers.}

%%%%%%
% Prop 7.11
%%%%%%
\pdfbookmark[1]{Proposition 7.11}{pdf7.11}
\begin{Parallel}{}{} 
\ParallelLText{
\begin{center}
{\large \ggn{11}.}
\end{center}\vspace*{-7pt}

\gr{>Ean >~h| <wc <'oloc pr`oc <'olon, o<'utwc >afaireje`ic pr`oc >afairej'enta,
ka`i <o loip`oc pr`oc t`on loip`on >'estai, <wc <'oloc pr`oc <'olon.}

\gr{>'Estw <wc <'oloc <o AB pr`oc <'olon t`on GD, o<'utwc >afaireje`ic
<o AE pr`oc >afairej'enta t`on GZ; l'egw, <'oti ka`i loip`oc <o EB pr`oc
loip`on t`on ZD >estin, <wc <'oloc <o AB pr`oc <'olon t`on GD.}\\~\\

\epsfysize=2in
\centerline{\epsffile{Book07/fig11g.eps}}

\gr{>Epe'i >estin <wc <o AB pr`oc t`on GD, o<'utwc <o AE pr`oc t`on GZ,
<`o >'ara m'eroc >est`in <o AB to~u GD >`h m'erh, t`o a\kern -.7pt >ut`o m'eroc
>est`i ka`i <o AE to~u GZ >`h t`a a\kern -.7pt >ut`a m'erh. ka`i loip`oc >'ara
<o EB loipo~u to~u ZD t`o a\kern -.7pt >ut`o m'eroc >est`in >`h m'erh, <'aper
<o AB to~u GD. >'estin >'ara <wc <o EB pr`oc t`on ZD, o<'utwc
<o AB pr`oc t`on GD; <'oper >'edei de~ixai.}}

\ParallelRText{
\begin{center}
{\large Proposition 11}
\end{center}

If as the whole (of a number) is to the whole (of
another), so
a (part) taken away (is) to a (part) taken away, (then) the remainder will also
be to the
remainder as the whole (is) to the whole.

Let the whole $AB$ be to the whole  $CD$ as the (part) taken
away $AE$ (is) to the (part) taken away $CF$. I say that the remainder
$EB$ is to the remainder $FD$ as the whole $AB$ (is) to the whole $CD$.

\epsfysize=2in
\centerline{\epsffile{Book07/fig11e.eps}}

(For) since as $AB$ is to $CD$, so $AE$ (is) to $CF$, thus which(ever) part, or parts,
$AB$ is of $CD$, $AE$ is also the same part, or the same parts, of $CF$ [Def.~7.20]. Thus, the remainder $EB$ is also the same
part, or parts, of the remainder $FD$ that $AB$ (is) of $CD$ [Props.~7.7, 7.8]. Thus, as $EB$ is to $FD$, so $AB$ (is) to $CD$ [Def.~7.20].
(Which is) the very thing it was required to show.}
\end{Parallel}
{\footnotesize\noindent$^\dag$ In modern notation, this
proposition states that if $a:b::c:d$ then $a:b::a-c:b-d$, where all
symbols denote numbers.}

%%%%%%
% Prop 7.12
%%%%%%
\pdfbookmark[1]{Proposition 7.12}{pdf7.12}
\begin{Parallel}{}{} 
\ParallelLText{
\begin{center}
{\large \ggn{12}.}
\end{center}\vspace*{-7pt}

\gr{>E`an >~wsin <oposoio~un >arijmo`i >an'alogon, >'estai <wc e<~ic t~wn
<hgoum'enwn pr`oc <'ena t~wn <epom'enwn, o<'utwc <'apantec o<i
<hgo'umenoi pr`oc <'apantac to`uc <epom'enouc.}\\

\epsfysize=2in
\centerline{\epsffile{Book07/fig12g.eps}}

\gr{>'Estwsan <oposoio~un >arijmo`i >an'alogon o<i A, B, G, D, <wc <o
A pr`oc t`on B, o<'utwc <o G pr`oc t`on D; l'egw, <'oti >est`in <wc <o
A pr`oc t`on B, o<'utwc o<i A, G pr`oc to`uc B, D.}

\gr{>Epe`i g'ar >estin <wc <o A pr`oc t`on B, o<'utwc <o G pr`oc t`on
D, <`o >'ara m'eroc >est`in <o A to~u B >`h m'erh, t`o a\kern -.7pt >ut`o m'eroc
>est`i ka`i <o G to~u D >`h m'erh. ka`i sunamf'oteroc >'ara <o A, G
sunamfot'erou to~u B, D t`o a\kern -.7pt >ut`o m'eroc >est`in >`h t`a a\kern -.7pt >ut`a m'erh,
<'aper <o A to~u B. >'estin >'ara <wc <o A pr`oc t`on B, o<'utwc o<i A,
G pr`oc to`uc B, D; <'oper >'edei de~ixai.}}

\ParallelRText{
\begin{center}
{\large Proposition 12}$^\dag$
\end{center}

If any multitude whatsoever of numbers are proportional,  (then) as one of the leading (numbers is) to one of the
following so (the sum of) all of the leading (numbers) will be to 
(the sum of) all of the following.

\epsfysize=2in
\centerline{\epsffile{Book07/fig12e.eps}}

Let any multitude  whatsoever of numbers, $A$, $B$, $C$,  $D$, be proportional,
(such that) as $A$ (is) to $B$, so $C$ (is) to $D$. I say that as $A$ is to $B$, so
$A$, $C$ (is) to $B$, $D$.

For since as $A$ is to $B$, so $C$ (is) to $D$, thus which(ever) part, or parts,
$A$ is of $B$, $C$ is also the same part, or parts, of $D$ [Def.~7.20]. Thus, the sum $A$, $C$ is also
the same part, or the same parts, of the sum $B$, $D$ that $A$ (is) of $B$  [Props.~7.5, 7.6]. Thus, as $A$ is to $B$, so $A$, $C$ (is) to
$B$, $D$ [Def.~7.20]. (Which is) the very thing it
was required to show.}
\end{Parallel}
{\footnotesize\noindent$^\dag$ In modern notation, this
proposition states that if $a:b::c:d$ then $a:b::a+c:b+d$, where all
symbols denote numbers.}

%%%%%%
% Prop 7.13
%%%%%%
\pdfbookmark[1]{Proposition 7.13}{pdf7.13}
\begin{Parallel}{}{} 
\ParallelLText{
\begin{center}
{\large \ggn{13}.}
\end{center}\vspace*{-7pt}

\gr{>E`an t'essarec >arijmo`i >an'alogon >~wsin, ka`i >enall`ax >an'alogon >'esontai.}

\epsfysize=2in
\centerline{\epsffile{Book07/fig12g.eps}}

\gr{>'Estwsan t'essarec >arijmo`i >an'alogon o<i A, B, G, D, <wc <o A
pr`oc t`on B, o<'utwc <o G pr`oc t`on D; l'egw, <'oti ka`i >enall`ax
>an'alogon >'esontai, <wc <o A pr`oc t`on G, o<'utwc <o B
pr`oc t`on D.}

\gr{>Epe`i g'ar >estin <wc <o A pr`oc t`on B, o<'utwc <o G
pr`oc t`on D, <`o >'ara m'eroc >est`in <o A to~u B >`h m'erh,
t`o a\kern -.7pt >ut`o m'eroc >est`i ka`i <o G to~u D >`h t`a a\kern -.7pt >ut`a m'erh.
>enall`ax >'ara, <`o m'eroc >est`in <o A to~u G >`h m'erh, t`o
a\kern -.7pt >ut`o m'eroc >est`i ka`i <o B to~u D >`h t`a a\kern -.7pt >ut`a m'erh. >'estin
>'ara <wc <o A pr`oc t`on G, o<'utwc <o B pr`oc t`on D;
<'oper >'edei de~ixai.}}

\ParallelRText{
\begin{center}
{\large Proposition 13}$^\dag$
\end{center}

If four numbers are proportional,  (then) they will
also be proportional alternately.

\epsfysize=2in
\centerline{\epsffile{Book07/fig12e.eps}}

Let the four numbers $A$, $B$, $C$, and $D$ be proportional, (such that) as
$A$ (is) to $B$, so $C$ (is) to $D$. I say that they will also be proportional
alternately, (such that) as $A$ (is) to $C$, so $B$ (is) to $D$.

For since as $A$ is to $B$, so $C$ (is) to $D$, thus which(ever) part, or parts,
$A$ is of $B$, $C$ is also the same part, or the same parts, of $D$  [Def.~7.20]. Thus, alternately, which(ever)
part, or parts, $A$ is of $C$, $B$ is also the same part, or the same parts, of $D$
[Props.~7.9, 7.10]. Thus, as $A$ is to $C$, so $B$ (is) to $D$
[Def.~7.20]. (Which is) the very thing it
was required to show.}
\end{Parallel}
{\footnotesize\noindent$^\dag$ In modern notation, this
proposition states that if $a:b::c:d$ then $a:c::b:d$, where all
symbols denote numbers.}

%%%%%%
% Prop 7.14
%%%%%%
\pdfbookmark[1]{Proposition 7.14}{pdf7.14}
\begin{Parallel}{}{} 
\ParallelLText{
\begin{center}
{\large \ggn{14}.}
\end{center}\vspace*{-7pt}

\gr{>E`an >~wsin <oposoio~un >arijmo`i ka`i >'alloi a\kern -.7pt >uto~ic >'isoi t`o
pl~hjoc s'unduo lamban'omenoi ka`i >en t~w| a\kern -.7pt >ut~w| l'ogw|, ka`i
di> >'isou >en t~w| a\kern -.7pt >ut~w| l'og~w| >'esontai.}\\

\epsfysize=0.65in
\centerline{\epsffile{Book07/fig14g.eps}}

\gr{>'Estwsan <oposoio~un >arijmo`i o<i A, B, G ka`i >'alloi a\kern -.7pt >uto~ic
>'isoi t`o pl~hjoc s'unduo lamban'omenoi >en t~w| a\kern -.7pt >ut~w| l'ogw|
o<i D, E, Z, <wc m`en <o A pr`oc t`on B, o<'utwc <o D pr`oc t`on
E, <wc d`e <o B pr`oc t`on G, o<'utwc
<o E pr`oc t`on Z; l'egw, <'oti
ka`i di> >'isou >est`in <wc <o A pr`oc t`on G, o<'utwc <o D pr`oc t`on Z.}

\gr{>Epe`i g'ar >estin <wc <o A pr`oc t`on B, o<'utwc <o D pr`oc t`on E,
>enall`ax >'ara >est`in <wc <o A pr`oc t`on D, o<'utwc <o B pr`oc
t`on E. p'alin, >epe'i >estin <wc <o B pr`oc t`on G, o<'utwc <o E pr`oc t`on
Z, >enall`ax >'ara >est`in <wc <o B pr`oc t`on E, o<'utwc
<o G pr`oc t`on Z. <wc d`e <o B pr`oc t`on E, o<'utwc <o A pr`oc
t`on D; ka`i <wc >'ara <o A pr`oc t`on D, o<'utwc <o G pr`oc t`on Z;
>enall`ax >'ara >est`in <wc <o A pr`oc t`on G, o<'utwc <o D pr`oc
t`on Z; <'oper >'edei de~ixai.
}}

\ParallelRText{
\begin{center}
{\large Proposition 14}$^\dag$
\end{center}

If there are any multitude of numbers whatsoever,
and (some) other (numbers) of  equal  multitude to them, (which are)
also in the same ratio taken two by two, (then) they will also
be in the same ratio via equality.

\epsfysize=0.65in
\centerline{\epsffile{Book07/fig14e.eps}}

Let there be any multitude of numbers whatsoever, $A$, $B$,  $C$, and 
(some) other (numbers), $D$, $E$, $F$, of equal multitude to them, (which are)
in the same ratio taken two by two, (such that) as $A$ (is) to $B$, so $D$ (is) to $E$,
and as $B$ (is) to $C$, so $E$ (is) to $F$. I say that also, via equality, 
as $A$ is to $C$, so $D$ (is) to $F$.

For since as $A$ is to $B$, so $D$ (is) to $E$, thus, alternately, as $A$ is to $D$, so
$B$ (is) to $E$  [Prop.~7.13]. Again, since
as $B$ is to $C$, so $E$ (is) to $F$, thus, alternately, as $B$ is to $E$, so $C$ (is) to $F$
[Prop.~7.13]. And as $B$ (is) to $E$, so
$A$ (is) to $D$. Thus, also, as $A$ (is) to $D$, so $C$ (is) to $F$. Thus,
alternately, as $A$ is to $C$, so $D$ (is) to $F$ [Prop.~7.13]. (Which is) the very thing it was required to show.}
\end{Parallel}
{\footnotesize\noindent$^\dag$ In modern notation, this
proposition states that if $a:b::d:e$ and $b:c::e:f$ then $a:c::d:f$, where
all symbols denote numbers.}

%%%%%%
% Prop 7.15
%%%%%%
\pdfbookmark[1]{Proposition 7.15}{pdf7.15}
\begin{Parallel}{}{} 
\ParallelLText{
\begin{center}
{\large \ggn{15}.}
\end{center}\vspace*{-7pt}

\gr{>E`an mon`ac >arijm'on tina metr~h|, >isakic d`e <'eteroc >arijm`oc >'allon tin`a >arijm`on metr~h|, ka`i >enall`ax >is'akic <h mon`ac t`on tr'iton >arijm`on metr'hsei
ka`i <o de'uteroc t`on t'etarton.}\\

\epsfysize=0.65in
\centerline{\epsffile{Book07/fig15g.eps}}

\gr{Mon`ac g`ar <h A >arijm'on tina t`on BG metre'itw, >is'akic d`e <'eteroc >arijm`oc
<o D >'allon tin`a >arijm`on t`on EZ metre'itw; l'egw, <'oti ka`i >enall`ax >is'akic
<h A mon`ac t`on D >arijm`on metre~i ka`i <o BG t`on EZ.}

\gr{>Epe`i g`ar >is'akic <h A mon`ac t`on BG >arijm`on metre~i ka`i <o D t`on  EZ,
<'osai >'ara e>is`in >en t~w| BG mon'adec, toso~uto'i e>isi ka`i >en t~w| 
EZ >arijmo`i >'isoi t~w| D. dih|r'hsjw <o m`en BG e>ic  t`ac >en <eaut~w|
mon'adac t`ac BH, HJ, JG, <o d`e EZ e>ic  to`uc t~w| D >'isouc to`uc EK, KL, LZ. >'estai d`h >'ison t`o pl~hjoc t~wn BH, HJ, JG t~w| pl'hjei t~wn EK, KL, LZ.
ka`i >epe`i >'isai e>is`in a<i BH, HJ, JG mon'adec >all'hlaic, e>is`i d`e ka`i o<i
EK, KL, LZ >arijmo`i >'isoi >all'hloic,
ka'i >estin 
>'ison t`o pl~hjoc t~wn BH, HJ, JG mon'adwn t~w| pl'hjei t~wn EK, KL, LZ
>arijm~wn, >'estai >'ara <wc <h BH mon`ac pr`oc t`on EK >arijm'on, o<'utwc
<h HJ mon`ac pr`oc t`on KL >arijm`on ka`i <h JG mon`ac pr`oc t`on LZ
>arijm'on. >'estai >'ara ka`i <wc e<~ic t~wn <hgoum'enwn pr`oc <'ena t~wn
<epom'enwn, o<'utwc <'apantec o<i <hgo'umenoi pr`oc <'apantac to`uc
<epom'enouc; >'estin >'ara <wc <h BH mon`ac pr`oc t`on EK >arijm'on, o<'utwc
<o BG pr`oc t`on EZ. >'ish d`e <h BH mon`ac t~h| A mon'adi, <o d`e EK 
>arijm`oc
t~w| D >arijm~w|. >'estin >'ara <wc <h A mon`ac pr`oc t`on D >arijm'on,
o<'utwc <o BG pr`oc t`on EZ. >is'akic >'ara <h A mon`ac t`on D >arijm`on
metre~i ka`i <o BG t`on EZ; <'oper >'edei de~ixai.}}

\ParallelRText{
\begin{center}
{\large Proposition 15}
\end{center}

If a unit measures some number, and another number
measures some other number as many times, (then), also, alternately, the unit will measure the
third number as many times as the second (number measures) the fourth.

\epsfysize=0.65in
\centerline{\epsffile{Book07/fig15e.eps}}

For let a unit $A$ measure some number $BC$, and let another number $D$ measure some
other number $EF$ as many times. I  say that, also, alternately,
the unit $A$ also measures the number $D$ as many times
as $BC$ (measures) $EF$.

For since the unit $A$ measures the number $BC$ as many times as 
$D$ (measures) $EF$, thus as many units as are in $BC$, so many 
numbers are also in  $EF$ equal to $D$. Let $BC$ be divided into its
constituent units, $BG$, $GH$, and $HC$, and $EF$ into the (divisions) $EK$, $KL$, and $LF$, equal
to $D$. So the multitude of (units) $BG$, $GH$, $HC$ will be equal to the multitude
of (divisions) $EK$, $KL$, $LF$. And since the units $BG$, $GH$, and $HC$ are equal to
one another, and the numbers $EK$, $KL$,  and $LF$ are also equal to one another, and the multitude of the (units) $BG$, $GH$, $HC$ is
equal to the multitude of the numbers $EK$, $KL$, $LF$,
thus as the unit $BG$ (is) to the number $EK$, so the unit $GH$ will be to the number $KL$, and
the unit $HC$ to the number $LF$. And thus, as one of the leading (numbers is) to
one of the following, so (the sum of) all of the leading will be to (the sum of) all of the following 
[Prop.~7.12]. Thus, as the unit $BG$ (is) to the number $EK$,
so $BC$ (is) to $EF$. And the unit $BG$ (is) equal to the unit $A$, and the number
$EK$ to the number $D$. Thus, as the unit $A$ is to the number $D$, so $BC$ (is) to
$EF$. Thus, the unit $A$ measures the number $D$ as many times as $BC$ (measures)
$EF$ [Def.~7.20]. (Which is) the very thing it was required to show.}
\end{Parallel}
{\footnotesize\noindent$^\dag$ This proposition is a
special case of Prop.~7.9.}

%%%%%%
% Prop 7.16
%%%%%%
\pdfbookmark[1]{Proposition 7.16}{pdf7.16}
\begin{Parallel}{}{} 
\ParallelLText{
\begin{center}
{\large \ggn{16}.}
\end{center}\vspace*{-7pt}

\gr{E`an d'uo >arijmo`i pollaplasi'asantec >all'hlouc poi~ws'i tinac, o<i gen'omenoi
>ex a\kern -.7pt >ut~wn >'isoi >all'hloic >'esontai.}\\

\epsfysize=1.6in
\centerline{\epsffile{Book07/fig16g.eps}}

\gr{>'Estwsan d'uo >arijmo`i o<i A, B, ka`i <o m`en A t`on B pollaplasi'asac t`on G
poie'itw, <o d`e B t`on A pollaplasi'asac t`on D poie'itw; l'egw, <'oti >'isoc
>est`in <o G t~w| D.}

\gr{>Epe`i g`ar <o A t`on B pollaplasi'asac t`on G pepo'ihken, <o B >'ara
t`on G metre~i kat`a t`ac >en t~w| A mon'adac. metre~i d`e ka`i <h E mon`ac
t`on A >arijm`on kat`a t`ac >en a\kern -.7pt >ut~w| mon'adac; >is'akic >'ara <h E mon`ac
t`on A >arijm`on metre~i ka`i <o B t`on G. >enall`ax >'ara >is'akic
<h E mon`ac t`on B >arijm`on metre~i ka`i <o A t`on G. p'alin, >epe`i <o
B t`on A pollaplasi'asac t`on D pepo'ihken, <o A >'ara t`on D metre~i kat`a
t`ac >en t~w| B mon'adac. metre~i d`e ka`i <h E mon`ac t`on B kat`a t`ac >en
a\kern -.7pt >ut~w| mon'adac; >is'akic >'ara <h E mon`ac t`on B >arijm`on metre~i 
ka`i <o A t`on D. >is'akic d`e <h E mon`ac t`on B >arijm`on >em'etrei ka`i <o A
t`on G; >is'akic >'ara <o A <ek'ateron t~wn G, D metre~i. >'isoc >'ara
>est`in <o G t~w| D; <'oper >'edei de~ixai. }}

\ParallelRText{
\begin{center}
{\large Proposition 16}$^\dag$
\end{center}

If two numbers multiplying one another make
some (numbers, then) the (numbers) generated from them  will be equal
to one another.

\epsfysize=1.6in
\centerline{\epsffile{Book07/fig16e.eps}}

Let $A$ and $B$ be two numbers. And let $A$  make $C$ (by)  multiplying $B$, and let $B$  make $D$ (by) multiplying
$A$. I say that $C$ is equal to $D$.

For since $A$ has made $C$ (by) multiplying $B$, $B$ thus measures $C$ according to the
units in $A$ [Def.~7.15]. And the unit $E$ also measures
the number $A$ according to the units in it. Thus, the unit $E$ measures the
number $A$ as many times as $B$ (measures) $C$. Thus, alternately, the unit
$E$ measures the number $B$ as many times as $A$ (measures) $C$ [Prop.~7.15]. Again, since $B$ has made
$D$ (by) multiplying $A$, $A$ thus measures $D$ according to the units in $B$ [Def.~7.15].  And the unit $E$ also measures $B$ according to
the units in it. Thus, the unit $E$ measures the number $B$ as many times
as $A$ (measures) $D$.  And the unit $E$ was measuring the number $B$ as many times as
$A$ (measures) $C$. Thus, $A$ measures each of $C$ and $D$ an equal number of times. Thus,
$C$ is equal to $D$. (Which is) the very thing it was required to show.}
\end{Parallel}
{\footnotesize\noindent$^\dag$ In modern notation, this proposition states that
$a\,b=b\,a$, where all symbols denote numbers.}

%%%%%%
% Prop 7.17
%%%%%%
\pdfbookmark[1]{Proposition 7.17}{pdf7.17}
\begin{Parallel}{}{} 
\ParallelLText{
\begin{center}
{\large \ggn{17}.}
\end{center}\vspace*{-7pt}

\gr{>E`an >arijm`oc d'uo >arijmo`uc pollaplasi'asac poi~h| tinac, o<i gen'omenoi
>ex a\kern -.7pt >ut~wn t`on a\kern -.7pt >ut`on <'exousi l'ogon to~ic pollaplasiasje~isin.}

\epsfysize=0.9in
\centerline{\epsffile{Book07/fig17g.eps}}

\gr{>Arijm`oc g`ar <o A d'uo >arijmo`uc to`uc B, G pollaplasi'asac to`uc
D, E poie'itw; l'egw, <'oti >est`in <wc <o B pr`oc t`on G, o<'utwc <o D
pr`oc t`on E.}

\gr{>Epe`i g`ar <o A t`on B  pollaplasi'asac t`on D pepo'ihken, <o B >'ara t`on
D metre~i kat`a t`ac >en t~w| A mon'adac. metre~i d`e ka`i <h Z mon`ac
t`on A >arijm`on kat`a t`ac >en a\kern -.7pt >ut~w| mon'adac; >is'akic >'ara <h
Z mon`ac t`on A >arijm`on metre~i ka`i <o  B t`on D. >'estin >'ara
<wc <h Z mon`ac pr`oc t`on A >arijm'on, o<'utwc <o B pr`oc
t`on D. di`a t`a a\kern -.7pt >ut`a d`h ka`i <wc <h Z mon`ac pr`oc t`on A >arijm'on,
o<'utwc <o G pr`oc t`on
E; ka`i <wc >'ara <o B pr`oc t`on D, o<'utwc <o G pr`oc t`on E.
>enall`ax >'ara >est`in <wc <o B pr`oc t`on G, o<'utwc <o D pr`oc
t`on E; <'oper >'edei de~ixai.}}

\ParallelRText{
\begin{center}
{\large Proposition 17}$^\dag$
\end{center}

If a number  multiplying two numbers makes
some (numbers, then) the (numbers) generated from them will have the
same ratio as the multiplied (numbers).

\epsfysize=0.9in
\centerline{\epsffile{Book07/fig17e.eps}}

For let the number $A$ make
(the numbers) $D$ and $E$ (by) multiplying the two numbers $B$ and $C$ (respectively). I say that as $B$ is to $C$, so $D$ (is) to $E$.

For since $A$  has made $D$ (by) multiplying $B$, $B$ thus measures $D$ according
to the units in $A$ [Def.~7.15]. And the unit $F$ also measures the number $A$
according to the units in it. Thus, the unit $F$ measures the number $A$ as many
times as $B$ (measures) $D$. Thus, as the unit $F$ is to the number $A$, 
so $B$ (is) to $D$ [Def.~7.20]. And so, for the same (reasons), as the unit $F$ (is) to the
number $A$,  so $C$ (is) to $E$. And thus, as $B$ (is) to $D$, so $C$ (is) to $E$. Thus,
alternately, as $B$ is to $C$, so $D$ (is) to $E$ [Prop.~7.13]. (Which is) the very thing it
was required to show.}
\end{Parallel}{\footnotesize\noindent$^\dag$ In modern notation, 
this proposition states that if $d = a\,b$ and $e=a\,c$ then
$d:e::b:c$, where all symbols denote numbers.}

%%%%%%
% Prop 7.18
%%%%%%
\pdfbookmark[1]{Proposition 7.18}{pdf7.18}
\begin{Parallel}{}{} 
\ParallelLText{
\begin{center}
{\large \ggn{18}.}
\end{center}\vspace*{-7pt}

\gr{>E`an d'uo >arijmo`i >arijm'on tina pollaplasi'asantec poi~ws'i tinac,
o<i gen'omenoi >ex a\kern -.7pt >ut~wn t`on a\kern -.7pt >ut`on <'exousi l'ogon to~ic
pollaplasi'asasin.}

\epsfysize=1.5in
\centerline{\epsffile{Book07/fig18g.eps}}

\gr{D'uo g`ar >arijmo`i o<i A, B >arijm'on tina t`on G pollaplasi'asantec
to`uc D, E poie'itwsan; l'egw, <'oti >est`in <wc <o A pr`oc t`on B, o<'utwc
<o D pr`oc t`on E.}

\gr{>Epe`i g`ar <o A t`on G pollaplasi'asac t`on D pepo'ihken, ka`i <o G
>'ara t`on A pollaplasi'asac t`on D pepo'ihken. di`a t`a a\kern -.7pt >ut`a d`h ka`i
<o G t`on B pollaplasi'asac t`on E pepo'ihken. >arijm`oc d`h <o G d'uo
>arijmo`uc to`uc A, B pollaplasi'asac to`uc D, E pepo'ihken. >'estin >'ara <wc
<o A pr`oc t`on B, o<'utwc <o D pr`oc t`on E; <'oper >'edei de~ixai.}}

\ParallelRText{
\begin{center}
{\large Proposition 18}$^\dag$
\end{center}

If two numbers  multiplying some
number make some (other numbers, then) the (numbers) generated from them
will have the same ratio as the multiplying (numbers).

\epsfysize=1.5in
\centerline{\epsffile{Book07/fig18e.eps}}

For let the two numbers $A$ and $B$ make (the numbers) $D$ and $E$ (respectively, by) multiplying
some number $C$. I say that as $A$ is to $B$, so $D$ (is) to $E$.

For since $A$ has made $D$ (by) multiplying $C$,  $C$ has thus also made $D$ (by)
multiplying $A$ [Prop.~7.16]. So, for
the same (reasons), $C$ has also made $E$ (by) multiplying $B$. So the
number $C$ has made  $D$ and $E$ (by) multiplying
the two numbers $A$ and $B$ (respectively). Thus, as $A$ is to $B$, so
$D$ (is) to $E$  [Prop.~7.17]. (Which is)
the very thing it was required to show.}
\end{Parallel}
{\footnotesize\noindent$^\dag$ In modern notation, this
propositions states that if $a\,c = d$ and $b\,c=e$ then $a:b::d:e$,
 where all symbols denote numbers.}
 
%%%%%%
% Prop 7.19
%%%%%%
\pdfbookmark[1]{Proposition 7.19}{pdf7.19}
\begin{Parallel}{}{} 
\ParallelLText{
\begin{center}
{\large \ggn{19}.}
\end{center}\vspace*{-7pt}

\gr{>E`an t'essarec >arijmo`i >an'alogon >~wsin, <o >ek pr'wtou ka`i
tet'artou gen'omenoc >arijm`oc >'isoc >'estai t~w| >ek deut'erou ka`i
tr'itou genom'enw| >arijm~w|; ka`i >e`an <o >ek pr'wtou ka`i tet'artou
gen'omenoc >arijm`oc >'isoc >~h| t~w| >ek deut'erou
ka`i tr'itou, o<i t'essasrec >arijmo`i >an'alogon >'esontai.}

\gr{>'Estwsan t'essarec >arijmo`i >an'alogon o<i A, B, G, D, <wc <o A
pr`oc t`on B, o<'utwc <o G pr`oc t`on D, ka`i <o m`en A t`on D pollaplasi'asac t`on E poie'itw, <o d`e B t`on G pollaplasi'asac t`on Z poie'itw;
l'egw, <'oti >'isoc >est`in <o E t~w| Z.}

\gr{<O g`ar A t`on G pollaplasi'asac t`on H poie'itw. >epe`i o>~un <o A
t`on G pollaplasi'asac t`on H pepo'ihken, t`on d`e D pollaplasi'asac t`on
E pepo'ihken, >arijm`oc d`h <o A d'uo >arijmo`uc to`uc G, D pollaplasi'asac
to'uc H, E pepo'ihken. >'estin >'ara <wc <o G pr`oc t`on D, o<'utwc
<o H pr`oc t`on E. >all> <wc <o G pr`oc t`on D, o<'utwc <o A pr`oc
t`on B; ka`i <wc >'ara <o A pr`oc t`on B, o<'utwc <o H pr`oc t`on E.
p'alin, >epe`i <o A t`on G pollaplasi'asac t`on H pepo'ihken, >all`a  m\kern -.7pt `hn
ka`i <o B t`on G pollaplasi'asac t`on Z pepo'ihken, d'uo d`h >arijmo`i
o<i A, B >arijm'on tina t`on G pollaplasi'asantec to`uc H, Z pepoi'hkasin.
>'estin >'ara <wc <o A pr`oc t`on B, o<'utwc <o H pr`oc t`on Z. >all`a
m\kern -.7pt `hn ka`i <wc <o A pr`oc t`on B, o<'utwc <o H pr`oc
t`on E; ka`i <wc >'ara <o H pr`oc t`on E, o<'utwc <o H pr`oc
t`on Z. <o H >'ara pr`oc <ek'ateron t~wn E, Z t`on a\kern -.7pt >ut`on >'eqei l'ogon;
>'isoc >'ara >est`in <o E t~w| Z.}

\epsfysize=2in
\centerline{\epsffile{Book07/fig19g.eps}}

\gr{>'Estw d`h p'alin >'isoc <o E t~w| Z; l'egw, <'oti >est`in <wc <o A pr`oc
t`on B, o<'utwc <o G pr`oc t`on D.}

\gr{T~wn g`ar a\kern -.7pt >ut~wn kataskeuasj'entwn, >epe`i >'isoc >est`in <o E
t~w| Z, >'estin >'ara <wc <o H pr`oc t`on E, o<'utwc <o H
pr`oc t`on Z. >all> <wc m`en <o H pr`oc t`on E, o<'utwc <o G
pr`oc t`on D, <wc d`e <o H pr`oc t`on Z, o<'utwc
<o A pr`oc t`on B. ka`i <wc >'ara <o A pr`oc t`on B, o<'utwc <o G pr`oc
t`on D; <'oper >'edei de~ixai.}}

\ParallelRText{
\begin{center}
{\large Proposition 19}$^\dag$
\end{center}

If four number are proportional, (then) the number
created from (multiplying) the first and fourth will be equal to the number
created from (multiplying) the second and third. And if the number
created from (multiplying) the first and fourth is equal to the
(number created) from (multiplying) the second and third, (then) the
four numbers will be proportional.

Let $A$, $B$, $C$, and $D$ be four proportional numbers, (such that) as $A$ (is) to
$B$, so $C$ (is) to $D$. And let $A$ make $E$ (by) multiplying $D$, and let
$B$ make $F$ (by) multiplying $C$. I say that $E$ is equal to $F$.

For let $A$ make $G$ (by) multiplying $C$. Therefore, since $A$ has made
$G$ (by) multiplying $C$, and has made $E$ (by) multiplying $D$, the number $A$
has made  $G$ and $E$ by multiplying the two numbers $C$ and $D$ (respectively).
Thus, as $C$ is to $D$, so $G$ (is) to $E$  [Prop.~7.17].
But, as $C$ (is) to $D$, so $A$ (is) to $B$. Thus, also, as $A$ (is) to $B$, so
$G$ (is) to $E$. Again, since $A$ has made $G$ (by) multiplying
$C$, but, in fact,  $B$ has also made $F$ (by)  multiplying $C$, the two
numbers $A$ and 
$B$ have made $G$ and $F$ (respectively, by) multiplying some
number $C$. Thus, as $A$ is to $B$, so $G$ (is) to $F$ [Prop.~7.18]. But, also, as $A$ (is) to $B$, so
$G$ (is) to $E$. And thus,  as $G$ (is) to $E$, so $G$ (is) to $F$. Thus, $G$ has the same
ratio to each of $E$ and $F$. Thus, $E$ is equal to $F$ [Prop.~5.9].

\epsfysize=2in
\centerline{\epsffile{Book07/fig19e.eps}}

So, again, let $E$ be equal to $F$. I say that as $A$ is to $B$, so
$C$  (is) to $D$.

For, with the same construction, since $E$ is equal to $F$, thus as
$G$ is to $E$, so $G$ (is) to $F$  [Prop.~5.7].
But, as $G$ (is) to $E$, so $C$ (is) to $D$ [Prop.~7.17].
And as $G$ (is) to $F$, so $A$ (is) to $B$ [Prop.~7.18].
And, thus, as $A$ (is) to $B$, so $C$ (is) to $D$. (Which is) the very thing it
was required to show.}
\end{Parallel}
{\footnotesize\noindent$^\dag$ In modern notation,  this
proposition reads that if $a:b::c:d$ then $a\,d=b\,c$, and
 vice versa, where all symbols denote numbers.}

%%%%%%
% Prop 7.20
%%%%%%
\pdfbookmark[1]{Proposition 7.20}{pdf7.20}
\begin{Parallel}{}{} 
\ParallelLText{
\begin{center}
{\large \ggn{20}.}
\end{center}\vspace*{-7pt}

\gr{O<i >el'aqistoi >arijmo`i t~wn t`on a\kern -.7pt >ut`on l'ogon >eq'ontwn a\kern -.7pt >uto~ic
metro~usi to`uc t`on a\kern -.7pt >ut`on l'ogon >'eqontac >is'akic <'o te
me'izwn t`on me'izona ka`i <o >el'asswn t`on >el'assona.}

\gr{>'Estwsan g`ar >el'aqistoi >arijmo`i t~wn t`on a\kern -.7pt >ut`on 
l'ogon >eq'ontwn to~ic A, B o<i GD, EZ; l'egw, <'oti >is'akic <o GD
t`on A metre~i ka`i <o EZ t`on B.}\\

\epsfysize=2in
\centerline{\epsffile{Book07/fig20g.eps}}

\gr{<O GD g`ar to~u A o>'uk >esti m'erh. e>i g`ar dunat'on, >'estw; ka`i
<o EZ >'ara to~u B t`a a\kern -.7pt >ut`a m'erh >est'in, <'aper
<o GD to~u A. <'osa >'ara >est`in >en t~w| GD m'erh to~u A, tosa\kern -.7pt ~ut'a
>esti ka`i >en t~w| EZ m'erh to~u B. dih|r'hsjw <o m`en GD e>ic
t`a to~u A m'erh t`a GH, HD, <o d`e EZ e>ic t`a to~u B m'erh t`a
EJ, JZ; >'estai d`h >'ison t`o pl~hjoc t~wn GH, HD t~w| pl'hjei
t~wn EJ, JZ. ka`i >epe`i >'isoi e>is`in o<i GH, HD >arijmo`i
 >all'hloic, e>is`i d`e ka`i o<i EJ, JZ >arijmo`i >'isoi >all'hloic, 
  ka'i >estin >'ison t`o pl~hjoc t~wn GH, HD
t~w| pl'hjei t~wn EJ, JZ, >'estin >'ara <wc <o GH pr`oc t`on EJ,
o<'utwc  <o HD pr`oc t`on JZ.
 >'estai >'ara ka`i <wc e<~ic
t~wn <hgoum'enwn pr`oc <'ena t~wn <epom'enwn, o<'utwc <'apantec
o<i
<hgo'umenoi pr`oc <'apantac to`uc <epom'enouc. 
>'estin >'ara <wc <o GH pr`oc t`on EJ, o<'utwc <o GD pr`oc t`on
EZ; o<i GH, EJ >'ara to~ic GD, EZ >en t~w| a\kern -.7pt >ut~w| l'ogw| e>is`in
>el'assonec >'ontec a\kern -.7pt >ut~wn; <'oper >est`in >ad'unaton; <up'okeintai
g`ar o<i GD, EZ >el'aqistoi t~wn t`on a\kern -.7pt >ut`on l'ogon >eq'ontwn
a\kern -.7pt >uto~ic. o>uk >'ara m'erh >est`in <o GD to~u A; m'eroc >'ara.
ka`i <o EZ to~u B t`o a\kern -.7pt >ut`o m'eroc >est'in, <'oper <o GD to~u A;
>is'akic >'ara <o GD t`on A metre~i ka`i <o EZ t`on B;
<'oper >'edei de~ixai.}}

\ParallelRText{
\begin{center}
{\large Proposition 20}
\end{center}

The  least numbers of those (numbers) having the
same ratio  measure those (numbers) having the same ratio as them an equal
number of times, the greater (measuring) the greater, and the lesser  the lesser.

For let $CD$ and $EF$ be the least numbers having the same ratio as $A$ and
$B$ (respectively). I say that $CD$ measures $A$ the same number of times
as $EF$ (measures) $B$.

\epsfysize=2in
\centerline{\epsffile{Book07/fig20e.eps}}

For $CD$ is not parts of $A$. For, if possible, let it be (parts of $A$). Thus,
$EF$ is also the same parts of $B$ that $CD$ (is) of $A$  [Def.~7.20, Prop.~7.13]. Thus, as many parts of $A$ as
are in $CD$, so many parts of $B$ are also in $EF$.  Let $CD$ be divided into the parts of $A$, $CG$ and $GD$, and $EF$ into the
parts of $B$, $EH$ and $HF$. So the multitude of (divisions) $CG$, $GD$
will be equal to the multitude of (divisions) $EH$, $HF$. And since the
numbers $CG$ and $GD$ are equal to one another, and the numbers
$EH$ and $HF$ are also equal to one another, and the multitude of (divisions)
$CG$, $GD$ is equal to the multitude of (divisions) $EH$, $HF$, thus as $CG$ is to $EH$, so
$GD$ (is) to $HF$. Thus,  as one of the
leading (numbers is) to one of the following, so will (the sum of) all
of the leading (numbers)  be  to (the sum of) all of the following [Prop.~7.12].  Thus, as $CG$ is to $EH$,
so $CD$ (is) to $EF$. Thus, $CG$ and $EH$ are in the same ratio as
$CD$ and $EF$, being less than them. The very thing is impossible. For
$CD$ and $EF$ were assumed (to be) the least of those (numbers) having
the same ratio as them. Thus, $CD$ is not parts of $A$. Thus, (it is) a
part (of $A$) [Prop.~7.4]. And $EF$ is the same part of $B$ that $CD$ (is) of $A$ [Def.~7.20, Prop~7.13]. Thus, $CD$ measures $A$ the same number of times that $EF$ (measures) $B$. (Which is) the very thing it was required to
show.}
\end{Parallel}

%%%%%%
% Prop 7.21
%%%%%%
\pdfbookmark[1]{Proposition 7.21}{pdf7.21}
\begin{Parallel}{}{} 
\ParallelLText{
\begin{center}
{\large\ggn{21}.}
\end{center}\vspace*{-7pt}

\gr{O<i pr~wtoi pr`oc >all'hlouc >arijmo`i >el'aqisto'i e>isi t~wn t`on a\kern -.7pt >ut`on
l'ogon >eq'ontwn a\kern -.7pt >uto~ic.}

\gr{>'Estwsan pr~wtoi pr`oc >all'hlouc >arijmo`i o<i A, B; l'egw, <'oti
o<i A, B >el'aqisto'i e>isi t~wn t`on a\kern -.7pt >ut`on l'ogon >eq'ontwn
a\kern -.7pt >uto~ic.}

\gr{ E>i g`ar m\kern -.7pt 'h, >'esonta'i tinec t~wn A, B >el'assonec >arijmo`i
>en t~w| a\kern -.7pt >ut~w| l'ogw| >'ontec to~ic A, B. >'estwsan
o<i G, D.}\\~\\

\epsfysize=1.8in
\centerline{\epsffile{Book07/fig21g.eps}}

\gr{>Epe`i o>~un o<i >el'aqistoi >arijmo`i t~wn t`on a\kern -.7pt >ut`on l'ogon >eq'ontwn
metro~usi to`uc t`on a\kern -.7pt >ut`on l'ogon >'eqontac >is'akic <'o te me'izwn
t`on me'izona ka`i <o >el'attwn t`on >el'attona, tout'estin <'o te <hgo'umenoc
t`on <hgo'umenon ka`i <o <ep'omenoc t`on <ep'omenon, >is'akic >'ara
<o G t`on A metre~i ka`i <o D t`on B. <os'akic d`h 
<o G t`on A metre~i, tosa\kern -.7pt ~utai mon'adec >'estwsan >en t~w| E.
ka`i <o D >'ara t`on B metre~i kat`a t`ac >en t~w| E mon'adac.
ka`i >epe`i <o G t`on A metre~i kat`a t`ac >en t~w| E mon'adac, ka'i
<o E >'ara t`on A metre~i kat`a t`ac >en t~w| G
mon'adac. di`a t`a a\kern -.7pt >ut`a d`h <o E ka`i t`on B metre~i kat`a t`ac >en t~w|
D mon'adac. <o E >'ara to`uc A, B metre~i pr'wtouc
>'ontac pr`oc >all'hlouc; <'oper >est`in >ad'unaton. o>uk >'ara >'esonta'i
tinec t~wn A, B >el'assonec >arijmo`i >en t~w| a\kern -.7pt >ut~w| l'ogw|
>'ontec to~ic A, B. o<i A, B >'ara >el'aqisto'i e>isi t~wn t`on a\kern -.7pt >ut`on
l'ogon >eq'ontwn a\kern -.7pt >uto~ic; <'oper >'edei de~ixai.}}

\ParallelRText{
\begin{center}
{\large Proposition 21}
\end{center}

Numbers prime to one another are the least of
those (numbers) having the same ratio as them.

Let $A$ and $B$ be numbers prime to one another. I say that $A$ and $B$
are the least of those (numbers) having the same ratio as them.

For if not then there will be some numbers less than $A$ and $B$ which are in
the same ratio as $A$ and $B$. Let them be $C$ and $D$.

\epsfysize=1.8in
\centerline{\epsffile{Book07/fig21e.eps}}

Therefore, since the least numbers of those (numbers) having the
same ratio measure those (numbers) having the same ratio (as them)
an equal number of times, the greater (measuring) the greater, and
the lesser  the lesser---that is to say, the leading (measuring) the leading, and
the following the following---$C$ thus measures $A$ the same number of
times that $D$ (measures) $B$  [Prop.~7.20].
So as many times as $C$ measures $A$, so many units let there be in $E$.
Thus, $D$ also measures $B$ according to the units in $E$. And since $C$ measures
$A$ according to the units in $E$, $E$ thus also measures $A$ according to the
units in $C$ [Prop.~7.16]. So, for the
same (reasons), $E$ also measures $B$ according to the units in $D$  [Prop.~7.16]. Thus, $E$ measures $A$ and $B$, which
are prime to one another. The very thing is impossible. Thus, there cannot
be any numbers less than $A$ and $B$ which are in the same ratio as $A$ and
$B$. Thus, $A$ and $B$ are the least of those (numbers) having the same ratio
as them. (Which is) the very thing it was required to show.}
\end{Parallel}

%%%%%%
% Prop 7.22
%%%%%%
\pdfbookmark[1]{Proposition 7.22}{pdf7.22}
\begin{Parallel}{}{} 
\ParallelLText{
\begin{center}
{\large\ggn{22}.}
\end{center}\vspace*{-7pt}

\gr{O<i >el'aqistoi >arijmo`i t~wn t`on a\kern -.7pt >ut`on l'ogon >eq'ontwn a\kern -.7pt >uto~ic
pr~wtoi pr`oc >all'hlouc e>is'in.}

\epsfysize=1.6in
\centerline{\epsffile{Book07/fig22g.eps}}

\gr{>'Estwsan >el'aqistoi >arijmo`i t~wn t`on a\kern -.7pt >ut`on l'ogon >eq'ontwn
a\kern -.7pt >uto~ic o<i A, B; l'egw, <'oti o<i A, B pr~wtoi pr`oc >all'hlouc
e>is'in.}

\gr{E>i g`ar m\kern -.7pt 'h e>isi pr~wtoi pr`oc >all'hlouc, metr'hsei tic a\kern -.7pt >uto`uc
>arijm'oc. metre'itw, ka`i >'estw <o G. ka`i <os'akic m`en <o G
t`on A metre~i, tosa\kern -.7pt ~utai mon'adec >'estwsan >en t~w| D, <os'akic
d`e <o G t`on B metre~i, tosa\kern -.7pt ~utai mon'adec >'estwsan >en t~w| E.}

\gr{>Epe`i <o G t`on A metre~i kat`a t`ac >en t~w| D mon'adac, <o G
>'ara t`on D pollaplasi'asac t`on A  pepo'ihken. di`a t`a a\kern -.7pt >ut`a d`h ka`i <o
G t`on E pollaplasi'asac t`on B pepo'ihken. >arijm`oc d`h <o G d'uo
>arijmo`uc to`uc D, E pollaplasi'asac to`uc A, B pepo'ihken; >'estin
>'ara <wc <o D pr`oc t`on E, o<'utwc <o A pr`oc t`on B; o<i D, E
>'ara to~ic A, B >en t~w| a\kern -.7pt >ut~w|
l'ogw| e>is`in >el'assonec >'ontec a\kern -.7pt >ut~wn;  <'oper >est`in >ad'unaton.
o>uk >'ara to`uc A, B >arijmo`uc >arijm'oc tic metr'hsei. o<i A, B
>'ara pr~wtoi pr`oc >all'hlouc e>is'in; <'oper >'edei de~ixai.}}

\ParallelRText{
\begin{center}
{\large Proposition 22}
\end{center}

The  least numbers of those (numbers) having
the same ratio as them are prime to one another.

\epsfysize=1.6in
\centerline{\epsffile{Book07/fig22e.eps}}

Let $A$ and $B$ be the least numbers of those (numbers) having the same
ratio as them. I say that $A$ and $B$ are prime to one another.

For if they are not prime to one another, (then) some number will measure them.
Let it (so measure them), and let it be $C$. And as many times as $C$ measures
$A$, so many units let there be in $D$. And as many times as $C$ measures $B$,
so many units let there be in $E$.

Since $C$ measures $A$ according to the units in $D$,  $C$ has thus made $A$ (by)
multiplying $D$ [Def.~7.15].
So, for the same (reasons), $C$ has also made $B$ (by) multiplying
$E$. So the number $C$ has made $A$ and $B$ (by) multiplying the two
numbers $D$ and $E$ (respectively).  Thus, as $D$ is to $E$, so $A$ (is) to $B$ [Prop.~7.17]. Thus, $D$ and $E$ are in the same
ratio as $A$ and $B$, being less than them. The very thing is impossible. Thus,
some number does not measure the numbers $A$ and $B$. Thus, $A$ and $B$ are prime to one another. (Which is) the very thing it was required to show.}
\end{Parallel}

%%%%%%
% Prop 7.23
%%%%%%
\pdfbookmark[1]{Proposition 7.23}{pdf7.23}
\begin{Parallel}{}{} 
\ParallelLText{
\begin{center}
{\large\ggn{23}.}
\end{center}\vspace*{-7pt}

\gr{>E`an d'uo >arijmo`i pr~wtoi pr`oc >all'hlouc >~wsin, <o t`on <'ena
a\kern -.7pt >ut~wn metr~wn >arijm`oc pr`oc t`on loip`on pr~wtoc >'estai.}

\gr{>'Estwsan d'uo >arijmo`i pr~wtoi pr`oc >all'hlouc o<i A, B, t`on
d`e A metre'itw tic >arijm`oc <o G; l'egw, <'oti ka`i o<i G, B
pr~wtoi pr`oc >all'hlouc e>is'in.}

\epsfysize=2in
\centerline{\epsffile{Book07/fig23g.eps}}

\gr{E>i g`ar m\kern -.7pt 'h e>isin o<i G, B pr~wtoi pr`oc >all'hlouc, metr'hsei [tic]
to`uc G, B >arijm'oc. metre'itw, ka`i >'estw <o D. >epe`i <o D t`on
G metre~i, <o d`e G t`on A metre~i, ka`i <o D >'ara t`on A metre~i.
metre~i d`e ka`i t`on B; <o D >'ara to`uc A, B metre~i pr'wtouc
>'ontac pr`oc >all'hlouc; <'oper >est`in >ad'unaton. o>uk >'ara
to`uc G, B >arijmo`uc >arijm'oc tic metr'hsei. o<i G, B >'ara
pr~wtoi pr`oc >all'hlouc e>is'in; <'oper >'edei de~ixai.}}

\ParallelRText{
\begin{center}
{\large Proposition 23}
\end{center}

If two numbers are prime to one another, (then)
a number measuring one of them will be prime to the remaining (one).

Let $A$ and $B$ be two numbers (which are) prime to one another, and let some
number $C$ measure $A$. I say that $C$ and $B$ are also prime to one another.

\epsfysize=2in
\centerline{\epsffile{Book07/fig23e.eps}}

For if $C$ and $B$ are not prime to one another, (then) [some] number will measure
$C$ and $B$. Let it (so) measure (them), and let it be $D$. Since $D$ measures $C$,
and $C$ measures $A$, $D$ thus also measures $A$. And  ($D$) also measures $B$. 
Thus, $D$ measures $A$ and $B$, which are prime to one another. The very thing is
impossible. Thus, some number does not measure the numbers $C$ and $B$.
Thus, $C$ and $B$ are prime to one another. (Which is) the very thing it was required to show.}
\end{Parallel}

%%%%%%
% Prop 7.24
%%%%%%
\pdfbookmark[1]{Proposition 7.24}{pdf7.24}
\begin{Parallel}{}{} 
\ParallelLText{
\begin{center}
{\large\ggn{24}.}
\end{center}\vspace*{-7pt}

\gr{>E`an d'uo >arijmo`i pr'oc tina >arijm`on pr~wtoi >~wsin, ka`i <o >ex
a\kern -.7pt >ut~wn gen'omenoc pr`oc t`on a\kern -.7pt >ut`on pr~wtoc >'estai.}\\

\epsfysize=2in
\centerline{\epsffile{Book07/fig24g.eps}}

\gr{D'uo g`ar >arijmo`i o<i A, B pr'oc tina >arijm`on t`on G pr~wtoi >'estwsan, 
ka`i <o A t`on B pollaplasi'asac t`on D poie'itw; l'egw, <'oti o<i G, D
pr~wtoi pr`oc >all'hlouc e>is'in.}

\gr{E>i g`ar m\kern -.7pt 'h e>isin o<i G, D pr~wtoi pr`oc >all'hlouc, metr'hsei
[tic] to`uc G, D >arijm'oc. metre'itw, ka`i >'estw <o E. ka`i >epe`i
o<i G, D pr~wtoi pr`oc >all'hlouc e>is'in, t`on d`e G metre~i tic
>arijm`oc <o E, o<i A, E >'ara pr~wtoi pr`oc >all'hlouc e>is'in. <os'akic
d`h <o E t`on D metre~i, tosa\kern -.7pt ~utai mon'adec >'estwsan >en t~w| Z;
ka`i <o Z >'ara t`on D metre~i kat`a  t`ac >en t~w|
 E mon'adac.
<o E >'ara t`on Z pollaplasi'asac t`on D pepo'ihken. >all`a m\kern -.7pt `hn ka`i
<o A t`on B  pollaplasi'asac t`on D pepo'ihken; >'isoc >'ara >est`in
<o <ek t~wn E, Z t~w| >ek t~wn A, B. >e`an d`e <o <up`o t~wn >'akrwn
>'isoc >~h| t~w| <up`o t~wn m'eswn, o<i t'essarec >arijmo`i >an'alog'on
e>isin; >'estin >'ara <wc <o E pr`oc t`on A, o<'utwc <o B pr`oc t`on Z. o<i d`e A, E pr~wtoi, o<i d`e pr~wtoi ka`i >el'aqistoi, o<i d`e >el'aqistoi
>arijmo`i t~wn t`on a\kern -.7pt >ut`on l'ogon >eq'ontwn a\kern -.7pt >uto~ic metro~usi 
to`uc t`on a\kern -.7pt >ut`on l'ogon
>'eqontac
>is'akic <'o te me'izwn t`on me'izona ka`i <o >el'asswn t`on >el'assona,
tout'estin <'o te <hgo'umenoc t`on <hgo'umenon ka`i <o <ep'omenoc
t`on <ep'omenon; <o E >'ara t`on B metre~i. metre~i d`e ka`i t`on G;
<o E >'ara to`uc B, G metre~i pr'wtouc >'ontac pr`oc >all'hlouc;
<'oper >est`in >ad'unaton. o>uk >'ara to`uc G, D >arijmo`uc
>arijm'oc tic metr'hsei. o<i G, D >'ara pr~wtoi pr`oc >all'hlouc
e>is'in; <'oper >'edei de~ixai.}}

\ParallelRText{
\begin{center}
{\large Proposition 24}
\end{center}

If two numbers are prime to some number, (then)
the number created from (multiplying) the former (two numbers) will also be prime to
the latter (number).

\epsfysize=2in
\centerline{\epsffile{Book07/fig24e.eps}}

For let $A$ and $B$ be two numbers (which are both) prime to some number $C$. And let $A$ make
$D$ (by) multiplying $B$. I say that $C$ and $D$ are prime to one another.

For if $C$ and $D$ are not prime to one another, (then) [some] number will
measure $C$ and $D$. Let it (so) measure them, and let it be $E$. And since
$C$ and $A$ are prime to one another, and some number $E$ measures $C$, $A$ and $E$
are thus prime to one another  [Prop.~7.23].
So as many times as $E$ measures $D$, so many units let there be in $F$.
Thus, $F$ also measures $D$ according to the units in $E$ [Prop.~7.16]. Thus, $E$ has made $D$ (by) multiplying
$F$ [Def.~7.15]. But, in fact, $A$ has also made
$D$ (by) multiplying $B$.
 Thus, the (number created) from (multiplying) $E$ and $F$ is equal to the  (number created) from (multiplying) $A$ and $B$. And if the (rectangle contained) by the (two)
outermost is equal to the (rectangle contained) by the middle (two then) the
four numbers are proportional [Prop.~6.15].
Thus, as $E$ is to $A$, so $B$  (is) to $F$. And $A$ and $E$ (are) prime (to one another).
And (numbers) prime (to one another) are also the least (of those numbers having the same ratio)
[Prop.~7.21]. And the least numbers of those
(numbers) having the same ratio  measure those (numbers)
having the same ratio as them an equal number of times,
the greater (measuring) the greater, and the lesser the lesser---that is
to say, the leading (measuring) the leading, and the following the
following [Prop.~7.20]. Thus, $E$
measures $B$.  And it also measures $C$. Thus, $E$ measures $B$ and $C$, which
are prime to one another. The very thing is impossible. Thus, some
number cannot measure the numbers $C$ and $D$. Thus, $C$ and
$D$ are prime to one another. (Which is) the very thing it was required
to show.}
\end{Parallel}

%%%%%%
% Prop 7.25
%%%%%%
\pdfbookmark[1]{Proposition 7.25}{pdf7.25}
\begin{Parallel}{}{} 
\ParallelLText{
\begin{center}
{\large \ggn{25}.}
\end{center}\vspace*{-7pt}

\gr{>E`an d'uo >arijmo`i pr~wtoi pr`oc >all'hlouc >~wsin, <o >ek to~u <en`oc
a\kern -.7pt >ut~wn gen'omenoc pr`oc t`on loip`on pr~wtoc >'estai.}

\gr{>'Estwsan d'uo >arijmo`i pr~wtoi pr`oc >all'hlouc o<i A, B, ka`i <o
A <eaut`on pollaplasi'asac t`on G poie'itw; l'egw, <'oti o<i B, G
pr~wtoi pr`oc >all`hlouc e>is'in.}\\

\epsfysize=2in
\centerline{\epsffile{Book07/fig25g.eps}}

\gr{Ke'isjw g`ar t~w| A >'isoc <o D. >epe`i o<i A, B pr~wtoi pr`oc >all'hlouc
e>is'in, >'isoc d`e <o A t~w| D, ka'i o<i D, B >'ara pr~wtoi pr`oc >all'hlouc
e>is'in. <ek'ateroc >'ara t~wn D, A pr`oc t`on B pr~wt'oc >estin;
 ka`i <o >ek t~wn D, A >'ara gen'omenoc pr`oc t`on B pr~wtoc
>'estai. <o d`e >ek t~wn D, A gen'omenoc >arijm'oc >estin <o G. o<i
G, B >'ara pr~wtoi pr`oc >all'hlouc e>is'in; <'oper >'edei
de~ixai.}}

\ParallelRText{
\begin{center}
{\large Proposition 25}
\end{center}

If two numbers are prime to one another, (then) the
number created from (squaring) one of them will be prime to
the remaining (number).


Let $A$ and $B$ be two numbers (which are) prime to one another. And let $A$ make $C$ (by) multiplying itself. I say that $B$ and $C$ are prime to one
another.

\epsfysize=2in
\centerline{\epsffile{Book07/fig25e.eps}}

For let $D$ be made equal to $A$. Since $A$ and $B$ are prime to one another, and
$A$ (is) equal to $D$, $D$ and $B$ are thus also prime to one another. Thus, $D$ and
$A$ are each prime to $B$. Thus, the (number) created from (multilying)
$D$ and $A$ will also be prime to $B$ [Prop.~7.24].
And $C$ is the number created from (multiplying) $D$ and $A$. Thus,
$C$ and $B$ are prime to one another. (Which is) the very thing it was required
to show.}
\end{Parallel}

%%%%%%
% Prop 7.26
%%%%%%
\pdfbookmark[1]{Proposition 7.26}{pdf7.26}
\begin{Parallel}{}{} 
\ParallelLText{
\begin{center}
{\large \ggn{26}.}
\end{center}\vspace*{-7pt}

\gr{>E`an d'uo >arijmo`i pr`oc d'uo >arijmo`uc >amf'oteroi pr`oc <ek'ateron
pr~wtoi >~wsin, ka`i o<i >ex a\kern -.7pt >ut~wn gen'omenoi pr~wtoi pr`oc
>all'hlouc >'esontai.}

\gr{D'uo g`ar >arijmo`i o<i A, B pr`oc d'uo >arijmo`uc to`uc G, D >amf'oteroi
pr`oc <ek'ateron pr~wtoi >'estwsan, ka`i <o m`en A t`on B pollaplasi'asac
t`on E poie'itw, <o d`e G t`on D pollaplasi'asac t`on Z poie'itw; l'egw, <'oti
o<i E, Z pr~wtoi pr`oc >all'hlouc e>is'in.}

\epsfysize=1in
\centerline{\epsffile{Book07/fig26g.eps}}

\gr{>Epe`i g`ar <ek'ateroc t~wn A, B pr`oc t`on G pr~wt'oc >estin, ka`i
<o >ek t~wn A, B >'ara gen'omenoc pr`oc t`on G pr~wtoc >'estai.
<o d`e >ek t~wn A, B gen'omen'oc >estin <o E; o<i E, G >'ara
pr~wtoi pr`oc >all'hlouc e>is'in. di`a t`a a\kern -.7pt >ut`a d`h ka`i o<i E, D
pr~wtoi pr`oc >all'hlouc e>is'in. <ek'ateroc >'ara t~wn G, D pr`oc t`on E
pr~wt'oc >estin. ka`i <o >ek t~wn G, D >'ara gen'omenoc pr`oc
t`on E pr~wtoc >'estai. <o d`e >ek t~wn G, D gen'omen'oc
>estin <o Z. o<i E, Z >'ara pr~wtoi pr`oc >all'hlouc e>is'in; <'oper
>'edei de~ixai.
}}

\ParallelRText{
\begin{center}
{\large Proposition 26}
\end{center}

If two numbers are both prime to each of two
numbers,  (then) the (numbers) created from (multiplying) them will
also be prime to one another.

For let two numbers, $A$ and $B$, both be prime to each of two numbers, $C$ and
$D$. And let $A$ make $E$ (by) multiplying $B$, and let $C$ make $F$ (by) multiplying
$D$. I say that $E$ and $F$ are prime to one another.

\epsfysize=1in
\centerline{\epsffile{Book07/fig26e.eps}}

For since $A$ and $B$ are each prime to $C$, the (number) created from (multiplying) $A$ and $B$ will thus also be prime to $C$ [Prop.~7.24]. And $E$ is the (number) created
from (multiplying) $A$ and $B$. Thus, $E$ and $C$ are prime to one another.
So, for the same (reasons), $E$ and $D$ are also prime to one another.
Thus, $C$ and $D$ are each prime to $E$. Thus, the (number) created from
(multiplying) $C$ and $D$ will also be prime to $E$ [Prop.~7.24]. And $F$ is the (number) created from
(multiplying) $C$ and $D$. Thus, $E$ and $F$ are prime to one another.
(Which is) the very thing it was required to show.}
\end{Parallel}

%%%%%%
% Prop 7.27
%%%%%%
\pdfbookmark[1]{Proposition 7.27}{pdf7.27}
\begin{Parallel}{}{} 
\ParallelLText{
\begin{center}
{\large\ggn{27}.}
\end{center}\vspace*{-7pt}

\gr{>E`an d'uo >arijmo`i pr~wtoi pr`oc >all'hlouc >~wsin, ka`i pollaplasi'asac
<ek'ateroc <eaut`on poi~h| tina, o<i gen'omenoi >ex a\kern -.7pt >ut~wn pr~wtoi
pr`oc >all'hlouc >'esontai, k>`an o<i >ex >arq~hc to`uc genom'enouc
pollaplasi'asantec poi~ws'i tinac, k>ake~inoi pr~wtoi pr`oc >all'hlouc
>'esontai [ka`i >ae`i per`i to`uc >'akrouc to~uto sumba'inei].}

\epsfysize=2in
\centerline{\epsffile{Book07/fig27g.eps}}

\gr{>'Estwsan d'uo >arijmo`i pr~wtoi pr`oc >all'hlouc o<i  A, B, ka`i <o A
<eaut`on m`en pollaplasi'asac t`on G poie'itw, t`on d`e G pollaplasi'asac
t`on D poie'itw, <o d`e B <eaut`on m`en pollaplasi'asac t`on E poie'itw,
t`on d`e E pollaplasi'asac t`on Z poie'itw; l'egw, <'oti o<'i te G, E ka`i
o<i D, Z pr~wtoi pr`oc >all'hlouc e>is'in.}

\gr{>Epe`i g`ar o<i A, B pr~wtoi pr`oc >all'hlouc e>is'in, ka`i <o A <eaut`on
pollaplasi'asac t`on G pepo'ihken, o<i G, B >'ara pr~wtoi pr`oc >all'hlouc
e>is'in. >epe`i o>~un o<i G, B pr~wtoi pr`oc >all'hlouc e>is'in, ka`i
<o B <eaut`on pollaplasi'asac t`on E pepo'ihken, o<i G, E >'ara pr~wtoi
pr`oc >all'hlouc e>is'in. p'alin, >epe`i o<i A, B pr~wtoi pr`oc
>all'hlouc e>is'in, ka`i <o B <eaut`on pollaplasi'asac t`on E pepo'ihken,
o<i A, E >'ara pr~wtoi pr`oc >all'hlouc e>is'in. >epe`i o>~un d'uo
>arijmo`i o<i A, G pr`oc d'uo >arijmo`uc to`uc B, E >amf'oteroi
pr`oc <ek'ateron pr~wto'i e>isin, ka`i <o >ek t~wn A, G >'ara gen'omenoc
pr`oc t`on >ek t~wn B, E pr~wt'oc >estin. ka'i >estin <o m`en >ek
t~wn A, G <o D, <o d`e >ek t~wn B, E <o Z. o<i D, Z >'ara pr~wtoi
pr`oc >all'hlouc e>is'in; <'oper >'edei de~ixai.}}

\ParallelRText{
\begin{center}
{\large Proposition 27}$^\dag$
\end{center}

If two numbers are prime to one another and each
makes some (number by) multiplying itself, (then) the numbers
created from them will be prime to one another, and if the original (numbers)
make some (more numbers by) multiplying the created (numbers, then) these 
will also be prime to one another [and this always happens with the
extremes].

\epsfysize=2in
\centerline{\epsffile{Book07/fig27e.eps}}

Let $A$ and $B$ be two numbers prime to one another, and let $A$ make
$C$ (by) multiplying itself, and let it make $D$ (by) multiplying $C$. 
And let $B$ make $E$ (by) multiplying itself, and let it make $F$ by multiplying $E$.
I say that $C$ and $E$, and $D$ and $F$, are prime to one another.

For since $A$ and $B$ are prime to one another, and $A$ has made $C$ (by)
multiplying itself, $C$ and $B$ are thus prime to one another [Prop.~7.25]. Therefore, since
$C$ and $B$ are prime to one another, and $B$ has made $E$ (by) multiplying itself,
$C$ and $E$ are thus prime to one another [Prop.~7.25].  Again, since $A$ and $B$ are prime to
one another, and $B$ has made $E$ (by) multiplying itself, $A$ and $E$ are thus
prime to one another [Prop.~7.25]. Therefore,
since the two numbers $A$ and $C$ are both prime to each of the two numbers
$B$ and $E$, the (number) created from (multiplying) $A$ and $C$ is thus
prime to the (number created) from (multiplying) $B$ and $E$ [Prop.~7.26]. And $D$ is the (number created)
from (multiplying) $A$ and $C$, and $F$ the (number created) from (multiplying)
$B$ and $E$. Thus, $D$ and $F$ are prime to one another. (Which is) the 
very thing it
was required to show.}
\end{Parallel}
{\footnotesize\noindent$^\dag$ In modern notation, this proposition
states that if $a$ is prime to $b$, then $a^2$ is also prime to $b^2$,
as well as $a^3$ to $b^3$, etc., where all symbols denote numbers.}

%%%%%%
% Prop 7.28
%%%%%%
\pdfbookmark[1]{Proposition 7.28}{pdf7.28}
\begin{Parallel}{}{} 
\ParallelLText{
\begin{center}
{\large \ggn{28}.}
\end{center}\vspace*{-7pt}

\gr{>E`an d'uo >arijmo`i pr~wtoi pr`oc >all'hlouc >~wsin, ka`i sunamf'oteroc
pr`oc <ek'ateron a\kern -.7pt >ut~wn pr~wtoc >'estai; ka`i >e`an sunamf'oteroc pr`oc <'ena tin`a
a\kern -.7pt >ut~wn pr~wtoc >~h|, ka`i o<i >ex >arq~hc >arijmo`i pr~wtoi pr`oc
>all'hlouc >'esontai.}

\epsfysize=0.7in
\centerline{\epsffile{Book07/fig28g.eps}}

\gr{Sugke'isjwsan g`ar d'uo >arijmo`i pr~wtoi pr`oc >all'hl\-ouc o<i AB, BG;
l'egw, <'oti ka`i sunamf'oteroc <o AG pr`oc <ek'ateron t~wn AB, BG pr~wt'oc
>estin.}

\gr{E>i g`ar m\kern -.7pt 'h e>isin o<i GA, AB pr~wtoi pr`oc >all'hlouc, metr'hsei tic
to`uc GA, AB >arijm'oc. metre'itw, ka`i >'estw <o D. >epe`i o>~un
<o D to`uc GA, AB metre~i, ka`i loip`on >'ara t`on BG metr'hsei.
metre~i d`e ka`i t`on BA; <o D >'ara to`uc AB, BG metre~i pr'wtouc
>'ontac pr`oc >all'hlouc; <'oper >est`in >ad'unaton. o>uk >'ara to`uc
GA, AB >arijmo`uc >arijm'oc tic metr'hsei; o<i GA, AB >'ara
pr~wtoi pr`oc >all'hlouc e>is'in. di`a t`a a\kern -.7pt >ut`a d`h ka`i o<i AG, GB
pr~wtoi pr`oc >all'hlouc e>is'in. <o GA >'ara pr`oc <ek'ateron t~wn
AB, BG pr~wt'oc >estin.}

\gr{>'Estwsan d`h p'alin o<i GA, AB pr~wtoi pr`oc >all'hlouc; l'egw,
<'oti ka`i o<i AB, BG pr~wtoi pr`oc >all'hlouc e>is'in.}

\gr{E>i g`ar m\kern -.7pt 'h e>isin o<i AB, BG pr~wtoi pr`oc >all'hlouc, metr'hsei
tic to`uc AB, BG >arijm'oc. metre'itw, ka`i >'estw <o D. ka`i >epe`i <o D
<ek'ateron t~wn AB, BG metre~i, ka`i <'olon >'ara t`on GA metr'hsei.
metre~i d`e ka`i t`on AB; <o D >'ara to`uc GA, AB metre~i pr'wtouc
>'ontac pr`oc >all'hlouc; <'oper >est`in >ad'unaton. o>uk >'ara to`uc
AB, BG >arijmo`uc >arijm'oc tic metr'hsei. o<i AB, BG >'ara pr~wtoi pr`oc
>all'hlouc e>is'in; <'oper >'edei de~ixai.}}

\ParallelRText{
\begin{center}
{\large Proposition 28}
\end{center}

If two numbers are prime to one another,
(then) their sum will also be prime to each of them. And if the sum (of two numbers) is
prime to any one of them, (then) the original numbers will also be prime
to one another.

\epsfysize=0.7in
\centerline{\epsffile{Book07/fig28e.eps}}

For let the two numbers, $AB$ and $BC$, (which are) prime to one another,
be laid down together. I say that their sum $AC$ is also prime to each of
$AB$ and $BC$.

For if $CA$ and $AB$ are not prime to one another, (then) some number
will measure $CA$ and $AB$. Let it (so) measure (them), and let it be $D$.
Therefore, since $D$ measures $CA$ and $AB$, it will thus also measure the remainder
$BC$. And it also measures $BA$. Thus, $D$ measures $AB$ and $BC$, which are
prime to one another. The very thing is impossible. Thus, some number cannot measure (both) the numbers $CA$ and $AB$. Thus, $CA$ and $AB$ are prime to
one another. So, for the same (reasons), $AC$ and $CB$ are also prime to one another. Thus, $CA$ is prime to each of $AB$ and $BC$.

So, again, let $CA$ and $AB$  be prime to one another. I say that $AB$ and $BC$
are also prime to one another.

For if $AB$ and $BC$ are not prime to one another, (then) some number will measure $AB$ and $BC$. Let it (so) measure (them), and let it be $D$. And since
$D$ measures each of $AB$ and $BC$, it will thus also measure the whole
of $CA$. And it also measures $AB$. Thus, $D$ measures $CA$ and $AB$, which are
prime to one another. The very thing is impossible. Thus, some number
cannot measure (both) the numbers $AB$ and $BC$. Thus, $AB$ and $BC$ are prime
to one another. (Which is) the very thing it was required to show.}
\end{Parallel}

%%%%%%
% Prop 7.29
%%%%%%
\pdfbookmark[1]{Proposition 7.29}{pdf7.29}
\begin{Parallel}{}{} 
\ParallelLText{
\begin{center}
{\large\ggn{29}.}
\end{center}\vspace*{-7pt}

\gr{<'Apac pr~wtoc >arijm`oc pr`oc <'apanta >arijm'on, <`on m\kern -.7pt `h metre~i, pr~wt'oc >estin.}

\gr{>'Estw pr~wtoc >arijm`oc <o A ka`i t`on B m\kern -.7pt `h metre'itw; l'egw, <'oti
o<i B, A pr~wtoi pr`oc >all'hlouc e>is'in.}

\epsfysize=0.8in
\centerline{\epsffile{Book07/fig29g.eps}}

\gr{E>i g`ar m\kern -.7pt 'h e>isin o<i B, A pr~wtoi pr`oc >all'hlouc, metr'hsei tic
a\kern -.7pt >uto`uc >arijm'oc. metre'itw <o G. >epe`i <o G t`on B metre~i, <o d`e A
t`on B o>u metre~i, <o G >'ara t~w| A o>'uk >estin <o a\kern -.7pt >ut'oc. ka`i
>epe`i <o G to`uc B, A metre~i, ka`i t`on A >'ara metre~i pr~wton
>'onta m\kern -.7pt `h >`wn a\kern -.7pt >ut~w| <o a\kern -.7pt >ut'oc; <'oper >est`in >ad'unaton. o>uk
>'ara to`uc B, A metr'hsei tic >arijm'oc. o<i A, B >'ara pr~wtoi
pr`oc >all'hlouc e>is'in; <'oper >'edei de~ixai.}}

\ParallelRText{
\begin{center}
{\large Proposition 29}
\end{center}

Every prime number is prime to every number which
it does not measure.

Let $A$ be a prime number, and let it not measure $B$. I say that $B$ and
$A$ are prime to one another.

\epsfysize=0.8in
\centerline{\epsffile{Book07/fig29e.eps}}

For if $B$ and $A$ are not prime to one another, (then) some number will measure
them. Let $C$ measure (them). Since $C$ measures $B$, and $A$ does not measure
$B$, $C$ is thus not the same as $A$.  And since $C$ measures $B$ and $A$, it thus also
measures $A$, which is prime, (despite) not being the same as it. The very thing is
impossible. Thus, some number cannot measure (both) $B$ and $A$.
Thus, $A$ and $B$ are prime to one another. (Which is) the very thing it
was required to show.}
\end{Parallel}

%%%%%%
% Prop 7.30
%%%%%%
\pdfbookmark[1]{Proposition 7.30}{pdf7.30}
\begin{Parallel}{}{} 
\ParallelLText{
\begin{center}
{\large \ggn{30}.}
\end{center}\vspace*{-7pt}

\gr{>E`an d'uo >arijmo`i pollaplasi'asantec >all'hlouc poi~ws'i tina, t`on d`e
gen'omenon >ex a\kern -.7pt >ut~wn metr~h| tic pr~wtoc >arijm'oc, ka`i <'ena
t~wn >ex >arq~hc metr'hsei.}\\

\epsfysize=1.6in
\centerline{\epsffile{Book07/fig30g.eps}}

\gr{D'uo g`ar >arijmo`i o<i A, B pollaplasi'asantec >all'hlouc t`on G
poie'itwsan, t`on d`e G metre'itw tic pr~wtoc >arijm`oc <o D;
l'egw, <'oti <o D <'ena t~wn A, B metre~i.}

\gr{T`on g`ar A m\kern -.7pt `h metre'itw; ka'i >esti pr~wtoc <o D;
o<i A, D >'ara pr~wtoi pr`oc >all'hlouc e>is'in. ka`i <os'akic <o D
t`on G metre~i, tosa\kern -.7pt ~utai mon'adec >'estwsan >en t~w| E. >epe`i o>~un
<o D t`on G metre~i kat`a t`ac >en t~w| E mon'adac, <o D >'ara t`on E
pollaplasi'asac t`on G pepo'ihken. >all`a m\kern -.7pt `hn ka`i <o A t`on B pollaplasi'asac
t`on G pepo'ihken; >'isoc >'ara >est`in <o >ek t~wn D, E t~w| >ek t~wn
A, B. >'estin >'ara <wc <o D pr`oc t`on A, o<'utwc <o B pr`oc t`on E.
o<i d`e D, A pr~wtoi, o<i d`e pr~wtoi ka`i >el'aqistoi, o<i d`e >el'aqistoi
metro~usi to`uc t`on a\kern -.7pt >ut`on l'ogon >'eqontac >is'akic <'o te me'izwn
t`on me'izona ka`i <o >el'asswn t`on >el'assona, tout'estin <'o te
<hgo'umenoc t`on <hgo'umenon ka`i <o >ep'omenoc t`on <ep'omenon;
<o D >'ara t`on B metre~i. <omo'iwc d`h de'ixomen, <'oti
ka`i >e`an t`on B m\kern -.7pt `h metr~h|, t`on A metr'hsei. <o D >'ara <'ena
t~wn A, B metre~i; <'oper >'edei de~ixai.}}

\ParallelRText{
\begin{center}
{\large Proposition 30}
\end{center}

If two numbers make some (number by) multiplying one another, and some prime number measures the number (so) created from them, (then) it will also measure one of the original (numbers).

\epsfysize=1.6in
\centerline{\epsffile{Book07/fig30e.eps}}

For let two numbers $A$ and $B$ make $C$ (by) multiplying one another, and
let some prime number $D$ measure $C$. I say that $D$ measures one of $A$ and
$B$.

For let it not measure $A$. And since $D$ is prime, $A$ and $D$ are thus prime to
one another [Prop.~7.29]. 
And as many times as $D$ measures $C$, so many units let there be in $E$.
Therefore, since $D$ measures $C$ according to the units $E$, $D$ has thus
made $C$ (by) multiplying $E$ [Def.~7.15].
But, in fact, $A$ has also made $C$ (by) multiplying $B$. Thus, the
(number created) from (multiplying) $D$ and $E$ is equal to
the (number created) from (multiplying) $A$ and $B$. Thus, as $D$ is to $A$,
so $B$ (is) to $E$ [Prop.~7.19]. And
$D$ and $A$ (are) prime (to one another), and (numbers) prime (to one another
are) also the least (of those numbers having the same ratio) [Prop.~7.21], and the least (numbers) measure
those (numbers) having the same ratio (as them) an equal number of times, the
greater (measuring) the greater, and the lesser the lesser---that is to
say, the leading (measuring) the leading, and the following the following
[Prop.~7.20]. Thus, $D$ measures $B$. So,
 similarly, we can also show that if ($D$) does not measure $B$,  (then) it will measure
 $A$. Thus, $D$ measures one of $A$ and $B$. (Which is) the very thing it was required to show.}
\end{Parallel}

%%%%%%
% Prop 7.31
%%%%%%
\pdfbookmark[1]{Proposition 7.31}{pdf7.31}
\begin{Parallel}{}{} 
\ParallelLText{
\begin{center}
{\large \ggn{31}.}
\end{center}\vspace*{-7pt}

\gr{<'Apac s'unjentoc >arijm`oc <up`o pr'wtou tin`oc >arijmo~u metre~itai.}

\gr{>'Estw s'unjentoc >arijm`oc <o A; l'egw, <'oti <o A <up`o pr'wtou tin`oc
>arijmo~u metre~itai.}

\epsfysize=0.9in
\centerline{\epsffile{Book07/fig31g.eps}}

\gr{>Epe`i g`ar s'unjet'oc >estin <o A, metr'hsei tic a\kern -.7pt >ut`on >arijm'oc.
metre'itw, ka`i >'estw <o B. ka`i e>i m`en pr~wt'oc >estin <o B, gegon`oc
>`an e>'ih t`o >epitaqj'en. e>i d`e s'unjetoc, metr'hsei tic a\kern -.7pt >ut`on >arijm'oc.
metre'itw, ka`i >'estw <o G. ka`i >epe`i <o G t`on B metre~i, <o d`e B
t`on A metre~i, ka`i <o G >'ara t`on A metre~i. ka`i e>i m`en pr~wt'oc
>estin <o G, gegon`oc >`an e>'ih t`o >epitaqj'en. e>i d`e 
s'unjetoc, metr'hsei tic a\kern -.7pt >ut`on >arijm'oc. toia\kern -.7pt 'uthc d`h
ginom'enhc 
>episk'eyewc lhfj'hseta'i tic pr~wtoc >arijm'oc, <`oc metr'hsei. e>i g`ar
o>u lhfj'hsetai, metr'hsousi t`on A >arijm`on >'apeiroi >arijmo'i, <~wn
<'eteroc <et'erou >el'asswn >est'in; <'oper >est`in >ad'unaton >en
>arijmo~ic. lhfj'hseta'i tic >'ara pr~wtoc >arijm'oc, <`oc metr'hsei
t`on pr`o <eauto~u, <`oc ka`i t`on A metr'hsei.}

 \gr{<'Apac >'ara s'unjentoc >arijm`oc <up`o pr'wtou tin`oc >arijmo~u metre~itai; <'oper >'edei de~ixai.}}

\ParallelRText{
\begin{center}
{\large Proposition 31}
\end{center}

Every composite number is measured by some
prime number.

Let $A$ be a composite number. I say that $A$ is measured by some prime
number.

\epsfysize=0.9in
\centerline{\epsffile{Book07/fig31e.eps}}

For since $A$ is composite, some number will measure it. Let it (so)
measure ($A$), and let it be $B$. And if $B$ is prime then that which was
prescribed has happened. And if ($B$ is) composite,  (then) some number
will measure it. Let it (so) measure ($B$), and let it be $C$. And since
$C$ measures $B$, and $B$ measures $A$, $C$ thus also measures $A$. And
if $C$ is prime then that which was prescribed has happened. 
And if ($C$ is) composite,  (then) some number will measure it. So, in this
manner of continued investigation, some prime number will be found
which will measure (the number preceding it, which will also
measure $A$). And if (such a number) cannot be found,  (then)
 an infinite (series of) numbers, each of which is less than the
preceding, will measure the number $A$. The very thing is impossible for numbers. Thus, some prime
number will (eventually) be found which will measure the (number) preceding it, 
which will also measure $A$.

Thus, every composite number is measured by some prime number.
(Which is) the very thing it was required to show.}
\end{Parallel}

%%%%%%
% Prop 7.32
%%%%%%
\pdfbookmark[1]{Proposition 7.32}{pdf7.32}
\begin{Parallel}{}{} 
\ParallelLText{
\begin{center}
{\large \ggn{32}.}
\end{center}\vspace*{-7pt}

\gr{<'Apac >arijm`oc >'htoi pr~wt'oc >estin >`h <up`o pr'wtou tin`oc 
>arijmo~u metre~itai.}

\epsfysize=0.2in
\centerline{\epsffile{Book07/fig32g.eps}}

\gr{>'Estw >arijm`oc <o A; l'egw, <'oti <o A >'htoi pr~wt'oc >estin >`h 
<up`o pr'wtou tin`oc >arijmo~u metre~itai.}

\gr{E>i m`en o>~un pr~wt'oc >estin <o A, gegon`oc >`an e>'ih t'o
>epitaqj'en. e>i d`e s'unjetoc, metr'hsei tic a\kern -.7pt >ut`on pr~wtoc
>arijm'oc.}

\gr{<'Apac >'ara >arijm`oc >'htoi pr~wt'oc >estin >`h <up`o pr'wtou 
tin`oc >arijmo~u metre~itai; <'oper >'edei de~ixai.}}

\ParallelRText{
\begin{center}
{\large Proposition 32}
\end{center}

Every number is either prime, or
is measured by some prime number.

\epsfysize=0.2in
\centerline{\epsffile{Book07/fig32e.eps}}

Let $A$ be a number. I say that $A$ is either prime, or is measured by some
prime number.

In fact, if $A$ is prime,  (then) that which was prescribed has happened.
And if (it is) composite,  (then)  some
prime number  will measure it [Prop.~7.31].

Thus, every number is either prime, or
is measured by some prime number. (Which is) the very thing it was required to
show.}
\end{Parallel}

%%%%%%
% Prop 7.33
%%%%%%
\pdfbookmark[1]{Proposition 7.33}{pdf7.33}
\begin{Parallel}{}{} 
\ParallelLText{
\begin{center}
{\large \ggn{33}.}
\end{center}\vspace*{-7pt}

\gr{>Arijm~wn doj'entwn <oposwno~un e<ure~in to`uc >elaq'isto\-uc t~wn
t`on a\kern -.7pt >ut`on l'ogon >eq'ontwn a\kern -.7pt >uto~ic.}

\gr{>'Estwsan o<i doj'entec <oposoio~un >arijmo`i o<i A, B, G; de~i d`h e<ure~in to`uc >elaq'istouc t~wn t`on a\kern -.7pt >ut`on l'ogon >eq'ontwn to~ic
A, B, G.}

\gr{O<i A, B, G g`ar >'htoi pr~wtoi pr`oc >all'hlouc e>is`in >`h o>'u. e>i m`en
o>~un o<i A, B, G pr~wtoi pr`oc >all'hlouc e>is'in, >el'aqisto'i e>isi t~wn
t`on a\kern -.7pt >ut`on l'ogon >eq'ontwn a\kern -.7pt >uto~ic.}\\

\epsfysize=1.3in
\centerline{\epsffile{Book07/fig33g.eps}}

\gr{E>i d`e o>'u, e>il'hfjw t~wn A, B, G t`o m'egiston koin`on m'etron <o D,
ka`i <os'akic <o D <'ekaston t~wn A, B, G metre~i, tosa\kern -.7pt ~utai mon'adec >'estwsan >en <ek'astw| t~wn E, Z, H. ka`i <'ekastoc >'ara t~wn E, Z, H <'ekaston t~wn A, B, G metre~i kat`a t`ac >en t~w| D mon'adac. o<i
E, Z, H >'ara to`uc A, B, G >is'akic metro~usin; o<i E, Z, H >'ara to~ic
A, B, G >en t~w| a\kern -.7pt >ut~w| l'ogw| e>is'in. l'egw d'h, <'oti ka`i >el'aqistoi.
e>i g`ar m\kern -.7pt 'h e>isin o<i E, Z, H >el'aqistoi t~wn t`on a\kern -.7pt >ut`on l'ogon >eq'ontwn to~ic A, B, G, >'esontai [tinec] t~wn E, Z, H >el'assonec
>arijmo`i >en t~w| a\kern -.7pt >ut~w| l'ogw| >'ontec to~ic A, B, G. >'estwsan o<i
J, K, L; >is'akic >'ara <o J t`on A metre~i ka`i <ek'ateroc t~wn K, L
<ek'ateron t~wn B, G. <os'akic d`e <o J t`on A metre~i, tosa\kern -.7pt ~utai mon'adec
>'estwsan >en t~w| M; ka`i <ek'ateroc >'ara t~wn K, L <ek'ateron t~wn
B, G metre~i kat`a t`ac >en t~w| M mon'adac. ka`i >epe`i <o J t`on A metre~i
kat`a t`ac >en t~w| M mon'adac, ka`i <o M >'ara t`on A metre~i kat`a t`ac
>en t~w| J mon'adac. di`a t`a a\kern -.7pt >ut`a d`h <o M ka`i <ek'ateron t~wn B, G
metre~i kat`a t`ac >en <ekat'erw| t~wn K, L mon'adac; <o M >'ara to`uc
A, B, G metre~i. ka`i >epe`i <o J t`on A metre~i kat`a t`ac >en t~w|
M mon'adac, <o J >'ara t`on M pollaplasi'asac t`on A pepo'ihken.
di`a t`a a\kern -.7pt >ut`a d`h ka`i <o E t`on D pollaplasi'asac t`on A pepo'ihken.
>'isoc >'ara >est`in <o >ek t~wn E, D t~w| >ek t~wn J, M.
>'estin >'ara <wc <o E pr`oc t`on J, o<'utwc <o M pr`oc t`on D. me'izwn
d`e <o E to~u J; me'izwn >'ara ka`i <o M to~u D. ka`i metre~i to`uc
A, B, G; <'oper >est`in >ad'unaton; <up'okeitai g`ar <o D t~wn A, B, G
t`o m'egiston koin`on m'etron. o>uk >'ara >'esonta'i tinec t~wn E, Z, H
>el'assonec >arijmo`i >en t~w| a\kern -.7pt >ut~w| l'ogw| >'ontec to~ic A, B, G. o<i
E, Z, H >'ara >el'aqisto'i e>isi t~wn t`on a\kern -.7pt >ut`on l'ogon >eq'ontwn
to~ic A, B, G; <'oper >'edei de~ixai.}}

\ParallelRText{
\begin{center}
{\large Proposition 33}
\end{center}

To  find the least of those (numbers) having the
same ratio as any given multitude of numbers.

Let $A$, $B$, and $C$ be any given multitude of numbers. So it is required to
find the least of those (numbers) having the same ratio as $A$, $B$, and $C$.

For $A$, $B$, and $C$ are either prime to one another, or not. In fact,
if $A$, $B$, and $C$ are prime to one another, (then) they are the least of those (numbers)
having the same ratio as them [Prop.~7.22].

\epsfysize=1.3in
\centerline{\epsffile{Book07/fig33e.eps}}

And if not, let the greatest common measure, $D$, of $A$, $B$, and $C$ have be taken
[Prop.~7.3]. And as many times as $D$
measures $A$, $B$,  $C$, so many units let there be in 
$E$, $F$, $G$, respectively. And thus  $E$, $F$, $G$ measure 
$A$, $B$, $C$, respectively, according to the units in $D$ [Prop.~7.15]. Thus, $E$, $F$, $G$ measure
$A$, $B$, $C$ (respectively) an equal number of times. Thus, $E$, $F$,  $G$
are in the same ratio as $A$, $B$, $C$ (respectively) [Def.~7.20]. So I say that (they
are) also the least (of those numbers having the same ratio as $A$, $B$, $C$).
For if $E$, $F$,  $G$ are not the least of those (numbers) having the same ratio
as $A$, $B$,  $C$ (respectively, then) there will be [some] numbers less than $E$, $F$,  $G$
which are in the same ratio as $A$, $B$, $C$ (respectively). Let them be $H$, $K$,  $L$.
Thus, $H$ measures $A$ the same number of times that 
$K$,  $L$ also measure $B$,  $C$, respectively. And as many
times as $H$ measures $A$, so many units let there be in $M$. Thus,  $K$, $L$
measure $B$, $C$, respectively, according to the units in $M$.
And since $H$ measures $A$ according to the units in $M$, $M$ thus also measures
$A$ according to the units in $H$ [Prop.~7.15].
So, for the same (reasons), $M$ also measures  $B$, $C$ according to
the units in  $K$, $L$, respectively. Thus, $M$ measures $A$, $B$, and $C$. And since
$H$ measures $A$ according to the units in $M$, $H$ has thus made $A$ (by)
multiplying $M$. So, for the same (reasons), $E$ has also made $A$ (by)
multiplying $D$. Thus, the (number created) from (multiplying) $E$ and $D$
is equal to the (number created) from (multiplying) $H$ and $M$. 
Thus, as $E$ (is) to $H$, so $M$ (is) to $D$ [Prop.~7.19]. And $E$ (is) greater than $H$. Thus,
$M$ (is) also greater than $D$ [Prop.~5.13].
And ($M$) measures $A$, $B$, and $C$.  The very thing is impossible.
For $D$ was assumed (to be) the greatest common measure of $A$, $B$, and $C$. 
Thus, there cannot be any numbers less than $E$, $F$, $G$ which are
in the same ratio as $A$, $B$,  $C$ (respectively). Thus, $E$, $F$,  $G$ are the
least of (those numbers) having the same ratio as $A$, $B$,  $C$ (respectively).
(Which is) the very thing it was required to show.\\~\\}
\end{Parallel}

%%%%%%
% Prop 7.34
%%%%%%
\pdfbookmark[1]{Proposition 7.34}{pdf7.34}
\begin{Parallel}{}{} 
\ParallelLText{
\begin{center}
{\large \ggn{34}.}
\end{center}\vspace*{-7pt}

\gr{D'uo >arijm~wn doj'entwn e<ure~in, <`on >el'aqiston metro~usin
>arijm'on.}

\gr{>'Estwsan o<i doj'entec d'uo >arijmo`i o<i A, B; de~i d`h e<ure~in, <`on
>el'aqiston metro~usin >arijm'on.}

\epsfysize=1.2in
\centerline{\epsffile{Book07/fig34g.eps}}

\gr{O<i A, B g`ar >'htoi pr~wtoi pr`oc >all'hlouc e>is`in >`h o>'u. >'estwsan
pr'oteron o<i A, B pr~wtoi pr`oc >all'hlouc, ka`i <o A t`on B pollaplasi'asac
t`on G poie'itw; ka`i <o B >'ara t`on A pollaplasi'asac t`on G pepo'ihken.
o<i A, B >'ara t`on G metro~usin. l'egw d'h, <'oti ka`i >el'aqiston.
e>i g`ar m\kern -.7pt 'h, metr'hsous'i tina >arijm`on o<i A, B  >el'assona >'onta to~u G.
metre'itwsan t`on D. ka`i <os'akic <o A t`on D metre~i, tosa\kern -.7pt ~utai
mon'adec >'estwsan >en t~w| E, <os'akic d`e <o B t`on D metre~i,
tosa\kern -.7pt ~utai mon'adec >'estwsan >en t~w| Z. <o m`en A >'ara t`on E pollaplasi'asac t`on D pepo'ihken, <o d`e B t`on Z pollaplasi'asac t`on
D pepo'ihken; >'isoc >'ara >est`in <o >ek t~wn A, E t~w| >ek t~wn
B, Z. >'estin >'ara <wc <o A pr`oc t`on B, o<'utwc <o Z pr`oc
t`on E. o<i d`e A, B pr~wtoi, o<i d`e pr~wtoi ka`i >el'aqistoi, o<i d`e
>el'aqistoi metro~usi to`uc t`on a\kern -.7pt >ut`on l'ogon >'eqontac >is'akic <'o te 
me'izwn t`on me'izona ka`i <o >el'asswn t`on >el'assona; <o B >'ara
t`on E metre~i, <wc <ep'omenoc <ep'omenon. ka`i >epe`i <o A to`uc
B, E pollaplasi'asac to`uc G, D pepo'ihken, >'estin >'ara <wc <o B pr`oc
t`on E, o<'utwc <o G pr`oc t`on D. metre~i d`e <o B t`on E; metre~i >'ara
ka`i <o G t`on D <o me'izwn t`on >el'assona; <'oper >est`in >ad'unaton.
o>uk >'ara o<i A, B metro~us'i tina >arijm`on >el'assona >'onta
to~u G. <o G >'ara >el'aqistoc >`wn <up`o t~wn A, B metre~itai.}\\~\\~\\~\\~\\

\epsfysize=1.5in
\centerline{\epsffile{Book07/fig34ag.eps}}

\gr{m\kern -.7pt `h >'estwsan d`h o<i A, B pr~wtoi pr`oc >all'hlouc, ka`i e>il'hfjwsan
>el'aqistoi >arijmo`i t~wn t`on a\kern -.7pt >ut`on l'ogon >eq'ontwn to~ic
A, B o<i Z, E; >'isoc >'ara >est`in <o >ek t~wn A, E t~w| >ek t~wn B, Z.
ka`i <o A t`on E pollaplasi'asac t`on G poie'itw; ka`i <o B >'ara t`on Z
pollaplasi'asac t`on G pepo'ihken; o<i A, B >'ara t`on G metro~usin.
l'egw d'h, <'oti ka`i >el'aqiston. e>i g`ar m\kern -.7pt 'h, metr'hsous'i tina >arijm`on
o<i A, B >el'assona >'onta to~u G. metre'itwsan t`on D. ka`i <os'akic
m`en <o A t`on D
metre~i, tosa\kern -.7pt ~utai mon'adec >'estwsan
>en t~w| H, <os'akic d`e <o B t`on D metre~i, tosa\kern -.7pt ~utai mon'adec >'estwsan
>en t~w| J. <o m`en A >'ara t`on H pollaplasi'asac t`on D pepo'ihken,
<o d`e B t`on J pollaplasi'asac t`on D pepo'ihken. >'isoc >'ara >est`in
<o >ek t~wn A, H t~w| >ek t~wn B, J; >'estin >'ara <wc <o A pr`oc t`on
B, o<'utwc <o J pr`oc t`on H. <wc d`e <o A pr`oc t`on B, o<'utwc <o Z
pr`oc t`on E; ka`i <wc >'ara <o Z pr`oc t`on E, o<'utwc <o J pr`oc
t`on H. o<i d`e Z, E >el'aqistoi, o<i d`e >el'aqistoi metro~usi to`uc t`on
a\kern -.7pt >ut`on l'ogon >'eqontac >is'akic <'o te me'izwn t`on me'izona ka`i
<o >el'asswn t`on >el'assona; <o E >'ara t`on H metre~i. ka`i >epe`i
<o A to`uc E, H pollaplasi'asac to`uc G, D pepo'ihken, >'estin >'ara
<wc <o E pr`oc t`on H, o<'utwc <o G pr`oc t`on D. <o d`e E t`on H
metre~i; ka`i <o G >'ara t`on D metre~i <o me'izwn t`on >el'assona; <'oper
>est`in >ad'unaton. o>uk >'ara o<i A, B metr'hsous'i tina >arijm`on
>el'assona >'onta to~u G. <o G >'ara >el'aqistoc >`wn <up`o
t~wn A, B metre~itai; <'oper >'epei de~ixai.}}

\ParallelRText{
\begin{center}
{\large Proposition 34}
\end{center}

To find the least
number which two given numbers  (both) measure.

Let $A$ and $B$ be the two given numbers. So it is required to find
the least number which they (both) measure.

\epsfysize=1.2in
\centerline{\epsffile{Book07/fig34e.eps}}

For $A$ and $B$ are either prime to one another, or not. Let them, first of all,
be prime to one another. And let $A$ make $C$ (by) multiplying $B$. Thus,
$B$ has also made $C$ (by) multiplying $A$ [Prop.~7.16]. Thus, $A$ and $B$ (both) measure $C$. So
I say that ($C$) is also the least (number which they both measure).
For if not, $A$ and $B$ will (both) measure some (other) number which is less than $C$. Let them (both) measure $D$ (which is less than $C$). And as many times
as $A$ measures $D$, so many units let there be in $E$. And as many times as $B$
measures $D$, so many units let there be in $F$. Thus, $A$ has made $D$ (by)
multiplying $E$, and $B$ has made $D$ (by) multiplying $F$. Thus, the
(number created) from (multiplying) $A$ and $E$ is equal to the (number created)
from (multiplying) $B$ and $F$. Thus, as $A$ (is) to $B$, so $F$ (is) to $E$ [Prop.~7.19]. And $A$ and $B$ are prime (to one another), and prime (numbers) are the least (of those numbers having
the same ratio)  [Prop.~7.21], and the
least (numbers) measure those (numbers) having the same
ratio (as them) an equal number of times, the greater (measuring) the greater,
and
the lesser the lesser [Prop.~7.20].
Thus, $B$ measures $E$, as the following (number measuring) the following.
And since $A$ has made $C$ and $D$ (by) multiplying $B$ and $E$ (respectively),
thus as $B$ is to $E$, so $C$ (is) to $D$ [Prop.~7.17]. 
And $B$ measures $E$. Thus, $C$ also measures $D$, the greater (measuring) the
lesser. The very thing is impossible. Thus, $A$ and $B$ do not (both) measure some
number which is less than $C$. Thus, $C$ is the least (number) which is
measured by (both) $A$ and $B$.

\epsfysize=1.5in
\centerline{\epsffile{Book07/fig34ae.eps}}

So let $A$ and $B$ be not prime to one another. And let the least numbers,
$F$ and $E$, be taken having the same ratio as $A$ and $B$ (respectively)
[Prop.~7.33]. Thus, the (number created)
from (multiplying) $A$ and $E$ is equal to the (number created) from (multiplying) $B$ and $F$ [Prop.~7.19].
And let $A$ make $C$ (by) multiplying $E$. Thus, $B$ has also made $C$
(by) multiplying $F$. Thus, $A$ and $B$ (both) measure $C$. So I say that
($C$) is also the least (number which they both measure). For if not, 
$A$ and $B$ will (both) measure some number which is less than $C$. Let them
(both) measure $D$ 
(which is less than $C$). And as many times as $A$ measures $D$, so many units
let there be in $G$.
And as many times as $B$ measures $D$,
so many units let there be in $H$. Thus, $A$ has made $D$ (by) multiplying $G$, and $B$
has made $D$ (by) multiplying $H$. Thus, the (number created) from (multiplying) $A$ and $G$ is equal to the (number created)
from (multiplying) $B$ and $H$. Thus, as $A$ is to $B$, so $H$ (is) to $G$ [Prop.~7.19]. And as $A$ (is) to $B$, so $F$ (is) to $E$.
Thus, also, as $F$ (is) to $E$, so $H$ (is) to $G$. And $F$ and $E$ are the least
(numbers having the same ratio as $A$ and $B$), and the least (numbers) measure
those (numbers) having the same ratio an equal number of times, the
greater (measuring) the greater, and the lesser the lesser [Prop.~7.20]. Thus, $E$ measures $G$. And since $A$
has made $C$ and $D$ (by) multiplying $E$ and $G$ (respectively), thus as
$E$ is to $G$, so $C$ (is) to $D$ [Prop.~7.17].
And $E$ measures $G$. Thus, $C$ also measures $D$, the greater (measuring) the
lesser. The very thing is impossible. Thus, $A$ and $B$ do not (both) measure some (number) which is less than $C$. Thus, $C$ (is) the least (number) which is measured by (both) $A$
and $B$. (Which is) the very thing it was required to show.}
\end{Parallel}

%%%%%%
% Prop 7.35
%%%%%%
\pdfbookmark[1]{Proposition 7.35}{pdf7.35}
\begin{Parallel}{}{} 
\ParallelLText{
\begin{center}
{\large \ggn{35}.}
\end{center}\vspace*{-7pt}

\gr{>E`an d'uo >arijmo`i >arijm'on tina metr~wsin, ka`i <o >el'aqistoc
<up> a\kern -.7pt >ut~wn metro'umenoc t`on a\kern -.7pt >ut`on metr'hsei.}\\

\epsfysize=1in
\centerline{\epsffile{Book07/fig35g.eps}}

\gr{D'uo g`ar >arijmo`i o<i A, B >arijm'on tina t`on GD metre'itwsan, >el'aqiston
d`e t`on E; l'egw, <'oti ka`i <o E t`on GD metre~i.}

\gr{E>i g`ar o>u metre~i <o E t`on GD, <o E t`on DZ metr~wn
leip'etw <eauto~u >el'assona t`on GZ. ka`i >epe`i o<i  A, B t`on E
metro~usin, <o d`e E t`on DZ metre~i, ka`i o<i A, B >'ara t`on DZ
metr'hsousin. metro~usi d`e ka`i <'olon t`on GD; ka`i
loip`on >'ara t`on GZ metr'hsousin >el'assona >'onta to~u E;
<'oper >est`in >ad'unaton. o>uk >'ara o>u metre~i <o E t`on GD;
metre~i >'ara; <'oper >'edei de~ixai.}}

\ParallelRText{
\begin{center}
{\large Proposition 35}
\end{center}

If two numbers (both) measure some number, 
(then) the least (number) measured by them will also measure the same (number).

\epsfysize=1in
\centerline{\epsffile{Book07/fig35e.eps}}

For let two numbers, $A$ and $B$, (both) measure some number $CD$, and
(let) $E$  (be the) least (number measured by both $A$ and $B$). I say that $E$
also measures $CD$.

For if $E$ does not measure $CD$,  (then) let $E$ leave $CF$ less than
itself (in) measuring $DF$. And since $A$ and $B$ (both) measure $E$, and
$E$ measures $DF$, $A$ and $B$ will thus also measure $DF$. And ($A$ and $B$) also
measure the whole of $CD$. Thus, they will also measure the
remainder $CF$, which is less than $E$. The very thing is impossible. 
Thus, $E$ cannot not measure $CD$. Thus, ($E$) measures ($CD$). 
(Which is) the very thing it was required to show.}
\end{Parallel}

%%%%%%
% Prop 7.36
%%%%%%
\pdfbookmark[1]{Proposition 7.36}{pdf7.36}
\begin{Parallel}{}{} 
\ParallelLText{
\begin{center}
{\large \ggn{36}.}
\end{center}\vspace*{-7pt}

\gr{Tri~wn >arijm~wn doj'entwn e<ure~in, <`on >el'aqiston metro~usin
>arijm'on.}

\gr{>'Estwsan o<i doj'entec tre~ic >arijmo`i o<i A, B, G; de~i d`h e<ure~in,
<`on >el'aqiston metro~usin >arijm'on.}

\epsfysize=1.8in
\centerline{\epsffile{Book07/fig36g.eps}}

\gr{E>il'hfjw g`ar <up`o d'uo t~wn A, B >el'aqistoc metro'umenoc <o D. <o d`h
G t`on D >'htoi metre~i >`h o>u metre~i. metre'itw pr'oteron. metro~usi
d`e ka`i o<i A, B t`on D; o<i A, B, G >'ara t`on D metro~usin. l'egw
d'h, <'oti ka`i >el'aqiston. e>i g`ar m\kern -.7pt 'h, metr'hsousin [tina] >arijm`on
o<i A, B, G >el'assona >'onta to~u D. metre'itwsan t`on E. >epe`i
o<i A, B, G t`on E metro~usin, ka`i o<i A, B >'ara t`on E metro~usin.
ka`i <o >el'aqistoc >'ara <up`o t~wn A, B metro'umenoc [t`on E] metr'hsei.
>el'aqistoc d`e <up`o t~wn A, B metro'umen'oc >estin <o D; <o D >'ara
t`on E metr'hsei <o me'izwn t`on >el'assona; <'oper >est`in >ad'unaton.
o>uk >'ara o<i A, B, G metr'hsous'i tina >arijm`on >el'assona >'onta
to~u D; o<i A, B, G >'ara >el'aqiston t`on D metro~usin.}

\gr{m\kern -.7pt `h metre'itw d`h p'alin <o G t`on D, ka`i e>il'hfjw <up`o t~wn G, D
>el'aqistoc metro'umenoc >arijm`oc <o E. >epe`i o<i A, B t`on
D metro~usin, <o d`e D t`on E metre~i, ka`i o<i A, B >'ara t`on E metro~usin.
metre~i d`e ka`i <o G [t`on E; ka`i] o<i A, B, G >'ara t`on E metro~usin.
l'egw d'h, <'oti ka`i >el'aqiston. e>i g`ar m\kern -.7pt 'h, metr'hsous'i tina o<i
A, B, G
>el'assona >'onta to~u E. metre'itwsan t`on Z. >epe`i
o<i A, B, G t`on Z metro~usin, ka`i o<i A, B >'ara t`on Z metro~usin; ka`i
<o >el'aqistoc >'ara <up`o t~wn A, B metro'umenoc t`on Z metr'hsei.
>el'aqistoc d`e <up`o t~wn A, B
metro'umen'oc >estin <o D; <o D
>'ara t`on Z metre~i. 
metre~i d`e ka`i <o G t`on Z; o<i D, G >'ara t`on
Z metro~usin; <'wste ka`i <o >el'aqistoc <up`o t~wn D, G metro'umenoc
t`on Z metr'hsei. <o d`e >el'aqistoc <up`o t~wn G, D metro'umen'oc
>estin <o E; <o E >'ara t`on Z metre~i <o me'izwn t`on >el'assona;
<'oper >est`in >ad'unaton. o>uk >'ara o<i A, B, G metr'hsous'i
tina >arijm`on >el'assona >'onta to~u E. <o E >'ara >el'aqistoc >`wn <up`o
t~wn A, B, G metre~itai; <'oper >'edei de~ixai.}}

\ParallelRText{
\begin{center}
{\large Proposition 36}
\end{center}

To  find the least
number which  three given numbers (all) measure.

Let $A$, $B$, and $C$ be the three given numbers. So it is required to
find the least number which they (all) measure.

\epsfysize=1.8in
\centerline{\epsffile{Book07/fig36e.eps}}

For let the least (number), $D$, measured by the two (numbers) $A$ and $B$
be taken [Prop. 7.34].
So $C$ either measures, or does not measure, $D$. Let it, first of
all, measure ($D$). And $A$ and $B$ also measure $D$. Thus, $A$, $B$, and $C$
(all) measure $D$. So I say that ($D$ is) also the least (number measured by
$A$, $B$, and $C$). For if not, $A$, $B$, and $C$ will (all) measure [some]
number which is less than $D$. Let them measure $E$ (which is less than $D$).
Since $A$, $B$, and $C$ (all) measure $E$, (then) $A$ and $B$ thus also measure $E$.
Thus, the least (number) measured by $A$ and $B$ will also
measure [$E$] [Prop.~7.35]. And $D$
is the least (number) measured by $A$ and $B$. Thus, $D$ will measure
$E$, the greater (measuring) the lesser. The very thing is impossible.
Thus, $A$, $B$, and $C$ cannot (all) measure some number which is less than $D$.
Thus, $A$, $B$, and $C$ (all) measure the least (number) $D$.

So, again, let $C$ not measure $D$. And let the least number, $E$, measured
by $C$ and $D$ be taken [Prop.~7.34].
Since $A$ and $B$ measure $D$, and $D$ measures $E$, $A$ and $B$ thus also
measure $E$. And $C$ also measures [$E$]. Thus, $A$, $B$, and $C$ [also] measure
$E$. So I say that ($E$ is) also the least (number measured by $A$, $B$, and $C$). For if not, $A$, $B$, and $C$ will (all) measure some (number) which
is less than $E$. Let them measure $F$ (which is less than $E$). Since
$A$, $B$, and $C$ (all) measure $F$, $A$ and $B$ thus also measure $F$. Thus,
the least (number) measured by $A$ and $B$ will also measure $F$  [Prop.~7.35]. And $D$ is the least (number) measured
by $A$ and $B$. Thus, $D$ measures $F$.  And $C$ also measures $F$. Thus, $D$ and $C$
(both) measure $F$. Hence, the least (number) measured by $D$ and $C$
will also measure $F$ [Prop.~7.35].
And $E$ is the least (number) measured by $C$ and $D$. Thus, $E$ measures $F$,
the greater (measuring) the lesser. The very thing is impossible.
Thus, $A$, $B$, and $C$ cannot measure some number which is less than $E$.
Thus, $E$ (is) the least (number) which is measured by $A$, $B$, and $C$.
(Which is) the very thing it was required to show.}
\end{Parallel}

%%%%%%
% Prop 7.37
%%%%%%
\pdfbookmark[1]{Proposition 7.37}{pdf7.37}
\begin{Parallel}{}{} 
\ParallelLText{
\begin{center}
{\large \ggn{37}.}
\end{center}\vspace*{-7pt}

\gr{>E`an  >arijm`oc <up'o tinoc >arijmo~u metr~htai, <o metro'umenoc
<om'wnumon m'eroc <'exei t~w| metro~unti.}\\

\epsfysize=1.2in
\centerline{\epsffile{Book07/fig37g.eps}}

\gr{>Arijm`oc g'ar <o A <up'o tinoc >arijmo~u to~u B metre'isjw; l'egw, <'oti
<o A <om'wnumon m'eroc >'eqei t~w| B.}

\gr{<Os'akic g`ar <o B t`on A metre~i, tosa\kern -.7pt ~utai mon'adec >'estwsan >en
t~w| G. >epe`i <o B t`on A metre~i kat`a t`ac >en t~w| G mon'adac, metre~i
d`e ka`i <h D mon`ac t`on G >arijm`on kat`a t`ac >en a\kern -.7pt >ut~w| mon'adac,
>is'akic >'ara <h D mon`ac t`on G >arijm`on metre~i ka`i <o
B t`on A. >enall`ax >'ara >is'akic <h D mon`ac t`on B >arijm`on
metre~i ka`i <o G t`on A; <`o >'ara m'eroc >est`in <h D mon`ac to~u
B >arijmo~u, t`o a\kern -.7pt >ut`o m'eroc >est`i ka`i <o G to~u A. <h d`e D
mon`ac to~u B >arijmo~u m'eroc >est`in <om'wnumon a\kern -.7pt >ut~w|;
ka`i <o G >'ara to~u A m'eroc >est`in <om'wnumon t~w| B.
<'wste <o A m'eroc >'eqei t`on G <om'wnumon >'onta t~w| B;
<'oper >'edei de~ixai.}}

\ParallelRText{
\begin{center}
{\large Proposition 37}
\end{center}

If a number is measured by some number,  (then)
the (number) measured will have a part called the same as the
measuring (number).

\epsfysize=1.2in
\centerline{\epsffile{Book07/fig37e.eps}}

For let the number $A$ be measured by some number $B$. I say that $A$
has a part called the same as $B$.

For as many times as $B$ measures $A$, so many units let there be in $C$.
Since $B$ measures $A$ according to the units in $C$, and the unit $D$ also
measures $C$ according to the units in it,   the unit
$D$ thus measures the number $C$ as many times as $B$ (measures) $A$. Thus, alternately, the unit $D$ measures the number $B$ as many times as $C$  (measures) $A$  [Prop.~7.15].
Thus, which(ever) part the unit $D$ is of the number $B$, $C$ is also the same
part of $A$. And the unit $D$ is a part of the number $B$ called the same as it
({\rm i.e.}, a $B$th part).
Thus, $C$ is also a part of $A$ called the same as $B$ ({\rm i.e.}, $C$ is the $B$th part of $A$). Hence, $A$ has a part $C$
which is called the same  as $B$ ({\rm i.e.}, $A$ has a $B$th part). (Which is) the very thing it was required to show.}
\end{Parallel}

%%%%%%
% Prop 7.38
%%%%%%
\pdfbookmark[1]{Proposition 7.38}{pdf7.38}
\begin{Parallel}{}{} 
\ParallelLText{
\begin{center}
{\large \ggn{38}.}
\end{center}\vspace*{-7pt}

\gr{>E`an >arijmoc m'eroc >'eqh| <otio~un, <up`o <omwn'umou >arijmo~u
metrhj'hsetai t~w| m'erei.}

\epsfysize=1.2in
\centerline{\epsffile{Book07/fig38g.eps}}

\gr{>Arijm`oc g`ar <o A m'eroc >eq'etw <otio~un t`on B,
ka`i t~w| B m'erei <om'wnumoc >'estw [>arijm`oc] <o G; l'egw,
<'oti <o G t`on A metre~i.}

\gr{>Epe`i g`ar <o B to~u A m'eroc >est`in <om'wnumon t~w| G, >'esti
d`e ka`i <h D mon`ac to~u G m'eroc <om'wnumon a\kern -.7pt >ut~w|, <`o
>'ara m'eroc >est`in <h D mon`ac to~u G >arijmo~u, t`o a\kern -.7pt >ut`o m'eroc
>est`i ka`i <o B to~u A; >is'akic >'ara <h D mon`ac t`on G >arijm`on
metre~i ka`i <o B t`on A. >enall`ax >'ara >is'akic <h D mon`ac t`on B
>arijm`on metre~i ka`i <o G t`on A. <o G >'ara t`on A metre~i;
<'oper >'edei de`ixai.}}

\ParallelRText{
\begin{center}
{\large Proposition 38}
\end{center}

If a number has any part whatever,  (then) it will
be measured by a number called the same as the part.

\epsfysize=1.2in
\centerline{\epsffile{Book07/fig38e.eps}}

For let the number $A$ have any part whatever, $B$. And let the [number]
$C$ be called the same as the part $B$ ({\rm i.e.}, $B$ is the $C$th part of $A$). I say that $C$ measures $A$.

For since $B$ is a part of $A$ called the same as $C$,  and the unit $D$ is also
a part of $C$ called the same as it ({\rm i.e.}, $D$ is the $C$th part of $C$), thus which(ever) part the unit $D$ is of the
number $C$, $B$ is also the same part of $A$. Thus, 
the unit $D$ measures the number $C$ as many times as $B$  (measures)
$A$. Thus, alternately, the unit $D$ measures the number $B$
as many times as $C$  (measures) $A$ [Prop.~7.15]. Thus, $C$ measures $A$. (Which is)
the very thing it was required to show.\\}
\end{Parallel}

%%%%%%
% Prop 7.39
%%%%%%
\pdfbookmark[1]{Proposition 7.39}{pdf7.39}
\begin{Parallel}{}{} 
\ParallelLText{
\begin{center}
{\large \ggn{39}.}
\end{center}\vspace*{-7pt}

\gr{>Arijm`on e>ure~in, <`oc >el'aqistoc >`wn <'exei t`a doj'enta m'erh.}\\

\epsfysize=2.2in
\centerline{\epsffile{Book07/fig39g.eps}}

\gr{>'Estw t`a doj'enta m'erh t`a A, B, G; de~i d`h >arijm`on e<ure~in,
<`oc >el'aqistoc >`wn <'exei t`a A, B, G m'erh.}

\gr{>'Estwsan g`ar to~ic A, B, G m'eresin <om'wnumoi >arijmo`i o<i D, E, Z,
ka`i e>il'hfjw <up`o t~wn D, E, Z >el'aqistoc metro'umenoc >arijm`oc
<o H.}

\gr{<O H >'ara <om'wnuma m'erh >'eqei to~ic D, E, Z. to~ic d`e D, E, Z
<om'wnuma m'erh >est`i t`a A, B, G; <o H >'ara >'eqei t`a A, B, G
m'erh. l'egw d'h, <'oti ka`i >el'aqistoc >'wn, e>i g`ar m\kern -.7pt 'h, >'estai tic 
to~u H >el'asswn >arijm'oc, <`oc <'exei t`a A, B, G m'erh. >'estw 
<o J. >epe`i <o J >'eqei t`a A, B, G m'erh, <o J >'ara <up`o <omwn'umwn
>arijm~wn metrhj'hsetai to~ic A, B, G m'eresin. to~ic d`e A, B, G m'eresin
<om'wnumoi >arijmo'i e>isin o<i D, E, Z;  <o J >'ara <up`o t~wn D, E, Z
metre~itai. ka'i >estin >el'asswn to~u H; <'oper >est`in >ad'unaton. o>uk
>'ara >'estai tic to~u H >el'asswn >arijm'oc, <`oc <'exei t`a A, B, G
m'erh; <'oper >'edei de~ixai.}}

\ParallelRText{
\begin{center}
{\large Proposition 39}
\end{center}

To find the least number that will have given
parts.\\

\epsfysize=2.2in
\centerline{\epsffile{Book07/fig39e.eps}}

Let $A$, $B$, and $C$ be the given parts. So it is required to find the
least  number which will have the parts $A$, $B$, and $C$ ({\rm i.e.}, an $A$th part, a $B$th part, and a $C$th part).

For let  $D$, $E$, and $F$ be numbers having the same names as
the parts $A$, $B$, and $C$ (respectively). And let the least number, $G$,
measured by $D$, $E$, and $F$, be taken [Prop.~7.36].

Thus, $G$ has parts called the same as $D$, $E$, and $F$ [Prop.~7.37]. And $A$, $B$, and $C$ are 
parts called the same as $D$, $E$, and $F$ (respectively). Thus, $G$ has the parts $A$, $B$, and $C$. So I say that ($G$) is also the least (number having the parts $A$, $B$, and $C$). For if not, there will be some number less than $G$ which will have the parts $A$, $B$, and $C$. Let it be
$H$. Since $H$ has the parts $A$, $B$, and $C$, $H$ will thus be measured
by numbers called the same as the parts $A$, $B$, and $C$  [Prop.~7.38]. And $D$, $E$, and $F$ are numbers called the same as the parts $A$, $B$, and $C$ (respectively). Thus, $H$ is measured by $D$, $E$, and $F$. And ($H$) is less than $G$. The
very thing is impossible. Thus, there cannot be some number less than $G$
which will have the parts $A$, $B$, and $C$. (Which is) the very thing it
was required to show.}
\end{Parallel}

\newpage
\thispagestyle{empty}
~\\