% !TEX root = ../Elements.tex
% Greek and English text last checked 15/03/2013
%%%%%%
% BOOK 3
%%%%%%
\pdfbookmark[0]{Book 3}{book3}
\pagestyle{empty}
\begin{center}
{\Huge ELEMENTS BOOK 3}\\
\spa\spa\spa
{\huge\it Fundamentals of Plane Geometry Involving}\\[0.5ex] {\huge\it Circles}
\end{center}
%\vspace{2cm}
%\epsfysize=5in
%\centerline{\epsffile{Book03/front.eps}}
\newpage

%%%%%%%
% Definitions
%%%%%%%
\pdfbookmark[1]{Definitions}{def3}
\pagestyle{fancy}
\cfoot{\gr{\thepage}}
\lhead{\large\gr{STOIQEIWN \ggn{3}.}}
\rhead{\large ELEMENTS BOOK 3}
\begin{Parallel}{}{} 
\ParallelLText{
\begin{center}
\large{\gr{<'Oroi}.}
\end{center}\vspace*{-7pt}

\ggn{1}.~\gr{>'Isoi k'ukloi e>is'in, <~wn a<i di'ametroi >'isai e>is'in,
>`h <~wn a<i >ek t~wn k'entrwn >'isai e>is'in.}

\ggn{2}.~\gr{E>uje~ia k'uklou >ef'aptesjai l'egetai, <'htic <aptom'enh to~u
k'uklou ka`i >ekballom'enh o>u t'emnei t`on k'uklon.}

\ggn{3}.~\gr{K'ukloi >ef'aptesjai >all'hlwn l'egontai o<'itinec <apt'o\-menoi >all'hlwn o>u t'emnousin >all'hlouc.}

\ggn{4}.~\gr{>En k'uklw| >'ison >ap'eqein >ap`o to~u k'entrou e>uje~iai
l'egontai, <'otan a<i >ap`o to~u k'entrou >ep> a\kern -.7pt >ut`ac k'ajetoi >ag'omenai
>'isai >~wsin.}

\ggn{5}.~\gr{Me~izon d`e >ap'eqein l'egetai, >ef> <`hn <h me'izwn k'ajetoc p'iptei.}

\ggn{6}.~\gr{Tm\kern -.7pt ~hma k'uklou >est`i t`o perieq'omenon sq~hma <up'o te e>uje'iac ka`i k'uklou perifere'iac.}

\ggn{7}.~\gr{Tm\kern -.7pt 'hmatoc d`e gwn'ia >est`in <h perieqom'enh <up'o
te e>uje'iac ka`i k'uklou perifere'iac.}

\ggn{8}.~\gr{>En tm\kern -.7pt 'hmati d`e gwn'ia >est'in, <'otan >ep`i t~hc perifere'iac to~u tm\kern -.7pt 'hmatoc lhfj~h| ti shme~ion ka`i >ap> a\kern -.7pt >uto~u >ep`i t`a p'erata
t~hc e>uje'iac, <'h >esti b'asic to~u tm\kern -.7pt 'hmatoc, >epizeuqj~wsin e>uje~iai,
<h perieqom'enh gwn'ia <up`o t~wn >epizeuqjeis~wn e>ujei~wn.}

\ggn{9}.~\gr{<'Otan d`e a<i peri'eqousai t`hn gwn'ian e>uje~iai >apolamb'anws'i tina perif'ereian, >ep> >eke'inhc l'egetai bebhk'enai <h gwn'ia.}

\ggn{10}.~\gr{Tome`uc d`e k'uklou >est'in, <'otan pr`oc t~w| k'entr~w| to~u
k'uklou sustaj~h| gwn'ia, t`o perieq'omenon sq~hma <up'o te t~wn t`hn
gwn'ian perieqous~wn
e>ujei~wn ka`i t~hc >apolambanom'enhc <up> a\kern -.7pt >ut~wn perifere'iac.}

\ggn{11}.~\gr{<'Omo'ia tm\kern -.7pt 'hmata k'uklwn >est`i t`a deq'omena gwn'iac >'isac, >'h >en o<~ic a<i gwn'iai >'isai >all'hlaic e>is'in.}}

\ParallelRText{
\begin{center}
{\large Definitions}
\end{center}

1.~Equal circles are (circles) whose diameters are equal,  or whose (distances)
from the centers (to the circumferences) are equal (i.e., whose radii are equal).

2.~A straight-line said to touch a circle  is any (straight-line) which, meeting the circle and being produced, does not cut the circle.

3.~Circles said to touch one another are any (circles) which, meeting one another, do not cut one another.

4.~In a circle, straight-lines are said to be equally far from the center
when the perpendiculars drawn to them from the center are equal.

5.~And (that straight-line) is said to be further (from the center) on
which the greater perpendicular falls (from the center).

6. ~A segment of a circle is the figure contained by a straight-line
and a circumference of a circle.

7.~And the angle of a segment is that contained by a straight-line
and a circumference of a circle.

8.~And the angle in a segment is the
angle contained by the joined straight-lines, when any
point  is taken on the circumference of a segment, 
and straight-lines are joined from it to the ends of the straight-line which is the base of the
segment.

9. ~And when the straight-lines containing an angle cut off some circumference, the angle is said to stand upon that (circumference).

10.~And a sector of a circle is the figure contained by the
straight-lines surrounding an angle, and the circumference cut off by them,
when the angle is constructed at the center of a circle.

11.~Similar segments of circles are those accepting equal angles, or
in which the angles are equal to one another.}
\end{Parallel}

%%%%%%
% Prop 3.1
%%%%%%
\pdfbookmark[1]{Proposition 3.1}{pdf3.1}
\begin{Parallel}{}{} 
\ParallelLText{
\begin{center}
{\large \ggn{1}.}
\end{center}\vspace*{-7pt}

\gr{To~u doj'entoc k'uklou t`o k'entron e<ure~in.}

\gr{>'Estw <o doje`ic k'ukloc <o ABG; de~i d`h to~u ABG k'uklou
t`o k'entron e<ure~in.}

\gr{Di'hqjw tic e>ic a\kern -.7pt >ut'on, <wc >'etuqen, e>uje~ia <h AB, ka`i tetm\kern -.7pt 'hsjw d'iqa
kat`a t`o D shme~ion, ka`i >ap`o to~u D t~h| AB pr`oc >orj`ac >'hqjw
<h DG ka`i di'hqjw >ep`i t`o E, ka`i tetm\kern -.7pt 'hsjw <h GE d'iqa kat`a t`o Z;
l'egw, <'oti t`o Z k'entron >est`i to~u ABG [k'uklou].}

\gr{m\kern -.7pt `h g'ar, >all> e>i dunat'on, >'estw t`o H, ka`i >epeze'uqjwsan a<i HA, HD, HB. ka`i >epe`i >'ish >est`in <h AD t~h| DB, koin`h d`e <h DH, d'uo d`h
a<i AD, DH d'uo ta~ic HD, DB >'isai e>is`in <ekat'era <ekat'era|;
ka`i b'asic <h HA b'asei t~h| HB >estin >'ish; >ek k'entrou g'ar;
gwn'ia >'ara <h <up`o ADH gwn'ia| t~h| <up`o HDB >'ish >est'in.
<'otan d`e e>uje~ia >ep> e>uje~ian staje~isa t`ac >efex~hc gwn'iac
>'isac >all'hlaic poi~h|, >orj`h <ekat'era t~wn >'iswn gwni~wn >estin; >orj`h
>'ara >est`in <h  <up`o HDB. >est`i d`e ka`i <h <up`o ZDB >orj'h;
>'ish >'ara <h <up`o ZDB t~h| <up`o HDB, <h me'izwn t~h| >el'attoni;
<'oper >est`in >ad'unaton. o>uk >'ara t`o H k'entron >est`i to~u
ABG k'uklou. <omo'iwc d`h de'ixomen, <'oti o>ud> >'allo
ti pl`hn to~u Z.}\\~\\~\\~\\

\epsfysize=2.2in
\centerline{\epsffile{Book03/fig01g.eps}}

\gr{T`o Z >'ara shme~ion k'entron >est`i to~u ABG [k'uklou].}

\begin{center}\mbox{}\\
{\large \gr{P'orisma}.}
\end{center}\vspace*{-7pt}

\gr{>Ek d`h to'utou faner'on, <'oti >e`an >en k'uklw| e>uje~i'a
tic e>uje~i'an tina d'iqa ka`i pr`oc >orj`ac t'emnh|, >ep`i
t~hc temno'ushc >est`i t`o k'entron to~u k'uklou.} --- \gr{<'oper >'edei
poi~hsai.}}

\ParallelRText{
\begin{center}
{\large Proposition 1}
\end{center}

To find the center of a given circle.

Let $ABC$ be the given circle. So it is required to find the center of circle $ABC$.

Let some straight-line $AB$ be drawn through ($ABC$), at random,
and let ($AB$) be cut in half at point $D$ 
[Prop.~1.9]. And let
$DC$ be drawn from $D$, at right-angles to $AB$ [Prop.~1.11].
And let ($CD$) be drawn through to $E$. And let $CE$ be cut in half
at $F$ [Prop.~1.9]. I say that (point) $F$ is the center of the [circle] $ABC$.

For (if) not (then), if possible, let $G$ (be the center of the circle), and
let $GA$, $GD$, and $GB$ be joined. And since $AD$ is equal to $DB$,
and $DG$ (is) common, the two (straight-lines) $AD$, $DG$
are equal to the two (straight-lines) $BD$, $DG$,$^\dag$ respectively. And the
base $GA$ is equal to the base $GB$. For (they are both) radii. Thus, angle
$ADG$ is equal to angle $GDB$ [Prop.~1.8]. And when a straight-line
stood upon (another) straight-line make  adjacent angles (which are)
equal to one another, each of the equal angles is a right-angle [Def.~1.10].
Thus, $GDB$ is a right-angle. And $FDB$ is also a right-angle. Thus,
$FDB$ (is) equal to $GDB$, the greater to the lesser. The very thing
is impossible. Thus, (point) $G$ is not the center of the circle $ABC$. 
So, similarly, we can show that neither is any other (point) except $F$.

\epsfysize=2.2in
\centerline{\epsffile{Book03/fig01e.eps}}

Thus, point $F$ is the center of the [circle] $ABC$.

\begin{center}\mbox{}\\
{\large Corollary}
\end{center}

So, from this, (it is) manifest that if any straight-line in a circle
cuts any (other) straight-line in half, and at right-angles, (then) the center
of the circle is on the former (straight-line). --- (Which is) the very thing it
was required to do.}
\end{Parallel}
{\footnotesize \noindent$^\dag$  The Greek text has ``$GD$, $DB$'', which is obviously a mistake.} 

%%%%%%
% Prop 3.2
%%%%%%
\pdfbookmark[1]{Proposition 3.2}{pdf3.2}
\begin{Parallel}{}{} 
\ParallelLText{
\begin{center}
{\large \ggn{2}.}
\end{center}\vspace*{-7pt}

\gr{>E`an k'uklou >ep`i t~hc perifere'iac lhfj~h| d'uo
tuq'onta shme~ia, <h >ep`i t`a shme~ia >epizeugnum'enh e>uje~ia
>ent`oc pese~itai to~u k'uklou.}

\gr{>'Estw k'ukloc <o ABG, ka`i >ep`i t~hc perifere'iac a\kern -.7pt >uto~u e>il'hfjw
d'uo tuq'onta shme~ia t`a A, B; l'egw, <'oti <h >ap`o to~u A >ep`i t`o B
>epizeugnum'enh e>uje~ia >ent`oc pese~itai to~u k'uklou.}

\gr{m\kern -.7pt `h g'ar, >all> e>i dunat'on, pipt'etw >ekt`oc <wc <h AEB, ka`i e>il'hfjw
t`o k'entron to~u ABG k'uklou, ka`i >'estw t`o D, ka`i >epeze'uqjwsan a<i
DA, DB, ka`i di'hqjw <h DZE.}

\gr{>Epe`i o>~un >'ish >est`in <h DA t~h| DB, >'ish >'ara
ka`i gwn'ia <h <up`o DAE t~h| <up`o DBE; ka`i >epe`i trig'wnou
to~u DAE m'ia pleur`a prosekb'eblhtai <h AEB, me'izwn
>'ara <h <up`o DEB gwn'ia t~hc <up`o DAE. >'ish d`e <h <up`o
DAE t~h|  <up`o DBE; me'izwn >'ara <h <up`o DEB t~hc <up`o
DBE. <up`o d`e t`hn me'izona gwn'ian <h me'izwn pleur`a
<upote'inei; me'izwn >'ara <h DB t~hc DE. >'ish d`e <h DB
t~h| DZ. me'izwn >'ara <h DZ t~hc DE <h >el'attwn t~hc
me'izonoc; <'oper >est`in >ad'unaton. o>uk >'ara <h >ap`o
to~u A >ep`i t`o B >epizeugnum'enh e>uje~ia >ekt`oc
pese~itai to~u k'uklou. <omo'iwc d`h de'ixomen, <'oti o>ud`e >ep>
a\kern -.7pt >ut~hc t~hc perifere'iac; >ent`oc >'ara.}\\~\\~\\

\epsfysize=2.2in
\centerline{\epsffile{Book03/fig02g.eps}}

\gr{>E`an >'ara k'uklou >ep`i t~hc perifere'iac lhfj~h| d'uo
tuq'onta shme~ia, <h >ep`i t`a shme~ia >epizeugnum'enh e>uje~ia
>ent`oc pese~itai to~u k'uklou; <'oper >'edei de~ixai.}}

\ParallelRText{
\begin{center}
{\large Proposition 2}
\end{center}

If two points are taken at random on the circumference of a circle,  (then)
the straight-line joining the points will fall inside the circle.

Let $ABC$ be a circle, and let two points $A$ and $B$ be taken
at random on its circumference. I say that the straight-line joining
$A$ to $B$ will fall inside the circle.

For (if) not (then), if possible, let it fall outside (the circle), like $AEB$ (in the figure). And let the center of the circle $ABC$ be found [Prop.~3.1], and let it be (at point) $D$. And let $DA$ and $DB$ be joined, and let $DFE$ be
drawn through.

Therefore, since $DA$ is equal to $DB$, the angle $DAE$ (is) thus also equal to
$DBE$ [Prop.~1.5]. And since in triangle $DAE$ the one side, $AEB$, has been
produced, angle $DEB$ (is) thus greater than $DAE$ [Prop.~1.16]. And
$DAE$ (is) equal to $DBE$ [Prop.~1.5]. Thus, $DEB$ (is) greater than $DBE$.
And the greater angle is subtended by the greater side [Prop.~1.19].
Thus, $DB$ (is) greater than $DE$. And $DB$ (is) equal to $DF$. Thus,
$DF$ (is) greater than $DE$, the lesser than the greater. The very thing is impossible.
Thus, the straight-line joining $A$ to $B$ will not fall outside the
circle. So, similarly, we can show that neither (will it fall) on the
circumference itself. Thus, (it will fall) inside (the circle).

\epsfysize=2.2in
\centerline{\epsffile{Book03/fig02e.eps}}

Thus, if two points are taken at random on the circumference of a circle, (then)
the straight-line joining the points will fall inside the circle. (Which is) the
very thing it was required to show.}
\end{Parallel}

%%%%%%
% Prop 3.3
%%%%%%
\pdfbookmark[1]{Proposition 3.3}{pdf3.3}
\begin{Parallel}{}{} 
\ParallelLText{
\begin{center}
{\large \ggn{3}.}
\end{center}\vspace*{-7pt}

\gr{>E`an >en k'uklw| e>uje~i'a tic di`a to~u k'entrou e>uje~i'an tina m\kern -.7pt `h di`a to~u k'entrou d'iqa t'emnh|, ka`i pr`oc >orj`ac a\kern -.7pt >ut`hn t'emnei; ka`i >e`an pr`oc >orj`ac a\kern -.7pt >ut`hn t'emnh|, ka`i d'iqa a\kern -.7pt >ut`hn t'emnei.}

\gr{>'Estw k'ukloc <o ABG, ka`i >en a\kern -.7pt >ut~w| e>uje~i'a tic di`a to~u k'entrou <h GD e>uje~i'an tina m\kern -.7pt `h di`a to~u k'entrou t`hn AB d'iqa temn'etw kat`a t`o
Z shme~ion; l'egw, <'oti ka`i pr`oc >orj`ac a\kern -.7pt >ut`hn t'emnei.}

\gr{E>il'hfjw g`ar t`o k'entron to~u ABG k'uklou, ka`i >'estw t`o E, ka`i >epeze'uqjwsan a<i EA, EB.}

\gr{Ka`i >epe`i >'ish >est`in <h AZ t~h| ZB, koin`h d`e <h ZE, d'uo dus`in >'isai [e>is'in]; ka`i b'asic <h EA b'asei t~h| EB >'ish; gwn'ia >'ara <h <up`o AZE gwn'ia| t~h| <up`o BZE >'ish >est'in. <'otan d`e e>uje~ia >ep> e>uje~ian
staje~isa t`ac >efex~hc gwn'iac >'isac >all'hlaic poi~h|,  >orj`h
<ekat'era t~wn >'iswn gwni~wn >estin; <ekat'era >'ara t~wn <up`o AZE, BZE >orj'h >estin. <h GD >'ara di`a to~u k'entrou o>~usa t`hn AB m\kern -.7pt `h
di`a to~u k'entrou o>~usan d'iqa t'emnousa ka`i pr`oc >orj`ac t'emnei.}

\gr{>All`a d`h <h GD t`hn AB pr`oc >orj`ac temn'etw; l'egw, <'oti ka`i d'iqa
a\kern -.7pt >ut`hn t'emnei, tout'estin, <'oti >'ish >est`in <h AZ t~h| ZB.}

\gr{T~wn g`ar a\kern -.7pt >ut~wn kataskeuasj'entwn, >epe`i >'ish >est`in <h EA t~h| EB,
>'ish >est`i ka`i gwn'ia <h <up`o EAZ t~h| <up`o EBZ. >est`i d`e ka`i
>orj`h <h <up`o AZE >orj~h| t~h| <up`o BZE >'ish; d'uo >'ara tr'igwn'a
>esti EAZ, EZB t`ac d'uo gwn'iac dus`i gwn'iaic >'isac >'eqonta ka`i
m'ian pleur`an mi~a| pleur~a| >'ishn koin`hn a\kern -.7pt >ut~wn t`hn EZ <upote'inousan <up`o m'ian t~wn >'iswn gwni~wn; ka`i t`ac loip`ac >'ara pleur`ac ta~ic
loipa~ic pleura~ic >'isac <'exei; >'ish >'ara <h AZ t~h| ZB.}\\~\\~\\~\\

\epsfysize=2.2in
\centerline{\epsffile{Book03/fig03g.eps}}

\gr{>E`an >'ara >en k'uklw| e>uje~i'a tic di`a to~u k'entrou e>uje~i'an tina m\kern -.7pt `h di`a to~u k'entrou d'iqa t'emnh|, ka`i pr`oc >orj`ac a\kern -.7pt >ut`hn t'emnei; ka`i >e`an pr`oc >orj`ac a\kern -.7pt >ut`hn t'emnh|, ka`i d'iqa a\kern -.7pt >ut`hn t'emnei; <'oper >'edei de~ixai.}}

\ParallelRText{
\begin{center}
{\large Proposition 3}
\end{center}

In a circle, if any straight-line through the center cuts in half any straight-line
not through the center, (then) it also cuts it at right-angles. And (conversely)
if it cuts it at right-angles, (then) it also cuts it in half.

Let $ABC$ be a circle, and, within it, let some straight-line through the center,
$CD$, 
cut in half some straight-line not through the center, $AB$, at the point $F$.
I say that ($CD$) also cuts ($AB$) at right-angles.

For let the center of the circle $ABC$ be found [Prop.~3.1], and let it
be (at point) $E$, and let $EA$ and $EB$ be joined.

And since  $AF$ is equal to $FB$, and $FE$ (is) common, two (sides  of triangle $AFE$) [are] equal to
two (sides of triangle $BFE$). And the base $EA$ (is) equal to the base $EB$. Thus, angle
$AFE$ is equal to angle $BFE$ [Prop.~1.8]. And when a straight-line
stood upon (another) straight-line makes adjacent angles (which are)
equal to one another, each of the equal angles is a right-angle [Def.~1.10].
Thus, $AFE$ and $BFE$ are each right-angles. Thus, the (straight-line) $CD$, which is
through the center
and cuts  in half the (straight-line) $AB$, which is not through the center, also cuts ($AB$) at right-angles.

And so let $CD$ cut $AB$ at right-angles. I say that it also cuts ($AB$) in half.
That is to say, that $AF$ is equal to $FB$.

For, with the same construction, since $EA$ is equal to $EB$, angle $EAF$ is also
equal to $EBF$ [Prop.~1.5]. And the right-angle $AFE$ is also equal to the
right-angle $BFE$. Thus, $EAF$ and $EFB$ are two triangles having two angles
equal to two angles, and one side equal to one side---(namely), their common (side)
$EF$, subtending one of the equal angles. Thus, they will also have the remaining sides equal to the (corresponding) remaining sides [Prop.~1.26]. Thus, $AF$ (is) equal to $FB$.

\epsfysize=2.2in
\centerline{\epsffile{Book03/fig03e.eps}}

Thus, in a circle, if any straight-line through the center cuts in half any straight-line
not through the center, (then) it also cuts it at right-angles. And (conversely)
if it cuts it at right-angles, (then) it also cuts it in half. (Which is) the very thing
it was required to show.}
\end{Parallel}

%%%%%%
% Prop 3.4
%%%%%%
\pdfbookmark[1]{Proposition 3.4}{pdf3.4}
\begin{Parallel}{}{} 
\ParallelLText{
\begin{center}
{\large \ggn{4}.}
\end{center}\vspace*{-7pt}

\gr{>E`an >en k'uklw| d'uo e>uje~iai t'emnwsin >all'hlac m\kern -.7pt `h d`ia to~u
k'entrou o>~usai, o>u t'emnousin >all'hlac d'iqa.}\\

\epsfysize=2in
\centerline{\epsffile{Book03/fig04g.eps}}

\gr{>'Estw k'ukloc <o ABGD, ka`i >en a\kern -.7pt >ut~w| d'uo e>uje~iai a<i AG,
BD temn'etwsan >all'hlac kat`a t`o E m\kern -.7pt `h di`a to~u k'entrou o>~usai;
l'egw, <'oti o>u t'emnousin >all'hlac d'iqa.}

\gr{E>i g`ar dunat'on, temn'etwsan >all'hlac d'iqa <'wste >'ishn e>~inai
t`hn m`en AE t~h| EG, t`hn d`e BE t~h| ED; ka`i e>il'hfjw t`o k'entron
to~u ABGD k'uklou, ka`i >'estw t`o Z, ka`i >epeze'uqjw <h ZE.}

\gr{>Epe`i o>~un e>uje~i'a tic di`a to~u k'entrou <h ZE e>uje~i'an tina
m\kern -.7pt `h di`a to~u k'entrou t`hn AG d'iqa t'emnei, ka`i pr`oc >orj`ac a\kern -.7pt >ut`hn t'emnei; >orj`h >'ara >est`in <h <up`o ZEA; p'alin, >epe`i e>uje~i'a tic <h ZE e>uje~i'an tina t`hn BD d'iqa t'emnei, ka`i pr`oc >orj`ac a\kern -.7pt >ut`hn t'emnei;
>orj`h >'ara <h <up`o ZEB. >ede'iqjh d`e ka`i <h <up`o ZEA >orj'h; >'ish >'ara <h <up`o ZEA t~h| <up`o ZEB <h >el'attwn t~h| me'izoni;
<'oper >est`in >ad'unaton. o>uk >'ara a<i AG, BD t'emnousin >all'hlac d'iqa.}

\gr{>E`an >'ara >en k'uklw| d'uo e>uje~iai t'emnwsin >all'hlac m\kern -.7pt `h d`ia to~u
k'entrou o>~usai, o>u t'emnousin >all'hlac d'iqa; <'oper >'edei de~ixai.}}

\ParallelRText{
\begin{center}
{\large Proposition 4}
\end{center}

In a circle, if two straight-lines, which are not through the center, cut
one another, (then) they do not cut one another in half.

\epsfysize=2in
\centerline{\epsffile{Book03/fig04e.eps}}

Let $ABCD$ be a circle, and within it, let two straight-lines, $AC$ and
$BD$, which are not through the center, cut one another at  (point) $E$. I say that they
do not cut one another in half.

For, if possible, let them cut one another in half, such that $AE$ is equal to $EC$, and
$BE$ to $ED$. And let the center of the circle $ABCD$ be found [Prop.~3.1], and let it be (at point) $F$, and let $FE$ be joined.

Therefore, since some straight-line through the center, $FE$, cuts in half
some straight-line not through the center, $AC$, it also cuts it at right-angles
[Prop.~3.3]. Thus, $FEA$ is a right-angle. Again, since some straight-line
$FE$ cuts in half some straight-line $BD$, it also cuts it at right-angles [Prop.~3.3]. Thus, $FEB$ (is) a right-angle. But $FEA$ was also shown (to be) a right-angle. Thus, $FEA$ (is) equal to $FEB$, the lesser to the greater.
The very thing is impossible. Thus, $AC$ and $BD$ do not cut one another in half.

Thus, in a circle, if two straight-lines, which are not through the center, cut
one another, (then) they do not cut one another in half. (Which is) the very
thing it was required to show.}
\end{Parallel}

%%%%%%
% Prop 3.5
%%%%%%
\pdfbookmark[1]{Proposition 3.5}{pdf3.5}
\begin{Parallel}{}{} 
\ParallelLText{
\begin{center}
{\large \ggn{5}.}
\end{center}\vspace*{-7pt}

\gr{>E`an d'uo k'ukloi t'emnwsin >all'hlouc, o>uk >'estai a\kern -.7pt >ut~wn t`o a\kern -.7pt >ut`o
k'entron.}

\epsfysize=2in
\centerline{\epsffile{Book03/fig05g.eps}}

\gr{D'uo g`ar k'ukloi o<i ABG, GDH temn'etwsan >all'hlouc kat`a t`a B, G shme~ia. l'egw, <'oti o>uk >'estai a\kern -.7pt >ut~wn t`o a\kern -.7pt >ut`o k'entron.}

\gr{E>i g`ar dunat'on, >'estw t`o E, ka`i >epeze'uqjw <h EG, ka`i di'hqjw
<h EZH, <wc >'etuqen. ka`i >epe`i t`o E shme~ion k'entron >est`i to~u ABG k'uklou, <'ish >est`in <h EG t~h| EZ. p'alin, >epe`i t`o E shme~ion k'entron >est`i to~u GDH k'uklou, >'ish >est`in <h EG t~h| EH; >ede'iqjh
d`e <h EG ka`i t~h| EZ >'ish; ka`i <h EZ >'ara t~h| EH >estin >'ish <h
>el'asswn t~h| me'izoni; <'oper >est`in >ad'unaton.  o>uk >'ara t`o E
shme~ion k'entron >est`i t~wn ABG, GDH k'uklwn.}

\gr{>E`an >'ara d'uo k'ukloi t'emnwsin >all'hlouc, o>uk >'estin a\kern -.7pt >ut~wn t`o
a\kern -.7pt >ut`o k'entron; <'oper >'edei de~ixai.}}

\ParallelRText{
\begin{center}
{\large Proposition 5}
\end{center}

If two circles cut one another,  (then) they will not have the same center.

\epsfysize=2in
\centerline{\epsffile{Book03/fig05e.eps}}

For let the two circles $ABC$ and $CDG$ cut one another at  points $B$ and $C$.
I say that they will not have the same center.

For, if possible, let $E$ be (the common center), and let $EC$ be joined,
and let $EFG$ be drawn through (the two circles), at random. And since point $E$
is the center of the circle $ABC$, $EC$ is equal to $EF$. Again, since point $E$ is
the center of the circle $CDG$, $EC$ is equal to $EG$. But $EC$  was also shown (to be) equal to $EF$. Thus, $EF$ is also equal to $EG$, the lesser to the
greater. The very thing is impossible. Thus, point $E$ is not the (common)
center of the circles $ABC$ and $CDG$.

Thus, if two circles cut one another, (then) they will not have the same
center. (Which is) the very thing it was required to show.}
\end{Parallel}

%%%%%%
% Prop 3.6
%%%%%%
\pdfbookmark[1]{Proposition 3.6}{pdf3.6}
\begin{Parallel}{}{} 
\ParallelLText{
\begin{center}
{\large \ggn{6}.}
\end{center}\vspace*{-7pt}

\gr{>E`an d'uo k'ukloi >ef'aptwntai >all'hlwn, o>uk >'estai a\kern -.7pt >ut~wn t`o a\kern -.7pt >ut`o k'entron.}

\gr{D'uo g`ar k'ukloi o<i ABG, GDE >efapt'esjwsan >all'hlwn kat`a t`o G
shme~ion; l'egw, <'oti o>uk >'estai a\kern -.7pt >ut~wn t`o a\kern -.7pt >ut`o k'entron.}

\gr{E>i g`ar dunat'on, >'estw t`o Z, ka`i >epeze'uqjw <h ZG, ka`i di'hqjw, <wc >'etuqen, <h ZEB.}

\gr{>Epe`i o~>un t`o Z shme~ion k'entron >est`i to~u ABG k'uklou, >'ish
>est`in <h ZG t~h| ZB. p'alin, >epe`i t`o Z shme~ion k'entron >est`i to~u GDE k'uklou, >'ish >est`in <h ZG t~h| ZE. >ede'iqjh d`e <h ZG t~h| ZB
>'ish; ka`i <h ZE >'ara t~h| ZB >estin >'ish, <h >el'attwn t~h| me'izoni;
<'oper >est`in >ad'unaton. o>uk >'ara t`o Z shme~ion k'entron >est`i
t~wn ABG, GDE k'uklwn.}

\epsfysize=2.2in
\centerline{\epsffile{Book03/fig06g.eps}}

\gr{>E`an >'ara d'uo k'ukloi >ef'aptwntai >all'hlwn, o>uk >'estai a\kern -.7pt >ut~wn t`o a\kern -.7pt >ut`o k'entron; <'oper >'edei de~ixai.}}

\ParallelRText{
\begin{center}
{\large Proposition 6}
\end{center}

If  two circles touch one another, (then) they  will not have the same center.

For let the two circles $ABC$ and $CDE$ touch one another at point $C$. I say that
they will not have the same center.

For, if possible, let $F$ be (the common center), and let $FC$ be
joined, and let $FEB$ be drawn through (the two circles), at random.

Therefore, since point $F$ is the center of the circle $ABC$, $FC$ is equal to
$FB$. Again, since point $F$ is the center of the circle $CDE$, $FC$ is
equal to $FE$. But $FC$ was shown (to be) equal to $FB$. Thus, $FE$ is
also equal to $FB$, the lesser to the greater. The very thing is impossible.
Thus, point $F$ is not the (common) center of the circles
$ABC$ and $CDE$.

\epsfysize=2.2in
\centerline{\epsffile{Book03/fig06e.eps}}

Thus, if two circles touch one another, (then) they  will not have the same center. (Which is) the very thing it was required to show.}
\end{Parallel}

%%%%%%
% Prop 3.7
%%%%%%
\pdfbookmark[1]{Proposition 3.7}{pdf3.7}
\begin{Parallel}{}{} 
\ParallelLText{
\begin{center}
{\large \ggn{7}.}
\end{center}\vspace*{-7pt}

\gr{>E`an k'uklou >ep`i t~hc diam'etrou lhfj~h| ti shme~ion, <`o m\kern -.7pt 'h >esti k'entron to~u k'uklou, >ap`o d`e to~u shme'iou pr`oc t`on k'uklon prosp'iptwsin e>uje~ia'i tinec, meg'isth m`en >'estai, >ef> <~hc
t`o k'entron, >elaq'isth d`e <h loip'h, t~wn d`e >'allwn >ae`i <h
>'eggion t~hc d`ia to~u k'entrou t~hc >ap'wteron me'izwn >est'in, d'uo
d`e m'onon >'isai >ap`o to~u shme'iou prospeso~untai pr`oc t`on
k'uklon >ef> <ek'atera t~hc >elaq'isthc.}

\gr{>'Estw k'ukloc <o ABGD, di'ametroc d`e a\kern -.7pt >uto~u >'estw <h AD, ka`i >ep`i
t~hc AD e>il'hfjw ti shme~ion t`o Z, <`o m\kern -.7pt 'h >esti k'entron to~u k'uklou, k'entron d`e to~u k'uklou >'estw t`o E, ka`i >ap`o to~u Z pr`oc t`on ABGD k'uklon prospipt'etwsan e>uje~ia'i tinec a<i ZB, ZG, ZH; l'egw, <'oti
meg'isth m'en >estin <h ZA, >elaq'isth d`e <h ZD, t~wn d`e >'allwn <h m`en
ZB t~hc ZG me'izwn, <h d`e ZG t~hc ZH.}

\gr{>Epeze'uqjwsan g`ar a<i BE, GE, HE. ka`i >epe`i pant`oc trig'wnou a<i d'uo pleura`i t~hc loip~hc me'izon'ec e>isin, a<i >'ara EB, EZ t~hc BZ me'izon'ec
e>isin. >'ish d`e <h AE t~h| BE [a<i >'ara BE, EZ >'isai e>is`i t~h| AZ];
me'izwn >'ara <h AZ t~hc BZ. p'alin, >epe`i >'ish >est`in <h BE t~h| GE,
koin`h d`e <h ZE, d'uo d`h a<i BE, EZ dus`i ta~ic GE, EZ >'isai e>is'in.
>all`a ka`i gwn'ia <h <up`o BEZ gwn'iac t~hc <up`o GEZ me'izwn; b'asic
>'ara <h BZ b'asewc t~hc GZ me'izwn >est'in. di`a t`a a\kern -.7pt >ut`a d`h ka`i
<h GZ t~hc ZH me'izwn >est'in.}

\gr{P'alin, >epe`i a<i HZ, ZE t~hc EH me'izon'ec e>isin, >'ish d`e <h EH
t~h| ED, a<i >'ara HZ, ZE t~hc ED me'izon'ec e>isin. koin`h >afh|r'hsjw <h EZ; loip`h >'ara <h HZ loip~hc t~hc ZD me'izwn >est'in. meg'isth m`en >'ara <h ZA, >elaq'isth d`e <h ZD, me'izwn d`e <h m`en ZB t~hc ZG, <h
d`e ZG t~hc ZH.}\\~\\~\\~\\~\\~\\~\\

\epsfysize=2.5in
\centerline{\epsffile{Book03/fig07g.eps}}

\gr{L'egw, <'oti ka`i >ap`o to~u Z shme'iou d'uo m'onon >'isai prospeso~untai pr`oc t`on ABGD k'uklon >ef> <ek'atera t~hc ZD >elaq'isthc. sunest'atw g`ar pr`oc t~h| EZ e>uje'ia| ka`i t~w| pr`oc a\kern -.7pt >ut~h| shme'iw| t~w| E t~h|
<up`o HEZ gwn'ia| >'ish <h <up`o ZEJ, ka`i >epeze'uqjw <h ZJ. >epe`i o>~un >'ish >est`in <h HE t~h| EJ, koin`h d`e <h EZ, d'uo
d`h a<i HE, EZ dus`i ta~ic JE, EZ >'isai e>is'in; ka`i gwn'ia <h <up`o HEZ
gwn'ia| t~h| <up`o JEZ >'ish; b'asic >'ara <h ZH b'asei t~h| ZJ >'ish
>est'in. l'egw d'h, <'oti t~h| ZH >'allh >'ish o>u prospese~itai pr`oc t`on
k'uklon >ap`o to~u Z shme'iou. e>i g`ar dunat'on, prospipt'etw <h ZK.
ka`i >epe`i <h ZK t~h| ZH >'ish >est'in, >all`a <h ZJ t~h| ZH
[>'ish >est'in], ka`i  <h  ZK
>'ara t~h| ZJ >estin >'ish, <h >'eggion
t~hc di`a to~u k'entrou t~h| >ap'wteron >'ish; <'oper >ad'unaton.
o>uk >'ara >ap`o to~u Z shme'iou <et'era tic prospese~itai pr`oc t`on
k'uklon >'ish t~h| HZ; m'ia >'ara m'onh.}

\gr{>E`an >'ara k'uklou >ep`i t~hc diam'etrou lhfj~h| ti shme~ion, <`o m\kern -.7pt 'h >esti k'entron to~u k'uklou, >ap`o d`e to~u shme'iou pr`oc t`on k'uklon prosp'iptwsin e>uje~ia'i tinec, meg'isth m`en >'estai, >ef> <~hc
t`o k'entron, >elaq'isth d`e <h loip'h, t~wn d`e >'allwn >ae`i <h
>'eggion t~hc d`ia to~u k'entrou t~hc >ap'wteron me'izwn >est'in, d'uo
d`e m'onon >'isai >ap`o to~u a\kern -.7pt >uto~u shme'iou prospeso~untai pr`oc t`on
k'uklon >ef> <ek'atera t~hc >elaq'isthc; <'oper >'edei de~ixai.}}

\ParallelRText{
\begin{center}
{\large Proposition 7}
\end{center}

If some point, which is not the
center of the circle, is taken on the diameter of a circle,
and some straight-lines radiate from the point towards
the (circumference of the) circle, (then) the greatest (straight-line) will be that on which
the center (lies), and the least the remainder (of the same diameter). And
for the others, a (straight-line) nearer$^\dag$ to the (straight-line) through the center is always
greater than a (straight-line) further away. And only two equal (straight-lines)
will radiate from the point towards the (circumference of the) circle, (one)
on each (side) of the least (straight-line).

Let $ABCD$ be a circle, and let $AD$ be its diameter, and
let some point $F$, which is not the center of the circle, be taken on $AD$.
Let $E$ be the center of the circle. And let some straight-lines, $FB$, $FC$,
and $FG$, radiate from $F$ towards (the circumference of) circle $ABCD$.
I say that $FA$ is the greatest (straight-line), $FD$ the least, and of the others, $FB$ (is)
greater than $FC$, and $FC$ than $FG$.

For let $BE$, $CE$, and $GE$ be joined. And since for every triangle
(any) two sides are greater than the remaining (side) [Prop.~1.20],
$EB$ and $EF$ is thus greater than $BF$. And $AE$ (is) equal to $BE$ [thus,
$BE$ and $EF$ is equal to $AF$]. Thus, $AF$ (is) greater than $BF$. Again,
since $BE$ is equal to $CE$, and $FE$ (is) common, the two (straight-lines)
$BE$, $EF$ are equal to the two (straight-lines) $CE$, $EF$ (respectively). But, angle $BEF$ (is)
also greater than angle $CEF$.$^\ddag$ Thus, the base $BF$ is greater than the base $CF$.
Thus, the base $BF$ is greater than the base $CF$ [Prop.~1.24]. So, for the same (reasons), $CF$ is also greater than $FG$.

Again, since $GF$ and $FE$ are greater than $EG$ [Prop.~1.20], and $EG$
(is) equal to $ED$, $GF$ and $FE$ are thus greater than $ED$. Let $EF$ be
taken from both. Thus, the remainder $GF$ is greater than the remainder $FD$.
Thus, $FA$ (is) the greatest  (straight-line), $FD$ the least, and $FB$ (is) greater than $FC$, and $FC$ than $FG$.

\epsfysize=2.5in
\centerline{\epsffile{Book03/fig07e.eps}}

I also say that from point $F$ only two equal (straight-lines) will radiate
towards (the circumference of) circle $ABCD$, (one) on each (side) of the
least (straight-line) $FD$. For let the (angle) $FEH$, equal to angle $GEF$, be constructed on the straight-line $EF$, at the point $E$ on it [Prop.~1.23], and
let $FH$ be joined. Therefore, since $GE$ is equal to $EH$, and $EF$ (is) common, the two (straight-lines) $GE$, $EF$ are equal to the two (straight-lines)
$HE$, $EF$ (respectively). And angle $GEF$ (is) equal to angle $HEF$. Thus, the base $FG$ is
equal to the base $FH$ [Prop.~1.4]. So I say that another (straight-line)
equal to $FG$
will not radiate towards (the circumference of)
the circle from point $F$. For, if possible, let $FK$ (so) radiate. And since
$FK$ is equal to $FG$, but $FH$ [is equal] to $FG$, $FK$ is thus also equal to
$FH$, the nearer to the (straight-line) through the center equal to the
further away. The very thing (is) impossible. Thus, another (straight-line)
equal to $GF$ will not radiate from the point $F$ towards (the circumference of) the circle.
Thus, (there is) only one (such straight-line).

Thus, if some point, which is not the
center of the circle, is taken on the diameter of a circle,
and some straight-lines radiate from the point towards
the (circumference of the) circle, (then) the greatest (straight-line) will be that on which
the center (lies), and the least the remainder (of the same diameter). And
for the others, a (straight-line) nearer to the (straight-line) through the center is always
greater than a (straight-line) further away. And only two equal (straight-lines)
will radiate from the same point towards the (circumference of the) circle,  (one)
on each (side) of the least (straight-line). (Which is) the very thing it was required to show.}
\end{Parallel}
{\footnotesize \noindent$^\dag$ Presumably, in an angular sense.\\
$^\ddag$ This is not proved, except by reference to the figure.}

%%%%%%
% Prop 3.8
%%%%%%
\pdfbookmark[1]{Proposition 3.8}{pdf3.8}
\begin{Parallel}{}{} 
\ParallelLText{
\begin{center}
{\large \ggn{8}.}
\end{center}\vspace*{-7pt}

\gr{>E`an k'uklou lhfj~h| ti shme~ion >ekt'oc, >ap`o d`e to~u shme'iou pr`oc
t`on k'uklon diaqj~wsin e>uje~ia'i tinec, <~wn m'ia m`en di`a to~u k'entrou,
a<i d`e loipa'i, <wc >'etuqen, t~wn m`en pr`oc t`hn ko'ilhn perif'ereian prospiptous~wn e>ujei~wn meg'isth m'en >estin <h di`a to~u k'entrou, t~wn d`e >'allwn >ae`i <h >'eggion t~hc di`a to~u k'entrou t~hc >ap'wteron me'izwn >est'in, t~wn d`e pr`oc t`hn kurt`hn perif'ereian prospiptous~wn e>ujei~wn
>elaq'isth m'en >estin <h metax`u to~u te shme'iou ka`i t~hc diam'etrou, t~wn d`e >'allwn >ae`i <h >'eggion t~hc >elaq'isthc t~hc >ap'wter'on >estin
>el'attwn, d'uo d`e m'onon >'isai >ap`o to~u shme'iou prospeso~untai
pr`oc t`on k'uklon >ef> <ek'atera t~hc >elaq'isthc.}

\gr{>'Estw k'ukloc <o ABG, ka`i to~u ABG e>il'hfjw ti shme~ion >ekt`oc t`o D,
ka`i >ap> a\kern -.7pt >uto~u di'hqjwsan e>uje~ia'i tinec a<i DA, DE, DZ, DG,
>'estw d`e <h DA di`a to~u k'entrou. l'egw, <'oti t~wn m`en pr`oc
t`hn AEZG ko'ilhn perif'ereian prospiptous~wn e>ujei~wn meg'isth
m'en >estin <h di`a to~u k'entrou <h DA, me'izwn d`e <h m`en DE t~hc DZ <h d`e DZ t~hc DG, t~wn d`e pr`oc t`hn JLKH kurt`hn perif'ereian prospiptous~wn e>ujei~wn >elaq'isth m'en >estin <h DH <h
metax`u to~u shme'iou ka`i t~hc diam'etrou t~hc AH, >ae`i d`e <h
>'eggion t~hc DH >elaq'isthc >el'attwn >est`i t~hc >ap'wteron, <h m`en
DK t~hc DL, <h d`e DL t~hc DJ.}

\gr{E>il'hfjw g`ar t`o k'entron to~u ABG k'uklou ka`i >'estw t`o M; ka`i
>epeze'uqjwsan a<i ME, MZ, MG, MK, ML, MJ.}

\gr{Ka`i >epe`i >'ish >est`in <h AM t~h| EM, koin`h proske'isjw <h MD;
<h >'ara AD >'ish >est`i ta~ic EM, MD. >all> a<i EM, MD t~hc ED me'izon'ec e>isin; ka`i <h AD >'ara t~hc ED me'izwn >est'in. p'alin, >epe`i
>'ish >est`in <h ME t~h| MZ, koin`h d`e <h MD, a<i EM, MD >'ara
ta~ic ZM, MD >'isai e>is'in; ka`i gwn'ia <h <up`o EMD gwn'iac t~hc <up`o ZMD me'izwn >est'in. b'asic >'ara <h ED b'asewc t~hc ZD me'izwn >est'in.
<omo'iwc d`h de'ixomen, <'oti ka`i <h ZD t~hc
GD me'izwn >est'in; meg'isth m`en >'ara <h DA, me'izwn d`e <h m`en DE
t~hc DZ, <h d`e DZ t~hc DG.}

\gr{Ka`i >epe`i a<i MK, KD t~hc MD me'izon'ec e>isin, >'ish d`e <h m\kern -.7pt h t~h| MK, loip`h >'ara <h KD loip~hc t~hc HD me'izwn >est'in; <'wste
<h HD t~hc KD >el'attwn >est'in; ka`i >epe`i trig'wnou to~u MLD
>ep`i mi~ac t~wn pleur~wn t~hc MD d'uo e>uje~iai >ent`oc
sunest'ajhsan a<i MK, KD, a<i >'ara MK, KD  t~wn ML, LD >el'atton'ec e>isin; >'ish d`e <h MK  
 t~h| ML; loip`h >'ara <h DK loip~hc t~hc DL >el'attwn >est'in. <omo'iwc d`h de'ixomen, <'oti ka`i <h DL t~hc DJ
>el'attwn >est'in; >elaq'isth m`en >'ara <h DH, >el'attwn d`e <h
m`en DK t~hc DL <h d`e DL t~hc DJ.}\\~\\~\\~\\~\\~\\~\\~\\

\epsfysize=2.5in
\centerline{\epsffile{Book03/fig08g.eps}}

\gr{L'egw, <'oti ka`i d'uo m'onon >'isai >ap`o to~u D shme'iou prospeso~untai
pr`oc t`on k'uklon >ef> <ek'atera t~hc DH >elaq'isthc; sunest'atw pr`oc
t~h| MD e>uje'ia| ka`i t~w| pr`oc a\kern -.7pt >ut~h| shme'iw| t~w| M t~h| <up`o KMD gwn'ia| >'ish gwn'ia <h <up`o DMB, ka`i >epeze'uqjw <h DB. ka`i >epe`i >'ish >est`in <h MK t~h| MB, koin`h d`e <h MD, d'uo d`h a<i KM, MD
d'uo ta~ic BM, MD >'isai e>is`in <ekat'era <ekat'era|; ka`i gwn'ia <h
<up`o KMD gwn'ia| t~h| <up`o BMD >'ish; b'asic  >'ara <h DK b'asei
t~h| DB >'ish >est'in. l'egw [d'h], <'oti t~h| DK e>uje'ia| >'allh >'ish
o>u prospese~itai pr`oc t`on k'uklon >ap`o to~u D shme'iou. e>i g`ar
dunat'on, prospipt'etw ka`i >'estw <h DN. >epe`i o>~un <h DK
t~h| DN >estin >'ish, >all> <h DK t~h| DB >estin >'ish, ka`i
<h DB >'ara t~h| DN >estin >'ish, <h >'eggion t~hc DH >elaq'isthc t~h|
>ap'wteron [>estin] >'ish; <'oper >ad'unaton >ede'iqjh. o>uk
>'ara ple'iouc >`h d'uo >'isai pr`oc t`on ABG k'uklon >ap`o to~u D shme'iou
>ef> <ek'atera t~hc DH >elaq'isthc prospeso~untai.}

\gr{>E`an >'ara k'uklou lhfj~h| ti shme~ion >ekt'oc, >ap`o d`e to~u shme'iou pr`oc
t`on k'uklon diaqj~wsin e>uje~ia'i tinec, <~wn m'ia m`en di`a to~u k'entrou
a<i d`e loipa'i, <wc >'etuqen, t~wn m`en pr`oc t`hn ko'ilhn perif'ereian prospiptous~wn e>ujei~wn meg'isth m'en >estin <h di`a to~u k'entou, t~wn d`e >'allwn >ae`i <h >'eggion t~hc di`a to~u k'entrou t~hc >ap'wteron me'izwn >est'in, t~wn d`e pr`oc t`hn kurt`hn perif'ereian prospiptous~wn e>ujei~wn
>elaq'isth m'en >estin <h metax`u to~u te shme'iou ka`i t~hc diam'etrou, t~wn d`e >'allwn >ae`i <h >'eggion t~hc >elaq'isthc t~hc >ap'wter'on >estin
>el'attwn, d'uo d`e m'onon >'isai >ap`o to~u shme'iou prospeso~untai
pr`oc t`on k'uklon >ef> <ek'atera t~hc >elaq'isthc; <'oper >'edei de~ixai.}}

\ParallelRText{
\begin{center}
{\large Proposition 8}
\end{center}

If some point is taken outside a circle, and some straight-lines are
drawn from the point to the (circumference of the) circle, one of which (passes) through the
center, the remainder  (being) random, (then) for the straight-lines
radiating towards the concave (part of the) circumference, the
greatest is  that (passing) through the center. For the others, a (straight-line)
nearer$^\dag$ to the (straight-line) through the center is always greater than one
further away. For the straight-lines radiating towards the convex (part of
the) circumference, the least is that between the point and the
diameter. For the others, a (straight-line) nearer to the least
(straight-line) is always less than one further away. And only
two equal (straight-lines) will radiate from the point towards the (circumference of the) circle, (one) on each (side)
of the least (straight-line).

Let $ABC$ be a circle, and let some point $D$ be taken outside $ABC$,
and from it let some straight-lines, $DA$, $DE$, $DF$, and $DC$, be
drawn through (the circle), and let $DA$ be through the center. I say 
that for the straight-lines radiating towards the concave (part of the)
circumference, $AEFC$, the greatest is the one (passing) through the center,  (namely) $AD$,
and (that) $DE$ (is) greater than $DF$, and $DF$ than $DC$. For the straight-lines
radiating towards the convex (part of the) circumference, $HLKG$, the
least is the one between the point and the diameter $AG$, (namely) $DG$, 
and  a (straight-line) nearer to the least (straight-line) $DG$ is
always less than one farther away, (so that) $DK$ (is less) than $DL$, and $DL$ than than
$DH$.

For let the center of the circle be found [Prop.~3.1], and let it be (at point) $M$
[Prop.~3.1]. And let $ME$, $MF$, $MC$, $MK$, $ML$, and $m\kern -.7pt h$ be joined.

And since $AM$ is equal to $EM$, let $MD$ be added to both. Thus, $AD$ is equal to $EM$ and $MD$. But, $EM$ and $MD$ is greater than
$ED$ [Prop.~1.20]. Thus, $AD$ is also greater than $ED$. Again, since $ME$ is
equal to $MF$, and $MD$ (is) common, the (straight-lines) $EM$, $MD$ are thus
equal to $FM$,
$MD$. And angle $EMD$ is greater than angle $FMD$.$^\ddag$
Thus, the
base $ED$ is greater than the base $FD$ [Prop.~1.24]. 
So, similarly,
we can show that $FD$ is also greater than $CD$. Thus, $AD$ (is) the greatest
(straight-line), and $DE$ (is) greater than $DF$, and $DF$ than $DC$.

And since $MK$ and $KD$ is greater than $MD$ [Prop. 1.20], and $MG$ (is)
equal to $MK$, the remainder $KD$ is thus greater than the remainder $GD$.
So $GD$ is less than $KD$. And since in triangle $MLD$, the two internal straight-lines
$MK$ and $KD$ were constructed on one of the sides, $MD$, (then) $MK$ and $KD$
are thus less than $ML$ and $LD$  [Prop.~1.21]. And $MK$ (is) equal to $ML$.
Thus, the remainder $DK$ is less than the remainder $DL$. So, similarly, we
can show that $DL$ is also less than $DH$. Thus, $DG$ (is) the least (straight-line),
and $DK$ (is) less than $DL$, and $DL$ than $DH$.

\epsfysize=2.5in
\centerline{\epsffile{Book03/fig08e.eps}}

I also say that only two equal (straight-lines) will radiate from point $D$ towards (the
circumference of) the circle, (one) on each (side) on the least (straight-line),
$DG$. Let the  angle $DMB$, equal to angle $KMD$, be constructed
on the straight-line $MD$, at the point $M$ on it [Prop.~1.23], and  let $DB$ be
joined. And since $MK$ is equal to $MB$, and $MD$ (is) common, the
two (straight-lines) $KM$, $MD$ are equal to the two (straight-lines) $BM$, $MD$,
respectively. And angle $KMD$ (is) equal to angle $BMD$. Thus, the base
$DK$ is equal to the base $DB$ [Prop.~1.4]. [So] I say that another (straight-line)
equal to $DK$ will not radiate towards the (circumference of the) circle
from point $D$. For, if possible, let (such a straight-line) radiate, and let
it be $DN$. Therefore, since $DK$ is equal to $DN$, but $DK$ is equal to $DB$, (then) $DB$
is thus also equal to $DN$, (so that) a (straight-line) nearer to the least (straight-line) $DG$ [is]
equal to one further away. The very thing was shown (to be)  impossible. 
Thus, not more than two equal (straight-lines) will radiate towards
(the circumference of) circle $ABC$ from point $D$, (one) on each side
of the least (straight-line) $DG$.

Thus, if some point is taken outside a circle, and some straight-lines are
drawn from the point to the (circumference of the) circle, one of which (passes) through the
center, the remainder  (being) random, (then) for the straight-lines
radiating towards the concave (part of the) circumference, the
greatest is  that (passing) through the center. For the others, a (straight-line)
nearer to the (straight-line) through the center is always greater than one
further away. For the straight-lines radiating towards the convex (part of
the) circumference, the least is that between the point and the
diameter. For the others, a (straight-line) nearer to the least
(straight-line) is always less than one further away. And only
two equal (straight-lines) will radiate from the point towards the (circumference of the) circle, (one) on each (side)
of the least (straight-line).  (Which is) the very thing it was required to show.}
\end{Parallel}
{\footnotesize \noindent$^\dag$ Presumably, in an angular sense.\\
$\ddag$  This is not
proved, except by reference to the figure.} 

%%%%%%
% Prop 3.9
%%%%%%
\pdfbookmark[1]{Proposition 3.9}{pdf3.9}
\begin{Parallel}{}{} 
\ParallelLText{
\begin{center}
{\large \ggn{9}.}
\end{center}\vspace*{-7pt}

\gr{>E`an k'uklou lhfj~h| ti shme~ion >ent'oc, >apo d`e to~u shme'iou pr`oc
t`on k'uklon prosp'iptwsi ple'iouc >`h d'uo >'isai e>uje~iai, t`o lhfj`en shme~ion k'entron >est`i to~u k'uklou.}

\gr{>'Estw k'ukloc <o ABG, >ent`oc d`e a\kern -.7pt >uto~u shme~ion t`o D, ka`i >ap`o
to~u D pr`oc t`on ABG k'uklon prospipt'etwsan ple'iouc >`h d'uo >'isai
e>uje~iai a<i DA, DB, DG; l'egw, <'oti t`o D shme~ion k'entron >est`i to~u
ABG k'uklou.}\\

\epsfysize=2.2in
\centerline{\epsffile{Book03/fig09g.eps}}

\gr{>Epeze'uqjwsan g`ar a<i AB, BG ka`i tetm\kern -.7pt 'hsjwsan d'iqa kat`a t`a E, Z shme~ia, ka`i >epizeuqje~isai a<i ED, ZD di'hqjwsan >ep`i t`a
H, K, J, L shme~ia.}

\gr{>Epe`i o>~un >'ish >est`in <h AE t~h| EB, koin`h d`e <h ED,
d'uo d`h a<i AE, ED d'uo ta~ic BE, ED >'isai e>is'in; ka`i b'asic <h
DA b'asei t~h| DB >'ish; gwn'ia >'ara <h <up`o AED gwn'ia|
t~h| <up`o BED >'ish >est'in; >orj`h >'ara <ekat'era t~wn <up`o AED,
BED gwni~wn; <h HK >'ara t`hn AB t'emnei d'iqa ka`i pr`oc >orj'ac. ka`i
>epe'i, >e`an >en k'uklw| e>uje~i'a tic e>uje~i'an tina d'iqa te ka`i pr`oc
>orj`ac t'emnh|, >ep`i t~hc temno'ushc >est`i t`o k'entron to~u k'uklou,
>ep`i t~hc HK >'ara >est`i t`o k'entron to~u k'uklou. di`a t`a
a\kern -.7pt >ut`a d`h ka`i >ep`i t~hc JL >esti t`o k'entron to~u ABG k'uklou.
ka`i o>ud`en <'eteron koin`on >'eqousin a<i HK, JL e>uje~iai >`h t`o D
shme~ion; t`o D >'ara shme~ion k'entron >est`i to~u ABG
k'uklou.}

\gr{>E`an >'ara k'uklou lhfj~h| ti shme~ion >ent'oc, >ap`o d`e to~u shme'iou pr`oc
t`on k'uklon prosp'iptwsi ple'iouc >`h d'uo >'isai e>uje~iai, t`o lhfj`en shme~ion k'entron >est`i to~u k'uklou; <'oper >'edei de~ixai.}}

\ParallelRText{
\begin{center}
{\large Proposition 9}
\end{center}

If  some point is taken inside a circle, and more than two equal straight-lines
radiate from the point towards the (circumference of the) circle, (then) the
point taken  is the center of the circle.

Let $ABC$ be a circle, and $D$ a point inside it, and let more than two equal
straight-lines, $DA$, $DB$, and $DC$, radiate from $D$ towards (the circumference of) circle
$ABC$. I say that point $D$ is the center of circle $ABC$.

\epsfysize=2.2in
\centerline{\epsffile{Book03/fig09e.eps}}

For let $AB$ and $BC$ be joined, and (then) be cut in half at points $E$ and $F$ 
(respectively) [Prop.~1.10]. And $ED$ and $FD$ being joined, let them be
drawn through to points $G$, $K$, $H$, and $L$.

Therefore, since $AE$ is equal to $EB$, and $ED$ (is) common, the two (straight-lines) $AE$, $ED$ are equal to the two (straight-lines) $BE$, $ED$ (respectively). And the 
base $DA$ (is) equal to the base $DB$. Thus, angle $AED$ is equal to
angle $BED$ [Prop.~1.8]. Thus, angles $AED$ and $BED$ (are) each right-angles [Def.~1.10]. Thus, $GK$ cuts $AB$ in half, and at right-angles. And since,
if some straight-line in a circle cuts some (other) straight-line in half,
and at right-angles, (then) the center of the circle is on the former (straight-line)
[Prop.~3.1~corr.], the center of the circle is thus on $GK$. So, for the
same (reasons), the center of circle $ABC$ is also on $HL$. And the straight-lines
$GK$ and $HL$ have no  common (point) other than point $D$. Thus, point
$D$ is the center of circle $ABC$.

Thus, if some point is taken inside a circle, and more than two equal straight-lines
radiate from the point towards the (circumference of the) circle, (then) the
point taken  is the center of the circle. (Which is) the very thing it was
required to show.}
\end{Parallel}

%%%%%%
% Prop 3.10
%%%%%%
\pdfbookmark[1]{Proposition 3.10}{pdf3.10}
\begin{Parallel}{}{} 
\ParallelLText{
\begin{center}
{\large \ggn{10}.}
\end{center}\vspace*{-7pt}

\gr{K'ukloc  k'uklon o>u t'emnei kat`a ple'iona shme~ia >`h d'uo.}

\gr{E>i g`ar dunat'on, k'ukloc <o ABG k'uklon t`on DEZ temn'etw kat`a ple'iona
shme~ia >`h d'uo t`a B, H, Z, J, ka`i >epizeuqje~isai a<i BJ, BH d'iqa temn'esjwsan kat`a t`a K, L shme~ia; ka`i >ap`o t~wn K, L ta~ic BJ, BH
pr`oc >orj`ac >aqje~isai a<i KG, LM di'hqjwsan >ep`i t`a A, E shme~ia.}\\~\\

\epsfysize=2.2in
\centerline{\epsffile{Book03/fig10g.eps}}

\gr{>Epe`i o>~un >en k'uklw| t~w| ABG e>uje~i'a tic <h AG e>uje~i'an tina
t`hn BJ d'iqa ka`i pr`oc >orj`ac t'emnei, >ep`i t~hc AG >'ara >est`i t`o k'entron to~u ABG k'uklou. p'alin, >epe`i >en k'uklw| t~w| a\kern -.7pt >ut~w|
t~w| ABG e>uje~i'a tic <h NX e>uje~i'an tina t`hn BH d'iqa ka`i pr`oc
>orj`ac t'emnei, >ep`i t~hc NX >'ara >est`i t`o k'entron to~u ABG k'uklou.
>ede'iqjh d`e ka`i >ep`i t~hc AG, ka`i kat> o>ud`en sumb'allousin
a<i AG, NX e>uje~iai >`h kat`a t`o O; t`o O >'ara shme~ion k'entron
>est`i to~u ABG k'uklou. <omo'iwc d`h de'ixomen, <'oti ka`i to~u DEZ k'uklou k'entron >est`i t`o O; d'uo >'ara k'uklwn temn'ontwn >all'hlouc
t~wn ABG, DEZ t`o a\kern -.7pt >ut'o >esti k'entron t`o O; 
<'oper >est`in >ad'unaton.}

\gr{O>uk >'ara k'ukloc  k'uklon t'emnei kat`a ple'iona shme~ia >`h d'uo;
<'oper >'edei de~ixai.}}

\ParallelRText{
\begin{center}
{\large Proposition 10}
\end{center}

A circle does not cut a(nother) circle at more than two points.

For, if possible, let the circle $ABC$ cut the circle $DEF$ at more than
two points, $B$, $G$, $F$, and $H$. And $BH$ and $BG$ being joined,
let them (then) be cut in half at points $K$ and $L$ (respectively). And $KC$ and $LM$
being drawn at right-angles to $BH$ and $BG$ from $K$ and $L$ (respectively) [Prop.~1.11], let them (then) be
drawn through to points $A$ and $E$ (respectively).

\epsfysize=2.2in
\centerline{\epsffile{Book03/fig10e.eps}}

Therefore, since in circle $ABC$ some straight-line $AC$ cuts some (other)
straight-line $BH$ in half, and at right-angles, the center of circle $ABC$
is thus on $AC$ [Prop.~3.1~corr.]. Again, since in the same circle $ABC$
some straight-line $NO$ cuts some (other straight-line) $BG$ in half, and
at right-angles, the center of circle $ABC$ is thus on $NO$ [Prop.~3.1~corr.].
And it was also shown (to be) on $AC$. And the straight-lines $AC$ and $NO$
meet at no other (point) than $P$. Thus, point $P$ is the center of circle $ABC$.
So, similarly, we can show that $P$ is also the center of circle $DEF$.
Thus,  two circles cutting one another, $ABC$ and $DEF$, have the same center
$P$. The very thing is impossible [Prop.~3.5].

Thus, a circle does not cut a(nother) circle at more than two points. (Which is)
the very thing it was required to show.}
\end{Parallel}

%%%%%%
% Prop 3.11
%%%%%%
\pdfbookmark[1]{Proposition 3.11}{pdf3.11}
\begin{Parallel}{}{} 
\ParallelLText{
\begin{center}
{\large \ggn{11}.}
\end{center}\vspace*{-7pt}

\gr{>E`an d'uo k'ukloi >ef'aptwntai >all'hlwn >ent'oc, ka`i lhfj~h| a\kern -.7pt >ut~wn
t`a k'entra, <h >ep`i t`a k'entra a\kern -.7pt >ut~wn >epizeugnum'enh e>uje~ia ka`i
>ekballom'enh >ep`i t`hn sunaf`hn pese~itai t~wn k'uklwn.}

\gr{D'uo g`ar k'ukloi o<i ABG, ADE >efapt'esjwsan >all'hlwn >ent`oc kat`a t`o
A shme~ion, ka`i e>il'hfjw to~u m`en ABG k'uklou k'entron t`o Z, to~u d`e
ADE t`o H; l'egw, <'oti <h >ap`o to~u H >ep`i t`o Z >epizeugnum'enh
e>uje~ia >ekballom'enh >ep`i t`o A pese~itai.}

\gr{m\kern -.7pt `h g'ar, >all> e>i dunat'on, pipt'etw <wc <h ZHJ, ka`i
>epeze'uqjwsan a<i AZ, AH.}

\gr{>Epe`i o>~un a<i AH, HZ t~hc ZA, tout'esti t~hc ZJ, me'izon'ec e>isin, 
koin`h >afh|r'hsjw <h ZH; loip`h >'ara <h AH loip~hc t~hc HJ me'izwn >est'in. >'ish d`e <h AH t~h| HD; ka`i <h HD >'ara t~hc HJ me'izwn >est`in
<h >el'attwn t~hc me'izonoc; <'oper >est`in >ad'unaton; o>uk >'ara <h >ap`o
to~u Z >ep`i t`o H >epizeugnum'enh e>uje`ia >ekt`oc pese~itai;
kat`a t`o A >'ara >ep`i t~hc sunaf~hc pese~itai.}\\~\\

\epsfysize=2.2in
\centerline{\epsffile{Book03/fig11g.eps}}

\gr{>E`an >'ara d'uo k'ukloi >ef'aptwntai >all'hlwn >ent'oc, [ka`i lhfj~h| a\kern -.7pt >ut~wn
t`a k'entra], <h >ep`i t`a k'entra a\kern -.7pt >ut~wn >epizeugnum'enh e>uje~ia [ka`i
>ekballom'enh] >ep`i t`hn sunaf`hn pese~itai t~wn k'uklwn; <'oper
>'edei de~ixai.}}

\ParallelRText{
\begin{center}
{\large Proposition 11}
\end{center}

If two circles  touch one another internally, and their
centers are found, (then)  the straight-line joining their
centers, being produced, will fall upon the point of union of
the circles.

For let two circles, $ABC$ and $ADE$, touch one another internally
at point $A$, and let the center $F$ of circle $ABC$ be
found [Prop.~3.1], and (the center) $G$ of (circle) $ADE$ [Prop.~3.1].
I say that the straight-line joining $G$ to $F$, being produced, will fall on $A$.

For (if) not (then), if possible, let it fall like $FGH$ (in the figure), and
let $AF$ and $AG$ be joined.

Therefore, since $AG$ and $GF$ is greater than $FA$, that is to
say $FH$ [Prop.~1.20], let $FG$ be taken from both.
Thus, the remainder $AG$ is greater than the remainder $GH$.
And $AG$ (is) equal to $GD$. Thus, $GD$ is also greater than $GH$, the lesser
than the greater. The very thing is impossible. Thus, the straight-line
joining $F$ to $G$ will not fall outside (one circle but inside the other). Thus, it will fall upon the point of
union (of the circles) at point $A$. 

\epsfysize=2.2in
\centerline{\epsffile{Book03/fig11e.eps}}

Thus, if two circles  touch one another internally, [and their
centers are found], (then)  the straight-line joining their
centers, [being produced], will fall upon the point of union of
the circles. (Which is) the very thing it was required to show.}
\end{Parallel}

%%%%%%
% Prop 3.12
%%%%%%
\pdfbookmark[1]{Proposition 3.12}{pdf3.12}
\begin{Parallel}{}{} 
\ParallelLText{
\begin{center}
{\large \ggn{12}.}
\end{center}\vspace*{-7pt}

\gr{>E`an d'uo k'ukloi >ef'aptwntai >all'hlwn >ekt'oc, <h >ep`i t`a k'entra
a\kern -.7pt >ut~wn >epizeugnum'enh di`a t~hc >epaf~hc >ele'usetai.}\\

\epsfysize=2.5in
\centerline{\epsffile{Book03/fig12g.eps}}

\gr{D'uo g`ar k'ukloi o<i ABG, ADE >efapt'esjwsan >all'hlwn >ekt`oc
kat`a t`o A shme~ion, ka`i e>il'hfjw to~u m`en ABG k'entron t`o Z,
to~u d`e ADE t`o H; l'egw, <'oti <h >ap`o to~u Z >ep`i t`o H
>epizeugnum'enh e>uje~ia di`a t~hc kat`a t`o A >epaf~hc >ele'usetai.}

\gr{m\kern -.7pt `h g'ar, >all> e>i dunat'on, >erq'esjw <wc <h ZGDH, ka`i >epeze'uqjwsan
a<i AZ, AH.}

\gr{>Epe`i o>~un t`o Z shme~ion k'entron >est`i to~u ABG k'uklou,
>'ish >est`in <h ZA t~h| ZG. p'alin, >epe`i t`o H shme~ion k'entron
>est`i to~u ADE k'uklou, >'ish >est`in <h HA t~h| HD. >ede'iqjh d`e
ka`i <h ZA t~h| ZG >'ish; a<i >'ara ZA, AH ta~ic ZG, HD >'isai e>is'in; <'wste <'olh <h ZH t~wn ZA, AH me'izwn >est'in; >all`a ka`i >el'attwn;
<'oper >est`in >ad'unaton. o>uk >'ara <h >ap`o to~u Z >ep`i t`o H
>epizeugnum'enh e>uje~ia di`a t~hc kat`a t`o A >epaf~hc o>uk >ele'usetai;
di> a\kern -.7pt >ut~hc >'ara.}

\gr{>E`an >'ara d'uo k'ukloi >ef'aptwntai >all'hlwn >ekt'oc, <h >ep`i t`a k'entra
a\kern -.7pt >ut~wn >epizeugnum'enh [e>uje~ia] di`a t~hc >epaf~hc >ele'usetai; <'oper
>'edei de~ixai.}}

\ParallelRText{
\begin{center}
{\large Proposition 12}
\end{center}

If two circles touch one another externally, (then) the (straight-line)
joining their centers will go through the point of union.

\epsfysize=2.5in
\centerline{\epsffile{Book03/fig12e.eps}}

For let two circles, $ABC$ and $ADE$, touch one another externally at point $A$,
and let the center $F$ of $ABC$ be found [Prop.~3.1], and
(the center) $G$ of $ADE$ [Prop.~3.1]. I say that the straight-line joining
$F$ to $G$ will go through the point of union at $A$.

For (if) not (then), if possible, let it go like $FCDG$ (in the figure), and let $AF$ and
$AG$ be joined.

Therefore, since point $F$ is the center of circle $ABC$, $FA$ is equal to $FC$.
Again, since point $G$ is the center of circle $ADE$, $GA$ is equal to $GD$.
And $FA$ was also shown (to be)  equal to $FC$. Thus, the (straight-lines)
$FA$ and $AG$ are equal to the (straight-lines) $FC$ and $GD$. So the whole
of $FG$ is greater than $FA$ and $AG$. But, (it is) also less [Prop.~1.20].
The very thing is impossible. Thus, the straight-line joining $F$ to $G$
cannot not go through the point of union at $A$. Thus, (it will go) through it.

Thus, if two circles touch one another externally, (then) the [straight-line]
joining their centers will go through the point of union. (Which is)
the very thing it was required to show.}
\end{Parallel}

%%%%%%
% Prop 3.13
%%%%%%
\pdfbookmark[1]{Proposition 3.13}{pdf3.13}
\begin{Parallel}{}{} 
\ParallelLText{
\begin{center}
{\large \ggn{13}.}
\end{center}\vspace*{-7pt}

\gr{K'ukloc k'uklou o>uk >ef'aptetai kat`a ple'iona shme~ia
>`h  kaj> <'en, >e'an te >ent`oc >e'an te >ekt`oc >ef'apthtai.}

\epsfysize=2.2in
\centerline{\epsffile{Book03/fig13g.eps}}

\gr{E>i g`ar dunat'on, k'ukloc <o ABGD k'uklou to~u EBZD >efapt'esjw
pr'oteron >ent`oc kat`a ple'iona shme~ia >`h <`en t`a D, B.}

\gr{Ka`i e>il'hfjw to~u m`en ABGD k'uklou k'entron t`o H, to~u d`e EBZD
t`o J.}

\gr{<H >'ara >ap`o to~u H >ep`i t`o J >epizeugnum'enh >ep`i t`a B, D pese~itai.
pipt'etw <wc <h BHJD. ka`i >epe`i t`o H shme~ion k'entron >est`i to~u
ABGD k'uklou, >'ish >est`in <h BH t~h| HD; me'izwn >'ara <h BH t~hc
JD; poll~w| >'ara me'izwn <h BJ t~hc JD. p'alin, >epe`i t`o J shme~ion
k'entron >est`i to~u EBZD k'uklou, >'ish >est`in <h BJ t~h| JD;
>ede'iqjh d`e a\kern -.7pt >ut~hc ka`i poll~w| me'izwn; <'oper >ad'unaton;
o>uk >'ara k'ukloc k'uklou >ef'aptetai >ent`oc kat`a ple'iona shme~ia
>`h <'en.}

\gr{L'egw d'h, <'oti o>ud`e >ekt'oc.}

\gr{E>i g`ar dunat'on, k'ukloc <o AGK k'uklou to~u ABGD >efapt'esjw >ekt`oc
kat`a ple'iona shme~ia >`h <`en t`a A, G, ka`i >epeze'uqjw <h AG.}

\gr{>'Epe`i o>~un k'uklwn t~wn ABGD, AGK e>'ilhptai >ep`i
t~hc perifere'iac <ekat'erou d'uo tuq'onta shme~ia t`a A, G, <h
>ep`i t`a shme~ia >epizeugnum'enh e>uje~ia >ent`oc <ekat'erou pese~itai;
>all`a to~u m`en ABGD >ent`oc >'epesen, to~u d`e AGK >ekt'oc; <'oper
>'atopon; o>uk >'ara k'ukloc k'uklou >ef'aptetai >ekt`oc kat`a ple'iona
shme~ia >`h <'en. >ede'iqjh d'e, <'oti o>ud`e >ent'oc.}

\gr{K'ukloc >'ara k'uklou o>uk >ef'aptetai kat`a ple'iona shme~ia >`h
[kaj>] <'en, >e'an te >ent`oc >e'an te >ekt`oc >ef'apthtai; <'oper >'edei
de~ixai.}}

\ParallelRText{
\begin{center}
{\large Proposition 13}
\end{center}

A circle does not touch a(nother) circle at more than one point, whether they
touch internally or externally.

\epsfysize=2.2in
\centerline{\epsffile{Book03/fig13e.eps}}

For, if possible, let circle $ABDC$$^{\,\dag}$ touch circle $EBFD$---first of all, internally---at more than one point, $D$ and $B$.

And let the center $G$ of circle $ABDC$ be found [Prop.~3.1], and
(the center) $H$ of $EBFD$ [Prop.~3.1].

Thus, the (straight-line) joining $G$ and $H$ will fall on $B$ and $D$ [Prop.~3.11].
Let it fall like $BGHD$ (in the figure). And since point $G$ is the center of
circle $ABDC$, $BG$ is  equal to $GD$. Thus, $BG$ (is) greater than $HD$.
Thus, $BH$ (is) much greater than $HD$. Again, since point $H$ is the
center of circle $EBFD$, $BH$ is equal to $HD$. But it was also shown
(to be) much greater than it. The very thing (is) impossible.
Thus, a circle does not touch a(nother) circle internally at more than one point.

So, I say that neither (does it touch) externally (at more than one point).

For, if possible, let circle $ACK$ touch circle $ABDC$ externally at more
than one point, $A$ and $C$. And let $AC$ be joined.

Therefore, since  two points, $A$ and $C$,
be taken at random on the circumference
of each of the circles $ABDC$ and $ACK$, the straight-line joining
the points will fall inside each (circle) [Prop.~3.2]. But, it fell
inside $ABDC$, and outside $ACK$ [Def.~3.3]. The very thing (is) absurd.
Thus, a circle does not touch a(nother) circle externally  at more than one
point. And it was shown that neither (does it) internally.

Thus, a circle does not touch a(nother) circle at more than one point, whether they touch internally or externally. (Which is) the very thing it was required to
show.}
\end{Parallel}
{\footnotesize \noindent$^\dag$ The Greek text has ``$ABCD$'', which is obviously a mistake.}

%%%%%%
% Prop 3.14
%%%%%%
\pdfbookmark[1]{Proposition 3.14}{pdf3.14}
\begin{Parallel}{}{} 
\ParallelLText{
\begin{center}
{\large \ggn{14}.}
\end{center}\vspace*{-7pt}

\gr{>En k'uklw| a<i >'isai e>uje~iai >'ison >ap'eqousin >ap`o to~u k'entrou, ka`i
a<i >'ison >ap'eqousai >ap`o to~u k'entrou >'isai >all'hlaic e>is'in.}\\

\epsfysize=2.2in
\centerline{\epsffile{Book03/fig14g.eps}}

\gr{>'Estw k'ukloc <o ABGD, ka`i >en a\kern -.7pt >ut~w| >'isai e>uje~iai >'estwsan a<i
AB, GD; l'egw, <'oti a<i AB, GD >'ison >ap'eqousin >ap`o to~u
k'entrou.}

\gr{E>il'hfjw g`ar t`o k'enton to~u ABGD k'uklou ka`i >'estw t`o E, ka`i
>ap`o to~u E >ep`i t`ac AB, GD k'ajetoi >'hqjwsan a<i EZ, EH, ka`i
>epeze'uqjwsan a<i AE, EG.}

\gr{>Epe`i o>~un e>uje~i'a tic d`ia to~u k'entrou <h EZ e>uje~i'an tina m\kern -.7pt `h
di`a to~u k'entrou t`hn AB pr`oc >orj`ac t'emnei, ka`i d'iqa a\kern -.7pt >ut`hn
t'emnei. >'ish >'ara <h AZ t~h| ZB; dipl~h >'ara <h AB t~hc AZ. di`a t`a
a\kern -.7pt >ut`a d`h ka`i <h GD t~hc GH >esti dipl~h; ka'i >estin >'ish <h AB
t~h| GD; >'ish >'ara ka`i <h AZ t~h| GH. ka`i >epe`i >'ish >est`in <h AE
t~h| EG, >'ison ka`i t`o >ap`o t~hc AE t~w| >ap`o t~hc EG.
>all`a t~w| m`en >ap`o t~hc AE >'isa t`a >ap`o t~wn AZ, EZ; >orj`h g`ar <h pr`oc t~w| Z gwn'ia; t~w| d`e >ap`o t~hc EG >'isa t`a >ap`o t~wn EH, HG;
>orj`h g`ar <h pr`oc t~w| H gwn'ia; t`a >'ara >ap`o t~wn AZ, ZE >'isa
>est`i to~ic >ap`o t~wn GH, HE, <~wn t`o >ap`o t~hc AZ >'ison >est`i
t~w| >ap`o t~hc GH; >'ish g'ar >estin <h AZ t~h| GH; loip`on >'ara
t`o >ap`o t~hc ZE t~w| >ap`o t~hc EH >'ison >est'in; >'ish >'ara <h
EZ t~h| EH. >en d`e k'uklw| >'ison >ap'eqein >ap`o to~u k'entrou
e>uje~iai l'egontai, <'otan a<i >ap`o to~u k'entrou >ep> a\kern -.7pt >ut`ac
k'ajetoi >ag'omenai >'isai >~wsin; a<i >'ara AB, GD >'ison >ap'eqousin
>ap`o to~u k'entrou.}

\gr{>All`a d`h a<i AB, GD e>uje~iai >'ison >apeq'etwsan >ap`o to~u k'entrou,
tout'estin >'ish >'estw <h EZ t~h| EH. l'egw, <'oti >'ish >est`i ka`i
<h AB t~h| GD.}

\gr{T~wn g`ar a\kern -.7pt >ut~wn kataskeuasj'entwn <omo'iwc de'ixomen, <'oti dipl~h
>estin <h m`en AB t~hc AZ, <h d`e GD t~hc GH; ka`i >epe`i >'ish >est`in <h AE t~h| GE, >'ison >est`i t`o >ap`o t~hc 
AE t~w| >ap`o t~hc GE; >all`a t~w|
m`en >ap`o t~hc AE >'isa >est`i t`a >ap`o t~wn EZ, ZA, t~w| d`e >ap`o
t~hc GE 
>'isa t`a >ap`o t~wn EH, HG. t`a >'ara >ap`o t~wn
EZ, ZA >'isa >est`i to~ic >ap`o t~wn EH, HG; <~wn t`o >ap`o t~hc
EZ t~w| >ap`o t~hc EH >estin >'ison; >'ish g`ar <h EZ t~h| EH; loip`on >'ara
t`o >ap`o t~hc AZ >'ison >est`i t~w| >ap`o t~hc GH; >'ish >'ara <h AZ
t~h| GH; ka'i >esti t~hc m`en AZ dipl~h <h AB, t~hc d`e GH dipl~h <h GD;
>'ish >'ara <h AB t~h| GD.}

\gr{>En k'uklw| >'ara a<i >'isai e>uje~iai >'ison >ap'eqousin >ap`o to~u k'entrou, ka`i
a<i >'ison >ap'eqousai >ap`o to~u k'entrou >'isai >all'hlaic e>is'in;
<'oper >'edei de~ixai.}}

\ParallelRText{
\begin{center}
{\large Proposition 14}
\end{center}

In  a circle, equal straight-lines are equally far from the center, and
(straight-lines) which are equally far from the center are equal to one another.

\epsfysize=2.2in
\centerline{\epsffile{Book03/fig14e.eps}}

Let $ABDC$$^{\,\dag}$ be a circle, and let $AB$ and $CD$ be equal straight-lines within it.
I say that $AB$ and $CD$ are equally far from the center.

For let the center of circle $ABDC$ be found [Prop.~3.1], and
let it be (at) $E$. And let $EF$ and $EG$ be drawn from (point) $E$, perpendicular
to $AB$ and $CD$ (respectively) [Prop.~1.12]. And let $AE$ and $EC$ be joined.

Therefore, since some straight-line, $EF$, through the center (of the circle),
cuts some (other) straight-line, $AB$, not through the center, at right-angles,
it also cuts it in half [Prop.~3.3]. Thus, $AF$ (is) equal to $FB$. Thus, 
$AB$ (is) double $AF$. So, for the same (reasons), $CD$ is also double $CG$.
And $AB$ is equal to $CD$. Thus, $AF$ (is) also equal to $CG$. And since
$AE$ is equal to $EC$, the (square) on $AE$ (is) also equal to the (square)
on $EC$. But, the (sum of the squares) on $AF$
and $EF$ (is) equal to the (square) on $AE$. For the angle at $F$ (is) a right-angle [Prop.~1.47]. And the
(sum of the squares) on $EG$ and $GC$ (is) equal to the (square) on $EC$. For the angle at $G$ (is) a right-angle [Prop.~1.47]. Thus, the (sum of the squares)
on $AF$ and $FE$ is equal to the (sum of the squares) on $CG$ and $GE$, 
of which
the (square) on $AF$ is equal to the (square) on $CG$. For $AF$ is equal to
$CG$. Thus, the remaining (square) on $FE$ is equal to the (remaining square) on  $EG$. Thus, $EF$ (is) equal to $EG$. And straight-lines in a circle are said to be equally far from the center when perpendicular (straight-lines)
which are drawn to them from the center  are equal [Def.~3.4]. Thus, $AB$ and
$CD$ are equally far from the center.

So, let the straight-lines $AB$ and $CD$ be equally far from the center. That is to say, let $EF$ be equal to $EG$. I say that $AB$ is also equal to $CD$.

For, with the same construction,  we can, similarly, show that
$AB$ is double $AF$, and $CD$ (double) $CG$. And since $AE$ is equal to $CE$, the (square) on $AE$ is equal to the (square) on $CE$. But, the (sum of the squares)
on $EF$ and $FA$ is equal to the (square) on $AE$ [Prop.~1.47]. And the (sum of
the squares) on $EG$ and $GC$ (is) equal to the (square) on $CE$ [Prop.~1.47]. Thus, the (sum of the squares) on $EF$ and $FA$ is equal to the (sum of the squares) on $EG$ and $GC$, of which the (square) on $EF$ is equal to the (square)
on $EG$. For $EF$ (is) equal to $EG$. Thus, the remaining (square) on  $AF$ is equal to the (remaining square) on  $CG$. Thus, $AF$ (is) equal to $CG$. And $AB$ is double $AF$, and
$CD$ double $CG$. Thus, $AB$ (is) equal to $CD$.

Thus, in a circle, equal straight-lines are equally far from the center, and
(straight-lines) which are equally far from the center are equal to one another.
(Which is) the very thing it was required to show.}
\end{Parallel}
{\footnotesize \noindent$^\dag$ The Greek text has ``$ABCD$'', which is obviously a
mistake.}

%%%%%%
% Prop 3.15
%%%%%%
\pdfbookmark[1]{Proposition 3.15}{pdf3.15}
\begin{Parallel}{}{} 
\ParallelLText{
\begin{center}
{\large \ggn{15}.}
\end{center}\vspace*{-7pt}

\gr{>En k'uklw| meg'isth m`en <h di'ametroc, t~wn d`e >'allwn >ae`i <h >'eggion to~u k'entrou t~hc >ap'wteron me'izwn >est'in.}

\gr{>'Estw k'ukloc <o ABGD, di'ametroc d`e a\kern -.7pt >uto~u >'estw <h AD, k'entron d`e t`o E, ka`i >'eggion m`en t~hc AD diam'etrou >'estw <h BG, >ap'wteron d`e <h ZH; l'egw, <'oti meg'isth m'en >estin <h AD, me'izwn d`e <h BG t~hc ZH.}

\gr{>'Hqjwsan g`ar >ap`o to~u E k'entrou >ep`i t`ac BG, ZH k'ajetoi a<i EJ, EK.
ka`i >epe`i >'eggion m`en to~u k'entrou >est`in <h BG, >ap'wteron d`e
<h ZH, me'izwn >'ara <h EK t~hc EJ. ke'isjw t~h| EJ >'ish <h EL,
ka`i di`a to~u L t~h| EK pr`oc >orj`ac >aqje~isa <h LM di'hqjw >ep`i
t`o N, ka`i >epeze'uqjwsan a<i ME, EN, ZE, EH.}

\gr{Ka`i >epe`i >'ish >est`in <h EJ t~h| EL, >'ish >est`i ka`i <h BG t~h| MN. p'alin, >epe`i >'ish >est`in <h m`en AE t~h| EM, <h d`e ED t~h| EN, <h >'ara
AD ta~ic ME, EN >'ish >est'in. >all> a<i m`en ME, EN t~hc MN me'izon'ec
e>isin [ka`i <h AD t~hc MN me'izwn >est'in], >'ish d`e <h MN t~h| BG;
<h AD >'ara t~hc BG me'izwn >est'in. ka`i >epe`i d'uo a<i ME, EN d'uo
ta~ic ZE, EH >'isai e>is'in, ka`i gwn'ia <h <up`o MEN gwn'iac t~hc <up`o
ZEH me'izwn [>est'in], b'asic >'ara <h MN b'asewc t~hc ZH me'izwn
>est'in. >all`a <h MN t~h| BG >ede'iqjh >'ish [ka`i <h BG t~hc ZH
me'izwn >est'in]. meg'isth m`en >'ara <h AD di'ametroc, me'izwn
d`e <h BG t~hc ZH.}~\\~\\~\\~\\

\epsfysize=2.2in
\centerline{\epsffile{Book03/fig15g.eps}}

\gr{>En k'uklw| >'ara meg'isth m`en 'estin <h di'ametroc, t~wn d`e >'allwn >ae`i <h >'eggion to~u k'entrou t~hc >ap'wteron me'izwn >est'in; <'oper
>'edei de~ixai.}}

\ParallelRText{
\begin{center}
{\large Proposition 15}
\end{center}

In a circle, a diameter (is) the greatest (straight-line), and for 
the others, a (straight-line) nearer to the center is always greater than one further away.

Let $ABCD$ be a circle, and let $AD$ be its diameter, and $E$ (its) center.
And let $BC$ be nearer to the diameter $AD$,$^\dag$ and $FG$ further away. I say that
$AD$ is the greatest (straight-line), and $BC$ (is) greater than $FG$.

For let $EH$ and $EK$ be drawn from the center $E$, at right-angles
to $BC$ and $FG$ (respectively) [Prop.~1.12]. And since $BC$ is nearer to the center, and $FG$ further away, $EK$ (is) thus greater than $EH$ [Def.~3.5]. Let $EL$ be made equal to
$EH$ [Prop.~1.3]. And $LM$ being drawn through $L$, at right-angles to $EK$ [Prop.~1.11], 
let it be drawn through to $N$. And let $ME$, $EN$, $FE$, and $EG$
be joined.

And since $EH$ is equal to $EL$, $BC$ is also equal to $MN$ [Prop.~3.14]. Again,
since $AE$ is equal to $EM$, and $ED$ to $EN$, $AD$ is thus equal to $ME$ and $EN$.
But, $ME$ and $EN$ is greater than $MN$ [Prop.~1.20] [also $AD$ is greater than $MN$], and $MN$ (is) equal to $BC$. Thus, $AD$ is greater than $BC$. And since
the two (straight-lines) $ME$, $EN$ are equal to the two (straight-lines) $FE$, $EG$ (respectively),
and angle $MEN$ [is] greater than angle $FEG$,$^\ddag$ the base $MN$ is thus
greater than the base $FG$ [Prop.~1.24]. But, $MN$ was shown (to be) equal to $BC$ [(so) $BC$ is also greater than $FG$]. Thus, the diameter $AD$ (is) the
greatest (straight-line), and $BC$ (is) greater than $FG$.

\epsfysize=2.2in
\centerline{\epsffile{Book03/fig15e.eps}}

Thus, in a circle, a diameter (is) the greatest (straight-line), and for 
the others, a (straight-line) nearer to the center is always greater than one further away. (Which is) the very thing it was required to show.}
\end{Parallel}
{\footnotesize \noindent$^\dag$ Euclid should have said ``to the center'', rather than "to the diameter $AD$", since $BC$, $AD$ and $FG$ are not necessarily parallel.\\
$^\ddag$  This is not proved,
except by reference to the figure.}

%%%%%%
% Prop 3.16
%%%%%%
\pdfbookmark[1]{Proposition 3.16}{pdf3.16}
\begin{Parallel}{}{} 
\ParallelLText{
\begin{center}
{\large \ggn{16}.}
\end{center}\vspace*{-7pt}

\gr{<H t~h| diam'etrw| to~u k'uklou pr`oc >orj`ac >ap> >'akrac >agom'enh
>ekt`oc pese~itai to~u k'uklou, ka`i e>ic t`on metax`u t'opon t~hc te
e>uje'iac ka`i t~hc perifere'iac <et'era e>uje~ia o>u parempese~itai,
ka`i <h m`en to~u <hmikukl'iou gwn'ia <ap'ashc gwn'iac >oxe'iac
e>ujugr'ammou me'izwn >est'in, <h d`e loip`h >el'attwn.}

\gr{>'Estw k'ukloc <o ABG per`i k'entron t`o D ka`i di'ametron t`hn
AB; l'egw, <'oti <h >ap`o to~u A t~h| AB pr`oc >orj`ac >ap> >'akrac >agom'enh >ekt`oc pese~itai to~u k'uklou.}

\gr{m\kern -.7pt `h g'ar, >all> e>i dunat'on, pipt'etw >ent`oc <wc <h GA, ka`i >epeze'uqjw <h DG.}

\gr{>Epe`i >'ish >est`in <h DA t~h| DG, >'ish >est`i ka`i gwn'ia <h <up`o DAG gwn'ia| t~h| <up`o AGD. >orj`h d`e <h <up`o DAG; >orj`h >'ara ka`i <h
<up`o AGD; trig'wnou d`h to~u AGD a<i d'uo gwn'iai a<i <up`o DAG, AGD
d'uo >orja~ic >'isai e>is'in; <'oper >est`in >ad'unaton. o>uk >'ara <h >ap`o
to~u A shme'iou t~h| BA pr`oc >orj`ac >agom'enh >ent`oc pese~itai
to~u k'uklou. <omo'iwc d`h de~ixomen, <'oti o>ud> >ep`i t~hc perifere'iac;
>ekt`oc >'ara.}

\gr{Pipt'etw <wc <h AE; l'egw d'h, <'oti e>ic t`on metax`u t'opon t~hc te AE
e>uje'iac ka`i t~hc GJA perifere'iac <et'era e>uje~ia o>u parempese~itai.}

\gr{E>i g`ar dunat'on, parempipt'etw <wc <h ZA, ka`i >'hqjw
>ap`o to~u D shme'iou >ep`i t`hn ZA k'ajetoc <h DH. ka`i >epe`i >orj'h
>estin <h <up`o AHD, >el'attwn d`e >orj~hc <h <up`o DAH, me'izwn >'ara
<h AD t~hc DH. >'ish d`e <h DA t~h| DJ; me'izwn >'ara <h DJ t~hc
DH, <h >el'attwn t~hc me'izonoc; <'oper >est`in >ad'unaton. o>uk >'ara
e>ic t`on metax`u t'opon t~hc te e>uje'iac ka`i t~hc perifere'iac <et'era
e>uje~ia parempese~itai.}\\~\\~\\~\\

\epsfysize=2in
\centerline{\epsffile{Book03/fig16g.eps}}

\gr{L'egw, <'oti ka`i <h m`en to~u <hmikukl'iou gwn'ia <h perieqom'enh <up'o
te t~hc BA e>uje'iac ka`i t~hc GJA perifere'iac <ap'ashc gwn'iac
>oxe'iac e>ujugr'ammou me'izwn >est'in, <h d`e loip`h <h perieqom'enh
<up'o te t~hc GJA perifere'iac ka`i t~hc AE e>uje'iac <ap'ashc gwn'iac
>oxe'iac e>ujugr'ammou >el'attwn >est'in.}

\gr{E>i g`ar >est'i tic gwn'ia e>uj'ugrammoc me'izwn m`en t~hc perieqom'enhc
<up'o te t~hc BA e>uje'iac ka`i t~hc GJA perifere'iac, >el'attwn d`e t~hc
perieqom'enhc <up'o te t~hc GJA perifere'iac ka`i t`hc AE e>uje'iac, e>ic
t`on metax`u t'opon t~hc te GJA perifere'iac ka`i t~hc AE e>uje'iac
e>uje~ia parempese~itai, <'htic poi'hsei me'izona
m`en t~hc perieqom'enhc <up`o te t~hc BA e>uje'iac ka`i t~hc  GJA
perifere'iac <up`o e>ujei~wn perieqom'enhn, >el'attona d`e t~hc perieqom'enhc <up'o te t~hc GJA perifere'iac ka`i t~hc AE e>uje'iac.
o>u paremp'iptei d'e; o>uk >'ara t~hc perieqom'enhc gwn'iac <up'o
te t~hc BA e>uje'iac ka`i t~hc GJA perifere'iac >'estai me'izwn >oxe~ia
<up`o e>ujei~wn perieqom'enh, o>ud`e m\kern -.7pt `hn >el'attwn t~hc perieqom'enhc
<up'o te t~hc GJA perifere'iac ka`i t~hc AE e>uje'iac.}\\~\\

\begin{center}
{\large \gr{P'orisma}.}
\end{center}\vspace*{-7pt}

\gr{>Ek d`h to'utou faner'on, <'oti <h t~h| diam'etrw| to~u k'uklou pr`oc
>orj`ac >ap> >'akrac >agom'enh >ef'aptetai to~u k'uklou [ka`i <'oti e>uje~ia k'uklou kaj> <`en m'onon >ef'aptetai shme~ion, >epeid'hper ka`i <h kat`a d'uo
a\kern -.7pt >ut~w| sumb'allousa >ent`oc a\kern -.7pt >uto~u p'iptousa >ede'iqjh]; <'oper
>'edei de~ixai.}}

\ParallelRText{
\begin{center}
{\large Proposition 16}
\end{center}

A (straight-line) drawn at right-angles to the diameter of a circle, from its
end, will fall outside the circle. And another straight-line cannot be inserted
into the space between the (aforementioned) straight-line and the circumference. And the
  angle of the semi-circle is greater than any acute rectilinear angle whatsoever, 
and the remaining (angle is) less (than any acute rectilinear angle).

Let $ABC$ be a circle around the center $D$ and the diameter $AB$. I say that
the (straight-line) drawn from $A$, at right-angles to $AB$ [Prop~1.11], from its end, will fall
outside the circle.

For (if) not (then), if possible, let it fall inside, like $CA$ (in the figure), and let
$DC$ be joined.

Since $DA$ is equal to $DC$, angle $DAC$ is also equal to angle $ACD$ [Prop.~1.5].
And $DAC$ (is) a right-angle. Thus, $ACD$ (is) also a right-angle. So, in triangle
$ACD$, the two angles $DAC$ and $ACD$ are equal to two right-angles. 
The very thing is
impossible [Prop.~1.17]. Thus, the (straight-line) drawn from point $A$,
at right-angles to $BA$, will not fall inside the circle. So, similarly, we can show 
that neither (will it fall) on the circumference. Thus, (it will fall) outside
(the circle).

Let it fall like $AE$ (in the figure). So, I say that another straight-line cannot
be inserted into the space between the straight-line $AE$ and the
circumference $CHA$.

For, if possible, let it be inserted like $FA$ (in the figure), and let $DG$ 
be drawn from point $D$, perpendicular to $FA$ [Prop.~1.12]. And since $AGD$ is a right-angle, and
 $DAG$ (is) less than a right-angle, $AD$ (is) thus greater than $DG$  [Prop.~1.19]. And $DA$ (is) equal to $DH$. Thus, $DH$ (is)
greater than $DG$, the lesser than the greater. The very thing is impossible.
Thus, another straight-line cannot be inserted into the space between the
straight-line ($AE$) and the circumference.

\epsfysize=2in
\centerline{\epsffile{Book03/fig16e.eps}}

And I also say that the  semi-circular angle  contained by the straight-line
$BA$ and the circumference $CHA$ is greater than any acute rectilinear angle whatsoever,
and the remaining (angle) contained by the circumference $CHA$ and the straight-line
$AE$ is less than any acute rectilinear angle whatsoever.

For if any rectilinear angle is greater than the (angle) contained by the
straight-line $BA$ and the circumference $CHA$, or  less than the (angle)
contained by the circumference $CHA$ and the straight-line $AE$, (then)
a straight-line can be inserted into the space between the circumference 
$CHA$ and the straight-line $AE$---anything which will make (an angle)
contained by straight-lines greater than the angle contained by the straight-line $BA$ and the circumference $CHA$, or less than the (angle) contained 
by the circumference $CHA$ and the straight-line $AE$. But (such a straight-line)
cannot be inserted. Thus, an acute (angle) contained by straight-lines
cannot be greater than the angle contained by the straight-line $BA$ and
the circumference $CHA$, neither (can it be) less than the (angle) contained by
the circumference $CHA$ and the straight-line $AE$.\\

\begin{center}
{\large Corollary}
\end{center}\vspace*{-7pt}

So, from this, (it is) manifest that a (straight-line) drawn at right-angles
to the diameter of a circle, from its extremity, touches the circle [and that the
straight-line  touches the circle at a single point, inasmuch as it was also
shown that a (straight-line) meeting (the circle) at two (points) falls inside it [Prop.~3.2]\,]. (Which is) the very thing it was required to show. }
\end{Parallel}

%%%%%%
% Prop 3.17
%%%%%%
\pdfbookmark[1]{Proposition 3.17}{pdf3.17}
\begin{Parallel}{}{} 
\ParallelLText{
\begin{center}
{\large \ggn{17}.}
\end{center}\vspace*{-7pt}

\gr{>Ap`o to~u doj'entoc shme'iou to~u doj'entoc k'uklou >efaptom'enhn
e>uje~ian gramm\kern -.7pt `hn >agage~in.}

\gr{>'Estw t`o m`en doj`en shme~ion t`o A, <o d`e doje`ic k'ukloc <o BGD; de~i d`h >ap`o to~u A shme'iou to~u BGD k'uklou >efaptom'enhn
e>uje~ian gramm\kern -.7pt `hn >agage~in.}

\epsfysize=2.2in
\centerline{\epsffile{Book03/fig17g.eps}}

\gr{E>il'hfjw g`ar t`o k'entron to~u k'uklou t`o E, ka`i >epeze'uqjw <h AE,
ka`i k'entrw| m`en t~w| E diast'hmati d`e t~w| EA k'ukloc gegr'afjw <o AZH,
ka`i >ap`o to~u  D t~h| EA pr`oc >orj`ac >'hqjw <h DZ, ka`i
>epeze'uqjwsan a<i EZ, AB; l'egw, <'oti >ap`o to~u A shme'iou to~u
BGD k'uklou >efaptom'enh >~hktai <h AB.}

\gr{>Epe`i g`ar t`o E k'entron >est`i t~wn BGD, AZH k'uklwn, >'ish
>'ara >est`in <h m`en EA t~h| EZ, <h d`e ED t~h| EB; d'uo d`h a<i AE, EB d'uo ta~ic ZE, ED >'isai e>is'in; ka`i gwn'ian koin`hn peri'eqousi t`hn
pr`oc t~w| E; b'asic >'ara <h DZ b'asei t~h| AB >'ish >est'in, ka`i t`o DEZ tr'igwnon t~w| EBA trig'wnw| >'ison >est'in, ka`i a<i loipa`i
gwn'iai ta~ic loipa~ic gwn'iaic; >'ish >'ara <h <up`o EDZ t~h| <up`o EBA.
>orj`h d`e <h <up`o EDZ; >orj`h >'ara ka`i <h <up`o EBA. ka'i >estin
<h EB >ek to~u k'entrou; <h d`e t~h| diam'etrw| to~u k'uklou pr`oc
>orj`ac >ap> >'akrac >agom'enh >ef'aptetai to~u k'uklou; <h AB >'ara
>ef'aptetai to~u BGD k'uklou.}

\gr{>Ap`o to~u  >'ara doj'entoc shme'iou to~u A to~u doj'entoc k'uklou to~u BGD >efaptom'enh
e>uje~ia gramm\kern -.7pt `h >~hktai <h AB; <'oper >'edei poi~hsai.}}

\ParallelRText{
\begin{center}
{\large Proposition 17}
\end{center}

To draw a straight-line touching a given circle from a given point.

Let $A$ be the given point, and $BCD$ the given circle. So it is required to draw a
straight-line touching  circle $BCD$ from point $A$.

\epsfysize=2.2in
\centerline{\epsffile{Book03/fig17e.eps}}

For let the center $E$ of the circle be found [Prop.~3.1], and let $AE$ be joined. And let (the circle) $AFG$ be drawn with center
$E$ and radius $EA$. And let $DF$ be drawn from from (point) $D$,
at right-angles to $EA$ [Prop.~1.11]. And let $EF$ and $AB$ 
be  joined. I say that the (straight-line) $AB$ has been drawn from  point $A$
touching circle $BCD$.

For since $E$ is the center of circles $BCD$ and $AFG$, $EA$ is thus equal to $EF$,
and $ED$ to $EB$. So the two (straight-lines) $AE$, $EB$ are equal to
the two (straight-lines) $FE$, $ED$ (respectively). And they contain a common angle at $E$.
Thus, the base $DF$ is equal to the base $AB$, and triangle $DEF$ is equal to
triangle $EBA$, and the remaining angles (are equal) to the (corresponding)
remaining angles [Prop.~1.4]. Thus, (angle) $EDF$ (is) equal to $EBA$. And
$EDF$ (is) a right-angle. Thus, $EBA$ (is) also a right-angle. And $EB$ is a radius.
And a (straight-line) drawn at right-angles to the diameter of a circle, from
its extremity,  touches the circle [Prop.~3.16~corr.]. Thus, $AB$ touches circle $BCD$.

Thus, the straight-line $AB$ has been drawn touching the given circle $BCD$ from the given point $A$. (Which is) the very thing it was required to do.}
\end{Parallel}

%%%%%%
% Prop 3.18
%%%%%%
\pdfbookmark[1]{Proposition 3.18}{pdf3.18}
\begin{Parallel}{}{} 
\ParallelLText{
\begin{center}
{\large \ggn{18}.}
\end{center}\vspace*{-7pt}

\gr{>E`an k'uklou >ef'apthta'i tic e>uje~ia, >ap`o d`e to~u k'entrou >ep`i t`hn <af`hn >epizeuqj~h| tic e>uje~ia, <h >epizeuqje~isa k'ajetoc >'estai
>ep`i t`hn >efaptom'enhn.}

\gr{K'uklou g`ar to~u ABG >efapt'esjw tic e>uje~ia <h DE kat`a t`o G
shme~ion, ka`i e>il'hfjw t`o k'entron to~u ABG k'uklou t`o
Z, ka`i >ap`o to~u Z >ep`i t`o G >epeze'uqjw <h ZG; l'egw, <'oti <h ZG
k'ajet'oc >estin >ep`i t`hn DE.}

\gr{E>i g`ar m\kern -.7pt 'h, >'hqjw >ap`o to~u Z >ep`i t`hn DE k'ajetoc <h ZH.}

\gr{>Epe`i o>~un <h <up`o ZHG gwn'ia >orj'h >estin, >oxe~ia >'ara
>est`in <h <up`o ZGH; <up`o d`e t`hn me'izona gwn'ian <h me'izwn
pleur`a <upote'inei; me'izwn >'ara <h ZG t~hc ZH; >'ish d`e <h ZG t~h| ZB;
me'izwn >'ara ka`i <h ZB t~hc ZH <h >el'attwn t~hc me'izonoc; <'oper
>est`in >ad'unaton. o>uk >'ara <h ZH k'ajet'oc >estin >ep`i t`hn DE.
<omo'iwc d`h de~ixomen, <'oti o>ud> >'allh tic pl`hn t~hc ZG;
<h ZG >'ara k'ajet'oc >estin >ep`i t`hn DE.}\\~\\~\\

\epsfysize=2.2in
\centerline{\epsffile{Book03/fig18g.eps}}

\gr{>E`an >'ara k'uklou >ef'apthta'i tic e>uje~ia, >ap`o d`e to~u k'entrou >ep`i t`hn <af`hn >epizeuqj~h| tic e>uje~ia, <h >epizeuqje~isa k'ajetoc >'estai
>ep`i t`hn >efaptom'enhn; <'oper >'edei de~ixai.}}

\ParallelRText{
\begin{center}
{\large Proposition 18}
\end{center}

If some straight-line touches a circle, and some (other) straight-line is
joined from the center (of the circle) to the point of contact, (then) the (straight-line)
so joined will be perpendicular to the tangent.

For let some straight-line $DE$  touch the circle $ABC$  at  point $C$, and
let the center $F$ of circle $ABC$ be found [Prop.~3.1], 
and let $FC$ be joined from $F$ to $C$. I say that $FC$ is perpendicular
to $DE$.

For if not, let $FG$ be drawn from $F$, perpendicular to $DE$ [Prop.~1.12].

Therefore, since angle $FGC$ is a right-angle, (angle) $FCG$ is thus acute [Prop.~1.17]. And the greater angle is subtended by the greater side [Prop.~1.19]. Thus, $FC$ (is) greater than $FG$. And $FC$ (is) equal to
$FB$. Thus, $FB$ (is)  also greater than $FG$, the lesser than the greater.
The very thing is impossible. Thus, $FG$ is not perpendicular to
$DE$. So, similarly, we can show that neither (is) any other
(straight-line) except $FC$. Thus, $FC$ is perpendicular  to $DE$.

\epsfysize=2.2in
\centerline{\epsffile{Book03/fig18e.eps}}

Thus, if some straight-line  touches a circle, and some (other) straight-line is
joined from the center (of the circle) to the point of contact, (then) the (straight-line)
so joined will be perpendicular to the tangent. (Which is)
the very thing it was required to show.}
\end{Parallel}

%%%%%%
% Prop 3.19
%%%%%%
\pdfbookmark[1]{Proposition 3.19}{pdf3.19}
\begin{Parallel}{}{} 
\ParallelLText{
\begin{center}
{\large \ggn{19}.}
\end{center}\vspace*{-7pt}

\gr{>E`an k'uklou >ef'apthta'i tic e>uje~ia, >ap`o d`e t~hc <af~hc t~h|
>efaptom'enh| pr`oc >orj`ac [gwn'iac] e>uje~ia gramm\kern -.7pt `h >aqj~h|,
>ep`i t~hc >aqje'ishc >'estai t`o k'entron to~u k'uklou.}\\

\epsfysize=2.2in
\centerline{\epsffile{Book03/fig19g.eps}}

\gr{K'uklou g`ar to~u ABG >efapt'esjw tic e>uje~ia <h DE kat`a t`o G
shme~ion, ka`i >ap`o to~u G t~h| DE pr`oc >orj`ac >'hqjw <h GA;
l'egw, <'oti >ep`i t~hc AG >esti t`o k'entron to~u k'uklou.}

\gr{m\kern -.7pt `h g'ar, >all> e>i dunat'on, >'estw t`o Z, ka`i >epeze'uqjw <h GZ.}

\gr{>Epe`i [o>~un] k'uklou to~u ABG >ef'apteta'i tic e>uje~ia <h DE,
>ap`o d`e to~u k'entrou >ep`i t`hn <af`hn >ep'ezeuktai <h ZG, <h ZG
>'ara k'ajet'oc >estin >ep`i t`hn DE; >orj`h >'ara >est`in <h <up`o ZGE. >est`i
d`e ka`i <h <up`o AGE >orj'h; >'ish >'ara >est`in <h <up`o ZGE
t~h| <up`o AGE <h >el'attwn t~h| me'izoni; <'oper >est`in >ad'unaton.
o>uk >'ara t`o Z k'entron >est`i to~u ABG k'uklou. <omo'iwc d`h de'ixomen,
<'oti o>ud> >'allo ti pl`hn >ep`i t~hc AG.}

\gr{>E`an >'ara k'uklou >ef'apthta'i tic e>uje~ia, >ap`o d`e t~hc <af~hc t~h|
>efaptom'enh| pr`oc >orj`ac e>uje~ia gramm\kern -.7pt `h >aqj~h|,
>ep`i t~hc >aqje'ishc >'estai t`o k'entron to~u k'uklou; <'oper >'edei
de~ixai.}}

\ParallelRText{
\begin{center}
{\large Proposition 19}
\end{center}

If some straight-line  touches a circle, and a straight-line is drawn from the
point of contact, at
right-[angles] to the tangent, (then) the center (of the circle) will be on the (straight-line) so drawn.

\epsfysize=2.2in
\centerline{\epsffile{Book03/fig19e.eps}}

For let some straight-line $DE$  touch the circle $ABC$  at point $C$. And let $CA$
be drawn from $C$, at right-angles to $DE$ [Prop.~1.11]. I say that the center of the circle is on $AC$.

For (if) not, if possible, let $F$ be (the center of the circle), and let
$CF$ be joined.

\mbox{[}Therefore], since some straight-line $DE$ touches the circle $ABC$, and
$FC$ has been joined from the center to the point of contact, $FC$ is thus
perpendicular to $DE$ [Prop.~3.18]. Thus, $FCE$ is a right-angle.
And $ACE$ is also a right-angle. Thus, $FCE$ is equal to $ACE$, the lesser to the
greater. The very thing is impossible. Thus, $F$ is not the center of circle $ABC$. 
So, similarly, we can show that neither is any (point) other (than one) on $AC$.

Thus, if some straight-line touches a circle, and a straight-line is drawn from the
point of contact, at
right-angles to the tangent, (then) the center (of the circle) will be on the (straight-line) so drawn. (Which is) the very thing it was required to show.}
\end{Parallel}

%%%%%%
% Prop 3.20
%%%%%%
\pdfbookmark[1]{Proposition 3.20}{pdf3.20}
\begin{Parallel}{}{} 
\ParallelLText{
\begin{center}
{\large \ggn{20}.}
\end{center}\vspace*{-7pt}

\gr{>En k'uklw| <h pr`oc t~w| k'entrw| gwn'ia diplas'iwn >est`i t~hc pr`oc
t~h| perifere'ia|, <'otan t`hn a\kern -.7pt >ut`hn perif'ereian b'asin >'eqwsin a<i
gwn'iai.}

\gr{>'Estw k'ukloc <o ABG, ka`i pr`oc m`en t~w| k'entrw| a\kern -.7pt >uto~u gwn'ia
>'estw <h <up`o BEG, pr`oc d`e t~h| perifere'ia| <h <up`o BAG, >eq'etwsan
d`e t`hn a\kern -.7pt >ut`hn perif'ereian b'asin t`hn BG; l'egw, <'oti diplas'iwn >est`in
<h <up`o BEG gwn'ia t~hc <up`o BAG.}

\gr{>Epizeuqje~isa g`ar <h AE di'hqjw >ep`i t`o Z.}

\gr{>Epe`i o>~un >'ish >est`in <h EA t~h| EB, >'ish ka`i gwn'ia
<h <up`o EAB t~h| <up`o EBA; a<i >'ara <up`o EAB, EBA gwn'iai
t~hc <up`o EAB diplas'iouc e>is'in. >'ish d`e <h <up`o BEZ ta~ic
<up`o EAB, EBA; ka`i <h <up`o BEZ >'ara t~hc <up`o EAB >esti dipl~h. di`a t`a a\kern -.7pt >ut`a d`h ka`i <h <up`o ZEG t~hc <up`o EAG >esti dipl~h. <'olh
>'ara <h <up`o BEG <'olhc t~hc <up`o BAG >esti dipl~h.}

\epsfysize=2.2in
\centerline{\epsffile{Book03/fig20g.eps}}

\gr{Kekl'asjw d`h p'alin, ka`i >'estw <et'era gwn'ia <h <up`o BDG, ka`i
>epizeuqje~isa <h DE >ekbebl'hsjw >ep`i t`o H. <omo'iwc
d`h de'ixomen, <'oti dipl~h >estin <h <up`o HEG gwn'ia t~hc <up`o
EDG, <~wn <h <up`o HEB dipl~h >esti t~hc <up`o EDB; loip`h
>'ara <h <up`o BEG dipl~h >esti t~hc <up`o BDG.}

\gr{>En k'uklw| >'ara <h pr`oc t~w| k'entrw| gwn'ia diplas'iwn >est`i t~hc pr`oc
t~h| perifere'ia|, <'otan t`hn a\kern -.7pt >ut`hn perif'ereian b'asin >'eqwsin [a<i
gwn'iai]; <'oper >'edei de~ixai.}}

\ParallelRText{
\begin{center}
{\large Proposition 20}
\end{center}

In a circle, the angle at the center is double that at the circumference, when
the angles have the same circumference base.

Let $ABC$ be a circle, and let $BEC$ be an angle at its center, and $BAC$
(one) at (its) circumference. And let them have the same circumference base
$BC$. I say that angle $BEC$ is double (angle) $BAC$.

For being joined, let $AE$ be drawn through to $F$.

Therefore, since $EA$ is equal to $EB$, angle $EAB$ (is) also equal to
$EBA$ [Prop.~1.5]. Thus, angle $EAB$ and $EBA$ is double (angle) $EAB$.
And $BEF$ (is) equal to $EAB$ and $EBA$ [Prop.~1.32]. Thus, $BEF$ is also
double $EAB$. So, for the same (reasons), $FEC$ is also double $EAC$. Thus,
the whole (angle) $BEC$ is double the whole (angle) $BAC$.

\epsfysize=2.2in
\centerline{\epsffile{Book03/fig20e.eps}}

So let another  (straight-line)  be inflected, and let there be another
angle, $BDC$. And $DE$ being joined, let it be produced to $G$.
So, similarly, we can show that angle $GEC$ is double $EDC$, of which
$GEB$ is double $EDB$. Thus, the remaining (angle) $BEC$ is double
the (remaining angle) $BDC$.

Thus, in a circle, the angle at the center is double that at the circumference, when
[the angles] have the same circumference base. (Which is) the very thing it
was required to show.\\}
\end{Parallel}

%%%%%%
% Prop 3.21
%%%%%%
\pdfbookmark[1]{Proposition 3.21}{pdf3.21}
\begin{Parallel}{}{} 
\ParallelLText{
\begin{center}
{\large \ggn{21}.}
\end{center}\vspace*{-7pt}

\gr{>En k'uklw| a<i >en t~w| a\kern -.7pt >ut~w| tm\kern -.7pt 'hmati gwn'iai >'isai >all'hlaic
e>is'in.}

\epsfysize=2.2in
\centerline{\epsffile{Book03/fig21g.eps}}

\gr{>'Estw k'ukloc <o ABGD, ka`i >en t~w| a\kern -.7pt >ut~w| tm\kern -.7pt 'hmati t~w| BAED gwn'iai >'estwsan a<i <up`o BAD, BED; l'egw, <'oti a<i <up`o BAD,
BED gwn'iai >'isai >all'hlaic e>is'in.}

\gr{E>il'hfjw g`ar to~u ABGD k'uklou t`o k'entron, ka`i >'estw t`o Z, ka`i
>epeze'uqjwsan a<i BZ, ZD.}

\gr{Ka`i >epe`i <h m`en <up`o BZD gwn'ia pr`oc t~w| k'entrw| >est'in,
<h d`e <up`o BAD pr`oc t~h| perifere'ia|, ka`i >'eqousi t`hn a\kern -.7pt >ut`hn
perif'ereian b'asin t`hn BGD, <h >'ara <up`o BZD gwn'ia
diplas'iwn >est`i t~hc <up`o BAD. di`a t`a a\kern -.7pt >ut`a d`h <h <up`o BZD ka`i t~hc <up`o BED >esti dipls'iwn; >'ish >'ara <h <up`o BAD t~h| <up`o
BED.}

\gr{>En k'uklw| >'ara a<i >en t~w| a\kern -.7pt >ut~w| tm\kern -.7pt 'hmati gwn'iai >'isai >all'hlaic
e>is'in; <'oper >'edei de~ixai.}}

\ParallelRText{
\begin{center}
{\large Proposition 21}
\end{center}

In a circle, angles in the same segment are equal to one another.

\epsfysize=2.2in
\centerline{\epsffile{Book03/fig21e.eps}}

Let $ABCD$ be a circle, and let $BAD$ and $BED$ be angles in the same segment
$BAED$. I say that angles $BAD$ and $BED$ are equal to one another.

For let the center of circle $ABCD$ be found [Prop.~3.1],
and let it be (at point) $F$. And let $BF$ and $FD$ be joined.

And since angle $BFD$ is at the center, and $BAD$ at the
circumference, and they have the same circumference base $BCD$, angle
$BFD$ is thus double $BAD$ [Prop.~3.20]. So, for the
same (reasons), $BFD$ is also double  $BED$. Thus, $BAD$ (is) equal to $BED$.

Thus, in a circle, angles in the same segment are equal to one another.
(Which is) the very thing it was required to show.}
\end{Parallel}

%%%%%%
% Prop 3.22
%%%%%%
\pdfbookmark[1]{Proposition 3.22}{pdf3.22}
\begin{Parallel}{}{} 
\ParallelLText{
\begin{center}
{\large \ggn{22}.}
\end{center}\vspace*{-7pt}

\gr{T~wn >en to~ic k'ukloic tetraple'urwn a<i >apenant'ion gwn'iai dus`in >orja~ic >'isai e>is'in.}

\epsfysize=2.2in
\centerline{\epsffile{Book03/fig22g.eps}}

\gr{>'Estw k'ukloc <o ABGD, ka`i >en a\kern -.7pt >ut~w| tetr'apleuron >'estw t`o ABGD;
l'egw, <'oti a<i >apenant'ion gwn'iai dus`in >orja~ic >'isai e>is'in.}

\gr{>Epeze'uqjwsan a<i AG, BD.}

\gr{>Epe`i o>~un pant`oc trig'wnou a<i tre~ic gwn'iai dus`in >orja~ic >'isai
e>is'in, to~u ABG >'ara trig'wnou a<i tre~ic gwn'iai a<i <up`o GAB, ABG, BGA dus`in >orja~ic >'isai e>is'in. >'ish d`e <h m`en <up`o GAB t~h| <up`o
BDG; >en g`ar t~w| a\kern -.7pt >ut~w| tm\kern -.7pt 'hmat'i e>isi t~w| BADG; <h d`e <up`o AGB
t~h| <up`o ADB; >en g`ar t~w| a\kern -.7pt >ut~w| tm\kern -.7pt 'hmat'i e>isi t~w| ADGB;
<'olh >'ara <h <up`o ADG ta~ic <up`o BAG, AGB >'ish >est'in.
koin`h proske'isjw <h <up`o ABG; a<i >'ara <up`o ABG, BAG, AGB ta~ic
<up`o ABG, ADG >'isai e>is'in. >all> a<i <up`o ABG, BAG, AGB dus`in >orja~ic >'isai e>is'in. ka`i a<i <up`o ABG, ADG >'ara dus`in >orja~ic
>'isai e>is'in. <omo'iwc d`h de'ixomen, <'oti ka`i a<i <up`o BAD, DGB gwn'iai dus`in >orja~ic >'isai e>is'in.}

\gr{T~wn >'ara >en to~ic k'ukloic tetraple'urwn a<i >apenant'ion gwn'iai dus`in >orja~ic >'isai e>is'in; <'oper >'edei de~ixai.}}

\ParallelRText{
\begin{center}
{\large Proposition 22}
\end{center}

For quadrilaterals within circles, the (sum of the) opposite angles is equal to two right-angles.

\epsfysize=2.2in
\centerline{\epsffile{Book03/fig22e.eps}}

Let $ABCD$ be a circle, and let $ABCD$ be a quadrilateral within it. I say that
the (sum of the) opposite angles is equal to two right-angles.

Let $AC$ and $BD$ be joined.

Therefore, since the three angles of any triangle are equal to two
right-angles [Prop.~1.32], the
three angles $CAB$, $ABC$, and $BCA$ of triangle $ABC$ are thus equal to two right-angles. And $CAB$ (is) equal to $BDC$. For they are in the
same segment $BADC$ [Prop.~3.21]. And $ACB$ (is equal) to $ADB$.
For they are in the same segment $ADCB$ [Prop.~3.21].
Thus, the whole of $ADC$ is equal to $BAC$ and $ACB$. Let $ABC$ be added to both. Thus, $ABC$, $BAC$, and
$ACB$ are equal to $ABC$ and $ADC$. But, $ABC$, $BAC$, and $ACB$ are equal to
two right-angles. Thus, $ABC$ and $ADC$ are also equal to two right-angles.
Similarly, we can show that angles $BAD$ and $DCB$ are also equal to
two right-angles.

Thus, for quadrilaterals within circles, the (sum of the) opposite angles is equal to two right-angles. (Which is) the very thing it was required to show.}
\end{Parallel}

%%%%%%
% Prop 3.23
%%%%%%
\pdfbookmark[1]{Proposition 3.23}{pdf3.23}
\begin{Parallel}{}{} 
\ParallelLText{
\begin{center}
{\large \ggn{23}.}
\end{center}\vspace*{-7pt}

\gr{>Ep`i t~hc a\kern -.7pt >ut~hc e>uje'iac d'uo tm\kern -.7pt 'hmata k'uklwn <'omoia ka`i >'anisa
o>u sustaj'hsetai >ep`i t`a a\kern -.7pt >ut`a m'erh.}

\gr{E>i g`ar dunat'on, >ep`i t~hc a\kern -.7pt >ut~hc e>uje'iac t~hc AB d'uo tm\kern -.7pt 'hmata
k'uklwn <'omoia ka`i >'anisa sunest'atw >ep`i t`a a\kern -.7pt >ut`a m'erh t`a AGB, ADB,
ka`i di'hqjw <h AGD, ka`i >epeze'uqjwsan a<i GB, DB.}

\epsfysize=1.25in
\centerline{\epsffile{Book03/fig23g.eps}}

\gr{>Epe`i o>~un <'omoi'on >esti t`o AGB tm\kern -.7pt ~hma t~w| ADB tm\kern -.7pt 'hmati,
<'omoia d`e tm\kern -.7pt 'hmata k'uklwn >est`i t`a deq'omena gwn'iac >'isac, >'ish
>'ara >est`in <h <up`o AGB gwn'ia t~h| <up`o ADB <h >ekt`oc t~h| >ent'oc;
<'oper >est`in >ad'unaton.}

\gr{O>uk >'ara >ep`i t~hc a\kern -.7pt >ut~hc e>uje'iac d'uo tm\kern -.7pt 'hmata k'uklwn <'omoia
ka`i >'anisa sustaj'hsetai >ep`i t`a a\kern -.7pt >ut`a m'erh; <'oper >'edei de~ixai.}}

\ParallelRText{
\begin{center}
{\large Proposition 23}
\end{center}

Two similar and unequal segments of circles cannot be constructed on the same side of the same straight-line.

For, if possible, let the two similar and unequal segments of circles, $ACB$ and
$ADB$, be constructed on the same side of the same straight-line $AB$. And let $ACD$ be drawn through (the segments), and let $CB$ and $DB$ be joined.

\epsfysize=1.25in
\centerline{\epsffile{Book03/fig23e.eps}}

Therefore, since segment $ACB$ is similar to segment $ADB$, and similar segments of circles are those accepting equal angles [Def.~3.11], 
angle $ACB$ is thus equal to $ADB$, the external to the internal. The very thing
is impossible [Prop.~1.16].

Thus,  two similar and unequal segments of circles cannot be constructed on the same side of the same straight-line.}
\end{Parallel}

%%%%%%
% Prop 3.24
%%%%%%
\pdfbookmark[1]{Proposition 3.24}{pdf3.24}
\begin{Parallel}{}{} 
\ParallelLText{
\begin{center}
{\large\ggn{24}.}
\end{center}\vspace*{-7pt}

\gr{T`a >ep`i >'iswn e>ujei~wn <'omoia tm\kern -.7pt 'hmata k'ulwn >'isa >all'hloic >est'in.}

\gr{>'Estwsan g`ar >ep`i >'iswn e>ujei~wn t~wn AB, GD <'omoia
tm\kern -.7pt 'hmata k'uklwn t`a AEB, GZD; l'egw, <'oti >'ison >est`i t`o AEB
tm\kern -.7pt ~hma t~w| GZD tm\kern -.7pt 'hmati.}

\gr{>Efarmozom'enou g`ar to~u AEB tm\kern -.7pt 'hmatoc >ep`i t`o GZD ka`i tijem'enou to~u m`en A shme'iou >ep`i t`o G t~hc d`e AB e>uje'iac >ep`i t`hn
GD, >efarm'osei ka`i t`o B shme~ion >ep`i t`o D shme~ion di`a t`o >'ishn
e>~inai t`hn AB t~h| GD; t~hc d`e AB >ep`i t`hn GD >efarmos'ashc >efarm'osei ka`i t`o AEB tm\kern -.7pt ~hma >ep`i t`o GZD. e>i g`ar <h AB e>uje~ia
>ep`i t`hn GD >efarm'osei, t`o d`e AEB tm\kern -.7pt ~hma >ep`i t`o GZD m\kern -.7pt `h >efarm'osei, >'htoi >ent`oc a\kern -.7pt >uto~u pese~itai >`h >ekt`oc >`h parall'axei,
<wc t`o GHD, ka`i k'ukloc k'uklon t'emnei kat`a ple'iona shme~ia >`h
d'uo; <'oper >est'in >ad'unaton. o>uk >'ara >efarmozom'enhc t~hc AB
e>uje'iac >ep`i t`hn GD o>uk >efarm'osei ka`i t`o AEB tm\kern -.7pt ~hma >ep`i
t`o GZD; >efarm'osei >'ara, ka`i >'ison a\kern -.7pt >ut~w| >'estai.}\\~\\

\epsfysize=2.4in
\centerline{\epsffile{Book03/fig24g.eps}}

\gr{T`a >'ara >ep`i >'iswn e>ujei~wn <'omoia tm\kern -.7pt 'hmata k'uklwn
>'isa >all'hloic >est'in; <'oper >'edei de~ixai.}}

\ParallelRText{
\begin{center}
{\large Proposition 24}
\end{center}

Similar segments of circles on equal straight-lines are equal to one another.

For let $AEB$ and $CFD$ be similar segments of circles on the equal straight-lines $AB$ and $CD$ (respectively). I say that segment $AEB$ is equal to segment
$CFD$.

For if the segment $AEB$ is applied to the segment $CFD$,  and point $A$
is placed on (point) $C$, and the straight-line $AB$ on $CD$, (then) point $B$ will also coincide with point $D$, on account of $AB$ being equal to $CD$. And
if $AB$ coincides with $CD$, (then) the segment $AEB$ will also coincide with $CFD$.
For if the straight-line $AB$ coincides with $CD$, and the segment $AEB$ does not
coincide with $CFD$, (then) it will surely either fall inside it, outside (it),$^\dag$ or
it will miss like $CGD$ (in the figure), and a circle (will) cut (another) circle
at more than two points. The very thing is impossible [Prop.~3.10].
Thus, if the straight-line $AB$ is applied to $CD$, the segment $AEB$ cannot
not also coincide with $CFD$. Thus, it will coincide, and will be equal to it [C.N.~4].\\

\epsfysize=2.4in
\centerline{\epsffile{Book03/fig24e.eps}}

Thus, similar segments of circles on equal straight-lines are equal to one another. (Which is) the very thing it was required to show.}
\end{Parallel}
{\footnotesize \noindent$^\dag$ Both this possibility, and the previous one, are precluded by Prop.~3.23.}

%%%%%%
% Prop 3.25
%%%%%%
\pdfbookmark[1]{Proposition 3.25}{pdf3.25}
\begin{Parallel}{}{} 
\ParallelLText{
\begin{center}
{\large \ggn{25}.}
\end{center}\vspace*{-7pt}

\gr{K'uklou tm\kern -.7pt 'hmatoc doj'entoc prosanagr'ayai t`on k'uklon, o<~up'er >esti tm\kern -.7pt ~hma.}

\gr{>'Estw t`o doj`en tm\kern -.7pt ~hma k'uklou t`o ABG; de~i d`h to~u ABG tm\kern -.7pt 'hmatoc
prosanagr'ayai t`on k'uklon, o>~up'er >esti tm\kern -.7pt ~hma.}

\gr{Tetm\kern -.7pt 'hsjw g`ar <h AG d'iqa kat`a t`o D, ka`i >'hqjw >ap`o to~u D
shme'iou t~h| AG pr`oc >orj`ac <h DB, ka`i >epeze'uqjw <h AB; <h <up`o
ABD gwn'ia >'ara t~hc <up`o BAD >'htoi me'izwn >est`in >`h >'ish >`h
>el'attwn.}

\gr{>'Estw pr'oteron me'izwn, ka`i sunest'atw pr`oc t~h| BA e>uje'ia| ka`i
t~w| pr`oc a\kern -.7pt >ut~h| shme'iw| t~w| A t~h| <up`o ABD gwn'ia| >'ish <h <up`o
BAE, ka`i di'hqjw <h DB >ep`i t`o E, ka`i >epeze'uqjw <h EG. >epe`i o>~un
>'ish >est`in <h <up`o ABE gwn'ia t~h| <up`o BAE, >'ish >'ara >est`i
ka`i <h EB e>uje~ia t~h| EA. ka`i >epe`i >'ish >est`in <h AD t~h|
DG, koin`h d`e <h DE, d'uo d`h a<i AD, DE d'uo ta~ic GD, DE >'isai
e>is`in <ekat'era <ekat'era|; ka`i gwn'ia <h <up`o ADE gwn'ia| t~h| <up`o
GDE >estin >'ish; >orj`h g`ar <ekat'era; b'asic >'ara <h AE b'asei t~h| GE
>estin >'ish. >all`a <h AE t~h| BE >ede'iqjh >'ish; ka`i <h BE >'ara
t~h| GE >estin >'ish; a<i tre~ic >'ara a<i AE, EB, EG >'isai
>all'hlaic e>is'in; <o >'ara k'entr~w| t~w| E diast'hmati d`e <en`i t~wn AE, EB, EG k'ukloc graf'omenoc <'hxei ka`i di`a t~wn loip~wn shme'iwn
ka`i >'estai prosanagegramm'enoc. k'uklou >'ara tm\kern -.7pt 'hmatoc doj'entoc
prosanag'egraptai <o k'ukloc. ka`i d~hlon, <wc t`o ABG tm\kern -.7pt ~hma >'elatt'on
>estin <hmikukl'iou di`a t`o t`o E k'entron >ekt`oc a\kern -.7pt >uto~u tugq'anein.}\\~\\~\\

\epsfysize=1.3in
\centerline{\epsffile{Book03/fig25g.eps}}

\gr{<Omo'iwc [d`e] k>`an >~h| <h <up`o ABD gwn'ia >'ish t~h| <up`o
BAD, t~hc AD >'ishc genom'enhc <ekat'era| t~wn BD, DG a<i
tre~ic a<i DA, DB, DG >'isai >all'hlaic >'esontai, ka`i >'estai t`o D k'entron
to~u prosanapeplhrwm'enou k'uklou, ka`i dhlad`h >'estai t`o ABG
<hmik'uklion.}

\gr{>E`an d`e <h <up`o ABD >el'attwn >~h| t~hc <up`o BAD,
ka`i susths'wmeja pr`oc t~h| BA e>uje'ia| ka`i t~w| pr`oc
a\kern -.7pt >ut~h| shme'iw| t~w| A t~h| <up`o ABD gwn'ia| >'ishn, >ent`oc
to~u ABG tm\kern -.7pt 'hmatoc pese~itai t`o k'entron >ep`i t~hc DB, ka`i
>'estai dhlad`h t`o ABG tm\kern -.7pt ~hma me~izon <hmikukl'iou.}

\gr{K'uklou >'ara tm\kern -.7pt 'hmatoc doj'entoc prosanag'egraptai <o k'ukloc;
<'oper >'edei poi~hsai.}}

\ParallelRText{
\begin{center}
{\large Proposition 25}
\end{center}

For a given segment of a circle, to complete the circle, the very one of which it is a segment.

Let $ABC$ be the given segment of a circle. So it is required
to complete the circle for segment $ABC$, the very one of which it is a segment.

For let $AC$ be cut in half at (point) $D$ [Prop.~1.10], and let $DB$ be drawn
from point $D$, at right-angles to $AC$ [Prop.~1.11]. And let $AB$ be joined.
Thus, angle $ABD$ is surely either greater than,  equal to, or less than (angle)
$BAD$.

First of all, let it be greater. And let (angle) $BAE$, equal to
angle $ABD$, be constructed on the straight-line $BA$, at the point $A$ on it [Prop.~1.23].
And let $DB$ be drawn through to $E$, and let $EC$ be joined.
Therefore, since angle $ABE$ is equal to $BAE$, the straight-line $EB$ is thus also
equal to $EA$ [Prop.~1.6]. And since $AD$ is equal to $DC$, and $DE$ (is)
common, the two (straight-lines) $AD$, $DE$ are equal to the two (straight-lines)
$CD$, $DE$, respectively. And angle $ADE$ is equal to angle $CDE$. For each (is)
a right-angle. Thus, the base $AE$ is equal to the base $CE$ [Prop.~1.4].
But, $AE$ was shown (to be) equal to $BE$. Thus, $BE$ is also equal
to $CE$. Thus, the three (straight-lines) $AE$, $EB$, and $EC$ are equal to
one another. Thus, if a circle is drawn with center $E$, and radius one of $AE$, $EB$, or $EC$,  it will also go through the remaining points (of the segment), and the (associated circle) will be
 completed [Prop.~3.9]. Thus, a circle has been completed from the given
 segment of a circle. And (it is) clear that the segment $ABC$ is less
 than a semi-circle, because the center $E$ happens to lie outside it.

\epsfysize=1.3in
\centerline{\epsffile{Book03/fig25e.eps}}
 
\mbox{[}And], similarly, even if angle $ABD$ is equal to $BAD$,  (since) $AD$ becomes equal to
 each of $BD$ [Prop.~1.6] and $DC$, the three (straight-lines) $DA$, $DB$, and $DC$
 will  be equal to one another. And point $D$ will be the center of the completed
 circle. And $ABC$ will manifestly be a semi-circle.
 
 And if $ABD$ is less than $BAD$, and we construct (angle $BAE$), equal to angle
 $ABD$,  on the straight-line $BA$, at the point $A$ on it [Prop.~1.23], then
 the center will fall on $DB$,  inside the segment $ABC$. And segment
 $ABC$ will manifestly be greater than a semi-circle.
 
 Thus, a circle has been completed from the given segment of a circle. (Which is) the very thing it was required to do.}
\end{Parallel}

%%%%%%
% Prop 3.26
%%%%%%
\pdfbookmark[1]{Proposition 3.26}{pdf3.26}
\begin{Parallel}{}{} 
\ParallelLText{
\begin{center}
{\large \ggn{26}.}
\end{center}\vspace*{-7pt}

\gr{>En to~ic >'isoic k'ukloic a<i >'isai gwn'iai >ep`i >'iswn periferei~wn beb'hkasin, >e'an te pr`oc to~ic k'entroic >e'an te pr`oc ta~ic perifere'iaic
>~wsi bebhku~iai.}

\gr{>'Estwsan >'isoi k'ukloi o<i ABG, DEZ ka`i >en a\kern -.7pt >uto~ic >'isai gwn'iai >'estwsan pr`oc m`en to~ic k'entroic a<i <up`o BHG, EJZ, pr`oc d`e ta~ic
perifere'iaic a<i <up`o BAG, EDZ; l'egw, <'oti >'ish >est`in <h BKG
perif'ereia t~h| ELZ perifere'ia|.}

\gr{>Epeze'uqjwsan g`ar a<i BG, EZ.}

\gr{Ka`i >epe`i >'isoi e>is`in o<i ABG, DEZ k'ukloi, >'isai
e>is`in a<i >ek t~wn k'entrwn; d'uo d`h a<i BH, HG
d'uo ta~ic EJ, JZ >'isai; ka`i gwn'ia <h pr`oc t~w| H gwn'ia| t~h| pr`oc
t~w| J >'ish; b'asic >'ara <h BG b'asei t~h| EZ >estin >'ish. ka`i >epe`i
>'ish >est`in <h pr`oc t~w| A gwn'ia t~h| pr`oc t~w| D, <'omoion >'ara >est`i t`o BAG tm\kern -.7pt ~hma t~w| EDZ tm\kern -.7pt 'hmati; ka'i e>isin >ep`i >'iswn e>ujei~wn [t~wn BG, EZ]; t`a d`e >ep`i >'iswn e>ujei~wn <'omoia
tm\kern -.7pt 'hmata k'uklwn >'isa >all'hloic >est'in; >'ison >'ara t`o BAG tm\kern -.7pt ~hma t~w|
EDZ. >'esti d`e ka`i <'oloc <o ABG k'ukloc <'olw| t~w| DEZ k'uklw|
>'isoc; loip`h >'ara <h BKG perif'ereia t~h| ELZ perifere'ia| >est`in >'ish.}\\~\\~\\

\epsfysize=1.5in
\centerline{\epsffile{Book03/fig26g.eps}}

\gr{>En >'ara to~ic >'isoic k'ukloic a<i >'isai gwn'iai >ep`i >'iswn
periferei~wn beb'hkasin, >e'an te pr`oc to~ic k'entroic >e'an
te pr`oc ta~ic perifere'iac >~wsi bebhku~iai; <'oper >'edei
de~ixai.}}

\ParallelRText{
\begin{center}
{\large Proposition 26}
\end{center}

In equal circles, equal angles stand upon equal circumferences  whether they
are standing at the center or at the circumference.

Let $ABC$ and $DEF$ be equal circles, and within them let $BGC$ and $EHF$ be equal angles at the center, and $BAC$ and $EDF$ (equal angles) at the circumference. I say that circumference $BKC$ is equal to circumference $ELF$.

For let $BC$ and $EF$ be joined.

And since circles $ABC$ and $DEF$ are equal, their radii are equal.
So the two (straight-lines) $BG$, $GC$ (are) equal to the two (straight-lines)
$EH$, $HF$ (respectively). And the angle at $G$ (is) equal to the angle at $H$.
Thus, the base $BC$ is equal to the base $EF$ [Prop.~1.4].
And since the angle at $A$ is equal to the (angle) at $D$, the segment $BAC$ is
thus similar to the segment $EDF$ [Def.~3.11]. And they are on
equal straight-lines [$BC$ and $EF$]. And similar segments of circles on equal
straight-lines are equal to one another [Prop.~3.24]. Thus, segment
$BAC$ is equal to (segment) $EDF$. And the whole circle $ABC$ is also
equal to the whole circle $DEF$. Thus, the remaining circumference $BKC$
is equal to the (remaining) circumference $ELF$.

\epsfysize=1.5in
\centerline{\epsffile{Book03/fig26e.eps}}

Thus,   in equal circles, equal angles stand upon equal circumferences, whether they
are standing  at the center or at the circumference. (Which is)
the very thing which it was required to show.}
\end{Parallel}

%%%%%%
% Prop 3.27
%%%%%%
\pdfbookmark[1]{Proposition 3.27}{pdf3.27}
\begin{Parallel}{}{} 
\ParallelLText{
\begin{center}
{\large\ggn{27}.}
\end{center}\vspace*{-7pt}

\gr{>En to~ic >'isoic k'ukloic a<i >ep`i >'iswn periferei~wn bebhku~iai gwn'iai
>'isai >all'hlaic e>is'in, >e'an te pr`oc to~ic k'entroic >e'an te pr`oc
ta~ic perifere'iaic >~wsi bebhku~iai.}

\epsfysize=1.5in
\centerline{\epsffile{Book03/fig27g.eps}}

\gr{>En g`ar >'isoic k'ukloic to~ic ABG, DEZ >ep`i >'iswn periferei~wn
t~wn BG, EZ pr`oc m`en to~ic H, J k'entroic gwn'iai bebhk'etwsan a<i
<up`o BHG, EJZ, pr`oc d`e ta~ic perifere'iaic a<i <up`o BAG, EDZ;
l'egw, <'oti <h m`en <up`o BHG gwn'ia t~h| <up`o EJZ >estin >'ish,
<h d`e <up`o BAG t~h| <up`o EDZ >estin >'ish.}

\gr{E>i g`ar >'anis'oc >estin <h <up`o BHG t~h| <up`o EJZ,
m'ia a\kern -.7pt >ut~wn me'izwn >est'in. >'estw me'izwn <h <up`o BHG, ka`i
sunest'atw pr`oc t~h| BH e>uje'ia| ka`i t~w| pr`oc a\kern -.7pt >ut~h| shme'iw|
t~w| H t~h| <up`o EJZ gwn'ia| >'ish <h <up`o BHK; a<i d`e >'isai
gwn'iai >ep`i >'iswn periferei~wn beb'hkasin, <'otan pr`oc to~ic k'entroic
>~wsin; >'ish >'ara <h BK perif'ereia t~h| EZ perifere'ia|. >all`a <h EZ t~h|
BG >estin >'ish; ka`i <h BK >'ara t~h| BG >estin >'ish <h >el'attwn t~h|
me'izoni; <'oper >est`in >ad'unaton. o>uk >'ara >'anis'oc >estin <h <up`o
BHG gwn'ia t~h| <up`o EJZ; >'ish >'ara. ka'i >esti t~hc m`en <up`o BHG
<hm'iseia <h pr`oc t~w| A, t~hc d`e <up`o  EJZ <hm'iseia
<h pr`oc t~w| D; >'ish >'ara ka`i <h pr`oc t~w| A gwn'ia t~h| pr`oc
t~w| D.}

\gr{>En >'ara to~ic >'isoic k'ukloic a<i >ep`i >'iswn periferei~wn bebhku~iai
gwn'iai >'isai >all'hlaic e>is'in, >e'an te pr`oc to~ic k'entroic >e'an
te pr`oc ta~ic perifere'iaic >~wsi bebhku~iai; <'oper >'edei de~ixai.}}

\ParallelRText{
\begin{center}
{\large Proposition 27}
\end{center}

In equal circles, angles standing upon equal circumferences  are equal to one
another, whether they are standing at the center or at the circumference.

\epsfysize=1.5in
\centerline{\epsffile{Book03/fig27e.eps}}

For let the angles $BGC$ and $EHF$ at the centers $G$ and $H$, and the (angles) $BAC$ and $EDF$
at the circumferences, stand upon the equal circumferences $BC$ and $EF$, in the equal circles $ABC$ and $DEF$ (respectively). I say that angle $BGC$ is
equal to (angle) $EHF$, and $BAC$ is equal to $EDF$.

For if $BGC$ is unequal to $EHF$, one of them is greater. Let $BGC$ be greater, 
and let the (angle) $BGK$, equal to  angle $EHF$, be constructed on the
straight-line $BG$, at the point $G$ on it [Prop.~1.23]. But equal angles (in equal circles) stand upon equal circumferences, when they are at the centers [Prop.~3.26]. Thus,
circumference $BK$ (is) equal to circumference $EF$. But, $EF$ is equal to $BC$.
Thus, $BK$ is also equal to $BC$, the lesser to the greater. The very thing is
impossible. Thus, angle $BGC$ is not unequal to $EHF$. Thus, (it is) equal.
And the (angle) at $A$ is half $BGC$, and the (angle) at $D$ half $EHF$ [Prop.~3.20]. Thus, the angle at $A$ (is) also equal to the (angle) at $D$.

Thus, in equal circles, angles standing upon equal circumferences  are equal to one
another, whether they are standing at the center or at the circumference.
(Which is) the very thing it was required to show.\\}
\end{Parallel}

%%%%%%
% Prop 3.28
%%%%%%
\pdfbookmark[1]{Proposition 3.28}{pdf3.28}
\begin{Parallel}{}{} 
\ParallelLText{
\begin{center}
{\large\ggn{28}.}
\end{center}\vspace*{-7pt}

\gr{>En to~ic >'isoic k'ukloic a<i >'isai e>uje~iai >'isac perifere'iac >afairo~usi
t`hn m`en me'izona  t~h| me'izoni t`hn d`e >el'attona t~h| >el'attoni.}

\gr{>'Estwsan >'isoi k'ukloi o<i ABG, DEZ, ka`i >en to~ic k'ukloic >'isai
e>uje~iai >'estwsan a<i AB, DE t`ac m`en AGB, AZE perifere'iac me'izonac
>afairo~usai t`ac d`e AHB, DJE >el'attonac; l'egw, <'oti <h m`en AGB
me'izwn perif'ereia >'ish >est`i t~h| DZE me'izoni perifere'ia|
<h d`e AHB >el'attwn perif'ereia t~h| DJE.}\\

\epsfysize=1.6in
\centerline{\epsffile{Book03/fig28g.eps}}

\gr{E>il'hfjw g`ar t`a k'entra t~wn k'uklwn t`a K, L, ka`i >epeze'uqjwsan a<i
AK, KB, DL, LE.}

\gr{Ka`i >epe`i >'isoi k'ukloi e>is'in, >'isai e>is`i ka`i a<i >ek t~wn k'entrwn;
d'uo d`h a<i AK, KB dus`i ta~ic DL, LE
>'isai e>is'in; ka`i b'asic <h AB b'asei t~h| DE >'ish; gwn'ia >'ara <h <up`o
AKB gwn'ia| t~h| <up`o DLE >'ish >est'in. a<i d`e >'isai gwn'iai >ep`i
>'iswn periferei~wn beb'hkasin, <'otan pr`oc to~ic k'entroic >~wsin; >'ish
>'ara <h AHB perif'ereia t~h| DJE. >est`i d`e ka`i <'oloc <o ABG
k'ukloc <'olw| t~w| DEZ k'uklw| >'isoc; ka`i loip`h >'ara <h AGB 
perif'ereia loip~h| t~h| DZE perifere'ia| >'ish >est'in.}

\gr{>En >'ara to~ic >'isoic k'ukloic a<i >'isai e>uje~iai >'isac perifere'iac >afairo~usi
t`hn m`en me'izona t~h| me'izoni t`hn d`e >el'attona t~h| >el'attoni; <'oper
>'edei de~ixai.}}

\ParallelRText{
\begin{center}
{\large Proposition 28}
\end{center}

 In equal circles, equal straight-lines cut off equal circumferences, 
the greater (circumference being equal) to the greater, and the lesser to the lesser.

Let $ABC$ and $DEF$ be equal circles, and let $AB$ and $DE$ be equal straight-lines
in these circles, cutting off the greater circumferences $ACB$ and $DFE$, 
and the lesser (circumferences) $AGB$ and $DHE$ (respectively). I say that
the greater circumference $ACB$ is equal to the greater circumference $DFE$,
and the lesser circumference $AGB$ to (the lesser) $DHE$.

\epsfysize=1.6in
\centerline{\epsffile{Book03/fig28e.eps}}

For let the centers of the circles, $K$ and $L$, be found [Prop.~3.1],
and let $AK$, $KB$, $DL$, and $LE$ be joined.

And since ($ABC$ and $DEF$) are equal circles, their radii are also equal [Def.~3.1]. So the two
(straight-lines) $AK$, $KB$ are equal to the two (straight-lines) $DL$, $LE$ (respectively).
And the base $AB$ (is) equal to the base $DE$. Thus, angle $AKB$ is equal
to angle $DLE$ [Prop.~1.8]. And equal angles stand upon equal circumferences,
when they are at the centers [Prop.~3.26]. Thus, circumference $AGB$
(is) equal to $DHE$. And the whole circle $ABC$ is also equal to the whole
circle $DEF$. Thus, the remaining circumference $ACB$
is also equal to the remaining circumference $DFE$.

Thus, in equal circles, equal straight-lines cut off equal circumferences, 
the greater (circumference being equal) to the greater, and the lesser to the lesser. (Which is) the very thing it was required to show.}
\end{Parallel}

%%%%%%
% Prop 3.29
%%%%%%
\pdfbookmark[1]{Proposition 3.29}{pdf3.29}
\begin{Parallel}{}{} 
\ParallelLText{
\begin{center}
{\large \ggn{29}.}
\end{center}\vspace*{-7pt}

\gr{>En to~ic >'isoic k'ukloic t`ac >'isac perifere'iac >'isai e>uje~iai <upote'inousin.}

\gr{>'Estwsan >'isoi k'ukloi o<i ABG, DEZ, ka`i >en a\kern -.7pt >uto~ic >'isai perif'ereiai
>apeil'hfjwsan a<i BHG, EJZ, ka`i >epeze'uqjwsan a<i BG, EZ e>uje~iai;
l'egw, <'oti >'ish >est`in <h BG t~h| EZ.}

\gr{E>il'hfjw g`ar t`a k'entra t~wn k'uklwn, ka`i >'estw t`a K, L, ka`i >epeze'uqjwsan a<i BK, KG, EL, LZ.}

\gr{Ka`i >epe`i >'ish >est`in <h BHG perif'ereia t~h| EJZ perifere'ia|, >'ish
>est`i ka`i gwn'ia <h <up`o BKG t~h| <up`o ELZ. ka`i >epe`i >'isoi
e>is`in o<i ABG, DEZ k'ukloi, >'isai e>is`i ka`i a<i >ek t~wn k'entrwn;
d'uo d`h a<i BK, KG dus`i ta~ic EL, LZ >'isai e>is'in; ka`i gwn'iac
>'isac peri'eqousin; b'asic >'ara <h BG b'asei t~h| EZ >'ish >est'in;}\\~\\~\\~\\

\epsfysize=1.7in
\centerline{\epsffile{Book03/fig29g.eps}}

\gr{>En >'ara to~ic >'isoic k'ukloic t`ac >'isac perifere'iac
>'isai e>uje~iai <upote'inousin; <'oper >'edei de~ixai.}}

\ParallelRText{
\begin{center}
{\large Proposition 29}
\end{center}

In equal circles, equal straight-lines subtend equal circumferences.

Let $ABC$ and $DEF$ be equal circles, and within them let the equal circumferences $BGC$ and $EHF$ be cut off. And let the straight-lines $BC$ and $EF$ be joined. I say that $BC$ is equal to $EF$.

For let the centers of the circles be found [Prop.~3.1], and
let them be (at) $K$ and $L$. And let $BK$, $KC$, $EL$, and $LF$ be joined.

And since the circumference $BGC$ is equal to the circumference $EHF$,
the angle $BKC$ is also equal to (angle) $ELF$ [Prop.~3.27]. And since the circles
$ABC$ and $DEF$ are equal, their radii are also equal [Def.~3.1]. 
So the two (straight-lines) $BK$, $KC$ are equal to the two (straight-lines)
$EL$, $LF$ (respectively). And they contain equal angles. Thus, the
base $BC$ is equal to the base $EF$ [Prop.~1.4].

\epsfysize=1.7in
\centerline{\epsffile{Book03/fig29e.eps}}

Thus,  in equal circles, equal straight-lines subtend equal circumferences.
(Which is) the very thing it was required to show.}
\end{Parallel}

%%%%%%
% Prop 3.30
%%%%%%
\pdfbookmark[1]{Proposition 3.30}{pdf3.30}
\begin{Parallel}{}{} 
\ParallelLText{
\begin{center}
{\large \ggn{30}.}
\end{center}\vspace*{-7pt}

\gr{T`hn doje~isan perif'ereian d'iqa teme~in.}

\gr{>'Estw <h doje~isa perif'ereia <h ADB; de~i d`h t`hn ADB perif'ereian
d'iqa teme~in.}

\gr{>Epeze'uqjw <h AB, ka`i tetm\kern -.7pt 'hsjw d'iqa kat`a t`o G, ka`i >ap`o to~u
G shme'iou t~h| AB e>uje'ia| pr`oc >orj`ac >'hqjw <h GD, ka`i >epeze'uqjwsan a<i AD, DB.}

\epsfysize=1.3in
\centerline{\epsffile{Book03/fig30g.eps}}

\gr{Ka`i >epe`i >'ish >est`in <h AG t~h| GB, koin`h d`e <h GD, d'uo d`h
a<i AG, GD dus`i ta~ic BG, GD >'isai e>is'in; ka`i gwn'ia <h <up`o
AGD gwn'ia| t~h| <up`o BGD >'ish; >orj`h g`ar <ekat'era; b'asic
>'ara <h AD b'asei t~h| DB >'ish >est'in. a<i d`e >'isai e>uje~iai >'isac perifere'iac >afairo~usi t`hn m`en me'izona t~h| me'izoni t`hn d`e
>el'attona t~h| >el'attoni; k'ai >estin <ekat'era t~wn AD, DB periferei~wn
>el'attwn <hmikukl'iou; >'ish >'ara <h AD perif'ereia t~h| DB perifere'ia|.}

\gr{<H >'ara doje~isa perif'ereia d'iqa t'etm\kern -.7pt htai kat`a t`o D shme~ion; <'oper
>'edei poi~hsai.}}

\ParallelRText{
\begin{center}
{\large Proposition 30}
\end{center}

To cut a given circumference in half.

Let $ADB$ be the given circumference. So it is required to cut circumference
$ADB$ in half.

Let $AB$ be joined, and let it be cut in half at (point) $C$ [Prop.~1.10]. And let $CD$ be drawn from point $C$, at right-angles
to $AB$ [Prop.~1.11]. And let $AD$, and $DB$ be joined.

\epsfysize=1.3in
\centerline{\epsffile{Book03/fig30e.eps}}

And since $AC$ is equal to $CB$, and $CD$ (is) common, the two
(straight-lines) $AC$, $CD$ are equal to the two (straight-lines) $BC$, $CD$ (respectively).
And angle $ACD$ (is) equal to angle $BCD$. For (they are) each right-angles.
Thus, the base $AD$ is equal to the base $DB$ [Prop.~1.4]. And equal
straight-lines cut off equal circumferences, the greater (circumference being
equal) to the greater, and the lesser to the lesser [Prop.~1.28].
And the circumferences $AD$ and $DB$ are each less than a semi-circle.
Thus, circumference $AD$ (is) equal to circumference $DB$.

Thus, the given circumference has been cut in half at point $D$. (Which is)
the very thing it was required to do. }
\end{Parallel}

%%%%%%
% Prop 3.31
%%%%%%
\pdfbookmark[1]{Proposition 3.31}{pdf3.31}
\begin{Parallel}{}{} 
\ParallelLText{
\begin{center}
{\large \ggn{31}.}
\end{center}\vspace*{-7pt}

\gr{>En k'uklw| <h m`en >en t~w| <hmikukl'iw| gwn'ia >orj'h >estin, <h d`e
>en t~w| me'izoni tm\kern -.7pt 'hmati >el'attwn >orj~hc, <h d`e >en t~w| >el'attoni
tm\kern -.7pt 'hmati me'izwn >orj~hc; ka`i >'epi <h m`en to~u me'izonoc tm\kern -.7pt 'hmatoc gwn'ia me'izwn >est`in >orj~hc, <h d`e to~u >el'attonoc tm\kern -.7pt 'hmatoc gwn'ia
>el'attwn >orj~hc.}\\

\epsfysize=2.2in
\centerline{\epsffile{Book03/fig31g.eps}}

\gr{>'Estw k'ukloc <o ABGD, di'ametroc d`e a\kern -.7pt >uto~u >'estw <h BG,
k'entron d`e t`o E, ka`i >epeze'uqjwsan a<i BA, AG, AD, DG; l'egw,
<'oti <h m`en >en t~w| BAG <hmikukl'iw| gwn'ia <h <up`o BAG >orj'h
>estin, <h d`e >en t~w| ABG me'izoni to~u <hmikukl'iou tm\kern -.7pt 'hmati
gwn'ia <h <up`o ABG >el'attwn >est`in >orj~hc, <h d`e >en t~w| ADG
>el'attoni to~u <hmikukl'iou tm\kern -.7pt 'hmati gwn'ia <h <up`o ADG me'izwn
>est`in >orj~hc.}

\gr{>Epeze'uqjw <h AE, ka`i di'hqjw <h BA >ep`i t`o Z.}

\gr{Ka`i >epe`i >'ish >est`in <h BE t~h| EA, >'ish >est`i ka`i gwn'ia <h <up`o
ABE t~h| <up`o BAE. p'alin, >epe`i >'ish >est`in <h GE t~h| EA, >'ish
>est`i ka`i <h <up`o AGE t~h| <up`o GAE; <'olh >'ara <h <up`o BAG
dus`i ta~ic <up`o ABG, AGB >'ish >est'in. >est`i d`e ka`i <h <up`o 
ZAG >ekt`oc to~u ABG trig'wnou dus`i ta~ic <up`o ABG, AGB gwn'iaic
>'ish; >'ish >'ara ka`i <h <up`o BAG gwn'ia t~h| <up`o ZAG;
>orj`h >'ara <ekat'era; <h >'ara >en t~w| BAG <hmikukl'iw| 
gwn'ia <h <up`o BAG >orj'h >estin.}

\gr{Ka`i >epe`i to~u ABG tr'igwnou d'uo gwn'iai a<i <up`o ABG, BAG d'uo
>orj~wn >el'atton'ec e>isin, >orj`h d`e <h <up`o BAG, >el'attwn >'ara
>orj~hc >estin <h <up`o ABG gwn'ia; ka'i >estin >en t~w| ABG me'izoni
to~u <hmikukl'iou tm\kern -.7pt 'hmati.}

\gr{Ka`i >epe`i >en k'uklw| tetr'apleur'on >esti t`o ABGD, t~wn d`e >en to~ic
k'ukloic tetraple'urwn a<i >apenant'ion gwn'iai dus`in >orja~ic >'isai
e>is'in [a<i >'ara <up`o ABG, ADG gwn'iai dus`in >orja~ic >'isac
e>is'in], ka'i >estin <h <up`o ABG >el'attwn >orj~hc; loip`h >'ara <h <up`o
ADG gwn'ia me'izwn >orj~hc >estin; ka'i >estin >en t~w| ADG
>el'attoni to~u <hmikukl'iou tm\kern -.7pt 'hmati.}

\gr{L'egw, <'oti ka`i <h m`en to~u me'izonoc tm\kern -.7pt 'hmatoc gwn'ia <h perieqom'enh
<up'o [te] t~hc ABG perifere'iac ka`i t~hc AG e>uje'iac me'izwn >est`in
>orj~hc, <h d`e to~u >el'attonoc tm\kern -.7pt 'hmatoc gwn'ia <h perieqom'enh
<up'o [te] t~hc AD[G] perifere'iac ka`i t~hc AG e>uje'iac >el'attwn >est`in
>orj~hc. ka'i >estin a\kern -.7pt >ut'ojen faner'on. >epe`i g`ar <h <up`o t~wn BA,
AG e>ujei~wn >orj'h >estin, <h >'ara <up`o t~hc ABG perifere'iac ka`i t~hc
AG e>uje'iac perieqom'enh me'izwn >est`in >orj~hc. p'alin, >epe`i
<h <up`o t~wn AG, AZ e>ujei~wn >orj'h >estin, <h >'ara <up`o
t~hc GA e>uje'iac ka`i t~hc AD[G] perifere'iac perieqom'enh >el'attwn
>est`in >orj~hc.}

\gr{>En k'uklw| >'ara <h m`en >en t~w| <hmikukl'iw| gwn'ia >orj'h >estin, <h d`e
>en t~w| me'izoni tm\kern -.7pt 'hmati >el'attwn >orj~hc, <h d`e >en t~w| >el'attoni
[tm\kern -.7pt 'hmati] me'izwn >orj~hc; ka`i >'epi <h m`en to~u me'izonoc tm\kern -.7pt 'hmatoc [gwn'ia] me'izwn [>est`in] >orj~hc, <h d`e to~u >el'attonoc tm\kern -.7pt 'hmatoc [gwn'ia] >el'attwn >orj~hc; <'oper >'edei de~ixai.}}

\ParallelRText{
\begin{center}
{\large Proposition 31}
\end{center}

In a circle, the angle in a semi-circle is a right-angle, and that in a
greater segment (is) less than a right-angle, and that in a
lesser segment (is) greater than a right-angle. And, further, the angle of
a segment greater (than a semi-circle) is greater than a right-angle, and
the angle of a segment less (than a semi-circle) is less than a right-angle.

\epsfysize=2.2in
\centerline{\epsffile{Book03/fig31e.eps}}

Let $ABCD$  be a circle, and let $BC$ be its diameter, and $E$ its center. And let $BA$, $AC$, $AD$, and $DC$ be joined. I say that the angle $BAC$ in the semi-circle $BAC$ is a right-angle, and the angle $ABC$ in the segment $ABC$, (which is) greater than a semi-circle, is less
than a right-angle, and the angle $ADC$ in the segment $ADC$, (which is) less than a semi-circle,
is greater than a right-angle.

Let $AE$ be joined, and let $BA$ be drawn through to $F$.

And since $BE$ is equal to $EA$, angle $ABE$ is also equal to $BAE$ [Prop.~1.5]. Again, since $CE$ is equal to $EA$, $ACE$ is also equal to  $CAE$ [Prop.~1.5].
Thus, the whole (angle) $BAC$ is equal to the two (angles) $ABC$ and $ACB$.
And $FAC$, (which is) external to triangle $ABC$, is also equal to the two angles
$ABC$ and $ACB$ [Prop.~1.32]. Thus, angle $BAC$ (is) also equal to
$FAC$. Thus, (they are) each right-angles. [Def.~1.10]. Thus, the angle
$BAC$ in the semi-circle $BAC$ is a right-angle.

And since the two angles $ABC$ and $BAC$ of triangle $ABC$ are less than 
two right-angles [Prop.~1.17], and $BAC$ is a right-angle, angle $ABC$ is
thus less than a right-angle. And it is in segment $ABC$, (which is) greater 
than
a semi-circle.

And since $ABCD$ is a quadrilateral within a circle, and for quadrilaterals within circles
the (sum of the) opposite angles is equal to two right-angles 
[Prop.~3.22] [angles $ABC$ and $ADC$ are thus equal to two right-angles],
and (angle) $ABC$ is  less than a right-angle. The remaining angle $ADC$
is thus greater than a right-angle. And it is in segment $ADC$, (which is)
less than a semi-circle.

I also say that the angle of the greater segment, (namely) that contained by
the circumference $ABC$ and the straight-line $AC$, is greater than a right-angle.
And the angle of the lesser segment, (namely) that contained by the circumference
$AD[C]$ and the straight-line $AC$, is less than a right-angle. And
this is immediately apparent. For since the (angle contained by) the
two straight-lines $BA$ and $AC$ is a right-angle, the (angle) contained
by the circumference $ABC$ and the straight-line $AC$ is thus greater than a right-angle. Again, since the (angle contained by) the straight-lines
$AC$ and $AF$ is a right-angle, the (angle) contained by the circumference
$AD[C]$ and the straight-line $CA$ is thus less than a right-angle.

Thus, in a circle, the angle in a semi-circle is a right-angle, and that in a
greater segment (is) less than a right-angle, and that in a
lesser [segment] (is) greater than a right-angle. And, further, the [angle] of
a segment greater (than a semi-circle) [is] greater than a right-angle, and
the [angle] of a segment less (than a semi-circle) is  less than a right-angle.
(Which is) the very thing it was required to show.}
\end{Parallel}

%%%%%%
% Prop 3.32
%%%%%%
\pdfbookmark[1]{Proposition 3.32}{pdf3.32}
\begin{Parallel}{}{} 
\ParallelLText{
\begin{center}
{\large \ggn{32}.}
\end{center}\vspace*{-7pt}

\gr{>E`an k'uklou >ef'apthta'i tic e>uje~ia, >ap`o d`e t~hc <af~hc e>ic t`on k'uklon diaqj~h| tic e>uje~ia t'emnousa t`on k'uklon, <`ac poie~i gwn'iac
pr`oc t~h| >efaptom'enh|, >'isai >'esontai ta~ic >en to~ic >enall`ax
to~u k'uklou tm\kern -.7pt 'hmasi gwn'iaic.}\\

\epsfysize=2.1in
\centerline{\epsffile{Book03/fig32e.eps}}

\gr{K'uklou g`ar to~u ABGD >efapt'esjw tic e>uje~ia <h EZ kat`a t`o B shme~ion, ka`i >ap`o to~u B shme'iou di'hqjw tic e>uje~ia e>ic t`on
ABGD k'uklon t'emnousa a\kern -.7pt >ut`on <h BD. l'egw, <'oti <`ac poie~i gwn'iac
<h BD met`a t~hc EZ  >efaptom'enhc, >'isac >'esontai ta~ic >en to~ic
>enall`ax tm\kern -.7pt 'hmasi to~u k'uklou gwn'iaic, tout'estin, <'oti <h m`en
<up`o ZBD gwn'ia >'ish >est`i t~h| >en t~w| BAD tm\kern -.7pt 'hmati sunistam'enh| gwn'ia|, <h d`e <up`o EBD gwn'ia >'ish >est`i t~h| >en t~w| DGB tm\kern -.7pt 'hmati sunistam'enh| gwn'ia|.}

\gr{>'Hqjw g`ar >ap`o to~u B t~h| EZ pr`oc >orj`ac <h BA, ka`i e>il'hfjw >ep`i
t~hc BD perifere'iac tuq`on shme~ion t`o G, ka`i >epeze'uqjwsan a<i AD, DG, GB.}

\gr{Ka`i >epe`i k'uklou to~u ABGD >ef'apteta'i tic e>uje~ia <h EZ kat`a
t`o B, ka`i >ap`o t~hc <af~hc >~hktai t~h| >efaptom'enh| pr`oc >orj`ac <h BA, >ep`i t~hc BA >'ara t`o k'entron >est`i to~u ABGD k'uklou. <h BA
>'ara di'amet'oc >esti to~u ABGD k'uklou; <h >'ara <up`o ADB gwn'ia
>en <hmikukl'iw| o>~usa >orj'h >estin. loipa`i >'ara a<i <up`o BAD, ABD
mi~a| >orj~h| >'isai e>is'in. >est`i d`e ka`i <h <up`o ABZ >orj'h; <h >'ara <up`o ABZ >'ish >est`i ta~ic <up`o BAD, ABD. koin`h >afh|r'hsjw <h <up`o
ABD; loip`h >'ara <h <up`o DBZ gwn'ia >'ish >est`i t~h| >en t~w| >enall`ax tm\kern -.7pt 'hmati to~u k'uklou gwn'ia| t~h| <up`o BAD. ka`i >epe`i >en k'uklw|
tetr'apleur'on >esti t`o ABGD, a<i >apenant'ion a\kern -.7pt >uto~u gwn'iai dus`in
>orja~ic >'isai  e>is'in. e>is`i d`e ka`i a<i <up`o DBZ, DBE dus`in >orja~ic
>'isai;
 a<i >'ara <up`o DBZ, DBE ta~ic <up`o BAD, BGD >'isai
e>is'in, ~<wn <h <up`o BAD t~h| <up`o DBZ >ede'iqjh >'ish;
loip`h >'ara <h <up`o DBE t~h| >en t~w| >enall`ax to~u k'uklou tm\kern -.7pt 'hmati
t~w| DGB t~h| <up`o DGB gwn'ia| >est`in >'ish.}

\gr{>E`an >'ara k'uklou >ef'apthta'i tic e>uje~ia, >ap`o d`e t~hc <af~hc e>ic t`on k'uklon diaqj~h| tic e>uje~ia t'emnousa t`on k'uklon, <`ac poie~i gwn'iac
pr`oc t~h| >efaptom'enh|, >'isai >'esontai ta~ic >en to~ic >enall`ax
to~u k'uklou tm\kern -.7pt 'hmasi gwn'iaic; <'oper >'edei de~ixai.}}

\ParallelRText{
\begin{center}
{\large Proposition 32}
\end{center}

If some straight-line touches a circle, and some (other) straight-line is drawn  
across, from the point of contact into the circle,  cutting the circle (in two), (then) those angles  the (straight-line) makes
with the tangent will be equal to the angles in
the alternate segments of the circle.

\epsfysize=2.1in
\centerline{\epsffile{Book03/fig32e.eps}}

For let some straight-line $EF$ touch the circle $ABCD$ at the point $B$, and let
some (other) straight-line $BD$ be drawn from point $B$ into the
circle $ABCD$, cutting it (in two). I say that the angles $BD$ makes with the tangent 
$EF$  will be equal to the angles in the alternate segments of the circle.
That is to say, that angle $FBD$ is equal to the angle constructed
in segment $BAD$, and angle $EBD$ is equal to the angle constructed in segment $DCB$.

For let $BA$ be drawn from $B$, at right-angles to $EF$ [Prop.~1.11].
And let the point $C$ be taken at random on the circumference $BD$. And
let $AD$, $DC$, and $CB$ be joined.

And since some straight-line $EF$ touches the circle $ABCD$ at point $B$, and
$BA$ has been drawn from the point of contact, at right-angles to the
tangent, the center of circle $ABCD$ is thus on $BA$ [Prop.~3.19].
Thus, $BA$ is a diameter of circle $ABCD$.
Thus, angle $ADB$, being in a semi-circle, is a right-angle [Prop.~3.31].
Thus, the remaining angles (of triangle $ADB$) $BAD$ and $ABD$ are equal to one right-angle [Prop.~1.32]. And $ABF$ is also a right-angle. Thus,
$ABF$ is equal to $BAD$ and $ABD$. Let $ABD$ be subtracted from both.
Thus, the remaining angle $DBF$ is equal to the angle $BAD$ in the alternate
segment of the circle. And since $ABCD$ is a quadrilateral in a circle,
(the sum of) its opposite angles is equal to two right-angles [Prop.~3.22]. And $DBF$ and $DBE$ is also equal to two right-angles [Prop.~1.13]. Thus, $DBF$ and $DBE$ is equal to $BAD$ and $BCD$, of which
$BAD$ was shown (to be) equal to $DBF$. Thus, the remaining (angle) $DBE$ is
equal to the angle $DCB$ in the alternate segment $DCB$ of the circle.

Thus, if some straight-line touches a circle, and some (other) straight-line is drawn  across, from the point of contact into the circle,  cutting the circle (in two), (then) those angles  the (straight-line) makes
with the tangent will be equal to the angles in
the alternate segments of the circle. (Which is) the very thing it was required to
show.}
\end{Parallel}

%%%%%%
% Prop 3.33
%%%%%%
\pdfbookmark[1]{Proposition 3.33}{pdf3.33}
\begin{Parallel}{}{} 
\ParallelLText{
\begin{center}
{\large \ggn{33}.}
\end{center}\vspace*{-7pt}

\gr{>Ep`i t~hc doje'ishc e>uje'iac gr'ayai tm\kern -.7pt ~hma k'uklou deq'ome\-non gwn'ian >'ishn t~h| doje'ish| gwn'ia| e>ujugr'ammw|.}

\epsfysize=1.35in
\centerline{\epsffile{Book03/fig33g.eps}}

\gr{>'Estw <h doje~isa e>uje~ia <h AB, <h d`e doje~isa gwn'ia e>uj'ugrammoc <h
pr`oc t~w| G; de~i d`h >ep`i t~hc doje'ishc e>uje'iac t~hc AB gr'ayai tm\kern -.7pt ~hma
k'uklou deq'omenon gwn'ian >'ishn t~h| pr`oc t~w| G.}

\gr{<H d`h pr`oc t~w| G [gwn'ia] >'htoi >oxe~i'a >estin >`h >orj`h >`h >amble~ia;
>'estw pr'oteron >oxe~ia, ka`i <wc >ep`i t~hc pr'wthc katagraf~hc sunest'atw
pr`oc t~h| AB e>uje'ia| ka`i t~w| A shme'iw| t~h| pr`oc t~w| G gwn'ia|
>'ish <h <up`o BAD; >oxe~ia >'ara >est`i ka`i <h <up`o BAD. >'hqjw
t~h| DA pr`oc >orj`ac <h AE, ka`i tetm\kern -.7pt 'hsjw <h AB d'iqa kat`a t`o Z, ka`i
>'hqjw >ap`o to~u Z shme'iou t~h| AB pr`oc >orj`ac <h ZH, ka`i
>epeze'uqjw <h HB.}

\gr{Ka`i >epe`i >'ish >est`in <h AZ t~h| ZB, koin`h d`e <h ZH, d'uo d`h a<i
AZ, ZH d'uo ta~ic BZ, ZH >'isai e>is'in; ka`i gwn'ia <h <up`o AZH [gwn'ia|]
t~h| <up`o BZH >'ish; b'asic >'ara <h AH b'asei t~h| BH >'ish >est'in. <o >'ara k'entrw| m`en t~w| H diast'hmati d`e t~w| HA k'ukloc graf'omenoc <'hxei ka`i di`a to~u B. gegr'afjw ka`i >'estw <o ABE, ka`i >epeze'uqjw <h EB.
>epe`i o>~un >ap> >'akrac t~hc AE diam'etrou >ap`o to~u A t~h| AE pr`oc >orj'ac
>estin <h AD, <h AD >'ara >ef'aptetai to~u ABE k'uklou; >epe`i o>~un
k'uklou to~u ABE >ef'apteta'i tic e>uje~ia <h AD, ka`i >ap`o t~hc kat`a
t`o A <af~hc  e>ic t`on ABE k'uklon di~hkta'i tic e>uje~ia <h AB, <h
>'ara <up`o DAB gwn'ia >'ish >est`i t~h| >en t~w| >enall`ax to~u 
k'uklou tm\kern -.7pt 'hmati gwn'ia| t~h| <up`o AEB. >all> <h <up`o DAB t~h|
pr`oc t~w| G >estin >'ish; ka`i <h pr`oc t~w| G >'ara gwn'ia >'ish
>est`i t~h| <up`o AEB.}

\gr{>Ep`i t~hc doje'ishc >'ara e>uje'iac t~hc AB tm\kern -.7pt ~hma k'uklou g'egraptai
t`o AEB deq'omenon gwn'ian t`hn <up`o AEB >'ishn t~h| doje'ish|
t~h| pr`oc t~w| G.}

\gr{>All`a d`h >orj`h >'estw <h pr`oc t~w| G; ka`i d'eon p'alin >'estw >ep`i
t~hc AB gr'ayai tm\kern -.7pt ~hma k'uklou deq'omenon gwn'ian >'ishn t~h| pr`oc
t~w| G >orj~h| [gwn'ia|]. sunest'atw [p'alin] t~h| pr`oc t~w| G >orj~h| gwn'ia| >'ish <h <up`o BAD, <wc >'eqei >ep`i t~hc deut'erac katagraf~hc, ka`i tetm\kern -.7pt 'hsjw <h AB d'iqa kat`a t`o Z, ka`i k'entrw| t~w| Z, diast'hmati d`e
<opot'erw| t~wn ZA, ZB, k'ukloc gegr'afjw <o AEB.}

\gr{>Ef'aptetai >'ara <h AD e>uje~ia to~u ABE k'uklou di`a t`o >orj`hn e>~inai
t`hn pr`oc t~w| A gwn'ian. ka`i >'ish >est`in <h <up`o BAD gwn'ia
t~h| >en t~w| AEB tm\kern -.7pt 'hmati; >orj`h g`ar ka`i a\kern -.7pt >ut`h >en <hmikukl'iw|
o>~usa. >all`a ka`i <h <up`o BAD t~h| pr`oc t~w| G >'ish >est'in. ka`i
<h >en t~w| AEB >'ara >'ish >est`i  t~h| pr`oc t~w| G.}

\gr{G'egraptai >'ara p'alin >ep`i t~hc AB tm\kern -.7pt ~hma k'uklou t`o AEB deq'omenon
gwn'ian >'ishn t~h| pr`oc t~w| G.}

\gr{>All`a d`h <h pr`oc t~w| G >amble~ia >'estw; ka`i sunest'atw a\kern -.7pt >ut~h|
>'ish pr`oc t~h| AB e>uje'ia| ka`i t~w| A shme'iw| <h <up`o BAD, <wc
>'eqei >ep`i t~hc tr'ithc katagraf~hc, ka`i t~h| AD pr`oc >orj`ac >'hqjw <h AE,
ka`i tetm\kern -.7pt 'hsjw p'alin <h AB d'iqa kat`a t`o Z, ka`i t~h| AB pr`oc >orj`ac
>'hqjw <h ZH, ka`i >epeze'uqjw <h HB.}

\gr{Ka`i >epe`i p'alin >'ish >est`in <h AZ t~h| ZB, ka`i koin`h <h ZH, d'uo d`h a<i
AZ, ZH d'uo ta~ic BZ, ZH >'isai e>is'in; ka`i gwn'ia <h <up`o
AZH gwn'ia| t~h| <up`o BZH >'ish; b'asic >'ara <h AH b'asei t~h| BH
>'ish >est'in; <o >'ara k'entrw| m`en t~w| H diast'hmati d`e t~w| HA k'ukloc
graf'omenoc <'hxei ka`i di`a to~u B. >erq'esjw <wc <o AEB. ka`i >epe`i
t~h| AE diam'etrw| >ap> >'akrac pr`oc >orj'ac >estin <h AD, <h AD >'ara
>ef'aptetai to~u AEB k'uklou. ka`i >ap`o t~hc kat`a t`o A >epaf~hc di~hktai
<h AB; <h >'ara <up`o BAD gwn'ia >'ish >est`i t~h| >en t~w| >enall`ax to~u
k'uklou tm\kern -.7pt 'hmati  t~w| AJB sunistam'enh| gwn'ia|. >all> <h <up`o BAD gwn'ia t~h| pr`oc t~w| G >'ish >est'in. ka`i <h >en t~w| AJB >'ara tm\kern -.7pt 'hmati
gwn'ia >'ish  >est`i t~h| pr`oc t~w| G.}

\gr{>Ep`i t~hc >'ara doje'ishc e>uje'iac t~hc AB g'egraptai tm\kern -.7pt ~hma k'uklou t`o
AJB deq'omenon gwn'ian >'ishn t~h| pr`oc t~w| G; <'oper >'edei poi~hsai.}}

\ParallelRText{
\begin{center}
{\large Proposition 33}
\end{center}

To draw a segment of a circle, accepting an angle equal to a given rectilinear angle, on a given straight-line.

\epsfysize=1.53in
\centerline{\epsffile{Book03/fig33e.eps}}

Let $AB$ be the given straight-line, and $C$ the given rectilinear angle. So
it is required to draw a segment of a circle, accepting an angle equal to $C$,
on the given straight-line $AB$.

So the [angle] $C$ is surely either acute, a right-angle, or obtuse. First of all,
let it be acute. And, as in the first diagram (from the left), let (angle) $BAD$,
equal to angle $C$,  be
constructed  on the straight-line $AB$,  at the point A (on it)
[Prop.~1.23]. Thus, $BAD$ is also acute. Let $AE$ be drawn, at
right-angles to $DA$ [Prop.~1.11]. And let $AB$ be cut in half
at $F$ [Prop.~1.10]. And let $FG$ be drawn from point $F$, at right-angles to $AB$
[Prop.~1.11]. And let $GB$ be joined.

And since $AF$ is equal to $FB$, and $FG$ (is) common, the two (straight-lines)
$AF$, $FG$ are equal to the two (straight-lines) $BF$, $FG$ (respectively). And
angle $AFG$ (is) equal to [angle] $BFG$. Thus, the base $AG$ is equal to the
base $BG$ [Prop.~1.4]. Thus, the circle drawn with center $G$, and radius $GA$, will also go through $B$ (as well as $A$). Let it be drawn, and let it be (denoted) $ABE$. And let $EB$ be joined. Therefore, since $AD$ is at the extremity
of diameter $AE$,  (namely, point) $A$,  at right-angles to $AE$, the (straight-line)
$AD$ thus touches the circle $ABE$ [Prop.~3.16~corr.]. Therefore,
since some straight-line $AD$ touches the circle $ABE$, and some (other)
straight-line $AB$ has been drawn across from the point of contact $A$ into circle
$ABE$, angle $DAB$ is thus equal to the angle $AEB$ in the alternate segment of the
circle [Prop.~3.32]. But, $DAB$ is equal to $C$. Thus,
angle $C$ is also equal to $AEB$.

Thus, a segment $AEB$ of a circle, accepting the angle $AEB$ (which is) equal to the
given (angle) $C$, has been drawn on the given straight-line $AB$.

And so let $C$ be a right-angle. And let it again be necessary to draw  a segment
of a circle on $AB$, accepting an angle equal to the right-[angle] $C$. Let the (angle)
$BAD$ [again] be constructed, equal to the right-angle $C$ [Prop.~1.23], as in the second diagram (from the left). And let $AB$ be cut in half at $F$ [Prop.~1.10]. And let the circle $AEB$ be
drawn with center $F$, and radius either  $FA$ or $FB$.

Thus, the straight-line $AD$ touches the circle $ABE$, on account of the angle at $A$ being a right-angle [Prop.~3.16 corr.]. And angle $BAD$ is equal
to the angle in segment $AEB$. For (the latter angle), being in a semi-circle, is also a right-angle [Prop.~3.31]. But,  $BAD$ is also equal to $C$. Thus, the (angle) in (segment) $AEB$ is also equal to $C$.

Thus, a segment $AEB$ of a circle, accepting an angle equal to $C$, has again been
drawn on $AB$.

And so let (angle) $C$ be obtuse. And let (angle) $BAD$, equal
to ($C$), be constructed  on the straight-line $AB$, at the point $A$ 
(on it) [Prop.~1.23], as in the third diagram (from the left). And let $AE$ be
drawn, at right-angles to $AD$ [Prop.~1.11]. And let $AB$ again be 
cut in half at $F$ [Prop.~1.10]. And let $FG$ be drawn,
at right-angles to $AB$ [Prop.~1.10]. And let $GB$ 
be joined.

And again, since $AF$ is equal to $FB$, and $FG$ (is) common, the two (straight-lines) $AF$, $FG$ are equal to the two (straight-lines) $BF$, $FG$ (respectively).
And angle $AFG$ (is) equal to angle $BFG$. Thus, the base $AG$ is equal to
the base $BG$ [Prop.~1.4]. Thus, a circle of center $G$, and radius $GA$,
being drawn, will also go through $B$ (as well as $A$). Let it go like $AEB$ (in the third diagram
from the left). And since $AD$ is at right-angles to the diameter $AE$, at its extremity, $AD$ thus touches circle $AEB$ [Prop.~3.16~corr.]. And $AB$ has been drawn across (the circle) from the point of contact $A$. Thus, angle $BAD$ is equal to the angle constructed in the alternate segment $AHB$ of the circle 
[Prop.~3.32]. But, angle $BAD$ is equal to $C$. Thus, the angle in segment
$AHB$ is also equal to $C$.

Thus, a segment $AHB$ of a circle, accepting an angle equal to $C$, has been
drawn on the given straight-line $AB$. (Which is) the very thing
it was required to do.}
\end{Parallel}

%%%%%%
% Prop 3.34
%%%%%%
\pdfbookmark[1]{Proposition 3.34}{pdf3.34}
\begin{Parallel}{}{} 
\ParallelLText{
\begin{center}
{\large\ggn{34}.}
\end{center}\vspace*{-7pt}

\gr{>Ap`o to~u doj'entoc k'uklou tm\kern -.7pt ~hma >afele~in deq'omenon gwn'ian
>'ishn t~h| doje'ish| gwn'ia| e>ujugr'ammw|.}

\gr{>'Estw <o doje`ic k'ukloc <o ABG, <h d`e doje~isa gwn'ia e>uj'ugrammoc
<h pr`oc  t~w| D; de~i d`h >ap`o to~u ABG k'uklou tm\kern -.7pt ~hma >afele~in
deq'omenon gwn'ian >'ishn t~h| doje'ish| gwn'ia| e>ujugr'ammw| t~h| pr`oc
t~w| D.}

\epsfysize=1.8in
\centerline{\epsffile{Book03/fig34g.eps}}

\gr{>'Hqjw to~u ABG >efaptom'enh <h EZ kat`a t`o B shme~ion, ka`i sunest'atw
pr`oc t~h| ZB e>uje'ia| ka`i t~w| pr`oc a\kern -.7pt >ut~h| shme'iw| t~w| B t~h| pr`oc
t~w| D gwn'ia| >'ish <h <up`o ZBG.}

\gr{>Epe`i o>~un k'uklou to~u ABG >ef'apteta'i tic e>uje~ia <h EZ, ka`i >ap`o
t~hc kat`a t`o B >epaf~hc di~hktai <h BG, <h <up`o ZBG >'ara gwn'ia
>'ish >est`i t~h| >en t~w| BAG >enall`ax tm\kern -.7pt 'hmati sunistam'enh| gwn'ia|.
>all> <h <up`o ZBG t~h| pr`oc t~w| D >estin >'ish; ka`i <h >en t~w| BAG
>'ara tm\kern -.7pt 'hmati >'ish >est`i t~h| pr`oc t~w| D [gwn'ia|].}

\gr{>Ap`o to~u doj'entoc >'ara k'uklou to~u ABG tm\kern -.7pt ~hma >af'h|rhtai t`o BAG
deq'omenon gwn'ian >'ishn t~h| doje'ish| gwn'ia| e>ujugr'am\-mw| t~h|
pr`oc t~w| D; <'oper >'edei poi~hsai.}}

\ParallelRText{
\begin{center}
{\large Proposition 34}
\end{center}

To cut off a segment, accepting an angle equal to a given rectilinear angle,
from a given circle.

Let $ABC$ be the given circle, and $D$ the given rectilinear angle. So it is
required to cut off a segment, accepting an angle equal to the given
rectilinear angle $D$, from the given circle $ABC$.

\epsfysize=1.8in
\centerline{\epsffile{Book03/fig34e.eps}}

Let $EF$ be drawn touching $ABC$ at point $B$.$^\dag$ And let (angle)
$FBC$, equal to angle $D$, be constructed  on the
straight-line $FB$, at the point $B$ on it [Prop.~1.23].

Therefore, since some straight-line $EF$ touches the circle $ABC$, and
$BC$ has been drawn across (the circle) from the point of contact $B$, angle
$FBC$ is thus equal to the angle constructed in the alternate segment
$BAC$ [Prop.~1.32]. But, $FBC$ is equal to $D$. Thus, the (angle)
in the segment $BAC$ is also equal to [angle] $D$.

Thus, the segment $BAC$, accepting an angle equal to the given
rectilinear angle $D$, has been cut off from the given circle $ABC$. (Which is)
the very thing it was required to do.}
\end{Parallel}
{\footnotesize \noindent$^\dag$ Presumably,
by finding the center of $ABC$ [Prop.~3.1], drawing a straight-line between
the center and point $B$, and then drawing $EF$ through point $B$, at right-angles
to the aforementioned straight-line [Prop.~1.11].}

%%%%%%
% Prop 3.35
%%%%%%
\pdfbookmark[1]{Proposition 3.35}{pdf3.35}
\begin{Parallel}{}{} 
\ParallelLText{
\begin{center}
{\large \ggn{35}.}
\end{center}\vspace*{-7pt}

\gr{>E`an >en k'uklw| d'uo e>uje~iai t'emnwsin >all'hlac, t`o <up`o t~wn
t~hc mi~ac tm\kern -.7pt hm'atwn perieq'omenon >orjog'wnion >'ison >est`i
t~w| <up`o t~wn t~hc <et'erac tm\kern -.7pt hm'atwn perieqom'enw| >orjogwn'iw|.}

\gr{>En g`ar k'uklw| t~w| ABGD d'uo e>uje~iai a<i AG, BD temn'etwsan >all'hlac kat`a t`o E shme~ion; l'egw, <'oti t`o <up`o t~wn AE, EG
perieq'omenon >orjog'wnion >'ison >est`i t~w| <up`o t~wn DE, EB
perieqom'enw| >orjogwn'iw|.}

\gr{E>i m`en o>~un a<i AG, BD di`a to~u k'entrou e>is`in <'wste t`o E
k'entron e>~inai to~u ABGD k'uklou, faner'on, <'oti
>'iswn o>us~wn t~wn AE, EG, DE, EB ka`i t`o <up`o t~wn AE,
EG perieq'omenon >orjog'wnion >'ison >est`i t~w| <up`o t~wn
DE, EB perieqom'enw| >orjogwn'iw|.}

\gr{m\kern -.7pt `h >'estwsan d`h a<i AG, DB di`a to~u k'entrou, ka`i e>il'hfjw t`o
 k'entron to~u ABGD, ka`i >'estw t`o Z,  ka`i >ap`o to~u Z >ep`i
t`ac AG, DB e>uje'iac k'ajetoi >'hqjwsan a<i ZH, ZJ, ka`i >epeze'uqjwsan
a<i ZB, ZG, ZE.}

\gr{Ka`i >epe`i e>uje~i'a tic di`a to~u k'entrou <h HZ e>uje~i'an tina m\kern -.7pt `h di`a
to~u k'entrou t`hn AG pr`oc >orj`ac t'emnei, ka`i d'iqa a\kern -.7pt >ut`hn t'emnei;
>'ish >'ara <h AH t~h| HG. >epe`i o>~un e>uje~ia <h AG t'etm\kern -.7pt htai e>ic m`en
>'isa kat`a t`o H, e>ic d`e >'anisa kat`a t`o E, t`o >'ara <up`o t~wn
AE, EG perieq'omenon >orjog'wnion met`a to~u >ap`o t~hc EH tetrag'wnou
>'ison >est`i t~w| >ap`o t~hc HG; [koin`on] proske'isjw t`o >ap`o t~hc HZ;
t`o >'ara <up`o t~wn AE, EG met`a t~wn >ap`o t~wn HE, HZ >'ison >est`i
to~ic >ap`o t~wn GH, HZ. >all`a to~ic m`en >ap`o t~wn EH, HZ >'ison
>est`i t`o >ap`o t~hc ZE, to`ic d`e >ap`o t~wn GH, HZ >'ison
>est`i t`o >ap`o t~hc ZG; t`o >'ara <up`o t~wn AE, EG  met`a to~u >ap`o t~hc ZE >'ison  >est`i t~w| >ap`o t~hc ZG. >'ish d`e <h ZG t~h| ZB; t`o
>'ara <up`o t~wn AE, EG met`a to~u >ap`o t~hc EZ >'ison >est`i
t~w| >ap`o t~hc ZB. di`a t`a a\kern -.7pt >ut`a d`h ka`i t`o 
 <up`o  t~wn DE, EB met`a to~u >ap`o t~hc
ZE >ison >est`i t~w| >ap`o t~hc ZB. >ede'iqjh d`e ka`i t`o <up`o
t~wn AE, EG met`a to~u >ap`o t~hc ZE >'ison t~w| >ap`o t~hc
ZB; t`o >'ara <up`o t~wn AE, EG met`a to~u >ap`o t~hc ZE >'ison
>est`i t~w| <up`o t~wn
DE, EB met`a to~u >ap`o t~hc ZE. koin`on >af~h|r'hsjw
t`o >ap`o t~hc ZE; loip`on >'ara t`o <up`o
t~wn AE, EG perieq'omenon >orjog'wnion >'ison >est`i t~w| <up`o
t~wn DE, EB perieqom'enw| >orjogwn'iw|.}\\~\\~\\~\\~\\~\\~\\

\epsfysize=1.5in
\centerline{\epsffile{Book03/fig35g.eps}}

\gr{>E`an >'ara >en k'uklw| e>uje~iai d'uo t'emnwsin >all'hlac, t`o <up`o t~wn
t~hc mi~ac tm\kern -.7pt hm'atwn perieq'omenon >orjog'wnion >'ison >est`i
t~w| <up`o t~wn t~hc <et'erac tm\kern -.7pt hm'atwn perieqom'enw| >orjogwn'iw|;
<'oper >'edei de~ixai.}}

\ParallelRText{
\begin{center}
{\large Proposition 35}
\end{center}

If two straight-lines in a circle cut one another, (then) the rectangle
contained by the pieces of one is equal to the rectangle contained
by the pieces of the other.

For let the two straight-lines $AC$ and $BD$, in the circle $ABCD$,
cut one another at  point $E$. I say that the rectangle contained by
$AE$ and $EC$ is equal to the rectangle contained  by $DE$ and $EB$.

In fact, if $AC$ and $BD$ are through the center (as in the first diagram from the left), so that $E$ is the center
of circle $ABCD$, (then it is) clear that, $AE$, $EC$, $DE$, and $EB$ being equal,
the rectangle contained by $AE$ and $EC$ is also equal to the rectangle
contained by $DE$ and $EB$.

So let $AC$ and $DB$ not be though the center (as in the second diagram from the left), and let the center of $ABCD$ 
be found [Prop.~3.1], and let it be (at) $F$. And let $FG$ and $FH$
be drawn from $F$, perpendicular to the straight-lines $AC$ and $DB$
(respectively) [Prop.~1.12]. And let $FB$, $FC$, and $FE$ 
be joined.

And since some straight-line, $GF$, through the center, cuts at right-angles  some (other) straight-line, $AC$, not through the center,  (then) it also cuts
it in half [Prop.~3.3]. Thus, $AG$ (is) equal to $GC$. Therefore,
since the straight-line $AC$ is cut equally at $G$, and unequally at $E$,
the rectangle contained by $AE$ and $EC$ plus the square on $EG$ is thus
equal to the (square) on $GC$ [Prop.~2.5]. Let the (square) on
$GF$ be added [to both]. Thus, the (rectangle contained) by 
$AE$ and $EC$ plus the (sum of the squares) on $GE$ and $GF$ is equal to
the (sum of the squares) on $CG$ and $GF$. But, the (square) on $FE$ is equal to the (sum of the squares)
on $EG$ and $GF$   [Prop.~1.47],
and the (square)
on $FC$  is equal to the (sum of the squares) on $CG$ and $GF$ [Prop.~1.47]. Thus, the (rectangle contained) by
$AE$ and $EC$ plus the (square) on $FE$ is equal to the (square) on $FC$.
And $FC$ (is) equal to $FB$. Thus, the (rectangle contained) by
$AE$ and $EC$ plus the (square) on $FE$ is equal to the (square) on $FB$.
So, for the same (reasons), the (rectangle contained) by
$DE$ and $EB$ plus the (square) on $FE$ is equal to the (square) on $FB$.
And the (rectangle contained) by
$AE$ and $EC$ plus the (square) on $FE$ was also shown (to be) equal to the (square) on $FB$. Thus,  the (rectangle contained) by
$AE$ and $EC$ plus the (square) on $FE$ is equal to the (rectangle contained) by
$DE$ and $EB$ plus the (square) on $FE$. Let the (square) on $FE$ be
taken from both. Thus, the remaining rectangle contained by $AE$ and
$EC$ is equal to the rectangle contained by $DE$ and $EB$.

\epsfysize=1.5in
\centerline{\epsffile{Book03/fig35e.eps}}

Thus, if two straight-lines in a circle cut one another, (then) the rectangle
contained by the pieces of one is equal to the rectangle contained
by the pieces of the other. (Which is) the very thing it was required to
show.}
\end{Parallel}

%%%%%%
% Prop 3.36
%%%%%%
\pdfbookmark[1]{Proposition 3.36}{pdf3.36}
\begin{Parallel}{}{} 
\ParallelLText{
\begin{center}
{\large \ggn{36}.}
\end{center}\vspace*{-7pt}

\gr{>E`an k'uklou lhfj~h| ti shme~ion >ekt'oc, ka`i >ap> a\kern -.7pt >uto~u pr`oc t`on
k'uklon prosp'iptwsi d'uo e>uje~iai, ka`i <h m`en a\kern -.7pt >ut~wn t'emnh| t`on
k'uklon, <h d`e >ef'apthtai, >'estai t`o <up`o <'olhc t~hc temno'ushc
ka`i t~hc >ekt`oc >apolambanom'enhc metax`u to~u te shme'iou ka`i
t~hc kurt~hc perifere'iac >'ison t~w| >ap`o t~hc >efaptom'enhc tetrag'wnw|.}

\gr{K'uklou g`ar to~u ABG e>il'hfjw ti shme~ion >ekt`oc t`o D, ka`i >ap`o
to~u D pr`oc t`on ABG k'uklon prospipt'etwsan d'uo e>uje~iai a<i DG[A],
DB; ka`i <h m`en DGA temn'etw t`on ABG k'uklon, <h d`e BD >efapt'esjw;
l'egw, <'oti t`o <up`o t~wn AD, DG perieq'omenon >orjog'wnion
>'ison >est`i t~w| >ap`o t~hc DB tetrag'wnw|.}\\

\epsfysize=1.8in
\centerline{\epsffile{Book03/fig36g.eps}}

\gr{<H >'ara [D]GA >'htoi di`a to~u k'entrou >est`in >`h o>'u. >'estw pr'oteron
di`a to~u k'entrou, ka`i >'estw t`o Z k'entron to~u ABG k'uklou, ka`i >epeze'uqjw <h ZB; >orj`h >'ara >est`in <h <up`o ZBD. ka`i >epe`i e>uje~ia
<h AG d'iqa t'etm\kern -.7pt htai kat`a t`o Z, pr'oskeitai d`e a\kern -.7pt >ut~h| <h GD, t`o
>'ara <up`o t~wn AD, DG met`a to~u >ap`o t~hc ZG >'ison
>est`i t~w| >ap`o t~hc ZD. >'ish d`e <h ZG t~h| ZB; t`o >'ara <up`o
t~wn AD, DG met`a to~u >ap`o t~hc ZB >'ison >est`i t~w| >ap`o t~hc ZD.
 t~w| d`e >ap`o t~hc ZD >'isa
 >est`i t`a >ap`o t~wn ZB, BD; t`o >'ara <up`o t~wn AD, DG met`a
to~u >ap`o t~hc ZB >'ison >est`i to~ic >ap`o t~wn ZB, BD. koin`on
>afh|r'hsjw t`o >ap`o  	t~hc ZB; loip`on >'ara t`o <up`o t~wn AD, DG
>'ison >est`i t~w| >ap`o t~hc DB >efaptom'enhc.}

\gr{>All`a d`h <h DGA m\kern -.7pt `h >'estw di`a to~u k'entrou to~u ABG k'uklou,
ka`i e>il'hfjw t`o k'entron t`o E, ka`i 
>ap`o to~u E >ep`i t`hn AG k'ajetoc >'hqjw <h EZ, ka`i >epeze'uqjwsan
a<i EB, EG, ED; >orj`h >'ara >est`in <h <up`o EBD. ka`i >epe`i e>uje~i'a
tic di`a to~u k'entrou <h EZ e>uje~i'an tina m\kern -.7pt `h di`a to~u k'entrou t`hn
AG pr`oc >orj`ac t'emnei, ka`i d'iqa a\kern -.7pt >ut`hn t'emnei; <h AZ >'ara t~h|
ZG >estin >'ish. ka`i >epe`i e>uje~ia <h AG t'etm\kern -.7pt htai d'iqa kat`a t`o Z
shme~ion, pr'oskeitai d`e a\kern -.7pt >ut~h| <h GD, t`o >'ara <up`o t~wn AD, DG met`a
to~u >ap`o t~hc ZG >'ison >est`i t~w| >ap`o t~hc ZD. koin`on proske'isjw t`o
>ap`o t~hc ZE; t`o >'ara <up`o t~wn AD, DG met`a t~wn >ap`o
t~wn GZ, ZE >'ison >est`i to~ic >ap`o t~wn ZD, ZE. to~ic d`e >ap`o
t~wn GZ, ZE >'ison >est`i t`o >ap`o t~hc EG; >orj`h g`ar [>estin]
 <h <up`o EZG [gwn'ia]; to~ic d`e >ap`o t~wn DZ, ZE >'ison >est`i
 t`o >ap`o t~hc ED; t`o >'ara <up`o t~wn AD, DG met`a to~u >ap`o
 t~hc EG >'ison >est`i t~w| >ap`o t~hc ED. >'ish
 d`e <h EG t`h| EB; 
 t`o >'ara <up`o t~wn AD, DG met`a to~u >ap`o t~hc EB >'ison
 >est`i t~w| >'ap`o t~hc ED.
  t~w| d`e >ap`o t~hc ED
 >'isa >est`i t`a >ap`o t~wn EB, BD; >orj`h g`ar <h <up`o EBD gwn'ia;
 t`o >'ara <up`o t~wn AD, DG met`a to~u >ap`o 
t~hc EB >'ison 
 >est`i to~ic >ap`o t~wn EB, BD. koin`on >afh|r'hsjw t`o >ap`o t~hc
 EB; loip`on >'ara t`o <up`o t~wn AD, DG >'ison >est`i t~w| >ap`o t~hc
 DB.}
 
 \gr{>E`an >'ara k'uklou lhfj~h| ti shme~ion >ekt'oc, ka`i >ap> a\kern -.7pt >uto~u pr`oc t`on
k'uklon prosp'iptwsi d'uo e>uje~iai, ka`i <h m`en a\kern -.7pt >ut~wn t'emnh| t`on
k'uklon, <h d`e >ef'apthtai, >'estai t`o <up`o <'olhc t~hc temno'ushc
ka`i t~hc >ekt`oc >apolambanom'enhc metax`u to~u te shme'iou ka`i
t~hc kurt~hc perifere'iac >'ison t~w| >ap`o t~hc >efaptom'enhc tetrag'wnw|;
<'oper >'edei de~ixai.}}

\ParallelRText{
\begin{center}
{\large Proposition 36}\end{center}

If some point is taken outside a circle, and  two straight-lines
radiate from it towards the circle, and (one) of them cuts the circle, and the (other)
touches (it), (then) the (rectangle contained) by the whole  (straight-line) cutting (the circle), and the (part of it) 
cut off outside (the circle),
between the point and the convex circumference, will be equal to the
square on the tangent (line).

For let some point $D$ be taken outside circle $ABC$, and let two
straight-lines, $DC[A]$ and $DB$, radiate from $D$ towards circle $ABC$. And
let $DCA$ cut circle $ABC$, and let $BD$ touch (it). I say that the rectangle
contained by $AD$ and $DC$ is equal to the square on $DB$.

\epsfysize=1.8in
\centerline{\epsffile{Book03/fig36e.eps}}

$[D]CA$ is surely either through the center, or not. Let it first of all be
through the center, and let $F$ be the center of circle $ABC$, and let 
$FB$ be joined. Thus, (angle) $FBD$ is a right-angle [Prop.~3.18].
And since straight-line $AC$ is cut in half at $F$, let $CD$ be added to it.
Thus, the (rectangle contained) by $AD$ and $DC$ plus the (square) on $FC$ is
equal to the (square) on $FD$ [Prop.~2.6]. And $FC$ (is) equal to $FB$.
Thus, the (rectangle contained) by $AD$ and $DC$ plus the (square) on $FB$
is equal to the (square) on $FD$. And the (square) on $FD$ is equal to
the (sum of the squares) on $FB$ and $BD$ [Prop.~1.47]. Thus, the
(rectangle contained) by $AD$ and $DC$ plus the (square) on $FB$ is equal
to the (sum of the squares) on $FB$ and $BD$. Let the (square) on $FB$ be subtracted
from both. Thus, the remaining (rectangle contained) by $AD$ and $DC$ 
is equal to the (square) on the tangent $DB$.

And so let $DCA$ not be through the center of circle $ABC$, and let the center $E$
be found, and let $EF$ be drawn from $E$, perpendicular to
$AC$ [Prop.~1.12]. And let $EB$, $EC$, and $ED$ be joined. (Angle)
$EBD$ (is) thus a right-angle [Prop.~3.18]. And since some straight-line, $EF$, through the
center, cuts some (other) straight-line, $AC$, not through the center, at
right-angles, it also cuts it in half [Prop.~3.3]. Thus, $AF$ is equal to
$FC$. And since the straight-line $AC$ is cut in half at point $F$, let $CD$ 
be added to it. Thus, the (rectangle contained) by $AD$ and
 $DC$ plus the
(square) on $FC$ is equal to the (square) on $FD$ [Prop.~2.6]. Let
the (square) on $FE$ be added to both. Thus, the (rectangle contained)
by $AD$ and $DC$ plus the (sum of the squares) on $CF$ and $FE$ is equal to
the (sum of the squares) on $FD$ and $FE$. But  the (square) on $EC$  is equal to the (sum of the squares)
on $CF$ and $FE$. For [angle] $EFC$ [is] a
right-angle [Prop.~1.47]. And the (square) on $ED$ is equal to the (sum of the squares) on $DF$ and
$FE$   [Prop.~1.47]. Thus, the (rectangle
contained) by $AD$ and $DC$ plus the (square) on $EC$ is equal to the (square)
on $ED$. And $EC$ (is) equal to $EB$. Thus, the (rectangle
contained) by $AD$ and $DC$ plus the (square) on $EB$ is equal to the (square)
on $ED$. And the (sum of the squares) on 
$EB$ and $BD$  is equal to  the (square) on $ED$. For $EBD$ (is) a right-angle [Prop.~1.47]. Thus, the (rectangle
contained) by $AD$ and $DC$ plus the (square) on $EB$ is equal to the (sum of the squares)
on $EB$ and $BD$. Let the (square) on $EB$ be subtracted from both. Thus, the remaining
(rectangle contained) by $AD$ and $DC$ is equal to the (square) on $BD$.

Thus, if some point is taken outside a circle, and  two straight-lines
radiate from it towards the circle, and (one) of them cuts the circle, and (the other) touches (it), (then) the (rectangle contained) by the whole  (straight-line) cutting (the circle), and the (part of it) 
cut off outside (the circle),
between the point and the convex circumference, will be equal to the
square on the tangent (line). (Which is) the very thing it was required to show.}
\end{Parallel}

%%%%%%
% Prop 3.37
%%%%%%
\pdfbookmark[1]{Proposition 3.37}{pdf3.37}
\begin{Parallel}{}{} 
\ParallelLText{
\begin{center}
{\large \ggn{37}.}
\end{center}\vspace*{-7pt}

\gr{>E`an k'uklou lhfj~h| ti shme~ion >ekt'oc, >ap`o d`e to~u shme'iou pr`oc t`on
k'uklon prosp'iptwsi d'uo e>uje~iai, ka`i <h m`en a\kern -.7pt >ut~wn t'emnh|
t`on k'uklon, <h d`e prosp'ipth|, >~h| d`e t`o <up`o [t~hc] <'olhc t~hc
temno'ushc ka`i t~hc >ekt`oc >apolambanom'enhc metax`u to~u te shme'iou ka`i
t~hc kurt~hc perifere'iac >'ison t~w| >ap`o t~hc prospipto'ushc, <h
prosp'iptousa >ef'ayetai to~u k'uklou.}

\gr{K'uklou g`ar to~u ABG e>il'hfjw ti shme~ion >ekt`oc t`o D, ka`i >ap`o
to~u D pr`oc t`on ABG k'uklon prospipt'etwsan d'uo e>uje~iai a<i
DGA, DB, ka`i <h m`en DGA temn'etw t`on k'uklon, <h d`e DB prospipt'etw,
>'estw d`e t`o <up`o t~wn AD, DG >'ison t~w| >ap`o t~hc DB. l'egw,
<'oti <h DB >ef'aptetai to~u ABG k'uklou.}\\~\\

\epsfysize=1.75in
\centerline{\epsffile{Book03/fig37g.eps}}

\gr{>'Hqjw g`ar to~u ABG >efaptom'enh <h DE, ka`i e>il'hfjw t`o k'entron
to~u ABG k'uklou, ka`i >'estw t`o Z, ka`i >epeze'uqjwsan a<i ZE, ZB, ZD.
<h >'ara <up`o ZED >orj'h >estin. ka`i >epe`i <h DE >ef'aptetai to~u
ABG k'uklou, t'emnei d`e <h DGA, t`o >'ara <up`o t~wn AD, DG
>'ison >est`i t~w| >ap`o t~hc DE. >~hn d`e ka`i t`o <up`o t~wn
AD, DG >'ison t~w| >ap`o t~hc DB; t`o >'ara >ap`o t~hc DE >'ison >est`i t~w| >ap`o
t~hc DB;
>'ish >'ara <h DE t~h| DB.
>est`i d`e ka`i <h ZE t~h| ZB >'ish; d'uo d`h a<i DE, EZ d'uo ta~ic DB, BZ
>'isai e>is'in; ka`i b'asic a\kern -.7pt >ut~wn koin`h <h ZD; gwn'ia >'ara <h <up`o
DEZ gwn'ia| t~h| <up`o DBZ >estin >'ish. >orj`h d`e <h <up`o DEZ; >orj`h
>'ara ka`i <h <up`o DBZ. ka'i >estin <h ZB >ekballom'enh di'ametroc;
<h d`e t~h| diam'etrw| to~u k'uklou pr`oc >orj`ac >ap>
>'akrac >agom'enh >ef'aptetai to~u k'uklou; <h DB >'ara >ef'aptetai
to~u ABG k'uklou. <omo'iwc d`h deiqj'hsetai, k>`an t`o k'entron  >ep`i
t~hc AG tugq'anh|.}

\gr{>E`an >'ara k'uklou lhfj~h| ti shme~ion >ekt'oc, >ap`o d`e to~u shme'iou pr`oc t`on
k'uklon prosp'iptwsi d'uo e>uje~iai, ka`i <h m`en a\kern -.7pt >ut~wn t'emnh|
t`on k'uklon, <h d`e prosp'ipth|, >~h| d`e t`o <up`o  <'olhc t~hc
temno'ushc ka`i t~hc >ekt`oc >apolambanom'enhc metax`u to~u te shme'iou ka`i
t~hc kurt~hc perifere'iac >'ison t~w| >ap`o t~hc prospipto'ushc, <h
prosp'iptousa >ef'ayetai to~u k'uklou; <'oper >'edei de~ixai.}}

\ParallelRText{
\begin{center}
{\large Proposition 37}
\end{center}

If some point is taken outside a circle, and two straight-lines radiate
from the point towards the circle, and one of them cuts the circle, and the
(other) meets (it), and the (rectangle contained) by the whole (straight-line)
cutting (the circle), and the (part of it) 
cut off outside (the circle),
between the point and the convex circumference,  is equal to the (square)
on the (straight-line) meeting (the circle), (then) the (straight-line)
meeting (the circle) will touch the circle.

For let some point $D$ be taken outside circle $ABC$, and let two
straight-lines, $DCA$ and $DB$, radiate from $D$ towards circle $ABC$, and
let $DCA$ cut the circle, and let $DB$ meet (the circle). And let the (rectangle
contained) by $AD$ and $DC$ be equal to the (square) on $DB$. I say that $DB$ touches circle $ABC$.

\epsfysize=1.75in
\centerline{\epsffile{Book03/fig37e.eps}}

For let $DE$ be drawn touching $ABC$ [Prop. 3.17], and let the center of the circle
$ABC$ be found, and let it be (at) $F$. And let $FE$, $FB$, and $FD$ 
be joined. (Angle) $FED$ is thus a right-angle [Prop.~3.18].
And since $DE$ touches circle $ABC$, and $DCA$ cuts (it), the (rectangle contained) by $AD$ and $DC$ is thus equal to the (square) on $DE$ [Prop.~3.36]. And the (rectangle contained) by $AD$ and $DC$ was also equal
to the (square) on $DB$.  Thus, the (square) on $DE$ is equal to the
(square) on $DB$. Thus, $DE$ (is) equal to $DB$. And $FE$ is also equal to
$FB$. So the two (straight-lines) $DE$, $EF$ are equal to the two (straight-lines)
$DB$, $BF$ (respectively).  And their base, $FD$, is common. Thus, angle $DEF$ is equal to
angle $DBF$ [Prop.~1.8]. And $DEF$ (is) a right-angle. Thus, $DBF$
(is) also a right-angle. And $FB$ produced is a diameter, And a (straight-line)
drawn at right-angles to a diameter of a circle, at its extremity, touches the
circle [Prop.~3.16~corr.]. Thus, $DB$ touches circle $ABC$. 
Similarly, (the same thing) can be shown, even if the center happens to be
on $AC$.

Thus, if some point is taken outside a circle, and two straight-lines radiate
from the point towards the circle, and one of them cuts the circle, and the
(other) meets (it), and the (rectangle contained) by the whole (straight-line)
cutting (the circle), and the (part of it) 
cut off outside (the circle),
between the point and the convex circumference,  is equal to the (square)
on the (straight-line) meeting (the circle), (then) the (straight-line)
meeting (the circle) will touch the circle. (Which is) the very thing it was
required to show.}
\end{Parallel}

\newpage~\thispagestyle{empty}
