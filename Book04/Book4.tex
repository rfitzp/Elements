%%%%%%
% BOOK 4
%%%%%%
\pdfbookmark[0]{Book 4}{book4}
\pagestyle{plain}
\begin{center}
{\Huge ELEMENTS BOOK 4}\\
\spa\spa\spa
{\huge\it Construction of Rectilinear Figures In and}\\[0.5ex] {\huge\it Around Circles}
\end{center}\newpage

%%%%%%%
% Definitions
%%%%%%%
\pdfbookmark[1]{Definitions}{def4}
\pagestyle{fancy}
\cfoot{\gr{\thepage}}
\lhead{\large\gr{STOIQEIWN \ggn{4}.}}
\rhead{\large ELEMENTS BOOK 4}
\begin{Parallel}{}{} 
\ParallelLText{
\begin{center}
\large{\gr{<'Oroi}.}
\end{center}\vspace*{-7pt}

\ggn{1}.~\gr{Sq~hma e>uj'ugrammon e>ic sq~hma e>uj'ugrammon >eggr'af\-esjai l'egetai, <'otan <ek'asth t~wn to~u >eggrafom'enou sq'hmat\-oc gwni~wn
<ek'asthc pleur~ac to~u, e>ic <`o >eggr'afetai, <'apthtai.}

\ggn{2}.~\gr{Sq~hma d`e <omo'iwc per`i sq~hma perigr'afesjai l'egetai, <'otan <ek'asth
pleur`a to~u perigrafom'enou <ek'asthc gwn'iac to~u, per`i <`o perigr'afetai,
<'apthtai.}

\ggn{3}.~\gr{Sq~hma e>uj'ugrammon e>ic k'uklon >eggr'afesjai l'egetai, <'otan
<ek'asth gwn'ia to~u >eggrafom'enou <'apthtai t~hc to~u k'uklou
perifere'iac.}

\ggn{4}.~\gr{Sq~hma d`e e>uj'ugrammon per`i k'uklon perigr'afe\-sjai l'egetai,
<'otan <ek'asth pleur`a to~u perigrafom'enou >ef'apthtai t~hc to~u k'uklou
perifere'iac.}

\ggn{5}.~\gr{K'ukloc d`e e>ic sq~hma <omo'iwc >eggr'afesjai l'egetai, <'otan
<h to~u k'uklou perif'ereia <ek'asthc pleur~ac to~u, e>ic <`o >eggr'afetai,
<'apthtai.}

\ggn{6}.~\gr{K'ukloc d`e per`i sq~hma perigr'afesjai l'egetai, <'otan <h to~u
k'uklou perif'ereia <ek'asthc gwn'iac to~u, per`i <`o perigr'afetai, <'apthtai.}

\ggn{7}.~\gr{E>uje~ia e>ic k'uklon >enarm'ozesjai l'egetai, <'otan t`a p'erata
a>ut~hc >ep`i t~hc perifere'iac >~h| to~u k'uklou.}}

\ParallelRText{
\begin{center}
{\large Definitions}
\end{center}

1.~A rectilinear figure is said to be inscribed in a(nother) rectilinear figure
when  the respective angles of the inscribed figure touch the respective
sides of the (figure) in which it is inscribed.

2.~And, similarly, a  (rectilinear) figure is said to be circumscribed about a(nother rectilinear) figure when the respective sides of the circumscribed
(figure) touch the respective angles of the (figure) about which it is
circumscribed.

3.~A rectilinear figure is said to be inscribed in a circle when
each angle of the inscribed (figure) touches the circumference of the
circle.

4.~And a rectilinear figure is said to be circumscribed about a circle
when each side of the circumscribed (figure) touches the
circumference of the circle.

5.~And, similarly, a circle is said to be inscribed in a (rectilinear) figure
when the circumference of the circle touches each  side of
the (figure) in which it is inscribed.

6.~And a circle is said to be circumscribed about a rectilinear (figure)
when the circumference of the circle touches each  angle of the
(figure) about which it is circumscribed.

7.~A straight-line is said to be inserted into a circle when its extemities
are on the circumference of the circle.}
\end{Parallel}

%%%%%%
% Prop 4.1
%%%%%%
\pdfbookmark[1]{Proposition 4.1}{pdf4.1}
\begin{Parallel}{}{} 
\ParallelLText{
\begin{center}
{\large \ggn{1}.}
\end{center}\vspace*{-7pt}

\gr{E>ic t`on doj'enta k'uklon t~h| doje'ish| e>uje'ia| m`h me'izoni o>'ush|
t~hc to~u k'uklou diam'etrou >'ishn e>uje~ian >enarm'osai.}\\

\epsfysize=2.2in
\centerline{\epsffile{Book04/fig01g.eps}}

\gr{>'Estw <o doje`ic k'ukloc <o ABG, <h d`e doje~isa e>uje~ia m`h
me'izwn t~hc to~u k'uklou diam'etrou <h D. de~i d`h e>ic t`on
ABG k'uklon t~h| D e>uje'ia| >'ishn e>uje~ian >enarm'osai.}

\gr{>'Hqjw to~u ABG k'uklou di'ametroc <h BG. e>i m`en
o>~un >'ish >est`in <h BG t~h| D, gegon`oc >`an e>'ih t`o
>epitaqj'en; >en'hrmostai g`ar e>ic t`on ABG k'uklon t~h|
D e>uje'ia| >'ish <h BG. e>i d`e me'izwn >est`in <h BG t~hc
D, ke'isjw t~h| D >'ish <h GE, ka`i k'entrw| t~w| G diast'hmati
d`e t~w| GE k'ukloc gegr'afjw <o EAZ, ka`i >epeze'uqjw
<h GA.}

\gr{>Epe`i o>~un to G shme~ion k'entron >est`i to~u EAZ k'uklou, >'ish
 >est`in <h GA t~h| GE. >all`a t~h| D <h GE >estin >'ish;
ka`i <h D >'ara t~h| GA >estin >'ish.}

\gr{E>ic >'ara t`on doj'enta k'uklon t`on ABG t~h| doje'ish| e>uje'ia| t~h|
D >'ish >en'hrmostai <h GA; <'oper >'edei poi~hsai.}}

\ParallelRText{
\begin{center}
{\large Proposition 1}
\end{center}

To insert a straight-line equal to
a given straight-line into a circle, (the latter straight-line) not being greater than the diameter of the circle.

\epsfysize=2.2in
\centerline{\epsffile{Book04/fig01e.eps}}

Let $ABC$ be the given circle, and $D$ the given straight-line (which is) not greater than
the diameter of the circle. So it is required to insert a straight-line, equal to the
straight-line $D$, into the circle $ABC$.

Let a diameter $BC$ of circle $ABC$ have been drawn.$^\dag$ Therefore, if
$BC$ is equal to $D$ then that (which)  was prescribed has taken place.
For the (straight-line) $BC$, equal to the straight-line $D$, has been inserted into the circle $ABC$.
And if $BC$ is greater than $D$ then let $CE$ be made equal to $D$ [Prop.~1.3], and let the circle $EAF$ have been drawn with center $C$ and radius $CE$. And let $CA$ have been joined.

Therefore, since the point $C$ is the center of  circle $EAF$, $CA$ is equal to $CE$. But,
$CE$ is equal to $D$. Thus, $D$ is also equal to $CA$.

Thus, $CA$, equal to the given straight-line $D$,  has been inserted into
the given circle $ABC$. (Which is) the very thing it was required to do.}
\end{Parallel}
{\footnotesize \noindent$^\dag$ Presumably, by finding the center of the circle [Prop.~3.1], and then
drawing a line through it.}

%%%%%%
% Prop 4.2
%%%%%%
\pdfbookmark[1]{Proposition 4.2}{pdf4.2}
\begin{Parallel}{}{} 
\ParallelLText{
\begin{center}
{\large \ggn{2}.}
\end{center}\vspace*{-7pt}

\gr{E>ic t`on doj'enta k'uklon t~w| doj'enti trig'wnw| >isog'wnion tr'igwnon
>eggr'ayai.}

\epsfysize=2in
\centerline{\epsffile{Book04/fig02g.eps}}

\gr{>'Estw <o doje`ic k'ukloc <o ABG, t`o d`e doj`en trigwnon t`o DEZ;
de~i d`h e>ic t`on ABG k'uklon t~w| DEZ trig'wnw| >isog'wnion
tr'igwnon >eggr'ayai.}

\gr{>'Hqjw to~u ABG k'uklou >efaptom'enh <h HJ kat`a t`o A,
ka`i sunest'atw pr`oc t~h| AJ e>uje'ia| ka`i t~w| pr`oc a>ut~h| shme'iw|
t~w| A t~h| <up`o DEZ gwn'ia| >'ish <h <up`o
JAG, pr`oc d`e t~h| AH e>uje'ia| ka`i t~w| pr`oc a>ut~h|
shme'iw| t~w| A t~h| <up`o DZE [gwn'ia|] >'ish <h <up`o HAB, ka`i
>epeze'uqjw <h BG.}

\gr{>Epe`i o>~un k'uklou to~u ABG >ef'apteta'i tic e>uje~ia <h AJ, ka`i
>ap`o t~hc kat`a t`o A >epaf~hc e>ic t`on k'uklon di~hktai e>uje~ia
<h AG, <h >'ara <up`o JAG >'ish >est`i t~h| >en t~w| >enall`ax to~u
k'uklou tm'hmati gwn'ia| t~h| <up`o ABG. >all> <h <up`o JAG t~h|
<up`o DEZ >estin >'ish; ka`i <h <up`o ABG >'ara gwn'ia t~h| <up`o
DEZ >estin >'ish. di`a t`a a>ut`a d`h ka`i <h <up`o AGB t~h| <up`o
DZE >estin >'ish; ka`i loip`h >'ara <h <up`o BAG loip~h| t~h| <up`o
EDZ >estin >'ish [>isog'wnion >'ara >est`i t`o ABG tr'igwnon t~w|
DEZ trig'wnw|, ka`i >egg'egraptai e>ic t`on ABG k'uklon].}

\gr{E>ic t`on doj'enta >'ara k'uklon t~w| doj'enti trig'wnw|
>isog'wnion tr'igwnon >egg'egraptai; <'oper >'edei poi~hsai.}}

\ParallelRText{
\begin{center}
{\large Proposition 2}
\end{center}

To inscribe a triangle, equiangular with a given triangle, in a given circle.

\epsfysize=2in
\centerline{\epsffile{Book04/fig02e.eps}}

Let $ABC$ be the given circle, and $DEF$ the given triangle. So it is required
to inscribe a triangle, equiangular with triangle $DEF$, in circle $ABC$.

Let $GH$ have been drawn touching circle $ABC$ at $A$.$^\dag$ And let  (angle) $HAC$, equal to angle $DEF$,  have been constructed
on the straight-line $AH$ at the point $A$ on it, and (angle) $GAB$, equal to
[angle] $DFE$,  on the straight-line $AG$ at the point $A$ on it
 [Prop.~1.23]. And let $BC$ have been joined.
 
 Therefore, since some straight-line $AH$ touches the circle $ABC$, and the
 straight-line $AC$ has been drawn across (the circle) from the point of
 contact $A$, (angle) $HAC$ is thus equal to the angle $ABC$ in the alternate
 segment of the circle [Prop.~3.32]. But, $HAC$ is equal to $DEF$.
 Thus, angle $ABC$ is also equal to $DEF$. So, for the same (reasons),
 $ACB$ is also equal to $DFE$.  Thus, the remaining (angle) $BAC$ is equal
 to the remaining (angle) $EDF$ [Prop.~1.32]. [Thus, triangle
 $ABC$ is equiangular with triangle $DEF$, and has been inscribed in circle
 $ABC$].
 
 Thus, a triangle, equiangular with the given triangle, has been inscribed in
 the given circle. (Which is) the very thing it was required to do.}
\end{Parallel}
{\footnotesize \noindent$^\dag$ See the footnote
to Prop.~3.34.}

%%%%%%
% Prop 4.3
%%%%%%
\pdfbookmark[1]{Proposition 4.3}{pdf4.3}
\begin{Parallel}{}{} 
\ParallelLText{
\begin{center}
{\large \ggn{3}.}
\end{center}\vspace*{-7pt}

\gr{Per`i t`on doj'enta k'uklon t~w| doj'enti trig'wnw| >isog'wnion tr'igwnon perigr'ayai.}

\epsfysize=1.8in
\centerline{\epsffile{Book04/fig03g.eps}}

\gr{>'Estw <o doje`ic k'ukloc <o ABG, t`o d`e doj`en tr'igwnon t`o DEZ; de~i
d`h per`i t`on ABG k'uklon t~w| DEZ trig'wnw| >isog'wnion tr'igwnon
perigr'ayai.}

\gr{>Ekbebl'hsjw <h EZ >ef> <ek'atera t`a m'erh kat`a t`a H, J shme~ia,
ka`i e>il'hfjw to~u ABG k'uklou k'entron t`o K, ka`i di'hqjw, <wc
>'etuqen, e>uje~ia <h KB, ka`i sunest'atw pr`oc t~h| KB e>uje'ia|
ka`i t~w| pr`oc a>ut~h| shme'iw| t~w| K t~h| m`en <up`o DEH gwn'ia|
>'ish <h <up`o BKA, t~h| d`e <up`o DZJ >'ish <h <up`o BKG, ka`i di`a
t~wn A, B, G shme'iwn >'hqjwsan >efapt'omenai to~u ABG k'uklou
a<i LAM, MBN, NGL.}

\gr{Ka`i >epe`i >ef'aptontai to~u ABG k'uklou a<i LM, MN,
NL kat`a t`a A, B, G shme~ia, >ap`o d`e to~u K k'entrou >ep`i
t`a A, B, G shme~ia
>epezeugm'enai e>is`in  a<i KA, KB, KG, >orja`i >'ara e>is`in a<i
pr`oc to~ic A, B, G shme'ioic gwn'iai. ka`i >epe`i to~u AMBK tetraple'urou a<i t'essarec gwn'iai t'etrasin >orja~ic >'isai e>is'in, >epeid'hper ka`i
e>ic d'uo tr'igwna diaire~itai t`o AMBK, ka'i e>isin >orja`i a<i <up`o
KAM, KBM gwn'iai, loipa`i >'ara a<i <up`o AKB, AMB dus`in >orja~ic
>'isai e>is'in. e>is`i d`e ka`i a<i <up`o DEH, DEZ dus`in >orja~ic >'isai;
a<i >'ara <up`o AKB, AMB ta~ic <up`o DEH, DEZ >'isai e>is'in,
<~wn <h <up`o AKB t~h| <up`o DEH >estin >'ish; loip`h >'ara <h <up`o
AMB loip~h| t~h| <up`o DEZ >estin >'ish.
 <omo'iwc d`h
deiqj'hsetai, <'oti ka`i <h <up`o LNB t~h| <up`o DZE >estin >'ish;
ka`i loip`h >'ara <h <up`o MLN [loip~h|] t~h| <up`o EDZ >estin
>'ish. >isog'wnion >'ara >est`i t`o LMN tr'igwnon t~w| DEZ
trig'wnw|; ka`i perig'egraptai per`i t`on ABG k'uklon.}

\gr{Per`i t`on doj'enta >'ara k'uklon t~w| doj'enti trig'wnw| >isog'wnion
tr'igwnon perig'egraptai; <'oper >'edei poi~hsai.}}

\ParallelRText{
\begin{center}
{\large Proposition 3}
\end{center}

To circumscribe a triangle, equiangular with a given triangle, about a given
circle.

\epsfysize=1.8in
\centerline{\epsffile{Book04/fig03e.eps}}

Let $ABC$ be the given circle, and $DEF$ the given triangle. So it is required
to circumscribe a triangle, equiangular with triangle  $DEF$, about circle $ABC$.

Let $EF$ have been produced in each direction to points $G$ and 
$H$. And
let the center $K$ of circle $ABC$ have been found [Prop.~3.1].
And let the straight-line $KB$ have been drawn, at random,  across ($ABC$).
And let (angle) $BKA$, equal to angle $DEG$, have been constructed on the straight-line $KB$ at the point $K$ on it, and (angle) $BKC$, equal to
$DFH$ [Prop.~1.23]. And let the (straight-lines) $LAM$, $MBN$, and $NCL$ have been 
drawn through the points $A$, $B$, and $C$ (respectively), touching the circle $ABC$.$^\dag$

And since $LM$, $MN$, and $NL$ touch circle $ABC$ at points $A$, $B$, and $C$
(respectively), and $KA$, $KB$, and $KC$ are joined from the center $K$
to  points $A$, $B$, and $C$ (respectively), the angles at points
$A$, $B$, and $C$ are thus right-angles [Prop.~3.18]. And
since the (sum of the) four angles of quadrilateral $AMBK$ is equal to
four right-angles,
inasmuch as $AMBK$ (can) also (be) divided into two triangles [Prop.~1.32],
and angles $KAM$ and $KBM$ are (both) right-angles, the (sum of the) remaining (angles), $AKB$ and $AMB$, is thus equal to two right-angles. 
And $DEG$ and $DEF$ is also equal to two right-angles [Prop.~1.13].
Thus, $AKB$ and $AMB$ is equal to $DEG$ and $DEF$, of which $AKB$ is
equal to $DEG$. Thus, the remainder $AMB$ is equal to the remainder $DEF$.
So, similarly, it can be shown that $LNB$ is also equal to $DFE$. Thus, the
remaining (angle) $MLN$ is also equal to the [remaining] (angle) $EDF$ [Prop.~1.32]. Thus,
triangle $LMN$ is equiangular with triangle $DEF$. And it has been drawn around
circle $ABC$.

Thus,  a triangle, equiangular with  the given triangle, has been circumscribed about the given
circle. (Which is) the very thing it was required to do.}
\end{Parallel}
{\footnotesize \noindent$^\dag$ See the footnote
to Prop.~3.34.}

%%%%%%
% Prop 4.4
%%%%%%
\pdfbookmark[1]{Proposition 4.4}{pdf4.4}
\begin{Parallel}{}{} 
\ParallelLText{
\begin{center}
{\large \ggn{4}.}
\end{center}\vspace*{-7pt}

\gr{E>ic t`o doj`en  tr'igwnon k'uklon >eggr'ayai.}

\epsfysize=2in
\centerline{\epsffile{Book04/fig04g.eps}}

\gr{>'Estw t`o doj`en tr'igwnon t`o ABG; de~i d`h e>ic t`o ABG tr'igwnon k'uklon >eggr'ayai.}

\gr{Tetm'hsjwsan a<i <up`o ABG, AGB gwn'iai d'iqa ta~ic BD, GD e>uje'iaic,
ka`i sumball'etwsan >all'hlaic kat`a t`o D shme~ion, ka`i >'hqjwsan >ap`o
to~u D >ep`i t`ac AB, BG, GA 	e>uje'iac k'ajetoi a<i DE, DZ, DH.}

\gr{Ka`i >epe`i >'ish >est`in <h <up`o ABD gwn'ia t~h| <up`o GBD, >est`i
d`e ka`i >orj`h <h <up`o BED >orj~h| t~h| <up`o BZD >'ish, d'uo
d`h tr'igwn'a >esti t`a EBD, ZBD t`ac d'uo gwn'iac ta~ic dus`i gwn'iaic
>'isac >'eqonta ka`i m'ian pleur`an mi~a| pleur~a| >'ishn t`hn <upote'inousan
<up`o m'ian
t~wn >'iswn gwni~wn koin`hn a>ut~wn t`hn
BD; ka`i t`ac loip`ac >'ara pleur`ac ta~ic loipa~ic pleura~ic >'isac
<'exousin; >'ish >'ara <h DE t~h| DZ. di`a t`a a>ut`a d`h ka`i <h
DH t~h| DZ >estin >'ish. a<i tre~ic >'ara e>uje~iai a<i DE, DZ, DH
>'isai >all'hlaic e>is'in; <o >'ara k'entr~w| t~w| D ka`i diast'hmati
<en`i t~wn E, Z, H k'ukloc graf'omenoc <'hxei ka`i di`a t~wn loip~wn
shme'iwn ka`i >ef'ayetai t~wn AB, BG, GA e>ujei~wn di`a t`o >orj`ac e>~inai t`ac pr`oc to~ic E, Z, H shme'ioic gwn'iac. e>i g`ar teme~i
a>ut'ac, >'estai <h t~h| diam'etrw| to~u k'uklou pr`oc >orj`ac
>ap> >'akrac >agom'enh >ent`oc p'iptousa to~u k'uklou; <'oper
>'atopon >ede'iqjh; o>uk >'ara <o k'entrw| t~w| D diast'hmati
d`e <en`i t~wn E, Z, H graf'omenoc k'ukloc teme~i t`ac AB, BG, GA e>uje'iac; >ef'ayetai >'ara a>ut~wn, ka`i >'estai <o 
k'ukloc >eggegramm'enoc e>ic t`o ABG tr'igwnon. 
>eggegr'afjw <wc <o ZHE.}

\gr{E>ic >'ara t`o doj`en tr'igwnon t`o ABG k'ukloc
>egg'egr\-aptai <o EZH; <'oper >'edei poi~hsai.}}

\ParallelRText{
\begin{center}
{\large Proposition 4}
\end{center}

To inscribe a circle in a given triangle.

\epsfysize=2in
\centerline{\epsffile{Book04/fig04e.eps}}

Let $ABC$ be the given triangle. So it is required to inscribe
a circle in triangle $ABC$.

Let the angles $ABC$ and $ACB$ have been cut in half by the
straight-lines $BD$ and $CD$ (respectively) [Prop.~1.9], and let them meet one another
at point $D$, and let $DE$, $DF$, and $DG$ have been drawn from
point $D$, perpendicular to the straight-lines $AB$, $BC$, and $CA$ (respectively) [Prop.~1.12].

And since angle $ABD$ is equal to $CBD$, and the right-angle
$BED$ is also equal to the right-angle $BFD$, $EBD$ and $FBD$ are thus two triangles
having two angles equal to two angles, and one side equal to one side---the
(one) subtending one of the equal angles (which is) common to the (triangles)---(namely), $BD$. Thus, they will also have the 
remaining sides equal to the (corresponding) remaining sides [Prop.~1.26]. Thus, $DE$ (is)
equal to $DF$. So, for the same (reasons), $DG$ is also equal
to $DF$. Thus, the three straight-lines $DE$, $DF$, and $DG$ are equal to one another.
Thus, the circle drawn with center $D$, and radius one of $E$, $F$, or $G$,$^\dag$ will also
go through the remaining points, and will  touch the straight-lines $AB$, $BC$, and $CA$, on account of the angles at $E$, $F$, and $G$ being right-angles. For if it cuts (one of) them then it will be a (straight-line)
drawn at right-angles to a diameter of the circle, from its extremity, falling inside
the circle. The very thing was shown (to be) absurd [Prop.~3.16].
Thus, the circle drawn with center $D$, and radius one of $E$, $F$, or $G$,
does not cut the straight-lines $AB$, $BC$, and $CA$. Thus, it will
touch them and will be the circle inscribed  in triangle $ABC$.
Let it have been (so) inscribed, like $FGE$ (in the figure).

Thus, the circle $EFG$ has been inscribed in the given triangle $ABC$.
(Which is) the very thing it was required to do.}
\end{Parallel}
{\footnotesize \noindent$^\dag$ Here, and in the following propositions, it
is understood that the radius is actually one of $DE$, $DF$, or $DG$.} 

%%%%%%
% Prop 4.5
%%%%%%
\pdfbookmark[1]{Proposition 4.5}{pdf4.5}
\begin{Parallel}{}{} 
\ParallelLText{
\begin{center}
{\large \ggn{5}.}
\end{center}\vspace*{-7pt}

\gr{Per`i t`o doj`en tr'igwnon k'uklon perigr'ayai.}

\epsfysize=1.05in
\centerline{\epsffile{Book04/fig05g.eps}}

\gr{>'Estw t`o doj`en tr'igwnon t`o ABG; de~i d`e per`i t`o doj`en tr'igwnon
t`o ABG k'uklon perigr'ayai.}

\gr{Tetm'hsjwsan a<i AB, AG e>uje~iai d'iqa kat`a t`a D, E shme~ia, ka`i
>ap`o t~wn D, E shme'iwn ta~ic AB, AG pr`oc >orj`ac >'hqjwsan
a<i DZ, EZ; sumpeso~untai d`h >'htoi >ent`oc to~u ABG trig'wnou
>`h >ep`i t~hc BG e>uje'iac >`h >ekt`oc t~hc BG.}

\gr{Sumpipt'etwsan pr'oteron >ent`oc kat`a t`o Z, ka`i >epeze'uqj\-wsan
a<i ZB, ZG, ZA. ka`i >epe`i >'ish >est`in <h AD t~h| DB, koin`h
d`e ka`i pr`oc >orj`ac <h DZ, b'asic >'ara <h AZ b'asei t~h| ZB
>estin >'ish. <omo'iwc d`h de'ixomen, <'oti ka`i <h GZ t~h| AZ >estin
>'ish; <'wste ka`i <h ZB t~h| ZG >estin >'ish; a<i tre~ic
>'ara a<i ZA, ZB, ZG >'isai  >all'hlaic e>is'in. <o >'ara k'entrw|
t~w| Z diast'hmati d`e <en`i t~wn A, B, G k'ukloc graf'omenoc <'hxei
ka`i di`a t~wn loip~wn shme'iwn, ka`i >'estai perigegramm'enoc
<o k'ukloc per`i t`o ABG tr'igwnon. perigegr'afjw <wc <o ABG.}

\gr{>All`a d`h a<i DZ, EZ sumpipt'etwsan >ep`i t~hc BG e>uje'iac
kat`a t`o Z, <wc >'eqei >ep`i t~hc deut'erac katagraf~hc, ka`i
>epeze'uqjw <h AZ. <omo'iwc d`h de'ixomen, <'oti t`o Z shme~ion
k'entron >est`i to~u per`i t`o ABG tr'igwnon perigrafom'enou k'uklou.}

\gr{>All`a d`h a<i DZ, EZ sumpipt'etwsan >ekt`oc to~u ABG trig'wnou kat`a
t`o Z p'alin, <wc >'eqei >ep`i t~hc tr'ithc katagraf~hc, ka'i 
>epeze'uqjwsan a<i AZ, BZ, GZ. ka`i >epe`i p'alin >'ish >est`in <h
AD t~h| DB, koin`h d`e ka`i pr`oc >orj`ac <h DZ, b'asic >'ara
<h AZ b'asei t~h| BZ >estin >'ish. <omo'iwc d`h de'ixomen, <'oti
ka`i <h GZ t~h| AZ >estin >'ish; <'wste ka`i <h BZ t~h| ZG >estin
>'ish; <o >'ara [p'alin] k'entrw| t~w| Z diast'hmati d`e <en`i t~wn
ZA, ZB, ZG k'ukloc graf'omenoc <'hxei ka`i di`a t~wn loip~wn shme'iwn,
ka`i >'estai perigegramm'enoc per`i t`o ABG tr'igwnon.}

\gr{Per`i t`o doj`en >'ara tr'igwnon k'ukloc perig'egraptai; <'oper >'edei poi~hsai.}}

\ParallelRText{
\begin{center}
{\large Proposition 5}
\end{center}

To circumscribe a circle about a given  triangle.

\epsfysize=1.05in
\centerline{\epsffile{Book04/fig05e.eps}}

Let $ABC$ be the given triangle. So it is required to circumscribe
a circle about the given triangle $ABC$.

Let the straight-lines $AB$ and $AC$ have been cut in half at points $D$ and $E$ (respectively) [Prop.~1.10]. And let $DF$ and $EF$ have been drawn from 
points $D$ and $E$, at right-angles to $AB$ and $AC$ (respectively) [Prop.~1.11].
So  ($DF$ and $EF$) will surely either meet inside triangle $ABC$, on the straight-line
$BC$, or beyond $BC$.

Let them, first of all, meet inside (triangle $ABC$) at (point) $F$, and let $FB$, 
$FC$, and $FA$ have been joined. And since $AD$ is equal to $DB$, and $DF$ is
common and at right-angles, the base $AF$ is thus equal to the base $FB$
[Prop.~1.4]. So, similarly, we can show that $CF$ is also equal to
$AF$. So that $FB$ is also equal to $FC$. Thus, the three (straight-lines)
$FA$, $FB$, and $FC$ are equal to one another. Thus, the circle drawn with
center $F$, and radius one of $A$, $B$, or $C$, will also go through the remaining points.
And the circle will have been circumscribed about triangle $ABC$. Let it
have been (so) circumscribed, like $ABC$ (in the first diagram from the left).

And so, let $DF$ and $EF$ meet on the straight-line $BC$ at (point) $F$, like
in the second diagram (from the left). And let $AF$ have been joined.
So, similarly, we can show that point $F$ is the center of the circle circumscribed about triangle $ABC$.

And so, let $DF$ and $EF$ meet outside triangle $ABC$, again at (point) $F$, like
in the third diagram (from the left). And let $AF$, $BF$, and $CF$ have been joined.
And, again, since $AD$ is equal to $DB$, and $DF$ is common and at right-angles,
the base $AF$ is thus equal to the base $BF$ [Prop.~1.4]. So, similarly,
we can show that $CF$ is also equal to $AF$. So that $BF$ is also equal to $FC$.
Thus, [again] the circle drawn with center $F$, and radius one of $FA$, $FB$, and
$FC$, will also go through the remaining points. And it will have been circumscribed
about triangle $ABC$.

Thus, a circle has been circumscribed about the given triangle. (Which is)
the very thing it was required to do.}
\end{Parallel}

%%%%%%
% Prop 4.6
%%%%%%
\pdfbookmark[1]{Proposition 4.6}{pdf4.6}
\begin{Parallel}{}{} 
\ParallelLText{
\begin{center}
{\large \ggn{6}.}
\end{center}\vspace*{-7pt}

\gr{E>ic t`on doj'enta k'uklon tetr'agwnon >eggr'ayai.}

\epsfysize=2.2in
\centerline{\epsffile{Book04/fig06g.eps}}

\gr{>'Estw <h doje`ic k'ukloc <o ABGD; de~i d`h e>ic t`on ABGD k'uklon
tetr'agwnon >eggr'ayai.}

\gr{>'Hqjwsan to~u ABGD k'uklou d'uo di'ametroi pr`oc >orj`ac >all'hlaic
a<i AG, BD, ka`i >epeze'uqjwsan a<i AB, BG, GD, DA.}

\gr{Ka`i >epe`i >'ish >est`in <h BE t~h| ED; k'entron g`ar t`o E; koin`h
d`e ka`i pr`oc >orj`ac <h EA, b'asic >'ara <h AB b'asei t~h| AD >'ish
>est'in. di`a t`a a>ut`a d`h ka`i <ekat'era t~wn BG, GD <ekat'era|  t~wn AB, 
AD >'ish >est'in; >is'opleuron >'ara >est`i t`o ABGD tetr'apleuron. l'egw
d'h, <'oti ka`i >orjog'wnion. >epe`i g`ar <h BD e>uje~ia  di'ametr'oc >esti
to~u ABGD k'uklou, <hmik'uklion >'ara >est`i t`o BAD; >orj`h >'ara
<h <up`o BAD gwn'ia. di`a t`a a>ut`a d`h ka`i <ek'asth t~wn <up`o ABG, BGD, GDA >orj'h >estin; >orjog'wnion >'ara >est`i t`o ABGD tetr'apleuron.
>ede'iqjh d`e ka`i >is'opleuron; tetr'agwnon >'ara >est'in. ka`i
>egg'egraptai e>ic t`on ABGD k'uklon.}

\gr{E>ic >'ara t`on doj'enta k'uklon tetr'agwnon >egg'egraptai t`o ABGD; <'oper
>'edei poi~hsai.}}

\ParallelRText{
\begin{center}
{\large Proposition 6}
\end{center}

To inscribe a square in a given circle.

\epsfysize=2.2in
\centerline{\epsffile{Book04/fig06e.eps}}

Let $ABCD$ be the given circle. So it is required to inscribe a square in circle
$ABCD$.

Let two diameters  of circle $ABCD$,  $AC$ and $BD$, have been drawn at right-angles to one another.$^\dag$ And let $AB$, $BC$, $CD$, and $DA$ have been joined.

And since $BE$ is equal to $ED$, for $E$ (is) the center (of the circle), and $EA$
is common and at right-angles, the base $AB$ is thus equal to the base $AD$ [Prop.~1.4]. So, for the same (reasons), each of $BC$ and $CD$ is equal
to each of $AB$ and $AD$. Thus, the quadrilateral $ABCD$ is equilateral. So I say
that (it is) also right-angled. For since the straight-line $BD$ is a diameter
of circle $ABCD$, $BAD$ is thus a semi-circle. Thus, angle $BAD$ (is) a right-angle
[Prop.~3.31]. So, for the same (reasons), (angles)
$ABC$, $BCD$, and $CDA$ are also each right-angles. Thus, the quadrilateral $ABCD$
is right-angled. And it was also shown (to be) equilateral. Thus, it is a square [Def.~1.22]. And it has been inscribed in circle
$ABCD$.

Thus, the square $ABCD$ has been inscribed in the given circle. (Which is)
the very thing it was required to do.}
\end{Parallel}
{\footnotesize \noindent$^\dag$ Presumably, by finding the center of the circle [Prop.~3.1], 
drawing a line through it, and then drawing a second line through it, at right-angles to the first [Prop.~1.11].} 

%%%%%%
% Prop 4.7
%%%%%%
\pdfbookmark[1]{Proposition 4.7}{pdf4.7}
\begin{Parallel}{}{} 
\ParallelLText{
\begin{center}
{\large \ggn{7}.}
\end{center}\vspace*{-7pt}

\gr{Per`i t`on doj'enta k'uklon tetr'agwnon perigr'ayai.}

\gr{>'Estw <o doje`ic k'ukloc <o ABGD; de~i d`h per`i t`on ABGD k'uklon
tetr'agwnon perigr'ayai.}

\epsfysize=2.2in
\centerline{\epsffile{Book04/fig07g.eps}}

\gr{>'Hqjwsan to~u ABGD k'uklou d'uo di'ametroi pr`oc >orj`ac >all'hlaic
a<i AG, BD, ka`i di`a t~wn A, B, G, D shme'iwn >'hqjwsan >efapt'omenai
to~u ABGD k'uklou a<i ZH, HJ, JK, KZ.}

\gr{>Epe`i o>~un >ef'aptetai <h ZH to~u ABGD k'uklou, >ap`o d`e to~u E k'entrou >ep`i t`hn kat`a t`o A >epaf`hn >ep'ezeuktai <h EA, a<i
>'ara pr`oc t~w| A gwn'iai >orja'i e>isin. di`a t`a a>ut`a d`h ka`i a<i
pr`oc to~ic B, G, D shme'ioic gwn'iai >orja'i e>isin. ka`i >epe`i
>orj'h >estin <h <up`o AEB gwn'ia, >est`i d`e >orj`h ka`i <h <up`o
EBH, par'allhloc >'ara >est`in <h HJ t~h| AG. di`a t`a a>ut`a
d`h ka`i <h AG t~h| ZK >esti par'allhloc. <'wste ka`i <h HJ t~h|
ZK >esti par'allhloc. <omo'iwc d`h de'ixomen, <'oti ka`i <ekat'era
t~wn HZ, JK t~h| BED >esti par'allhloc. parallhl'ogramma >'ara >est`i
t`a HK, HG, AK, ZB, BK; >'ish >'ara >est`in <h m`en HZ t~h| JK,
<h d`e HJ t~h| ZK. ka`i >epe`i >'ish >est`in <h AG t~h| BD, >all`a
ka`i <h m`en AG <ekat'era| t~wn HJ, ZK, <h d`e BD <ekat'era|
t~wn HZ, JK >estin >'ish [ka`i <ekat'era >'ara t~wn HJ, ZK <ekat'era|
t~wn HZ, JK >estin >'ish], >is'opleuron >'ara >est`i t`o ZHJK
tetr'apleuron. l'egw d'h, <'oti
ka`i >orjog'wnion. >epe`i g`ar parallhl'ogramm'on >esti t`o HBEA,
ka'i >estin >orj`h <h <up`o AEB, >orj`h >'ara ka`i <h <up`o AHB. <omo'iwc
d`h de'ixomen, <'oti ka`i a<i pr`oc to~ic J, K, Z gwn'iai >orja'i
e>isin. >orjog'wnion >'ara >est`i t`o ZHJK. >ede'iqjh d`e ka`i >is'opleuron;
tetr'agwnon >'ara >est'in. ka`i perig'egraptai per`i t`on ABGD k'uklon.}

\gr{Per`i t`on doj'enta >'ara k'uklon tetr'agwnon perig'egraptai;
<'oper >'edei poi~hsai.}}

\ParallelRText{
\begin{center}
{\large Proposition 7}
\end{center}

To circumscribe a square about a given circle.

Let $ABCD$ be the given circle. So it is required to circumscribe a square
about circle $ABCD$.

\epsfysize=2.2in
\centerline{\epsffile{Book04/fig07e.eps}}

Let two diameters of circle $ABCD$,  $AC$ and $BD$,  have been drawn at right-angles to one another.$^\dag$
And let $FG$, $GH$, $HK$, and $KF$ have been drawn through points $A$, $B$, $C$, and
$D$ (respectively), touching circle $ABCD$.$^\ddag$

Therefore, since $FG$ touches circle $ABCD$, and $EA$ has been joined from the
center $E$ to the point of contact $A$, the angles at $A$ are thus  right-angles  
[Prop.~3.18]. So, for the same (reasons), the angles at points $B$, $C$, and $D$ are also
right-angles. And since angle $AEB$ is a right-angle, and $EBG$ is also a right-angle, $GH$ is thus parallel to $AC$ [Prop.~1.29]. So, for the same
(reasons), $AC$ is also parallel to $FK$. So that $GH$ is also parallel to
$FK$ [Prop.~1.30]. So, similarly, we can show that $GF$ and $HK$ are each parallel to
$BED$. Thus, $GK$, $GC$, $AK$, $FB$, and $BK$ are (all) parallelograms. Thus, 
$GF$ is equal to $HK$, and $GH$ to $FK$ [Prop.~1.34]. And since
$AC$ is equal to $BD$, but $AC$ (is) also (equal) to each of $GH$ and $FK$,
and $BD$ is equal to each of $GF$ and $HK$ [Prop.~1.34] [and each of $GH$ and $FK$ is thus equal to each of $GF$ and $HK$], the quadrilateral $FGHK$ is
thus equilateral.  So I say that (it is) also right-angled. For since $GBEA$
is a parallelogram, and $AEB$ is a right-angle, $AGB$ is thus also a right-angle [Prop.~1.34]. So, similarly, we can show that the angles at $H$, $K$, and
$F$ are also right-angles. Thus, $FGHK$ is right-angled. And it was also shown
(to be) equilateral. Thus,  it is a square [Def.~1.22].
And it has been circumscribed about circle $ABCD$.

Thus, a square has been circumscribed about the given circle. (Which is)
the very thing it was required to do.}
\end{Parallel}
{\footnotesize \noindent$^\dag$ See the footnote to the previous
 proposition.\\
 $^\ddag$ See the footnote
to Prop.~3.34.}\\

%%%%%%
% Prop 4.8
%%%%%%
\pdfbookmark[1]{Proposition 4.8}{pdf4.8}
\begin{Parallel}{}{} 
\ParallelLText{
\begin{center}
{\large \ggn{8}.}
\end{center}\vspace*{-7pt}

\gr{E>ic t`o doj`en tetr'agwnon k'uklon >eggr'ayai.}

\gr{>'Estw t`o doj`en tetr'agwnon t`o ABGD. de~i d`h e>ic t`o ABGD
tetr'agwnon k'uklon >eggr'ayai.}

\epsfysize=2.2in
\centerline{\epsffile{Book04/fig08g.eps}}

\gr{Tetm'hsjw <ekat'era t~wn AD, AB d'iqa kat`a t`a E, Z shme~ia, ka`i
di`a m`en to~u E <opot'era| t~wn AB, GD par'allhloc >'hqjw
<o EJ, di`a d`e to~u Z <opot'era| t~wn AD, BG par'allhloc >'hqjw
<h ZK; parallhl'ogrammon >'ara >est`in <'ekaston t~wn AK, KB,
AJ, JD, AH, HG, BH, HD, ka`i a<i >apenant'ion a>ut~wn pleura`i
dhlon'oti >'isai [e>is'in]. ka`i >epe`i >'ish >est`in <h AD t~h| AB,
ka'i >esti t~hc m`en AD <hm'iseia <h AE, t~hc d`e AB <hm'iseia <h AZ,
>'ish >'ara ka`i <h AE t~h| AZ; <'wste ka`i a<i >apenant'ion; >'ish >'ara
ka`i <h ZH t~h| HE. <omo'iwc d`h de'ixomen, <'oti ka`i <ekat'era
t~wn HJ, HK
<ekat'era| t~wn ZH, HE >estin >'ish; a<i t'essarec >'ara a<i HE, HZ, HJ, HK >'isai >all'hlaic [e>is'in]. <o >'ara k'entrw| m`en t~w| H diast'hmati
d`e <en`i t~wn E, Z, J, K k'ukloc graf'omenoc <'hxei ka`i di`a t~wn loip~wn
shme'iwn; ka`i >ef'ayetai t~wn AB, BG, GD, DA e>ujei~wn di`a
t`o >orj`ac e>~inai t`ac pr`oc to~ic E, Z, J, K gwn'iac; e>i g`ar teme~i
<o k'ukloc t`ac AB, BG, GD, DA, <h t~h| diam'etrw| to~u k'uklou
pr`oc >orj`ac >ap> >'akrac >agom'enh >ent`oc pese~itai to~u k'uklou;
<'oper >'atopon >ede'iqjh. o>uk
>'ara <o k'entrw| t~w| H diast'hmati d`e <en`i t~wn E, Z, J, K
k'ukloc graf'omenoc teme~i t`ac AB, BG, GD, DA e>uje'iac. >ef'ayetai
>'ara a>ut~wn ka`i >'estai >eggegramm'enoc e>ic t`o ABGD tetr'agwnon.}

\gr{E>ic  >'ara t`o doj`en tetr'agwnon k'ukloc >egg'egraptai; <'oper
>'edei poi~hsai.}}

\ParallelRText{
\begin{center}
{\large Proposition 8}
\end{center}

To inscribe a circle in a given square.

Let the given square be $ABCD$. So it is required to inscribe a circle in
square $ABCD$.

\epsfysize=2.2in
\centerline{\epsffile{Book04/fig08e.eps}}

Let  $AD$ and $AB$ each have been cut in half at points $E$ and $F$ (respectively)
[Prop.~1.10]. And let $EH$ have been drawn through $E$, parallel
to either of $AB$ or $CD$, and let $FK$ have been drawn through $F$, parallel to either of $AD$ or $BC$ [Prop.~1.31].
Thus, $AK$, $KB$, $AH$, $HD$, $AG$, $GC$, $BG$, and $GD$ are each parallelograms,
and their opposite sides [are] manifestly equal [Prop.~1.34].
And since $AD$ is equal to $AB$, and $AE$ is half of $AD$, and $AF$ half of
$AB$, $AE$ (is) thus also equal to $AF$. So that the opposite (sides are) also (equal).
Thus, $FG$ (is) also equal to $GE$. So, similarly, we can also show that each of
$GH$ and $GK$ is equal to each of $FG$ and $GE$. Thus, the four (straight-lines)
$GE$, $GF$, $GH$, and $GK$ [are] equal to one another. Thus, the circle drawn
with center $G$, and radius one of $E$, $F$, $H$, or $K$, will also go through the remaining points. And it will touch the straight-lines $AB$, $BC$, $CD$, and $DA$,
on account of the angles at $E$, $F$, $H$, and $K$ being right-angles. For if the
circle cuts $AB$, $BC$, $CD$, or $DA$, then a (straight-line) drawn at right-angles to a diameter of the circle, from its extremity, will fall inside the circle. 
The very thing was shown (to be) absurd [Prop.~3.16].  Thus, the
circle drawn with center $G$, and radius one of $E$, $F$, $H$, or $K$, does not
cut the straight-lines $AB$, $BC$, $CD$, or $DA$. Thus, it will touch them, and will
have been inscribed in the square $ABCD$.

Thus, a circle has been inscribed in the given square. (Which is) the very thing it was required to do.}
\end{Parallel}

%%%%%%
% Prop 4.9
%%%%%%
\pdfbookmark[1]{Proposition 4.9}{pdf4.9}
\begin{Parallel}{}{} 
\ParallelLText{
\begin{center}
{\large \ggn{9}.}
\end{center}\vspace*{-7pt}

\gr{Per`i t`o doj`en tetr'agwnon k'uklon perigr'ayai.}

\gr{>'Estw t`o doj`en tetr'agwnon t`o ABGD; de~i d`h per`i t`o ABGD tetr'agwnon k'uklon perigr'ayai.}

\gr{>Epizeuqje~isai g`ar a<i AG, BD temn'etwsan >all'hlac kat`a t`o E.}

\epsfysize=2.2in
\centerline{\epsffile{Book04/fig06g.eps}}

\gr{Ka`i >epe`i >'ish >est`in <h DA t~h| AB, koin`h d`e <h AG, d'uo
d`h a<i DA, AG dus`i ta~ic BA, AG >'isai e>is'in; ka`i b'asic <h DG b'asei
t~h| BG >'ish; gwn'ia >'ara <h <up`o DAG gwn'ia| t~h| <up`o BAG >'ish
>est'in; <h >'ara <up`o DAB gwn'ia d'iqa t'etmhtai <up`o t~hc AG. <omo'iwc
d`h de'ixomen, <'oti ka`i <ek'asth t~wn <up`o ABG, BGD, GDA d'iqa
t'etmhtai <up`o t~wn AG, DB e>ujei~wn. ka`i >epe`i >'ish
>est`in <h <up`o DAB gwn'ia t~h| <up`o ABG, ka'i >esti t~hc m`en <up`o
DAB <hm'iseia <h <up`o EAB, t~hc  d`e <up`o ABG <hm'iseia <h <up`o
EBA, ka`i <h <up`o EAB >'ara t~h| <up`o EBA >estin >'ish; <'wste
ka`i pleur`a <h EA t~h| EB >estin >'ish. <omo'iwc d`h de'ixomen,
<'oti ka`i <ekat'era t~wn EA, EB [e>ujei~wn] <ekat'era| t~wn EG, ED
>'ish >est'in. a<i t'essarec >'ara a<i EA,  EB, EG, ED >'isai >all'hlaic
e>is'in. <o >'ara k'entrw| t~w| E ka`i diast'hmati <en`i t~wn A, B, G, D
k'ukloc graf'omenoc <'hxei ka`i di`a t~wn loip~wn shme'iwn ka`i
>'estai perigegramm'enoc per`i t`o ABGD tetr'agwnon. perigegr'afjw
<wc <o ABGD.}

\gr{Per`i t`o doj`en >'ara tetr'agwnon k'ukloc perig'egraptai; <'oper >'edei poi~hsai.}}

\ParallelRText{
\begin{center}
{\large Proposition 9}
\end{center}

To  circumscribe a circle about a given square.

Let $ABCD$ be the given square. So it is required to circumscribe a circle about
square $ABCD$.

$AC$ and $BD$ being joined, let them cut one another at $E$.

\epsfysize=2.2in
\centerline{\epsffile{Book04/fig06e.eps}}

And since $DA$ is equal to $AB$, and $AC$ (is) common, the two (straight-lines)
$DA$, $AC$ are thus equal to the two (straight-lines) $BA$, $AC$. And the base $DC$
(is) equal to the base $BC$. Thus, angle $DAC$ is equal to angle $BAC$ [Prop.~1.8]. Thus, the angle $DAB$ has been cut in half by $AC$. So, similarly,
we can show that $ABC$, $BCD$, and $CDA$ have each been cut in half
by the straight-lines $AC$ and $DB$. And since angle $DAB$ is equal
 to $ABC$, and $EAB$ is half of $DAB$, and $EBA$ half of $ABC$, $EAB$ is thus
 also equal to $EBA$. So that side $EA$ is also equal to  $EB$ [Prop.~1.6]. So, similarly, we can show that each of the [straight-lines] $EA$ and $EB$ are also equal to each of $EC$ and $ED$. Thus, the four (straight-lines)
 $EA$, $EB$, $EC$, and $ED$ are equal to one another. Thus, the circle drawn with
 center $E$, and radius one of $A$, $B$, $C$, or $D$, will also go through the remaining points, and will have been circumscribed about the square $ABCD$. Let
 it have been (so) circumscribed, like $ABCD$ (in the figure).
 
 Thus, a circle has been circumscribed about the given square.
 (Which is) the very thing it was required to do.}
\end{Parallel}

%%%%%%
% Prop 4.10
%%%%%%
\pdfbookmark[1]{Proposition 4.10}{pdf4.10}
\begin{Parallel}{}{} 
\ParallelLText{
\begin{center}
{\large \ggn{10}.}
\end{center}\vspace*{-7pt}

\gr{>Isoskel`ec tr'igwnon sust'hsasjai >'eqon <ekat'eran t~wn pr`oc t~h|
b'asei gwni~wn diplas'iona t~hc loip~hc.}

\gr{>Ekke'isjw tic e>uje~ia <h AB, ka`i tetm'hsjw kat`a t`o G shme~ion,
<'wste t`o <up`o t~wn AB, BG perieq'omenon >orjog'wnion >'ison
e>~inai t~w| >ap`o t~hc GA tetrag'wnw|; ka`i k'entrw| t~w| A ka`i
diast'hmati t~w| AB k'ukloc gegr'afjw <o BDE, ka`i >enhrm'osjw
e>ic t`on BDE k'uklon t~h| AG e>uje'ia| m`h me'izoni o>'ush| t~hc
to~u BDE k'uklou diam'etrou >'ish e>uje~ia <h BD; ka`i >epeze'uqjwsan
a<i AD, DG, ka`i perigegr'afjw per`i t`o AGD tr'igwnon k'ukloc <o
AGD.}\\~\\

\epsfysize=2.2in
\centerline{\epsffile{Book04/fig10g.eps}}

\gr{Ka`i >epe`i t`o <up`o t~wn AB, BG >'ison >est`i t~w| >ap`o
t~hc AG, >'ish d`e <h AG t~h| BD, t`o >'ara <up`o t~wn AB, BG >'ison
>est`i t~w| >ap`o t~hc BD. ka`i >epe`i k'uklou to~u AGD e>'ilhpta'i
ti shme~ion >ekt`oc t`o B, ka`i >ap`o to~u B pr`oc t`on AGD k'uklon
prospept'wkasi d'uo e>uje~iai a<i BA, BD, ka`i <h m`en a>ut~wn
t'emnei, <h d`e prosp'iptei, ka'i >esti t`o <up`o t~wn AB, BG >'ison t~w| >ap`o t~hc BD, <h BD >'ara >ef'aptetai to~u AGD k'uklou.
>epe`i o>~un >ef'aptetai m`en <h BD, >ap`o d`e t~hc kat`a t`o D
>epaf~hc di~hktai <h DG, <h >'ara <up`o BDG gwni'a >'ish >est`i
t~h| >en t~w| >enall`ax to~u k'uklou tm'hmati gwn'ia|
t~h| <up`o DAG. >epe`i o>~un >'ish >est`in <h <up`o BDG t~h|
<up`o DAG, koin`h proske'isjw <h <up`o GDA; <'olh >'ara <h
<up`o BDA >'ish >est`i dus`i ta~ic <up`o GDA, DAG. >all`a ta~ic
<up`o GDA, DAG >'ish >est`in <h >ekt`oc <h <up`o BGD;
ka`i <h <up`o BDA >'ara >'ish >est`i t~h| <up`o BGD. >all`a
<h <up`o BDA t~h| <up`o GBD >estin >'ish, >epe`i ka`i pleur`a
<h AD t~h| AB >estin >'ish; <'wste ka`i <h <up`o DBA t~h|
<up`o BGD >estin >'ish.
a<i tre~ic >'ara a<i <up`o BDA,
DBA, BGA >'isai >all'hlaic e>is'in. ka`i >epe`i >'ish >est`in
<h <up`o DBG gwn'ia t~h| <up`o BGD, >'ish >est`i ka`i pleur`a
<h BD pleur~a| t~h| DG. >all`a <h BD t~h| GA <up'okeitai >'ish;
ka`i <h GA >'ara t~h| GD >estin >'ish; <'wste ka`i
gwn'ia <h <up`o GDA gwn'ia| t~h| <up`o DAG >estin >'ish; a<i
>'ara <up`o GDA, DAG t~hc <up`o DAG e>isi diplas'iouc.
>'ish d`e <h <up`o BGD ta~ic <up`o GDA, DAG; ka`i
<h <up`o BGD >'ara t~hc <up`o GAD >esti dipl~h. >'ish
d`e <h <up`o BGD <ekat'era| t~wn <up`o BDA, DBA; ka`i
<ekat'era >'ara t~wn <up`o BDA, DBA t~hc <up`o DAB
>esti dipl~h.}

\gr{>Isoskel`ec >'ara tr'igwnon sun'estatai t`o ABD >'eqon
<ekat'eran t~wn pr`oc t~h| DB b'asei gwni~wn diplas'iona t~hc
loip~hc; <'oper >'edei poi~hsai.}}

\ParallelRText{
\begin{center}
{\large Proposition 10}
\end{center}

To construct an isosceles triangle having each of the angles at the base
double the remaining (angle).

Let some straight-line $AB$ be taken, and let it have been cut at point $C$
so that the rectangle contained by $AB$ and $BC$ is equal to the square
on $CA$ [Prop.~2.11]. And let the circle $BDE$ have been drawn with
center $A$, and radius $AB$. And let the straight-line $BD$, equal to the straight-line $AC$, being not greater than the diameter of circle $BDE$, have been inserted into circle $BDE$ [Prop.~4.1]. And let $AD$ and $DC$ have been joined. 
And let the circle $ACD$ have been circumscribed about triangle $ACD$ [Prop.~4.5].

\epsfysize=2.2in
\centerline{\epsffile{Book04/fig10e.eps}}

And since the (rectangle contained) by $AB$ and $BC$ is equal to the
(square) on $AC$, and $AC$ (is) equal to $BD$, the (rectangle contained) by
$AB$ and $BC$ is thus equal to the (square) on $BD$. And since some point
$B$ has been taken outside of circle $ACD$, and two straight-lines $BA$ and
$BD$ have radiated from $B$ towards the circle $ACD$, and (one) of them
cuts (the circle), and (the other) meets (the circle), and the (rectangle contained)
by $AB$ and $BC$ is equal to the (square) on  $BD$,   $BD$ thus touches
circle $ACD$ [Prop.~3.37]. Therefore, since $BD$ touches (the circle),
and $DC$ has been drawn across (the circle) from the point of contact $D$,
the angle $BDC$ is thus equal to the angle $DAC$ in the alternate segment
of the circle [Prop.~3.32]. Therefore, since $BDC$ is equal to
$DAC$, let $CDA$ have been added to both. Thus, the whole of $BDA$ is equal
to the two (angles) $CDA$ and $DAC$. But, the
external (angle)  $BCD$  is equal to $CDA$ and $DAC$  [Prop.~1.32]. Thus, $BDA$ is also equal to
$BCD$. But, $BDA$ is equal to $CBD$, since the side $AD$ is also equal to
$AB$ [Prop.~1.5]. So that $DBA$ is also equal to $BCD$. Thus, the three
(angles) $BDA$, $DBA$, and $BCD$ are equal to one another. And since angle
$DBC$ is equal to $BCD$, side $BD$ is also equal to side $DC$ [Prop.~1.6].
But, $BD$ was assumed (to be) equal to $CA$. Thus, $CA$ is also equal to $CD$.
So that angle $CDA$ is also equal to angle $DAC$ [Prop.~1.5]. 
Thus, $CDA$ and $DAC$ is double $DAC$. But $BCD$ (is) equal to $CDA$ and
$DAC$. Thus, $BCD$ is also double $CAD$. And $BCD$ (is) equal to to each of
$BDA$ and $DBA$. Thus, $BDA$ and $DBA$ are each double $DAB$.

Thus, the isosceles triangle $ABD$ has been constructed having each of the
angles at the base $BD$ double the remaining (angle). (Which is) the
very thing it was required to do.}
\end{Parallel}

%%%%%%
% Prop 4.11
%%%%%%
\pdfbookmark[1]{Proposition 4.11}{pdf4.11}
\begin{Parallel}{}{} 
\ParallelLText{
\begin{center}
{\large \ggn{11}.}
\end{center}\vspace*{-7pt}

\gr{E>ic t`on doj'enta k'uklon pent'agwnon >is'opleur'on te ka`i >isog'wnion
>eggr'ayai.}

\epsfysize=1.8in
\centerline{\epsffile{Book04/fig11g.eps}}

\gr{>'Estw <o doje`ic k'ukloc <o ABGDE; de~i d`h e>ic t`on ABGDE k'uklon pent'agwnon >is'opleur'on te ka`i >isog'wnion >eggr'ayai.}

\gr{>Ekke'isjw tr'igwnon >isoskel`ec t`o ZHJ diplas'iona >'eqon <ekat'eran
t~wn pr`oc to~ic H, J gwni~wn t~hc pr`oc t~w| Z, ka`i >eggegr'afjw
e>ic t`on ABGDE k'uklon t~w| ZHJ trig'wnw| >isog'wnion tr'igwnon
t`o AGD, <'wste t~h| m`en pr`oc t~w| Z gwn'ia| >'ishn e>~inai
t`hn <up`o GAD, <ekat'eran d`e t~wn pr`oc to~ic H, J >'ishn <ekat'era|
t~wn <up`o AGD, GDA; ka`i <ekat'era >'ara t~wn <up`o AGD, GDA
t~hc <up`o GAD >esti dipl~h. tetm'hsjw d`h <ekat'era t~wn <up`o
AGD, GDA d'iqa <up`o <ekat'erac t~wn GE, DB e>ujei~wn, ka`i
>epeze'uqjwsan a<i AB, BG,  DE, EA.}

\gr{>Epe`i o>~un <ekat'era t~wn <up`o AGD, GDA gwni~wn diplas'iwn
>est`i t~hc <up`o GAD, ka`i tetmhm'enai e>is`i d'iqa <up`o t~wn GE,
DB e>ujei~wn, a<i p'ente >'ara gwn'iai a<i <up`o DAG, AGE, EGD, GDB,
BDA >'isai >all'hlaic e>is'in. a<i d`e >'isai gwn'iai >ep`i >'iswn
periferei~wn beb'hkasin; a<i p'ente  >'ara perif'ereiai a<i AB, BG, GD, DE, EA >'isai >all'hlaic e>is'in. <up`o d`e t`ac >'isac
perifere'iac >'isai e>uje~iai <upote'inousin; a<i p'ente >'ara e>uje~iai
a<i AB, BG, GD, DE, EA >'isai >all'hlaic e>is'in; >is'opleuron >'ara
>est`i t`o ABGDE pent'agwnon. l'egw d'h, <'oti ka`i >isog'wnion. >epe`i
g`ar <h AB perif'ereia t~h| DE perifere'ia| >est`in >'ish, koin`h
proske'isjw <h BGD; <'olh >'ara <h ABGD perif'eria  <'olh|
t~h| EDGB
perifere'ia| >est`in >'ish. ka`i b'ebhken >ep`i m`en t~hc ABGD
perifere'iac gwn'ia <h <up`o AED, >ep`i d`e t~hc EDGB perifere'iac
gwn'ia <h <up`o BAE; ka`i <h <up`o BAE >'ara gwn'ia
t~h| <up`o AED >estin >'ish. di`a t`a a>ut`a d`h ka`i <ek'asth t~wn
<up`o ABG, BGD, GDE gwni~wn <ekat'era| t~wn <up`o BAE, AED
>estin >'ish; >isog'wnion >'ara >est`i t`o ABGDE pent'agwnon.
>ede'iqjh d`e ka`i >is'opleuron.}

\gr{E>ic >'ara t`on doj'enta k'uklon pent'agwnon >is'opleur'on te ka`i >isog'wnion
>egg'egraptai; <'oper >'edei poi~hsai.}}

\ParallelRText{
\begin{center}
{\large Proposition 11}
\end{center}

To inscribe an equilateral and equiangular pentagon in a given circle.

\epsfysize=1.8in
\centerline{\epsffile{Book04/fig11e.eps}}

Let $ABCDE$ be the given circle. So it is required to inscribed an equilateral
and equiangular pentagon in circle $ABCDE$.

Let the the isosceles triangle $FGH$ be set up having each of the angles
at $G$ and $H$ double the (angle) at $F$ [Prop.~4.10]. And let triangle $ACD$, equiangular
to $FGH$, have been inscribed in circle $ABCDE$, such that $CAD$ is equal 
to the angle at $F$, and  the (angles) at $G$ and $H$ (are) equal to $ACD$ and $CDA$, respectively [Prop.~4.2]. Thus,
$ACD$ and $CDA$ are each double $CAD$. So let $ACD$ and $CDA$ have 
been cut in half by the straight-lines $CE$ and $DB$, respectively
[Prop.~1.9]. And let $AB$, $BC$, $DE$ and
$EA$ have been joined.

Therefore, since angles $ACD$ and $CDA$ are each double $CAD$, and are cut
in half by the straight-lines $CE$ and $DB$, the five angles
$DAC$, $ACE$, $ECD$, $CDB$, and $BDA$ are thus equal to one another.
And equal angles stand upon equal circumferences [Prop.~3.26]. Thus, the five circumferences
$AB$, $BC$, $CD$, $DE$, and $EA$ are equal to one another [Prop.~3.29]. Thus, the pentagon
$ABCDE$ is equilateral. So I say that (it is) also equiangular. For since the
circumference $AB$ is equal to the circumference $DE$, let $BCD$ have been added to both. Thus, the whole circumference $ABCD$ is equal to the
whole circumference $EDCB$. And the angle $AED$ stands upon circumference
$ABCD$, and angle $BAE$ upon circumference $EDCB$. Thus, angle $BAE$ is
also equal to $AED$ [Prop.~3.27]. 
So, for the same (reasons), each of the angles $ABC$, $BCD$,  and $CDE$ is also equal
to each of $BAE$ and $AED$. Thus, pentagon $ABCDE$ is equiangular.
And it was also shown (to be) equilateral.

Thus, an equilateral and equiangular pentagon has been inscribed in the
given circle. (Which is) the very thing it was required to do.}
\end{Parallel}

%%%%%%
% Prop 4.12
%%%%%%
\pdfbookmark[1]{Proposition 4.12}{pdf4.12}
\begin{Parallel}{}{} 
\ParallelLText{
\begin{center}
{\large \ggn{12}.}
\end{center}\vspace*{-7pt}

\gr{Per`i t`on doj'enta k'uklon pent'agwnon >is'opleur'on te ka`i >isog'wnion
perigr'ayai.}

\epsfysize=2in
\centerline{\epsffile{Book04/fig12g.eps}}

\gr{>'Estw <o doje`ic k'ukloc <o ABGDE; de~i d`e per`i t`on ABGDE k'uklon
pent'agwnon >is'opleur'on te ka`i >isog'wnion perigr'ayai.}

\gr{Neno'hsjw to~u >eggegramm'enou pentag'wnou t~wn gwni~wn shme~ia
t`a A, B, G, D, E, <'wste >'isac e>~inai t`ac AB, BG, GD, DE, EA
perifere'iac; ka`i di`a t~wn A, B, G, D, E >'hqjwsan to~u k'uklou
>efapt'omenai a<i HJ, JK, KL, LM, MH, ka`i e>il'hfjw to~u ABGDE
k'uklou k'entron t`o Z, ka`i >epeze'uqjwsan a<i ZB, ZK, ZG, ZL, ZD.}

\gr{Ka`i >epe`i <h m`en KL e>uje~ia >ef'aptetai to~u ABGDE kat`a t`o
G, >ap`o d`e to~u Z k'entrou >ep`i t`hn kat`a t`o G >epaf`hn >ep'ezeuktai
<h ZG, <h ZG >'ara k'ajet'oc >estin >ep`i t`hn KL; >orj`h >'ara >est`in
<ekat'era t~wn pr`oc t~w| G gwni~wn. di`a t`a a>ut`a d`h ka`i a<i 
pr`oc to~ic B, D shme'ioic gwn'iai  >orja'i e>isin. ka`i >epe`i >orj'h 
>estin <h <up`o ZGK gwn'ia, t`o >'ara >ap`o t~hc ZK >'ison 
>est`i to~ic >ap`o t~wn ZG, GK. di`a t`a a>ut`a d`h ka`i to~ic 
>ap`o t~wn ZB, BK >'ison >est`i t`o >ap`o t~hc ZK; <'wste t`a >ap`o
t~wn ZG, GK to~ic >ap`o t~wn ZB, BK >estin >'isa, <~wn t`o
>ap`o t~hc ZG t~w| >ap`o t~hc ZB >estin >'ison; loip`on >'ara
t`o >ap`o t~hc GK t~w| >ap`o t~hc BK >estin >'ison. >'ish
>'ara <h BK t~h| GK. ka`i >epe`i >'ish >est`in <h ZB t~h| ZG, ka`i koin`h 
<h ZK, d'uo d`h a<i BZ, ZK dus`i ta~ic GZ, ZK >'isai e>is'in;
ka`i b'asic <h BK b'asei t~h| GK [>estin] >'ish; gwn'ia >'ara <h m`en 
<up`o BZK [gwn'ia|] t~h| <up`o KZG >estin >'ish; <h d`e <up`o BKZ
t~h| <up`o ZKG; dipl~h >'ara <h m`en <up`o BZG t~hc <up`o KZG,
<h d`e <up`o BKG t~hc <up`o ZKG. di`a t`a a>ut`a d`h ka`i <h
m`en <up`o GZD t~hc <up`o GZL >esti dipl~h, <h d`e <up`o DLG
t~hc <up`o ZLG. ka`i >epe`i >'ish >est`in <h BG perif'ereia t~h| GD,
>'ish >est`i ka`i gwn'ia <h <up`o BZG t~h| <up`o GZD.
ka'i >estin <h m`en <up`o BZG t~hc <up`o KZG dipl~h,
<h d`e <up`o DZG t~hc <up`o LZG; >'ish >'ara ka`i <h <up`o
KZG t~h| <up`o LZG; >est`i d`e ka`i <h <up`o ZGK gwn'ia
t~h| <up`o ZGL >'ish. d'uo d`h tr'igwn'a >esti t`a ZKG,
ZLG t`ac  d'uo gwn'iac ta~ic dus`i gwn'iaic >'isac >'eqonta ka`i
m'ian pleur`an mi~a| pleur~a| >'ishn koin`hn a>ut~wn
t`hn ZG; ka`i t`ac loip`ac >'ara pleur`ac ta~ic loipa~ic pleura~ic >'isac
<'exei ka`i t`hn loip`hn gwn'ian t~h| loip~h| gwn'ia|; >'ish >'ara <h m`en KG e>uje~ia t~h| GL, <h d`e <up`o
ZKG gwn'ia t~h| <up`o ZLG. ka`i >epe`i >'ish >est`in
<h KG t~h| GL, dipl~h >'ara <h KL t~hc KG. di`a t`a a>uta
d`h deiqj'hsetai ka`i <h 
JK t~hc BK dipl~h. ka'i >estin <h BK t~h| KG
>'ish; ka`i <h JK >'ara t~h| KL >estin >'ish. <omo'iwc d`h deiqj'hsetai
ka`i <ek'asth t~wn JH, HM,
ML <ekat'era| t~wn JK, KL >'ish; >is'opleuron >'ara >est`i t`o HJKLM
pent'agwnon. l'egw d'h, <'oti ka`i >isog'wnion. >epe`i g`ar >'ish
>est`in <h <up`o ZKG gwn'ia t~h| <up`o ZLG, ka`i >ede'iqjh t~hc
m`en <up`o ZKG dipl~h <h <up`o
JKL, t~hc d`e <up`o ZLG dipl~h <h <up`o KLM, ka`i <h <up`o
JKL >'ara t~h| <up`o KLM >estin >'ish. <omo'iwc d`h deiqj'hsetai ka`i
<ek'asth t~wn <up`o KJH, JHM, HML <ekat'era| t~wn <up`o JKL, KLM
>'ish; a<i p'ente >'ara gwn'iai a<i <up`o HJK, JKL, KLM, LMH,
MHJ
>'isai >all'hlaic e>is'in. >isog'wnion >'ara >est`i t`o HJKLM
pent'agwnon. >ede'iqjh d`e ka`i >is'opleuron, ka`i
perig'egraptai per`i t`on ABGDE k'uklon.}

\gr{\mbox{[}Per`i t`on doj'enta >'ara k'uklon pent'agwnon >is'opleur\-'on
te ka`i >isog'wnion perig'egraptai]; <'oper >'edei
poi~hsai.}}

\ParallelRText{
\begin{center}
{\large Proposition 12}
\end{center}

To circumscribe an equilateral and equiangular
pentagon about a given circle.

\epsfysize=2in
\centerline{\epsffile{Book04/fig12e.eps}}

Let $ABCDE$ be the given circle. So it is required to circumscribe an
equilateral and equiangular pentagon about circle $ABCDE$. 

Let $A$, $B$, $C$, $D$,  and $E$ have been conceived as the angular points of a
pentagon having been inscribed (in circle $ABCDE$) [Prop.~3.11], such that the circumferences $AB$, $BC$, $CD$, $DE$, and
$EA$ are equal. And let $GH$, $HK$, $KL$, $LM$, and $MG$  have been drawn through
(points) $A$, $B$, $C$, $D$, and $E$ (respectively), touching the circle.$^\dag$ And let the center $F$
of the circle $ABCDE$ have been found [Prop.~3.1].
And let $FB$, $FK$, $FC$, $FL$, and $FD$ have been joined.

And since the straight-line $KL$ touches (circle) $ABCDE$ at $C$, and $FC$ has been
joined from the center $F$ to the point of contact $C$, $FC$ is thus perpendicular 
to $KL$ [Prop.~3.18]. Thus, each of the angles
at $C$ is a right-angle. So, for the same (reasons), the angles at $B$ and $D$
are also right-angles. And since angle $FCK$ is a right-angle, the (square)
on $FK$ is thus equal to the (sum of the squares) on $FC$ and $CK$ [Prop.~1.47]. So, for the same (reasons), 
the (square) on $FK$ is also equal to the (sum of the squares) on $FB$ and $BK$.
So that the (sum of the squares) on $FC$ and $CK$ is equal to the (sum of the squares) on $FB$ and $BK$, of which the (square) on $FC$ is equal to the
(square) on $FB$. Thus, the remaining (square) on $CK$ is equal to
the remaining (square) on $BK$. Thus, $BK$ (is) equal to $CK$. And since
$FB$ is equal to $FC$, and $FK$ (is) common, the two (straight-lines)
$BF$, $FK$ are equal to the two (straight-lines) $CF$, $FK$. And the base $BK$
[is] equal to the base $CK$. Thus, angle $BFK$ is equal to [angle] $KFC$ 
[Prop.~1.8]. And $BKF$ (is equal) to $FKC$ [Prop.~1.8].
Thus, $BFC$ (is) double $KFC$,
and $BKC$ (is double) $FKC$. So, for the same (reasons), $CFD$ is also double
$CFL$, and $DLC$ (is also double) $FLC$. And since circumference $BC$
is equal  to $CD$, angle $BFC$ is also  equal  to $CFD$ [Prop.~3.27]. And $BFC$ is double $KFC$,
and $DFC$ (is double) $LFC$.  Thus, $KFC$ is also equal to $LFC$. 
And
angle $FCK$ is also equal to $FCL$. So, $FKC$ and $FLC$ are two triangles
having two angles equal to two angles, and one side equal to one side,
(namely) their common (side) $FC$. Thus, they will also have the
remaining sides equal to the (corresponding) remaining sides, and
the remaining angle to the remaining angle [Prop.~1.26]. Thus, the straight-line $KC$
(is) equal to $CL$, and the angle $FKC$ to $FLC$. And since $KC$ is
equal to $CL$, $KL$ (is) thus double $KC$. So, for the
same (reasons), it can be shown that $HK$ (is) also double $BK$. And
$BK$ is equal to $KC$. Thus, $HK$ is also equal to $KL$. So, similarly,  each of $HG$, $GM$, and $ML$ can also be shown (to be) equal to each of
$HK$ and $KL$. Thus, pentagon $GHKLM$ is equilateral. So I say that
(it is) also equiangular. For since angle $FKC$ is equal to $FLC$, 
and $HKL$ was shown (to be) double $FKC$, and $KLM$ double 
$FLC$, $HKL$ is thus also equal to $KLM$. So, similarly, each of $KHG$, $HGM$, and $GML$ can also be shown (to be) equal to each of
$HKL$ and $KLM$. Thus, the five angles $GHK$, $HKL$, $KLM$, $LMG$, and
$MGH$ are equal to one another. Thus, the pentagon $GHKLM$ is
equiangular. And it was also shown (to be) equilateral, and has been circumscribed 
about circle $ABCDE$.

\mbox{[}Thus, an equilateral and equiangular pentagon has been circumscribed
about the given circle]. (Which is) the very thing it was required to do. }
\end{Parallel}
{\footnotesize \noindent$^\dag$ See the footnote to Prop.~3.34.} 

%%%%%%
% Prop 4.13
%%%%%%
\pdfbookmark[1]{Proposition 4.13}{pdf4.13}
\begin{Parallel}{}{} 
\ParallelLText{
\begin{center}
{\large \ggn{13}.}
\end{center}\vspace*{-7pt}

\gr{E>ic t`o doj`en pent'agwnon, <'o >estin >is'opleur'on te ka`i >isog'wnion, k'uklon >eggr'ayai.}

\epsfysize=2.2in
\centerline{\epsffile{Book04/fig13g.eps}}

\gr{>'Estw t`o doj`en pent'agwnon >is'opleur'on te ka`i >isog'wni\-on
t`o ABGDE; de~i d`h e>ic t`o ABGDE pent'agwnon k'uklon >eggr'ayai.}

\gr{Tetm'hsjw g`ar <ekat'era t~wn <up`o BGD, GDE gwni~wn d'iqa <up`o
<ekat'erac t~wn GZ, DZ e>ujei~wn; ka`i >ap`o to~u Z shme'iou, kaj> <`o
sumb'allousin >all'hlaic a<i GZ, DZ e>uje~iai, >epeze'uqjwsan a<i ZB,
ZA, ZE e>uje~iai. ka`i >epe`i >'ish >est`in <h BG t~h| GD, koin`h
d`e <h GZ, d'uo d`h a<i BG, GZ dus`i ta~ic DG, GZ >'isai
e>is'in; ka`i gwn'ia <h <up`o BGZ gwn'ia| t~h| <up`o DGZ [>estin]
>'ish; b'asic >'ara <h BZ b'asei t~h| DZ >estin >'ish, 
ka`i t`o BGZ tr'igwnon t~w| DGZ trig'wnw| >estin >'ison,
ka`i a<i loipa`i
gwn'iai ta~ic loipa~ic gwn'iaic >'isai >'esontai, <uf> <`ac a<i
>'isai pleura`i <upote'inousin; >'ish >'ara <h <up`o GBZ gwn'ia 
t~h|  <up`o GDZ. ka`i >epe`i dipl~h >estin <h <up`o GDE t~hc 
<up`o GDZ, >'ish d`e <h m`en <up`o GDE t~h| <up`o ABG, <h d`e 
<up`o GDZ t~h| <up`o GBZ, ka`i <h <up`o GBA >'ara t~hc <up`o
GBZ >esti dipl~h; >'ish >'ara <h <up`o ABZ gwn'ia t~h| <up`o
ZBG; <h >'ara <up`o ABG gwn'ia d'iqa t'etmhtai <up`o t~hc
BZ e>uje'iac. <omo'iwc d`h deiqj'hsetai, <'oti ka`i <ekat'era t~wn
<up`o BAE, AED d'iqa t'etmhtai <up`o <ekat'erac t~wn ZA, ZE
e>ujei~wn. >'hqjwsan d`h >ap`o to~u Z shme'iou >ep`i t`ac AB, BG,
GD, DE, EA e>uje'iac k'ajetoi a<i ZH, ZJ, ZK, ZL, ZM. ka`i >epe`i
>'ish >est`in <h <up`o JGZ gwn'ia t~h| <up`o KGZ, >est`i d`e 
ka`i >orj`h <h <up`o ZJG
[>orj~h|] t~h| <up`o ZKG >'ish,
d'uo d`h tr'igwn'a >esti t`a ZJG, ZKG t`ac d'uo gwn'iac dus`i gwn'iaic >'isac
>'eqonta ka`i m'ian pleur`an mi~a| pleur~a| >'ishn koin`hn a>ut~wn 
t`hn ZG <upote'inousan <up`o m'ian t~wn >'iswn gwni~wn; ka`i
t`ac loip`ac >'ara pleur`ac ta~ic loipa~ic pleura~ic >'isac <'exei;
>'ish >'ara <h ZJ k'ajetoc t`h| ZK kaj'etw|. <omo'iwc
d`h deiqj'hsetai, <'oti ka`i <ek'asth t~wn ZL, ZM, ZH
<ekat'era| t~wn ZJ, ZK >'ish >est'in; a<i p'ente >'ara e>uje~iai a<i
ZH, ZJ, ZK, ZL, ZM  >'isai >all'hlaic e>is'in. <o >'ara
k'entrw| t~w| Z diast'hmati d`e <en`i t~wn H, J, K, L, M
k'ukloc graf'omenoc <'hxei ka`i di`a t~wn loip~wn shme'iwn
ka`i >ef'ayetai t~wn 
AB, BG, GD, DE, EA e>ujei~wn di`a t`o >orj`ac e>~inai
t`ac pr`oc to~ic H, J, K, L, M shme'ioic gwn'iac. e>i g`ar o>uk
>ef'ayetai a>ut~wn, >all`a teme~i a>ut'ac, sumb'hsetai t`hn t~h|
diam'etrw| to~u k'uklou pr`oc >orj`ac >ap> >'akrac >agom'enhn
>ent`oc p'iptein to~u k'uklou; <'oper >'atopon >ede'iqjh.
o>uk >'ara <o k'entrw| t~w| Z diast'hmati d`e <en`i t~wn
H, J, K, L, M shme'iwn graf'omenoc k'ukloc teme~i t`ac AB, BG, GD, DE,
EA e>uje'iac;  >ef'ayetai >'ara a>ut~wn. gegr'afjw <wc <o HJKLM.}

\gr{E>ic >'ara t`o doj`en pent'agwnon, <'o >estin >is'opleur'on
te ka`i >isog'wnion, k'ukloc >egg'egraptai; <'oper >'edei poi~hsai.}}

\ParallelRText{
\begin{center}
{\large Proposition 13}
\end{center}

To inscribe a circle in a given
pentagon, which is equilateral and equiangular.

\epsfysize=2.2in
\centerline{\epsffile{Book04/fig13e.eps}}

Let $ABCDE$ be the given equilateral and equiangular pentagon. So
it is required to inscribe a circle in pentagon $ABCDE$.

For let angles $BCD$ and $CDE$ have each been cut in half by each
of the straight-lines $CF$ and $DF$ (respectively) [Prop.~1.9]. And from the point $F$, at which the
straight-lines $CF$ and $DF$ meet one another, let the straight-lines
$FB$, $FA$, and $FE$ have been joined. And since $BC$ is equal to $CD$,
and $CF$ (is) common, the two (straight-lines) $BC$, $CF$ 
are equal to the two (straight-lines) $DC$, $CF$. And angle $BCF$ [is]
equal to angle $DCF$. Thus, the base $BF$ is equal to the base $DF$,
and triangle $BCF$ is equal to triangle $DCF$, and the remaining angles
will be equal to the (corresponding) remaining angles which the
equal sides subtend
[Prop.~1.4]. Thus, angle $CBF$ (is) equal
to $CDF$. And since $CDE$ is double $CDF$, and $CDE$ (is) equal to $ABC$,
and $CDF$ to $CBF$, $CBA$ is thus also double $CBF$. Thus, angle $ABF$
is equal to $FBC$. Thus, angle $ABC$ has been cut in half by the straight-line
$BF$. So, similarly, it can be shown that $BAE$ and $AED$ have 
been cut in half by the straight-lines $FA$ and $FE$, respectively.
So let $FG$, $FH$, $FK$, $FL$, and $FM$ have been drawn from point $F$,
perpendicular to the straight-lines $AB$, $BC$, $CD$, $DE$, and $EA$ 
(respectively)
[Prop.~1.12]. And since angle
$HCF$ is equal to $KCF$, and the right-angle $FHC$ is also equal to
the [right-angle] $FKC$, $FHC$ and $FKC$ are two triangles having two
angles equal to two angles, and one side equal to one side, (namely)
their common (side) $FC$, subtending one of the equal angles. 
Thus, they will also have the remaining sides equal to the (corresponding)
remaining sides [Prop.~1.26].
Thus, the perpendicular $FH$ (is) equal to the perpendicular $FK$. So,
similarly, it can be shown that $FL$, $FM$, and $FG$ are each equal to
each  of $FH$ and $FK$. Thus, the five straight-lines $FG$, $FH$, $FK$, $FL$,
and $FM$ are equal to one another.
Thus, the circle drawn with center $F$, and
radius one of $G$, $H$, $K$, $L$, or $M$, will also go through the remaining
points, and will touch the straight-lines $AB$, $BC$, $CD$, $DE$, and $EA$,
on account of the angles at points $G$, $H$, $K$, $L$, and $M$ being
right-angles. For if it does not touch them, but cuts them, 
 it follows that a (straight-line) drawn at right-angles to the diameter
 of the circle, from its extremity, falls inside the circle.
 The very thing was shown (to be) absurd [Prop.~3.16]. Thus, the circle drawn
 with center $F$, and radius one of $G$, $H$, $K$, $L$,  or $M$, does not
 cut the straight-lines $AB$, $BC$, $CD$, $DE$, or $EA$. Thus, it will
 touch them. Let it have been drawn, like $GHKLM$ (in the figure).
 
 Thus, a circle has been inscribed in the given pentagon which is
 equilateral and equiangular. (Which is) the very thing it was required to
 do.}
\end{Parallel}

%%%%%%
% Prop 4.14
%%%%%%
\pdfbookmark[1]{Proposition 4.14}{pdf4.14}
\begin{Parallel}{}{} 
\ParallelLText{
\begin{center}
{\large \ggn{14}.}
\end{center}\vspace*{-7pt}

\gr{Per`i t`o doj`en pent'agwnon, <'o >estin >is'opleur'on te ka`i 
>isog'wnion, k'uklon perigr'ayai.}

\gr{>'Estw t`o doj`en pent'agwnon, <'o >estin >is'opleur'on te ka`i
>isog'wnion, t`o ABGDE; de~i d`h per`i t`o ABGDE pent'agwnon
k'uklon perigr'ayai.}

\epsfysize=2.2in
\centerline{\epsffile{Book04/fig14g.eps}}

\gr{Tetm'hsjw d`h <ekat'era t~wn <up`o BGD, GDE gwni~wn d'iqa
<up`o <ekat'erac t~wn GZ, DZ, ka`i >ap`o to~u Z shme'iou,
kaj> <`o sumb'allousin a<i e>uje~iai, >ep`i t`a B, A, E shme~ia
>epeze'uqjwsan e>uje~iai a<i ZB, ZA, ZE. <omo'iwc d`h t~w| pr`o to'utou
deiqj'hsetai, <'oti ka`i <ek'asth t~wn <up`o GBA, BAE, AED
gwni~wn d'iqa t'etmhtai <up`o <ek'asthc t~wn ZB, ZA, ZE
e>ujei~wn. ka`i >epe`i >'ish >est`in <h <up`o BGD gwn'ia t~h|
<up`o GDE, ka'i >esti t~hc m`en <up`o BGD  <hm'iseia <h <up`o ZGD,
t~hc d`e <up`o GDE <hm'iseia <h <up`o GDZ, ka`i <h <up`o
ZGD >'ara t~h| <up`o ZDG >estin >'ish; <'wste ka`i pleur`a <h ZG
pleur~a| t~h| ZD >estin >'ish. <omo'iwc d`h deiqj'hsetai, <'oti ka`i
<ek'asth t~wn ZB, ZA, ZE <ekat'era| t~wn ZG, ZD >estin >'ish; a<i
p'ente >'ara e>uje~iai a<i ZA, ZB, ZG, ZD, ZE  >'isai >all'hlaic
e>is'in. >o >'ara k'entrw| t~w| Z ka`i diast'hmati <en`i t~wn ZA, ZB, 
ZG, ZD, ZE k'ukloc graf'omenoc <'hxei ka`i di`a t~wn loip~wn
shme'iwn ka`i >'estai perigegramm'enoc. perigegr'afjw
ka`i >'estw <o ABGDE.}

\gr{Per`i >'ara t`o doj`en pent'agwnon, <'o >estin >is'opleur'on te ka`i
>isog'wnion, k'ukloc perig'egraptai; <'oper >'edei poi~hsai.}}

\ParallelRText{
\begin{center}
{\large Proposition 14}
\end{center}

To circumscribe a circle about a given pentagon which is equilateral and equiangular.

Let $ABCDE$ be the given pentagon  which is equilateral and equiangular.
So it is required to circumscribe a circle about the pentagon 
$ABCDE$.

\epsfysize=2.2in
\centerline{\epsffile{Book04/fig14e.eps}}

So let angles $BCD$ and $CDE$ have  been cut
in half by  the (straight-lines) $CF$ and $DF$, respectively [Prop.~1.9]. And let the straight-lines
$FB$, $FA$, and $FE$ have been joined from point $F$, at which the
straight-lines meet, to the points $B$, $A$, and $E$ (respectively).
So, similarly, to the (proposition) before this (one), it can be shown that
angles $CBA$, $BAE$, and $AED$ have also  been cut in half by
the straight-lines $FB$, $FA$, and $FE$, respectively. And since
angle $BCD$ is equal to $CDE$, and $FCD$ is half of $BCD$, and $CDF$ half
of $CDE$, $FCD$ is thus also equal to $FDC$. So that side $FC$ is also equal to
side $FD$ [Prop.~1.6]. So, similarly, it can be
shown that $FB$, $FA$, and $FE$ are also each equal to each of $FC$ and $FD$. 
Thus, the five straight-lines $FA$, $FB$, $FC$, $FD$, and $FE$ are equal to
one another. Thus, the circle drawn with center $F$, and radius one of
$FA$, $FB$, $FC$, $FD$, or $FE$, will also go through the remaining points, and
will have been circumscribed. Let it have been (so) circumscribed, and let
it be $ABCDE$.

Thus, a circle has been circumscribed about the given pentagon, which
is equilateral and equiangular. (Which is) the very thing it was required to
do.}
\end{Parallel}

%%%%%%
% Prop 4.15
%%%%%%
\pdfbookmark[1]{Proposition 4.15}{pdf4.15}
\begin{Parallel}{}{} 
\ParallelLText{
\begin{center}
{\large \ggn{15}.}
\end{center}\vspace*{-7pt}

\gr{E>ic t`on doj'enta k'uklon <ex'agwnon >is'opleur'on te ka`i >isog'wnion
>eggr'ayai.}

\gr{>'Estw <o doje`ic k'ukloc <o ABGDEZ; de~i d`h e>ic t`on ABGDEZ
k'uklon <ex'agwnon >is'opleur'on te ka`i >isog'wnion >eggr'ayai.}

\gr{>'Hqjw to~u ABGDEZ k'uklou di'ametroc <h AD, ka`i e>il'hfjw t`o
k'entron to~u k'uklou t`o H, ka`i k'entrw| m`en t~w| D diast'hmati d`e
t~w| DH k'ukloc gegr'afjw <o EHGJ, ka`i >epizeuqje~isai a<i EH, GH
di'hqjwsan >ep`i t`a B, Z shme~ia, ka`i >epeze'uqjwsan a<i AB, BG, GD,
DE, EZ, ZA; l'egw, <'oti t`o ABGDEZ <ex'agwnon >is'opleur'on t'e
>esti ka`i >isog'wnion.}\\~\\

\epsfysize=2.4in
\centerline{\epsffile{Book04/fig15g.eps}}

\gr{>Epe`i g`ar t`o H shme~ion k'entron >est`i to~u ABGDEZ k'uklou,
>'ish >est`in <h HE t~h| HD. p'alin, >epe`i t`o D shme~ion k'entron
>est`i to~u HGJ k'uklou, >'ish >est`in <h DE t~h| DH. >all>
<h HE t~h| HD >ede'iqjh >'ish; ka`i <h HE >'ara t~h| ED >'ish >est'in;
>is'opleuron >'ara >est`i t`o EHD tr'igwnon; ka`i a<i tre~ic >'ara a>uto~u
gwn'iai a<i <up`o EHD, HDE, DEH >'isai >all'hlaic e>is'in,
>epeid'hper t~wn >isoskel~wn trig'wnwn a<i pr`oc t~h| b'asei gwn'iai
>'isai >all'hlaic e>is'in; ka'i e>isin a<i tre~ic to~u trig'wnou gwn'iai
dus`in >orja~ic >'isai; <h >'ara <up`o EHD gwn'ia tr'iton >est`i
d'uo >orj~wn. <omo'iwc d`h deiqj'hsetai ka`i <h <up`o DHG tr'iton
d'uo >orj~wn.
ka`i >epe`i <h GH e>uje~ia >ep`i t`hn EB staje~isa
t`ac >efex~hc gwn'iac t`ac <up`o EHG, GHB dus`in >orja~ic
>'isac poie~i, ka`i loip`h >'ara <h <up`o GHB tr'iton >est`i d'uo
>orj~wn; a<i >'ara <up`o EHD, DHG, GHB gwn'iai >'isai
>all'hlaic e>is'in; <'wste ka`i a<i kat`a koruf`hn a>uta~ic a<i <up`o
BHA, AHZ, ZHE >'isai e>is`in [ta~ic <up`o EHD, DHG, GHB].
a<i <`ex >'ara gwn'iai a<i <up`o EHD, DHG, GHB, BHA, AHZ, ZHE
>'isai >all'hlaic e>is'in. a<i d`e >'isai gwn'iai >ep`i >'iswn periferei~wn
beb'hkasin; a<i <`ex >'ara perif'ereiai a<i AB, BG, GD, DE, EZ, ZA
>'isai >all'hlaic e>is'in. <up`o d`e t`ac >'isac perifere'iac a<i >'isai
e>uje~iai <upote'inousin; a<i <`ex >'ara e>uje~iai >'isai 
>all'hlaic e>is'in; >is'opleuron >'ara >est`i to ABGDEZ <ex'agwnon.
l'egw d'h, <'oti ka`i >isog'wnion. >epe`i g`ar >'ish >est`in <h ZA
perif'ereia t~h| ED perifere'ia|, koin`h proske'isjw <h ABGD perif'ereia; <'olh >'ara <h ZABGD <'olh| t~h| EDGBA  >estin >'ish; ka`i b'ebhken
>ep`i m`en t~hc ZABGD perifere'iac <h <up`o 
ZED gwn'ia, >ep`i
d`e t~hc EDGBA perifere'iac <h <up`o AZE gwn'ia;  >'ish 
>'ara <h <up`o 
AZE gwn'ia t~h| <up`o DEZ.  <omo'iwc d`h deiqj'hsetai, <'oti 
ka`i a<i loipa`i gwn'iai to~u ABGDEZ <exag'wnou kat`a m'ian 
>'isai e>is`in <ekat'era| t~wn <up`o AZE, ZED gwni~wn; >isog'wnion
>'ara >est`i t`o ABGDEZ <ex'agwnon. >ede'iqjh d`e ka`i >is'opleuron;
ka`i >egg'egraptai e>ic t`on ABGDEZ k'uklon.}

\gr{E>ic >'ara t`on doj'enta k'uklon <ex'agwnon >is'opleur'on te ka`i
>isog'wnion >egg'egraptai; <'oper >'edei poi~hsai.}\\~\\~\\~\\~\\~\\~\\~\\~\\

\begin{center}
{\large \gr{P'orisma}.}
\end{center}\vspace*{-7pt}

\gr{>Ek d`h to'utou faner'on, <'oti <h to~u <exag'wnou pleur`a >'ish >est`i
t~h| >ek to~u k'entrou to~u k'uklou.}

\gr{<Omo'iwc d`e to~ic >ep`i to~u pentag'wnou >e`an di`a t~wn kat`a
t`on k'uklon diair'esewn >efaptom'enac to~u k'uklou >ag'agwmen,
perigraf'hsetai per`i t`on k'uklon <ex'agwnon >is'opleur'on te
ka`i >isog'wnion >akolo'ujwc to~ic >ep`i to~u pentag'wnou
e>irhm'enoic. ka`i >'eti di`a t~wn <omo'iwn to~ic
>ep`i to~u pentag'wnou e>irhm'enoic e>ic t`o doj`en <ex'agwnon k'uklon
>eggr'ayom'en te ka`i perigr'ayomen; <'oper >'edei poi~hsai.}}

\ParallelRText{
\begin{center}
{\large Proposition 15}
\end{center}

To inscribe an equilateral and equiangular hexagon
in a given circle.

Let $ABCDEF$ be the given circle. So it is required to inscribe an equilateral
and equiangular hexagon in circle $ABCDEF$.

Let the diameter $AD$ of circle $ABCDEF$ have been drawn,$^\dag$ and let the center $G$ of
the circle have been found [Prop.~3.1].
And let the circle $EGCH$ have been drawn, with center $D$, and radius
$DG$.  And $EG$ and $CG$ being joined, let them have been drawn across (the
circle) to points $B$ and $F$ (respectively). 
And let $AB$, $BC$, $CD$, $DE$, $EF$, and $FA$ have been joined. 
I say that the hexagon $ABCDEF$ is equilateral and equiangular.

\epsfysize=2.4in
\centerline{\epsffile{Book04/fig15e.eps}}

For since point $G$ is the center of circle $ABCDEF$, $GE$ is equal to $GD$.
Again, since point $D$ is the center of circle $GCH$, $DE$ is equal to
$DG$. But, $GE$ was shown (to be) equal to $GD$. Thus, $GE$ is also equal to
$ED$. Thus, triangle $EGD$ is equilateral. Thus, its three angles
$EGD$, $GDE$, and $DEG$ are also equal to one another, inasmuch as the angles at the base of 
 isosceles triangles are equal to one another [Prop.~1.5]. And the three angles of the
 triangle are equal to
 two right-angles [Prop.~1.32].
 Thus, angle $EGD$ is one third of two right-angles. So, similarly,
  $DGC$ can also be shown (to be) one third of two right-angles.
   And since the straight-line $CG$, standing on $EB$, makes 
 adjacent angles $EGC$ and $CGB$ equal to two right-angles [Prop.~1.13], the remaining angle
 $CGB$ is thus also one third of two right-angles.
 Thus, angles $EGD$, $DGC$, and $CGB$ are equal to one another.
 And hence the (angles) opposite to them $BGA$, $AGF$, and $FGE$
 are also equal [to $EGD$, $DGC$, and $CGB$ (respectively)] [Prop.~1.15]. Thus,
 the six angles $EGD$, $DGC$, $CGB$, $BGA$, $AGF$, and $FGE$ are equal to one
 another. And equal angles stand on equal circumferences [Prop.~3.26]. Thus, the six circumferences
 $AB$, $BC$, $CD$, $DE$, $EF$, and $FA$ are equal to one another. And
  equal circumferences
  are subtended by equal straight-lines [Prop.~3.29]. Thus, the six
 straight-lines ($AB$, $BC$, $CD$, $DE$, $EF$, and $FA$) are equal to one another.
 Thus, hexagon $ABCDEF$ is equilateral. 
 So, I say that (it is) also equiangular. For since circumference $FA$
 is equal to circumference $ED$, let circumference $ABCD$ have been added
 to both. Thus, the whole of $FABCD$ is equal to the whole of
 $EDCBA$. And angle $FED$ stands on circumference $FABCD$,
 and angle $AFE$ on circumference $EDCBA$. Thus, angle
 $AFE$ is equal to $DEF$ [Prop.~3.27].
 Similarly, it can also be shown that the remaining angles of hexagon
 $ABCDEF$ are individually equal to each of the angles $AFE$ and $FED$.
 Thus, hexagon $ABCDEF$ is equiangular. And it was also shown (to be)
 equilateral. And it has been inscribed in circle $ABCDE$.
 
 Thus, an equilateral and equiangular hexagon has been inscribed
 in the given circle. (Which is) the very thing it was required to do.\\
 
 \begin{center}
{\large Corollary}
\end{center}\vspace*{-7pt}

So, from this, (it is) manifest that a side of the hexagon is equal
to the radius of the circle.

And similarly to  a pentagon, if we draw  tangents to the circle through
the (sixfold) divisions of the (circumference of the) circle, an equilateral and equiangular hexagon
can be circumscribed about the circle, analogously to the aforementioned pentagon. And, further, by (means) similar to the aforementioned pentagon,
we can inscribe and circumscribe a circle in (and about) a given hexagon. 
(Which
is) the very thing it was required to do.}
\end{Parallel}
{\footnotesize \noindent$^\dag$ See the
footnote to Prop.~4.6.} 

%%%%%%
% Prop 4.16
%%%%%%
\pdfbookmark[1]{Proposition 4.16}{pdf4.16}
\begin{Parallel}{}{} 
\ParallelLText{
\begin{center}
{\large \ggn{16}.}
\end{center}\vspace*{-7pt}

\gr{E>ic t`on doj'enta k'uklon pentekaidek'agwnon >is'opleur\-'on te ka`i
>isog'wnion >eggr'ayai.}

\epsfysize=2.2in
\centerline{\epsffile{Book04/fig16g.eps}}

\gr{>'Estw <o doje`ic k'ukloc <o ABGD; de~i d`h e>ic t`on ABGD
k'uklon pentekaidek'agwnon >is'opleur'on te ka`i >isog'wnion
>eggr'ayai.}

\gr{>Eggegr'afjw e>ic t`on ABGD k'uklon trig'wnou m`en >isople'urou
to~u e>ic a>ut`on >eggrafom'enou pleur`a <h AG, pentag'wnou d`e
>isople'urou <h AB; o<'iwn >'ara >est`in <o ABGD k'ukloc >'iswn
tm'hmatwn dekap'ente, toio'utwn <h m`en ABG perif'ereia tr'iton
o>~usa to~u k'uklou >'estai p'ente, <h d`e AB perif'ereia p'emton
o>~usa to~u k'uklou >'estai tri~wn; loip`h >'ara <h BG t~wn
>'iswn d'uo. tetm'hsjw <h BG d'iqa kat`a t`o E; <ekat'era >'ara t~wn
BE, EG periferei~wn pentekaid'ekat'on >esti to~u ABGD k'uklou.}

\gr{>E`an >'ara >epize'uxantec t`ac BE, EG >'isac a>uta~ic kat`a 
t`o suneq`ec e>uje'iac >enarm'oswmen e>ic t`on ABGD[E]
k'uklon, >'estai e>ic a>ut`on >eggegramm'enon pentekaidek'agwnon
>is'opleu\-r'on te ka`i >isog'wnion; <'oper >'edei poi~hsai.}

\gr{<Omo'iwc d`e to~ic >ep`i to~u pentag'wnou >e`an di`a t~wn kat`a
t`on k'uklon diair'esewn  >efaptom'enac to~u  k'uklou >ag'agwmen,
perigraf'hsetai per`i t`on k'uklon pentekaidek'agwnon >is'opleur'on
te ka`i >isog'wnion. >'eti d`e di`a t~wn <omo'iwn to~ic
>ep`i to~u pentag'wnou de'ixewn ka`i e>ic t`o doj`en pentekaidek'agwnon
k'uklon >eggr'ayom'en te ka`i perigr'ayomen; <'oper >'edei poi~hsai.}}

\ParallelRText{
\begin{center}
{\large Proposition 16}
\end{center}

To inscribe an equilateral and equiangular fifteen-sided figure in a given circle.

\epsfysize=2.2in
\centerline{\epsffile{Book04/fig16e.eps}}

Let $ABCD$ be the given circle. So it is required to inscribe an equilateral
and equiangular fifteen-sided figure in circle $ABCD$.

Let the side $AC$ of an equilateral triangle inscribed in (the circle) [Prop.~4.2],
and (the side) $AB$ of an (inscribed) equilateral pentagon [Prop.~4.11],
 have been inscribed in circle $ABCD$.
 Thus,   just as the circle $ABCD$ is (made up) of fifteen equal pieces, the
 circumference $ABC$, being a third of the circle, will be (made up) of five
 such (pieces), and the
 circumference $AB$, being a fifth of the circle, will be (made up) of  three.
 Thus, the remainder $BC$ (will be made up) of two equal (pieces).
 Let (circumference) $BC$ have been cut in half at $E$ [Prop.~3.30].
 Thus, each of the circumferences $BE$ and $EC$ is one fifteenth of the circle
 $ABCDE$.
 
 Thus, if, joining $BE$ and $EC$, we continuously insert straight-lines
 equal to them into circle $ABCD[E]$ [Prop.~4.1],
 then an equilateral and equiangular fifteen-sided figure will have been
 inserted into (the circle). (Which is) the very thing it was required to do.
 
 And similarly to the pentagon, if we draw  tangents to the circle through
the (fifteenfold) divisions of the (circumference of the) circle, we can
circumscribe an equilateral and equiangular fifteen-sided figure about
the circle. 
 And, further, through similar proofs to the  pentagon,
 we can also inscribe and circumscribe  a circle in (and about)  a given fifteen-sided 
  figure. (Which is) the very thing it was required to do.}
\end{Parallel}
\newpage
\thispagestyle{plain}~\\