%%%%%%%%%%%%%%%%
% TITLEPAGE
%%%%%%%%%%%%%%%%
\thispagestyle{empty}
\begin{center}
{\Huge\sf EUCLID'S ELEMENTS OF GEOMETRY}\\
\mbox{}\spa\spa\spa {\Large
{\sf The Greek text of J.L.~Heiberg (1883--1885)\nline
from {\em Euclidis Elementa, edidit et Latine interpretatus est I.L.~Heiberg, in aedibus B.G.~Teubneri, 1883--1885}\nline edited, and provided with a modern English translation, by\nline  {\em Richard Fitzpatrick}}}
\end{center}

%%%%%%%%%%%
%COPYRIGHT PAGE
%%%%%%%%%%%
\newpage
\thispagestyle{empty}
\begin{flushleft}
First edition - 2007\\
Revised and corrected - 2008\\
~\\
ISBN 978-0-6151-7984-1
\end{flushleft}

%%%%%%%%%%%%%%
%TABLE OF CONTENTS
%%%%%%%%%%%%%
\newpage~\\
\thispagestyle{empty}
\begin{center}
{\Large\bf Contents}
\end{center}
\begin{tabular}{lcr}
&&\\[0.5ex]
{\bf Introduction}&\mbox{\hspace{12cm}}&{\bf 4}\\[3.ex]
{\bf Book 1}  && {\bf    5}\\[3.ex]
{\bf Book 2}  && {\bf  49}\\[3.ex]
{\bf Book 3}  && {\bf  69}\\[3.ex]
{\bf Book 4}  && {\bf 109}\\[3.ex]
{\bf Book 5}  && {\bf 129}\\[3.ex]
{\bf Book 6}  && {\bf  155}\\[3.ex]
{\bf Book 7}  && {\bf  193}\\[3.ex]
{\bf Book 8}  && {\bf  227}\\[3.ex]
{\bf Book 9}  && {\bf 253}\\[3.ex]
{\bf Book 10} &&{\bf 281}\\[3.ex]
{\bf Book 11}  && {\bf  423}\\[3.ex]
{\bf Book 12}  && {\bf 471}\\[3.ex]
{\bf Book 13} &&{\bf 505}\\[3.ex]
{\bf Greek-English Lexicon}&&{\bf 539}
\end{tabular}
\newpage

%%%%%%%%%%
% PREFACE
%%%%%%%%%%
\pdfbookmark[0]{Introduction}{intro}
\begin{center}
{\Large Introduction}
\end{center}

Euclid's Elements is by far the most famous mathematical work of classical antiquity, and also has the distinction of being the world's oldest continuously used mathematical textbook. Little is known about the author, beyond the fact that he lived in Alexandria around 300 BCE. The main subjects  of the work 
are geometry, proportion, and number theory. 

Most of the theorems appearing in the Elements were not discovered by Euclid himself, but were the work of earlier Greek mathematicians such as Pythagoras (and his school), Hippocrates of Chios, Theaetetus of Athens, and Eudoxus of Cnidos. However, Euclid is generally credited with arranging these theorems in a logical manner, so as to demonstrate (admittedly, not always with the rigour demanded by modern mathematics) that they necessarily follow from five simple axioms. Euclid is also credited with devising a number of particularly ingenious proofs of previously discovered theorems: {\em e.g.}, Theorem 48 in Book 1.

The geometrical constructions employed in the Elements are restricted to
those which can be achieved using a straight-rule and a compass.
Furthermore, empirical proofs by means of measurement are strictly forbidden:
{\em i.e.}, 
any comparison of two magnitudes is restricted to saying that the magnitudes
are either equal, or that one is greater than the other.

The Elements consists of thirteen books. Book~1 outlines the fundamental
propositions of plane geometry, including the three cases in which
triangles are congruent, various theorems involving parallel lines, the theorem regarding the sum of the angles in a triangle, and the
Pythagorean theorem. Book~2 is commonly said to deal with ``geometric algebra'', since most of the theorems contained within it have simple algebraic
interpretations.
Book~3 investigates circles and their properties, and includes
theorems on tangents and inscribed angles. Book~4
is concerned with regular polygons inscribed in, and circumscribed around, circles.
Book~5 develops the arithmetic theory of proportion.
Book~6 applies the theory of proportion to plane geometry, and
contains theorems on similar figures. Book~7 deals
with elementary number theory: {\em e.g.}, prime numbers,
greatest common denominators, {\em etc.} Book~8 is concerned with geometric
series. Book~9 contains various applications of  results in the
previous two books, and includes theorems on the
infinitude of prime numbers, as well as the sum of a geometric series.
Book~10 attempts to classify incommensurable ({\em i.e.}, irrational)
magnitudes using the so-called ``method of exhaustion'', an ancient precursor to integration. Book~11 deals with the fundamental propositions of
three-dimensional geometry. Book~12 calculates the relative volumes of
cones, pyramids, cylinders, and spheres using the method of exhaustion.
Finally, Book~13 investigates the five so-called Platonic solids.

This  edition of Euclid's Elements presents the definitive Greek text---{\em i.e.}, that edited by J.L. Heiberg 
(1883--1885)---accompanied by a modern English translation, as well as a Greek-English
lexicon. Neither the spurious
books 14 and 15, nor the extensive scholia which have been added to
the Elements over the centuries, are included.
The aim of the translation is to make the mathematical argument as clear and unambiguous as possible, whilst still adhering closely to  the meaning of the original Greek. Text within square parenthesis (in both Greek and English) indicates material identified by Heiberg as being later interpolations to the original text (some particularly obvious or unhelpful interpolations have been omitted  
altogether). Text within round parenthesis (in English) indicates material which is implied, but not actually present, in the Greek text. 

My thanks to Mariusz Wodzicki (Berkeley) for typesetting advice, and
to Sam Watson \& Jonathan Fenno (U.\ Mississippi), and  Gregory Wong (UCSD) for pointing out a number of errors in Book 1.