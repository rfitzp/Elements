% !TEX root = ../Elements.tex
%%%%%%
% BOOK 8
%%%%%%
\pdfbookmark[0]{Book 8}{book8}
\pagestyle{empty}
\begin{center}
{\Huge ELEMENTS BOOK 8}\\
\spa\spa\spa
{\huge\it Continued Proportion\symbolfootnote[2]{The propositions contained in Books 7--9 are generally attributed to the
school of Pythagoras.}}
\end{center}
%\vspace{2cm}
%\epsfysize=4in
%\centerline{\epsffile{Book08/front.eps}}
\newpage

%%%%%%
% Prop 8.1
%%%%%%
\pdfbookmark[1]{Proposition 8.1}{pdf8.1}
\pagestyle{fancy}
\cfoot{\gr{\thepage}}
\lhead{\large\gr{STOIQEIWN \ggn{8}.}}
\rhead{\large ELEMENTS BOOK 8}
\begin{Parallel}{}{} 
\ParallelLText{
\begin{center}
{\large \ggn{1}.}
\end{center}\vspace*{-7pt}

\gr{>E`an >~wsin <osoidhpoto~un >arijmo`i <ex~hc >an'alogon, o<i  d`e >'akroi a\kern -.7pt >ut~wn pr~wtoi pr`oc >all'hlouc >~wsin, >el'aqisto'i e>isi t~wn t`on
a\kern -.7pt >ut`on l'ogon >eq'ontwn a\kern -.7pt >uto~ic.}\\

\epsfysize=0.9in
\centerline{\epsffile{Book08/fig01g.eps}}

\gr{>'Estwsan <oposoio~un >arijmo`i <ex~hc >an'alogon o<i A, B, G,
D, o<i d`e >'akroi a\kern -.7pt >ut~wn o<i A, D, pr~wtoi pr`oc >all'hlouc
>'estwsan; l'egw, <'oti o<i A, B, G, D >el'aqisto'i e>isi
t~wn t`on a\kern -.7pt >ut`on l'ogon >eq'ontwn a\kern -.7pt >uto~ic.}

\gr{E>i g`ar m\kern -.7pt 'h, >'estwsan >el'attonec t~wn A, B, G, D
o<i E, Z, H, J >en t~w| a\kern -.7pt >ut~w| l'ogw| >'ontec a\kern -.7pt >uto~ic. ka`i
>epe`i o<i A, B, G, D >en t~w| a\kern -.7pt >ut~w| l'ogw|
e>is`i to~ic
E, Z, H, J, ka'i >estin >'ison t`o pl~hjoc [t~wn A, B,
G, D] t~w| pl'hjei [t~wn E, Z, H, J], di> >'isou >'ara
>est`in <wc <o A pr`oc t`on D, <o E pr`oc t`on J. o<i d`e A, D
pr~wtoi, o<i d`e pr~wtoi ka`i >el'aqistoi, o<i
d`e >el'aqistoi >arijmo`i metro~usi to`uc t`on a\kern -.7pt >ut`on l'ogon
>'eqontac >is'akic <'o te me'izwn t`on me'izona ka`i <o >el'asswn
t`on >el'assona, tout'estin <'o te <hgo'umenoc t`on <hgo'umenon
ka`i <o <ep'omenoc t`on <ep'omenon. metre~i >'ara <o A t`on
E <o me'izwn t`on >el'assona; <'oper >est`in >ad'unaton. o>uk
>'ara  o<i E, Z, H, J >el'assonec >'ontec t~wn A, B,
G, D >en t~w| a\kern -.7pt >ut~w| l'ogw| e>is`in a\kern -.7pt >uto~ic. o<i A, B, G,
D >'ara >el'aqisto'i e>isi t~wn t`on a\kern -.7pt >ut`on l'ogon >eq'ontwn a\kern -.7pt >uto~ic;
<'oper >'edei de~ixai.}}

\ParallelRText{
\begin{center}
{\large Proposition 1}
\end{center}

If there are any multitude whatsoever of continuously proportional
numbers, and the outermost of them are prime to one another, then
the (numbers) are the least of those (numbers) having the same ratio as them.

\epsfysize=0.9in
\centerline{\epsffile{Book08/fig01e.eps}}

Let $A$, $B$, $C$,  $D$ be any multitude whatsoever of continuously proportional
numbers. And let the outermost of them, $A$ and $D$, be prime to
one another. I say that $A$, $B$, $C$, $D$ are the least of those
(numbers) having the same ratio as them.

For if not, let $E$, $F$, $G$, $H$ be less than $A$, $B$, $C$,  $D$ (respectively),
being in the same ratio as them. And since $A$, $B$, $C$, $D$ are
in the same ratio as $E$, $F$, $G$,  $H$, and
the multitude [of $A$, $B$, $C$, $D$] is equal to the
multitude [of $E$, $F$, $G$, $H$], thus, via equality, as
$A$ is to $D$, (so) $E$ (is) to $H$ [Prop.~7.14]. 
And $A$ and $D$ (are) prime (to one another). And prime (numbers are)
also the least of those (numbers having the same ratio as them) [Prop.~7.21]. And the least numbers measure those
(numbers) having the same ratio (as them) an equal number of times, the
greater (measuring) the greater, and the lesser the lesser---that is to say,
the leading (measuring) the leading, and the following the following [Prop.~7.20]. Thus, $A$ measures $E$, the
greater (measuring) the lesser. The very thing is impossible. Thus, $E$,
$F$, $G$,  $H$, being less than $A$, $B$, $C$, $D$, are not
in the same ratio as them. Thus, $A$, $B$, $C$,  $D$ are the least of those
(numbers) having the same ratio as them. (Which is) the very thing it was required to show.}
\end{Parallel}

%%%%%%
% Prop 8.2
%%%%%%
\pdfbookmark[1]{Proposition 8.2}{pdf8.2}
\begin{Parallel}{}{} 
\ParallelLText{
\begin{center}
{\large\ggn{2}.}
\end{center}\vspace*{-7pt}

\gr{Arijmo`uc e<ure~in <ex~hc >an'alogon >elaq'istouc, <'osouc >`an >epit'axh|
tic, >en t~w| doj'enti l'ogw|.}

\gr{>'Estw <o doje`ic l'ogoc >en >el'aq'istoic >arijmo~ic <o to~u A pr`oc
t`on B; de~i d`h >arijmo`uc e<ure~in <ex~hc >an'alogon >elaq'istouc,
<'osouc >'an tic >epit'axh|, >en t~w| to~u A pr`oc t`on B
l'ogw|.}

\gr{>Epitet'aqjwsan d`h t'essarec, ka`i <o A <eaut`on pollaplasi'asac t`on
G poie'itw, t`on d`e B pollaplasi'asac t`on D poie'itw, ka`i >'eti <o B
 <eaut`on pollaplasi'asac t`on E poie'itw,
ka`i >'eti <o A to`uc G, D, E pollaplasi'asac to`uc Z, H, J
poie'itw, <o d`e B t`on E pollaplasi'asac t`on K poie'itw.}

\epsfysize=2in
\centerline{\epsffile{Book08/fig02g.eps}}

\gr{Ka`i >epe`i <o A <eaut`on m`en pollaplasi'asac t`on G pepo'ihken,
t`on d`e B pollaplasi'asac t`on D pepo'ihken, >'estin >'ara <wc <o
A pr`oc t`on B, [o<'utwc] <o G pr`oc t`on D. p'alin, >epe`i <o m`en
A t`on B pollaplasi'asac t`on D pepo'ihken, <o d`e B <eaut`on
pollaplasi'asac t`on E pepo'ihken, <ek'ateroc >'ara t~wn A, B t`on B
pollaplasi'asac <ek'ateron t~wn D, E pepo'ihken. >'estin >'ara <wc <o A
pr`oc t`on B, o<'utwc <o D pr`oc t`on E. >all> <wc <o A pr`oc t`on B,
<o G pr`oc t`on D; ka`i <wc >'ara <o G pr`oc t`on D, <o D pr`oc t`on
E. ka`i >epe`i <o A to`uc G, D pollaplasi'asac to`uc Z, H pepo'ihken, >'estin >'ara
<wc <o G pr`oc t`on D, [o<'utwc] <o Z pr`oc t`on H. <wc d`e <o G pr`oc
t`on D, o<'utwc >~hn <o A pr`oc t`on B; ka`i <wc >'ara <o A pr`oc t`on
B, <o Z pr`oc t`on H. p'alin, >epe`i <o A to`uc D, E pollaplasi'asac to`uc
H, J pepo'ihken, >'estin >'ara <wc <o D pr`oc t`on E, <o H pr`oc t`on J. >all>
<wc <o D pr`oc t`on E, <o A pr`oc t`on B. ka`i <wc >'ara <o A pr`oc t`on B,
o<'utwc <o H pr`oc t`on J. ka`i >epe`i o<i A, B t`on E pollaplasi'asantec
to`uc J, K pepoi'hkasin, >'estin >'ara <wc <o A pr`oc t`on B, o<'utwc <o J
pr`oc t`on K. >all> <wc <o A pr`oc t`on B, o<'utwc <'o te Z pr`oc t`on H
ka`i <o H pr`oc t`on J. ka`i <wc >'ara <o Z pr`oc t`on H, o<'utwc <'o
te H pr`oc t`on J ka`i <o J pr`oc t`on K; o<i G, D, E >'ara ka`i o<i Z, H,
J, K >an'alog'on e>isin >en t~w| to~u A pr`oc t`on B l'ogw|. l'egw d'h,
<'oti ka`i >el'aqistoi. >epe`i g`ar o<i A, B >el'aqisto'i e>isi t~wn t`on a\kern -.7pt >ut`on
l'ogon >eq'ontwn a\kern -.7pt >uto~ic, o<i d`e >el'aqistoi t~wn t`on a\kern -.7pt >ut`on l'ogon
>eq'ontwn pr~wtoi pr`oc >all'hlouc e>is'in, o<i A, B >'ara pr~wtoi pr`oc
>all'hlouc e>is'in. ka`i <ek'ateroc m`en t~wn A, B <eaut`on pollaplasi'asac
<ek'ateron t~wn G, E pepo'ihken, <ek'ateron d`e t~wn G, E pollaplasi'asac
<ek'ateron t~wn Z, K pepo'ihken; o<i G, E >'ara ka`i o<i Z, K pr~wtoi
pr`oc >all'hlouc e>is'in. >e`an d`e >~wsin <oposoio~un >arijmo`i
<ex~hc >an'alogon, o<i d`e >'akroi a\kern -.7pt >ut~wn pr~wtoi pr`oc >all'hlouc
>~wsin, >el'aqisto'i e>isi t~wn t`on a\kern -.7pt >ut`on l'ogon >eq'ontwn a\kern -.7pt >uto~ic.
o<i G, D, E >'ara ka`i o<i Z, H, J, K >el'aqisto'i e>isi t~wn t`on a\kern -.7pt >ut`on
l'ogon >eq'ontwn to~ic A, B; <'oper >'edei de~ixai.}\\

\begin{center}
{\large \gr{P'orisma}.}
\end{center}\vspace*{-7pt}

\gr{>Ek d`h to'utou faner'on, <'oti >e`an tre~ic >arijmo`i <ex~hc >an'alogon
>el'aqistoi >~wsi t~wn t`on a\kern -.7pt >ut`on l'ogon >eq'ontwn a\kern -.7pt >uto~ic, o<i
>'akron a\kern -.7pt >ut~wn tetr'agwno'i e>isin, >e`an d`e t'essarec, k'uboi.
}}

\ParallelRText{
\begin{center}
{\large Proposition 2}
\end{center}

To find the least numbers, as many as may be prescribed,  (which are) continuously proportional
in a given ratio.

Let the given ratio, (expressed) in the least numbers, be that of $A$ to $B$.
So it is required to find the least numbers, as many as may be prescribed, 
(which are) in the ratio of $A$ to $B$. 

Let four (numbers) be prescribed. And let $A$ make $C$ (by) multiplying itself, and let it make $D$ (by) multiplying $B$.
And, further, let $B$ make $E$ (by) multiplying itself. And, further, 
let $A$ make $F$, $G$,  $H$ (by) multiplying $C$, $D$,  $E$. And let $B$ make $K$ (by) multiplying $E$.

\epsfysize=2in
\centerline{\epsffile{Book08/fig02e.eps}}

And since $A$ has made $C$ (by) multiplying itself, and has made $D$ (by)
multiplying $B$, thus as $A$ is to $B$, [so] $C$ (is) to $D$ [Prop.~7.17]. Again, since $A$ has made $D$
(by) multiplying $B$, and $B$ has made $E$ (by) multiplying itself,
 $A$, $B$ have thus made  $D$, $E$, respectively, (by)
multiplying $B$.  Thus, as $A$ is to $B$, so $D$ (is) to $E$ [Prop.~7.18]. But, as $A$ (is) to $B$, (so)
$C$ (is) to $D$. And thus as $C$ (is) to $D$, (so) $D$ (is) to $E$.
And since $A$ has made $F$, $G$ (by) multiplying $C$, $D$, thus as $C$ is to $D$, [so] $F$ (is) to $G$ [Prop.~7.17]. And as $C$ (is) to $D$, so
$A$ was to $B$. And thus as $A$ (is) to $B$, (so) $F$ (is) to $G$. Again,
since $A$ has made $G$, $H$ (by) multiplying $D$, $E$, thus as $D$ is to $E$, (so) $G$ (is) to $H$ [Prop.~7.17]. But, as $D$ (is) to $E$, (so) $A$ (is)
to $B$.  And thus as $A$ (is) to $B$, so $G$ (is) to $H$. And since
$A$, $B$ have made $H$, $K$ (by) multiplying
$E$, thus as $A$ is to $B$, so $H$ (is) to $K$. But, as $A$ (is) to $B$, so
$F$ (is) to $G$, and $G$ to $H$.  And thus as $F$ (is) to $G$, so $G$ (is)
to $H$, and $H$ to $K$. Thus, $C$, $D$, $E$ and $F$, $G$, $H$, $K$
are (both continuously) proportional in the ratio of $A$ to $B$. So I
say that (they are) also the least (sets of numbers continuously proportional in that
ratio). For since $A$ and $B$ are the least of those (numbers) having the
same ratio as them, and the least of those (numbers) having the same
ratio are prime to one another [Prop.~7.22],
$A$ and $B$ are thus prime to one another. And  $A$, $B$
have  made  $C$, $E$, respectively, (by) multiplying themselves, and have made $F$, $K$ by multiplying $C$, $E$, respectively. Thus,
$C$, $E$ and $F$, $K$ are prime to one another [Prop.~7.27]. And if there are any multitude whatsoever of continuously  proportional numbers, and the outermost
of them are prime to one another,  then the (numbers)  are the least of those
(numbers) having the same ratio as them [Prop.~8.1].
Thus, $C$, $D$, $E$ and $F$, $G$, $H$, $K$ are the least of those
(continuously proportional sets of numbers) having the same ratio as $A$ and $B$. (Which is) the very thing it was required to show.\\~\\

\begin{center}
{\large Corollary}
\end{center}\vspace*{-7pt}

So  it is clear, from this,  that if three continuously proportional numbers are the
least of those (numbers) having the same ratio as them then the outermost
of them are square, and, if four (numbers), cube.}
\end{Parallel}

%%%%%%
% Prop 8.3
%%%%%%
\pdfbookmark[1]{Proposition 8.3}{pdf8.3}
\begin{Parallel}{}{} 
\ParallelLText{
\begin{center}
{\large \ggn{3}.}
\end{center}\vspace*{-7pt}

\gr{>E`an >~wsin <oposoio~un >arijmo`i <ex~hc >an'alogon >el'aqist\-oi
t~wn t`on a\kern -.7pt >ut`on l'ogon >eq'ontwn a\kern -.7pt >uto~ic, o<i >'akroi
a\kern -.7pt >ut~wn pr~wtoi pr`oc >all'hlouc e>is'in.}\\

\epsfysize=2in
\centerline{\epsffile{Book08/fig03g.eps}}

\gr{>'Estwsan <oposoio~un >arijmo`i <ex~hc >an'alogon >el'aqistoi
t~wn t`on a\kern -.7pt >ut`on l'ogon >eq'ontwn a\kern -.7pt >uto~ic o<i A, B, G, D;
l'egw, <'oti o<i >'akroi a\kern -.7pt >ut~wn o<i A, D pr~wtoi pr`oc >all'hlouc
e>is'in.}

\gr{E>il'hfjwsan g`ar d'uo m`en >arijmo`i >el'aqistoi >en t~w| t~wn A, B, G,
D l'ogw| o<i E, Z, tre~ic d`e o<i H, J, K, ka`i <ex~hc <en`i
ple'iouc, <'ewc t`o lamban'omenon pl~hjoc >'ison g'enhtai t~w| pl'hjei
t~wn A, B, G, D. e>il'hfjwsan ka`i >'estwsan o<i L, M, N, X.}

\gr{Ka`i >epe`i o<i E, Z >el'aqisto'i e>isi  t~wn t`on a\kern -.7pt >ut`on l'ogon >eq'ontwn
a\kern -.7pt >uto~ic, pr~wtoi pr`oc >all'hlouc e>is'in. ka`i >epe`i <ek'ateroc t~wn
E, Z <eaut`on m`en pollaplasi'asac <ek'ateron t~wn H, K pepo'ihken,
<ek'ateron d`e t~wn H, K pollaplasi'asac <ek'ateron t~wn L, X
pepo'ihken, ka`i o<i H, K >'ara ka`i o<i L, X pr~wtoi pr`oc
>all'hlouc e>is'in. ka`i >epe`i o<i A, B, G, D >el'aqisto'i
e>isi t~wn t`on a\kern -.7pt >ut`on l'ogon >eq'ontwn a\kern -.7pt >uto~ic, e>is`i
d`e ka`i o<i L, M, N, X >el'aqistoi >en t~w| a\kern -.7pt >ut~w| l'ogw| >'ontec
to~ic A, B, G, D, ka'i >estin >'ison t`o pl~hjoc t~wn A, B, G, D t~w|
pl'hjei t~wn L, M, N, X, <'ekastoc >'ara t~wn A, B, G, D <ek'astw| t~wn
L, M, N, X
 >'isoc >est'in; >'isoc >'ara >est`in <o m`en
A t~w| L, <o d`e D t~w| X. ka'i e>isin o<i L, X pr~wtoi pr`oc >all'hlouc.
ka`i o<i A, D >'ara pr~wtoi pr`oc >all'hlouc e>is'in; <'oper
>'edei de~ixai.}}

\ParallelRText{
\begin{center}
{\large Proposition 3}
\end{center}

If there are any multitude whatsoever of continuously proportional
numbers (which are)  the least of those (numbers) having the same ratio as them then the outermost of them are prime to one another.

\epsfysize=2in
\centerline{\epsffile{Book08/fig03e.eps}}

 Let $A$, $B$, $C$, $D$ be  any multitude whatsoever of continuously proportional
numbers (which are)  the least of those (numbers) having the same ratio as them. I say that the outermost of them, $A$ and $D$, are prime to
one another.

For let the two 
least (numbers) $E$, $F$ (which are) in the same ratio as $A$, $B$, $C$,
$D$
be taken [Prop.~7.33]. And the three
(least numbers) $G$, $H$, $K$ [Prop.~8.2].
And (so on), successively increasing by one, until the multitude of
(numbers) taken is made equal to the multitude of $A$, $B$, $C$,  $D$.
Let them be taken, and let them be $L$, $M$, $N$,  $O$.

And since $E$ and $F$ are the least of those (numbers) having the
same ratio as them they are prime to one another [Prop.~7.22]. And since  $E$, $F$ have made
 $G$,  $K$, respectively, (by) multiplying themselves [Prop.~8.2~corr.], and  have made $L$,  $O$ 
(by) multiplying  $G$, $K$, respectively,  $G$, $K$ and $L$, $O$ are thus
 also prime to one another [Prop.~7.27]. 
And since $A$, $B$, $C$, $D$ are the least of those (numbers) having
the same ratio as them, and $L$, $M$, $N$,  $O$ are also the
least (of those numbers having the same ratio as them), being in the
same ratio as $A$, $B$, $C$, $D$, and the
multitude of $A$, $B$, $C$,  $D$ is equal to the multitude
of $L$, $M$, $N$,  $O$,  thus $A$, $B$, $C$,  $D$ are
equal to  $L$, $M$, $N$,  $O$, respectively. Thus, $A$ is equal to $L$, and $D$ to $O$. And $L$ and $O$ are prime to one another. Thus,
$A$ and $D$ are also prime to one another. (Which is) the very thing it was required to show.}
\end{Parallel}

%%%%%%
% Prop 8.4
%%%%%%
\pdfbookmark[1]{Proposition 8.4}{pdf8.4}
\begin{Parallel}{}{} 
\ParallelLText{
\begin{center}
{\large \ggn{4}.}
\end{center}\vspace*{-7pt}

\gr{L'ogwn doj'entwn <oposwno~un >en >elaq'istoic >arijmo~ic >arijmo`uc
e<ure~in <ex~hc >an'alogon >elaq'istouc >en to~ic doje~isi l'ogoic.}\\

\epsfysize=1.7in
\centerline{\epsffile{Book08/fig04g.eps}}

\gr{>'Estwsan o<i doj'entec l'ogoi >en >elaq'istoic >arijmo~ic <'o te to~u A
pr`oc t`on B ka`i <o to~u G pr`oc t`on D ka`i >'eti <o to~u E pr`oc
t`on Z; de~i d`h >arijmo`uc e<ure~in <ex~hc >an'alogon >elaq'istouc
>'en te t~w| to~u A pr`oc t`on B l'ogw| ka`i >en t~w| to~u G pr`oc
t`on D ka`i >'eti t~w| to~u E pr`oc t`on Z.}

\gr{E>il'hfjw g`ar <o <up`o t~wn B, G >el'aqistoc metro'umenoc >arijm`oc
<o H. ka`i <os'akic m`en <o B t`on H metre~i, tosaut'akic ka`i <o A t`on
J metre'itw, <os'akic d`e <o G t`on H metre~i, tosaut'akic ka`i <o D t`on K
metre'itw. <o d`e E t`on K >'htoi metre~i >`h o>u metre~i. metre'itw pr'oteron. ka`i <os'akic <o E t`on K metre~i, tosaut'akic ka`i <o Z t`on L metre'itw. ka`i >epe`i >is'akic <o A t`on J metre~i ka`i <o B t`on H, >'estin
>'ara <wc <o A pr`oc t`on B, o<'utwc <o J pr`oc t`on H. di`a t`a a\kern -.7pt >ut`a d`h ka`i <wc <o G pr`oc t`on D, o<'utwc <o H pr`oc t`on K, ka`i >'eti <wc <o
E pr`oc t`on Z, o<'utwc <o K pr`oc t`on L; o<i J, H, K, L >'ara <ex~hc >an'alog'on e>isin >'en te t~w| to~u A pr`oc t`on B ka`i >en t~w| to~u
G pr`oc t`on D ka`i >'eti >en t~w| to~u E pr`oc t`on Z l'ogw|. l'egw d'h, <'oti
ka`i >el'aqistoi. e>i g`ar m\kern -.7pt 'h e>isin o<i J, H, K, L
<ex~hc >an'alogon >el'aqistoi >'en te to~ic to~u A pr`oc t`on B ka`i to~u
G pr`oc t`on D ka`i >en t~w| to~u E pr`oc t`on Z l'ogoic, >'estwsan o<i
N, X, M, O. ka`i >epe'i >estin <wc <o A pr`oc t`on B, o<'utwc <o N
pr`oc t`on X, o<i d`e A, B >el'aqistoi, o<i d`e >el'aqistoi metro~usi to`uc t`on
a\kern -.7pt >ut`on l'ogon >'eqontac >is'akic <'o te me'izwn t`on me'izona ka`i <o >el'asswn t`on >el'assona, tout'estin <'o te <hgo'umenoc t`on <hgo'umenon ka`i <o <ep'omenoc t`on <ep'omenon, <o B >'ara t`on X metre~i. di`a t`a
a\kern -.7pt >ut`a d`h ka`i <o G t`on X metre~i; o<i B, G >'ara t`on X metro~usin;
ka`i <o >el'aqistoc >'ara <up`o t~wn B, G metro'umenoc t`on X metr'hsei.
>el'aqistoc d`e <up`o t~wn B, G metre~itai <o H; <o H >'ara t`on X metre~i
<o me'izwn t`on >el'assona; <'oper >est`in >ad'untaton. o>uk >'ara >'esonta'i tinec t~wn J, H, K, L >el'assonec >arijmo`i <ex~hc >'en te t~w| to~u A pr`oc
t`on B ka`i t~w| to~u G pr`oc t`on D ka`i >'eti t~w| to~u E pr`oc t`on Z
l'og~w|.}\\~\\~\\~\\~\\

\epsfysize=2.5in
\centerline{\epsffile{Book08/fig04ag.eps}}

\gr{m\kern -.7pt `h metre'itw d`h <o E t`on K, ka`i e>il'hfjw <up`o t~wn E, K >el'aqistoc
metro'umenoc >arijm`oc <o M. ka`i <os'akic m`en <o K t`on M
metre~i, tosaut'akic ka`i <ek'ateroc t~wn J, H <ek'ateron t~wn N, X metre'itw,
<os'aakic d`e <o E t`on M metre~i, tosaut'akic ka`i <o Z t`on O metre'itw.
>epe`i >is'akic <o J t`on N metre~i ka`i <o H t`on X, >'estin >'ara
<wc <o J pr`oc t`on H, o<'utwc <o N pr`oc t`on X. <wc d`e <o J pr`oc t`on H,
o<'utwc <o A pr`oc t`on B; ka`i <wc >'ara <o A pr`oc t`on B, o<'utwc
<o N pr`oc t`on X.  di`a
t`a a\kern -.7pt >ut`a d`h ka`i <wc <o G pr`oc t`on D, o<'utwc <o X pr`oc t`on M. p'alin,
>epe`i >is'akic <o E  t`on M metre~i ka`i <o Z t`on O, >'estin >'ara
<wc <o E pr`oc  t`on Z, o<'utwc <o M pr`oc t`on O; o<i N, X, M, O
>'ara <ex~hc >an'alog'on e>isin >en to~ic to~u te A pr`oc t`on B ka`i to~u
G pr`oc  t`on D ka`i >'eti to~u E pr`oc t`on Z l'ogoic. l'egw d'h, <'oti ka`i
>el'aqistoi >en to~ic A B, G D, E Z l'ogoic. e>i  g`ar m\kern -.7pt 'h, >'esonta'i tinec t~wn N, X, M, O
>el'assonec >arijmo`i <ex~hc >an'alogon >en to~ic A B, G D, E Z
l'ogoic. >'estwsan o<i P, R, S, T. ka`i >epe'i >estin <wc <o P pr`oc t`on R,
o<'utwc <o A pr`oc t`on B, o<i d`e A, B >el'aqistoi, o<i d`e >el'aqistoi
metro~usi to`uc t`on a\kern -.7pt >ut`on l'ogon >'eqontac
a\kern -.7pt >uto~ic >is'akic <'o te <hgo'umenoc t`on <hgo'umenon ka`i <o <ep'omenoc
t`on <ep'omenon, <o B >'ara t`on R metre~i. di`a t`a a\kern -.7pt >ut`a d`h ka`i <o G t`on
R metre~i; o<i B, G >'ara t`on R metro~usin. ka`i <o >el'aqistoc >'ara <up`o
t~wn B, G meto'umenoc t`on R metr'hsei. >el'aqistoc d`e <up`o t~wn
B, G metro'umenoc >estin <o H; <o H >'ara t`on R metre~i.
ka'i >estin <wc <o H pr`oc t`on  R, o<'utwc <o K pr`oc t`on S; ka`i
<o K >'ara t`on S metre~i. metre~i d`e ka`i <o E t`on S; o<i E, K
>'ara t`on S
metro~usin. ka`i <o >el'aqistoc >'ara <up`o t~wn
E, K metro'umenoc t`on S metr'hsei. >el'aqistoc d`e <up`o t~wn E, K
metro'umen'oc >estin <o M; <o M >'ara t`on S metre~i <o me'izwn
t`on >el'assona; <'oper >est`in >ad'unaton. o>uk >'ara >'esonta'i tinec t~wn
N, X, M, O >el'assonec >arijmo`i <ex~hc >an'alogon >'en te to~ic
to~u A pr`oc t`on B ka`i to~u G pr`oc t`on D ka`i >'eti to~u E pr`oc t`on
Z l'ogoic; o<i N, X, M, O >'ara <ex~hc >an'alogon >el'aqisto'i e>isin
>en to~ic A B, G D, E Z l'ogoic; <'oper >'edei de~ixai.}}

\ParallelRText{
\begin{center}
{\large Proposition 4}
\end{center}

For any multitude whatsoever of given ratios, (expressed) in the least numbers, to find the least numbers continuously proportional 
in these given ratios.

\epsfysize=1.7in
\centerline{\epsffile{Book08/fig04e.eps}}

Let the given ratios, (expressed) in the least numbers, be the (ratios) of
$A$ to $B$, and of $C$ to $D$, and, further,  of $E$ to $F$. So it is
required to find the least numbers continuously proportional  in the
ratio of $A$ to $B$, and  of $C$ to $B$, and, further,
 of $E$ to $F$.
 
For let the least number, $G$,  measured by (both) $B$ and $C$ have
be taken [Prop.~7.34]. And as many times
as $B$ measures $G$, so many times let $A$ also measure $H$.
And as many times as $C$ measures $G$, so many times let
$D$ also measure $K$. And $E$ either measures, or does not measure, $K$.
Let it, first of all, measure ($K$). And as many times as $E$ measures
$K$, so many times let $F$ also measure $L$. And since $A$
measures $H$ the same number of times that $B$ also (measures) $G$, thus as $A$ is to $B$, so $H$ (is) to $G$ [Def.~7.20, Prop.~7.13].
And so, for the same (reasons), as $C$ (is) to $D$, so
$G$ (is) to $K$, and, further, as $E$ (is) to $F$, so $K$ (is) to $L$.
Thus, $H$, $G$, $K$, $L$ are continuously proportional in the ratio
of $A$ to $B$, and of $C$ to $D$, and, further,  of $E$ to $F$. So I say that (they are) also the least
(numbers continuously proportional in these ratios). For if
$H$, $G$, $K$, $L$ are not the least numbers continuously proportional
 in the ratios of $A$ to $B$, and of $C$ to $D$, and of
$E$ to $F$, let $N$, $O$, $M$, $P$ be (the least such numbers). 
And since as $A$ is to $B$, so $N$ (is) to $O$, and $A$ and $B$ are the
least (numbers which have the same ratio as them), and the least (numbers) measure those (numbers)
having the same ratio (as them) an equal number of times, the greater (measuring) the greater, and the lesser the lesser---that is to say, the leading (measuring) the leading, and the
following the following [Prop.~7.20], $B$ thus
measures $O$. So, for the same (reasons), $C$ also measures $O$. Thus,
$B$ and $C$ (both) measure $O$. Thus, the least number measured by
(both) $B$ and $C$ will also measure $O$ [Prop.~7.35]. And $G$ (is) the least number measured by (both) $B$ and $C$. Thus, $G$ measures $O$, the greater
(measuring) the lesser. The very thing is impossible. Thus, there
cannot be any numbers less than $H$, $G$, $K$, $L$ 
(which are) continuously (proportional) in the ratio of $A$ to $B$, and of $C$ to $D$, and, further, of $E$ to $F$.

\epsfysize=2.5in
\centerline{\epsffile{Book08/fig04ae.eps}}

So let $E$ not measure $K$. And let the least number, $M$, measured by 
(both) $E$ and $K$ be taken [Prop.~7.34]. And as
many times as $K$ measures $M$, so many times let  $H$, $G$
also measure  $N$, $O$, respectively.  And as many times as $E$ measures $M$,
so many times let $F$ also measure $P$. Since $H$ measures $N$ the same
number of times as $G$
(measures) $O$, thus as $H$ is to $G$, so
$N$ (is) to $O$ [Def.~7.20, Prop.~7.13]. And as $H$ (is) to $G$, so $A$ (is)
to $B$. And thus as $A$ (is) to $B$, so $N$ (is) to $O$. And so, for the
same (reasons),  as $C$ (is) to $D$, so $O$ (is) to $M$. Again, since
$E$ measures $M$ the same number of times as $F$ (measures) $P$,
thus as $E$ is to $F$, so $M$ (is) to $P$ [Def.~7.20, Prop.~7.13]. Thus, $N$, $O$, $M$, $P$ are continuously proportional in the ratios of $A$ to $B$, and of $C$ to $D$, and,
further, of $E$ to $F$. So I say that (they are) also the least (numbers)
in the ratios of $A$ $B$, $C$ $D$, $E$ $F$.
For if not, then there will be some numbers less than $N$, $O$, $M$, $P$
(which are) continuously proportional in the ratios of $A$ $B$, $C$ $D$,
$E$ $F$. Let them be $Q$, $R$, $S$, $T$.
And since as $Q$ is to $R$, so $A$ (is) to $B$, and $A$ and $B$ (are) the
least (numbers having the same ratio as them), and the least (numbers) measure those
(numbers) having the same ratio as them an equal number of times, the leading (measuring) the leading, and the following the following [Prop.~7.20], $B$ thus measures $R$. So, for the
same (reasons), $C$ also measures $R$.  Thus, $B$ and $C$ (both)
measure $R$. Thus, the least (number) measured by (both) $B$ and $C$ will
also measure $R$ [Prop.~7.35]. And $G$ is the
least number measured by (both) $B$ and $C$. Thus, $G$ measures $R$.
And as $G$ is to $R$, so $K$ (is) to $S$. Thus, $K$ also measures $S$
[Def.~7.20]. And $E$ also measures $S$ [Prop.~7.20]. Thus, $E$ and $K$ (both) measure
$S$. Thus, the least (number) measured by (both) $E$ and $K$ will also
measure $S$   [Prop.~7.35]. And $M$  is the least (number) measured by (both)
$E$ and $K$. Thus, $M$ measures $S$, the greater (measuring) the lesser. The very thing is impossible.
Thus there cannot be any numbers less than $N$, $O$, $M$, $P$ (which
are) continuously proportional in the ratios of $A$ to $B$, and of
$C$ to $D$, and, further, of $E$ to $F$. Thus, $N$, $O$, $M$, $P$
are the least (numbers) continuously proportional in the ratios
of $A$ $B$, $C$ $D$, $E$ $F$. (Which is) the very thing it
was required to show.}
\end{Parallel}

%%%%%%
% Prop 8.5
%%%%%%
\pdfbookmark[1]{Proposition 8.5}{pdf8.5}
\begin{Parallel}{}{} 
\ParallelLText{
\begin{center}
{\large \ggn{5}.}
\end{center}\vspace*{-7pt}

\gr{O<i >ep'ipedoi >arijmo`i pr`oc >all'hlouc l'ogon >'eqousi t`on sugke'imenon
>ek t~wn pleur~wn.}

\epsfysize=2in
\centerline{\epsffile{Book08/fig05g.eps}}

\gr{>'Estwsan >ep'ipedoi >arijmo`i o<i A, B, ka`i to~u m`en A pleura`i
>'estwsan o<i G, D >arijmo'i, to~u d`e B o<i E, Z; l'egw, <'oti <o A
pr`oc t`on B l'ogon >'eqei t`on sugke'imenon >ek t~wn pleur~wn.}

\gr{L'ogwn g`ar doj'entwn to~u te <`on >'eqei <o G pr`oc t`on E ka`i
<o D pr`oc t`on Z e>il'hfjwsan >arijmo`i <ex~hc >el'aqistoi >en to~ic
G E, D Z l'ogoic, o<i H, J, K, <'wste e>~inai <wc m`en t`on G pr`oc t`on E,
o<'utwc t`on H pr`oc t`on J, <wc d`e t`on D pr`oc t`on Z, o<'utwc
t`on J pr`oc t`on K. ka`i <o D t`on E pollaplasi'asac t`on L poie'itw.}

\gr{Ka`i >epe`i <o D t`on m`en G pollaplasi'asac t`on A pepo'ihken, t`on
d`e E pollaplasi'asac t`on L pepo'ihken, >'estin >'ara <wc <o G pr`oc
t`on E, o<'utwc <o A pr`oc t`on L. <wc d`e <o G pr`oc t`on E, o<'utwc
<o H pr`oc t`on J; ka`i <wc >'ara <o H pr`oc t`on J, o<'utwc <o A pr`oc
t`on L. p'alin, >epe`i <o E t`on D pollaplasi'asac t`on L pepo'ihken,
>all`a m\kern -.7pt `hn ka`i t`on Z pollaplasi'asac t`on B 
pepo'ihken, >'estin
>'ara <wc <o D pr`oc t`on Z, o<'utwc <o L pr`oc t`on B. >all> <wc <o D
pr`oc t`on Z, o<'utwc <o J pr`oc t`on K; ka`i <wc >'ara <o J pr`oc t`on K,
o<'utwc <o L pr`oc t`on B. >ede'iqjh d`e ka`i <wc <o H pr`oc t`on J, o<'utwc
<o A pr`oc t`on L; di> >'isou >'ara >est`in <wc <o H pr`oc t`on K, [o<'utwc]
<o A pr`oc t`on B. <o d`e H pr`oc t`on K l'ogon >'eqei t`on sugke'imenon
>ek t~wn pleur~wn; ka`i <o A >'ara pr`oc t`on B l'ogon >'eqei t`on sugke'imenon >ek t~wn pleur~wn; <'oper >'edei de~ixai.}}

\ParallelRText{
\begin{center}
{\large Proposition 5}
\end{center}

Plane numbers have to one another the ratio compoun\-ded$^\dag$ out of (the ratios of)
 their sides.
 
 \epsfysize=2in
\centerline{\epsffile{Book08/fig05e.eps}}

Let $A$ and $B$ be plane numbers, and let the numbers $C$, $D$ be the
sides of $A$, and (the numbers) $E$, $F$ (the sides) of $B$. I say that $A$ has to $B$ the ratio compounded out of (the ratios of)  their sides.

For given the ratios which $C$ has to $E$, and $D$ (has) to $F$, let the least numbers, $G$, $H$, $K$, continuously proportional in the
ratios $C$ $E$, $D$ $F$ be taken [Prop.~8.4], so that as $C$ is to $E$, so
$G$ (is) to $H$, and as $D$ (is) to $F$, so $H$ (is) to $K$. And let $D$ make $L$ (by) multiplying $E$. 

And since $D$ has made $A$ (by) multiplying $C$, and has made $L$ (by)
multiplying $E$, thus as $C$ is to  $E$, so $A$ (is) to $L$ [Prop.~7.17]. And as $C$ (is) to $E$, so $G$ (is) to
$H$. And thus as $G$ (is) to $H$, so $A$ (is) to $L$. Again, since
$E$ has made $L$ (by) multiplying $D$ [Prop.~7.16],
but, in fact, has also made $B$ (by) multiplying $F$, thus as $D$ is to 
$F$, so $L$ (is) to $B$  [Prop.~7.17]. But,
as $D$ (is) to $F$, so $H$ (is) to $K$.  And thus as $H$ (is) to $K$, so
$L$ (is) to $B$. And it was also shown that as $G$ (is) to $H$, so $A$ (is)
to $L$. Thus, via equality, as $G$ is to $K$, [so] $A$ (is) to $B$ [Prop.~7.14]. 
And $G$ has to $K$ the ratio compounded out of (the ratios
of) the sides (of $A$ and $B$). Thus, $A$ also has to $B$ the ratio
compounded out of (the ratios of) the sides (of $A$ and $B$). (Which is)
the very thing it was required to show.}
\end{Parallel}
{\footnotesize\noindent$^\dag$ {\rm i.e.}, multiplied.}

%%%%%%
% Prop 8.6
%%%%%%
\pdfbookmark[1]{Proposition 8.6}{pdf8.6}
\begin{Parallel}{}{} 
\ParallelLText{
\begin{center}
{\large \ggn{6}.}
\end{center}\vspace*{-7pt}

\gr{>E`an >~wsin <oposoio~un >arijmo`i <ex~hc >an'alogon, <o d`e pr~wtoc
t`on de'uteron m\kern -.7pt `h metr~h|, o>ud`e >'alloc o>ude`ic o>ud'ena metr'hsei.}

\gr{>'Estwsan <oposoio~un >arijmo`i <ex~hc >an'alogon o<i A, B, G, D, E,
<o d`e A t`on B m\kern -.7pt `h metre'itw; l'egw, <'oti o>ud`e >'alloc o>ude`ic
o>ud'ena metr'hsei.}\\~\\

\epsfysize=2in
\centerline{\epsffile{Book08/fig06g.eps}}

\gr{<'Oti m`en o>~un o<i A, B, G, D, E <ex~hc >all'hlouc o>u metro~usin,
faner'on; o>ud`e g`ar <o A t`on B metre~i. l'egw d'h, <'oti o>ud`e >'alloc
o>ude`ic o>ud'ena metr'hsei. e>i g`ar dunat'on, metre'itw <o A t`on G.
ka`i <'osoi e>is`in o<i A, B, G, toso~utoi e>il'hfjwsan >el'aqistoi
>arijmo`i t~wn t`on a\kern -.7pt >ut`on l'ogon >eq'ontwn to~ic A, B, G o<i Z, H,
J. ka`i >epe`i o<i Z, H, J >en t~w| a\kern -.7pt >ut~w| l'ogw| e>is`i to~ic A, B, G,
ka'i >estin >'ison t`o pl~hjoc t~wn A, B, G t~w| pl'hjei t~wn Z, H, J, di>
>'isou >'ara >est`in <wc <o A pr`oc t`on G, o<'utwc <o Z pr`oc t`on
J. ka`i >epe'i >estin <wc <o A pr`oc t`on B, o<'utwc <o Z pr`oc t`on H,
o>u metre~i d`e <o A t`on B, o>u metre~i >'ara o>ud`e <o Z t`on 
H; o>uk >'ara mon'ac >estin <o Z; <h g`ar mon`ac p'anta >arijm`on
metre~i. ka'i e>isin o<i Z, J pr~wtoi pr`oc >all'hlouc [o>ud`e <o Z
>'ara t`on J metre~i]. ka'i >estin <wc <o Z pr`oc t`on J, o<'utwc
<o A pr`oc t`on G; o>ud`e <o A >'ara t`on G metre~i. <omo'iwc
d`h de'ixomen, <'oti o>ud`e >'alloc o>ude`ic o>ud'ena metr'hsei;
<'oper >'edei de~ixai.}}

\ParallelRText{
\begin{center}
{\large Proposition 6}
\end{center}

If there are any multitude whatsoever of continuously
proportional numbers, and the first does not measure the second, then no other (number) will
 measure any other (number) either.
 
 Let $A$, $B$, $C$, $D$, $E$ be any multitude whatsoever of continuously
 proportional numbers, and let $A$ not measure $B$.  I say that 
 no other (number) will measure any other (number) either.
 
 \epsfysize=2in
\centerline{\epsffile{Book08/fig06e.eps}}

 
 Now, (it is) clear that $A$, $B$, $C$, $D$, $E$ do not successively measure
 one another. For $A$ does not even measure $B$. So I say that  
 no other (number) will measure any other (number) either. For, if possible, let $A$ measure
 $C$. And as many (numbers) as  are $A$, $B$, $C$, let  so many of the least numbers,
 $F$, $G$, $H$, be taken of those (numbers) having the same ratio as $A$, $B$, $C$ [Prop.~7.33]. And since $F$, $G$, $H$ are in the same ratio as $A$, $B$, $C$, and the multitude of
 $A$, $B$, $C$ is equal to the multitude of $F$, $G$, $H$, thus, via
 equality, as $A$ is to $C$, so $F$ (is) to $H$ [Prop.~7.14]. And since as $A$ is to $B$, so $F$ (is) to $G$, and $A$ does not measure $B$, $F$ does not measure $G$ either
 [Def.~7.20]. Thus, $F$ is not a unit. For a unit measures all numbers. And $F$ and $H$ are prime to one another [Prop.~8.3] [and thus $F$ does not measure $H$].
 And as $F$ is to $H$, so $A$ (is) to $C$. And thus $A$ does not measure
 $C$ either [Def.~7.20]. So, similarly, we can show
 that no other (number) can measure any other (number) either.
 (Which is) the very thing it was required to show.}
\end{Parallel}

%%%%%%
% Prop 8.7
%%%%%%
\pdfbookmark[1]{Proposition 8.7}{pdf8.7}
\begin{Parallel}{}{} 
\ParallelLText{
\begin{center}
{\large \ggn{7}.}
\end{center}\vspace*{-7pt}

\gr{>E`an >~wsin <oposoio~un >arijmo`i [<ex~hc] >an'alogon,
<o d`e pr~wtoc t`on >'esqaton metr~h|, ka`i t`on de'uteron metr'hsei.}\\

\epsfysize=1.2in
\centerline{\epsffile{Book08/fig07g.eps}}

\gr{>'Estwsan <oposoio~un >arijmo`i <ex~hc >an'alogon o<i A, B, G, D,
<o d`e A t`on D metre'itw; l'egw, <'oti ka`i <o A t`on B metre~i.}

\gr{E>i g`ar o>u metre~i <o A t`on B, o>ud`e >'alloc o>ude`ic o>ud'ena
metr'hsei; metre~i d`e <o A t`on D. metre~i >'ara ka`i <o A t`on B;
<'oper >'edei de~ixai.}}

\ParallelRText{
\begin{center}
{\large Proposition 7}
\end{center}

If there are any multitude whatsoever of [continuously] proportional numbers, and the first measures the last, then (the first)
will also measure the second.

\epsfysize=1.2in
\centerline{\epsffile{Book08/fig07e.eps}}

Let $A$, $B$, $C$, $D$ be any number whatsoever of continuously proportional numbers. And let $A$ measure $D$. I say that $A$ also measures $B$.

For if $A$ does not measure $B$ then no other (number) will measure
any other (number) either [Prop.~8.6]. 
But $A$ measures $D$. Thus, $A$ also measures $B$. (Which is) the
very thing it was required to show.}
\end{Parallel}

%%%%%%
% Prop 8.8
%%%%%%
\pdfbookmark[1]{Proposition 8.8}{pdf8.8}
\begin{Parallel}{}{} 
\ParallelLText{
\begin{center}
{\large \ggn{8}.}
\end{center}\vspace*{-7pt}

\gr{>E`an  d'uo >arijm~wn metax`u kat`a t`o suneq`ec >an'alogon >emp'iptwsin
>arijmo'i, <'osoi e>ic a\kern -.7pt >uto`uc metax`u kat`a t`o suneq`ec >an'ologon >emp'iptousin >arijmo'i, toso~utoi ka`i e>ic to`uc t`on a\kern -.7pt >ut`on l'ogon
>'eqontac [a\kern -.7pt >uto~ic] metax`u kat`a t`o sun`eqec >an'alogon >empeso~untai}

\epsfysize=2.2in
\centerline{\epsffile{Book08/fig08g.eps}}

\gr{D'uo g`ar >arijm~wn t~wn A, B metax`u kat`a t`o suneq`ec >an'alogon
>empipt'etwsan >arijmo`i o<i G, D,  ka`i pepoi'hsjw <wc <o A pr`oc
t`on B, o<'utwc <o E pr`oc t`on Z; l'egw, <'oti <'osoi e>ic to`uc A, B
metax`u kat`a t`o suneq`ec >an'alogon >empept'wkasin >arijmo'i,
toso~utoi ka`i e>ic to`uc E, Z metax`u kat`a t`o suneq`ec >an'alogon
>empeso~untai.}

\gr{<'Osoi g'ar e>isi t~w| pl'hjei o<i A, B, G, D, toso~utoi e>il'hfjwsan
>el'aqistoi >arijmo`i t~wn t`on a\kern -.7pt >ut`on l'ogon >eq'ontwn to~ic
A, G, D, B o<i H, J, K, L; o<i >'ara >'akroi a\kern -.7pt >ut~wn o<i H, L pr~wtoi pr`oc
>all'hlouc e>is'in. ka`i >epe`i o<i A, G, D, B to~ic H, J, K, L >en t~w| a\kern -.7pt >ut~w| l'ogw| e>is'in, ka'i >estin >'ison t`o pl~hjoc t~wn
A, G, D, B t~w| pl'hjei t~wn H, J, K, L, di> >'isou >'ara >est`in
<wc <o A pr`oc t`on B, o<'utwc <o H pr`oc t`on L. <wc d`e <o A
pr`oc t`on B, o<'utwc <o E pr`oc t`on Z; ka`i <wc >'ara <o H pr`oc t`on L,
o<'utwc <o E pr`oc t`on Z. o<i d`e H, L pr~wtoi, o<i d`e pr~wtoi ka`i
>el'aqistoi, o<i d`e >el'aqistoi
>arijmo`i metro~usi to`uc t`on a\kern -.7pt >ut`on l'ogon >'eqontac >is'akic <'o te me'izwn t`on me'izona ka`i <o >el'asswn t`on >el'assona, tout'estin <'o te
<hgo'umenoc t`on <hgo'umenon ka`i <o <ep'omenoc t`on  <ep'omenon.
>is'akic >'ara <o H t`on E metre~i ka`i <o L t`on Z. <os'akic d`h <o H
t`on E metre~i, tosaut'akic ka`i <ek'ateroc t~wn J, K <ek'ateron t~wn
M, N metre'itw; o<i H, J, K, L
>'ara to`uc E, M, N, Z >is'akic metro~usin. o<i H, J, K, L
>'ara to~ic E, M, N, Z >en t~w| a\kern -.7pt >ut~w| l'ogw| e>is'in. >all`a o<i 
H, J, K, L to~ic A, G, D, B >en t~w| a\kern -.7pt >ut~w| l'ogw| e>is'in; ka`i
o<i A, G,  D, B >'ara to~ic E, M, N, Z >en t~w| a\kern -.7pt >ut~w| l'ogw|
e>is'in. o<i d`e A, G, D, B <ex~hc >an'alog'on
e>isin; ka`i o<i E, M, N, Z  >'ara <ex~hc >an'alog'on e>isin.
<'osoi >'ara e>ic to`uc A, B metax`u kat`a t`o suneq`ec >an'alogon
>empept'wkasin >arijmo'i, toso~utoi ka`i e>ic to`uc
E, Z metax`u kat`a t`o suneq`ec >an'alogon >empept'wkasin
>arijmo'i; <'oper >'edei de~ixai.}}

\ParallelRText{
\begin{center}
{\large Proposition 8}
\end{center}

If between two numbers there fall (some) numbers in continued proportion then, as many  numbers as fall in between them in continued proportion, so many (numbers)  will also fall in between (any two numbers) having the same ratio [as them] in continued  proportion.

\epsfysize=2.2in
\centerline{\epsffile{Book08/fig08e.eps}}

For let the numbers, $C$ and $D$,
 fall between two numbers, $A$ and $B$, in continued proportion, and let it be contrived
(that) as $A$ (is) to $B$, so $E$ (is) to $F$. I say that, as many numbers  as have fallen in between $A$ and $B$ in continued proportion, so many (numbers)  will also fall in between $E$ and $F$ in continued proportion.

For as many  as $A$, $B$, $C$, $D$ are in multitude, let so many
of the least numbers, $G$, $H$, $K$, $L$, having the
same ratio as $A$, $B$, $C$, $D$, be taken [Prop.~7.33]. Thus, the outermost of them, $G$ and
$L$, are prime to one another [Prop.~8.3].
And since $A$, $B$, $C$, $D$ are in the same ratio as $G$, $H$, $K$, $L$,
and the multitude of $A$, $B$, $C$, $D$ is equal to the
multitude of $G$, $H$, $K$, $L$, thus, via equality, as $A$ is to $B$,
so $G$ (is) to $L$ [Prop.~7.14].
And as $A$ (is) to $B$, so $E$ (is) to $F$. And thus as $G$ (is) to $L$,
so $E$ (is) to $F$. And $G$ and $L$ (are) prime (to one another). And (numbers) prime (to one another are) also the least (numbers having the same ratio as them) [Prop.~7.21]. And the least numbers measure
those (numbers) having the same ratio (as them) an equal number of times, the greater (measuring) the greater, and the lesser the lesser---that is to say, the leading (measuring) the leading, and the following the following [Prop.~7.20]. Thus, $G$ measures $E$ the same
number of times as $L$ (measures) $F$. So as many times as $G$ measures $E$, so many times let $H$, $K$ also measure  $M$, $N$, respectively.
Thus, $G$, $H$, $K$, $L$  measure $E$, $M$, $N$, $F$ (respectively) an equal
number of times. Thus, $G$, $H$, $K$, $L$ are in the same
ratio as $E$, $M$, $N$, $F$ [Def.~7.20]. But,
$G$, $H$, $K$, $L$ are in the same ratio as $A$, $C$, $D$, $B$. 
Thus, $A$, $C$, $D$, $B$ are also in the same ratio as $E$, $M$, $N$, $F$.
And $A$, $C$, $D$, $B$ are continuously proportional. Thus, $E$, $M$, 
$N$, $F$  are also continuously proportional. Thus, as many numbers  as have fallen in between $A$ and $B$ in
continued proportion, so many
numbers  have also fallen in between $E$ and $F$ in continued proportion.
(Which is) the very thing it was required to show.}
\end{Parallel}

%%%%%%
% Prop 8.9
%%%%%%
\pdfbookmark[1]{Proposition 8.9}{pdf8.9}
\begin{Parallel}{}{} 
\ParallelLText{
\begin{center}
{\large \ggn{9}.}
\end{center}\vspace*{-7pt}

\gr{>E`an d'uo >arijmo`i pr~wtoi pr`oc >all'hlouc >~wsin, ka`i e>ic
a\kern -.7pt >uto`uc metax`u kat`a t`o suneq`ec >an'alogon >emp'iptwsin
>arijmo'i, <'osoi e>ic a\kern -.7pt >uto`uc metax`u kat`a t`o suneq`ec >an'alogon
>emp'iptousin >arijmo'i, toso~utoi ka`i <ekat'erou a\kern -.7pt >ut~wn ka`i mon'adoc
metax`u kat`a t`o suneq`ec >an'alogon >empeso~untai.}

\epsfysize=2in
\centerline{\epsffile{Book08/fig09g.eps}}

\gr{>'Estwsan d'uo >arijmo`i pr~wtoi pr`oc >all'hlouc o<i A, B, ka`i e>ic
a\kern -.7pt >uto`uc metax`u kat`a t`o suneq`ec >an'alogon >empipt'etwsan
o<i G, D, ka`i >ekke'isjw <h E mon'ac; l'egw, <'oti <'osoi e>ic to`uc
A, B metax`u kat`a t`o suneq`ec >an'alogon >empept'wkasin
>arijmo'i, toso~utoi ka`i <ekat'erou t~wn A, B ka`i t~hc mon'adoc
metax`u kat`a t`o suneq`ec >an'alogon >empeso~untai.}

\gr{E>il'hfjwsan g`ar d'uo m`en >arijmo`i >el'aqistoi >en t~w| t~wn A, G, D, B
l'ogw| >'ontec o<i Z, H, tre~ic d`e o<i J, K, L, ka`i >ae`i <ex~hc <en`i
ple'iouc, <'ewc >`an >'ison g'enhtai t`o pl~hjoc a\kern -.7pt >ut~wn t~w|
pl'hjei t~wn A, G, D, B. e>il'hfjwsan, ka`i >'estwsan o<i M, N, X, O.
faner`on d'h, <'oti <o m`en Z <eaut`on pollaplasi'asac t`on J pepo'ihken,
t`on d`e J pollaplasi'asac t`on M pepo'ihken, ka`i <o H
<eaut`on m`en pollaplasi'asac t`on L pepo'ihken, t`on d`e L pollaplasi'asac t`on O pepo'ihken.  ka`i >epe`i o<i M, N, X, O >el'aqisto'i e>isi t~wn
t`on a\kern -.7pt >ut`on l'ogon >eq'ontwn to~ic Z, H, e>is`i d`e ka`i o<i A, G, D, B
>el'aqistoi t~wn t`on a\kern -.7pt >ut`on l'ogon >eq'ontwn to~ic
Z, H, ka'i >estin >'ison t`o pl~hjoc t~wn M, N, X, O t~w| pl'hjei
t~wn A, G, D, B, <'ekastoc >'ara t~wn M, N, X, O <ek'astw|
t~wn A, G, D, B >'isoc >est'in; >'isoc >'ara >est`in <o m`en M t~w|
A, <o d`e O t~w| B. ka`i >epe`i <o Z <eaut`on pollaplasi'asac t`on J pepo'ihken, <o Z >'ara t`on J metre~i kat`a t`ac >en t~w| Z mon'adac. metre~i
d`e ka`i <h E mon`ac t`on Z kat`a t`ac >en a\kern -.7pt >ut~w| mon'adac; >is'akic
>'ara <h E mon`ac t`on Z >arijm`on metre~i ka`i <o Z t`on J. >'estin
>'ara <wc <h E mon`ac pr`oc t`on Z >arijm'on, o<'utwc <o Z pr`oc t`on J.
p'alin, >epe`i <o Z t`on J pollaplasi'asac t`on M pepo'ihken, <o J >'ara
t`on M metre~i kat`a t`ac >en t~w| Z mon'adac. metre~i d`e ka`i <h E
mon`ac t`on Z >arijm`on kat`a t`ac >en a\kern -.7pt >ut~w| mon'adac; >is'akic
>'ara <h E mon`ac t`on Z >arijm`on metre~i ka`i <o J t`on M. >'estin
>'ara <wc <h E mon`ac pr`oc t`on Z >arijm'on, o<'utwc <o J pr`oc t`on M.
>ede'iqjh d`e ka`i <wc <h E mon`ac pr`oc t`on Z >arijm'on, o<'utwc
<o Z pr`oc t`on J; ka`i <wc >'ara <h E mon`ac pr`oc t`on Z >arijm'on, o<'utwc <o Z pr`oc t`on J ka`i <o J pr`oc t`on M. >'isoc d`e <o M 	t~w|
A; >'estin >'ara <wc <h E mon`ac pr`oc t`on Z >arijm'on, o<'utwc
<o Z pr`oc t`on J ka`i <o J pr`oc t`on
A. di`a t`a a\kern -.7pt >ut`a d`h
ka`i <wc <h E mon`ac pr`oc t`on H >arijm'on, o<'utwc <o H pr`oc t`on L
ka`i <o L pr`oc t`on B. <'osoi >'ara e>ic to`uc A, B
metax`u kat`a t`o
suneq`ec >an'alogon >empept'wkasin >arijmo'i, toso~utoi ka`i <ekat'erou
t~wn A, B ka`i mon'adoc t~hc E metax`u kat`a t`o suneq`ec >an'alogon
>empept'wkasin >arijmo'i; <'oper >'edei de~ixai.}}

\ParallelRText{
\begin{center}
{\large Proposition 9}
\end{center}

If two numbers are prime to one another and there fall  in  between them (some) numbers in continued proportion then,
as many numbers as fall in between them  in continued proportion, so many
(numbers) will also fall between each of them and a unit in continued proportion.

\epsfysize=2in
\centerline{\epsffile{Book08/fig09e.eps}}

Let $A$ and $B$ be two numbers (which are) prime to one another, and let the
(numbers) $C$ and $D$ fall in  between them in continued proportion. And let the unit $E$ be set out. I say that, as many numbers   as have fallen in between $A$ and $B$ in continued proportion, so many (numbers)  will also
fall between each of $A$ and $B$ and the unit in continued proportion.

For let the least two numbers, $F$ and $G$, which are in the
ratio of $A$, $C$, $D$, $B$, be taken [Prop.~7.33]. And the (least) three (numbers), $H$, $K$, $L$. And so on, successively increasing by one, until the multitude
of the (least numbers taken) is made  equal to the multitude of $A$, $C$, $D$, $B$
[Prop.~8.2]. Let them be taken, and let
them be $M$, $N$, $O$, $P$. So (it is) clear that $F$ has made $H$ (by)
multiplying itself, and has made $M$ (by) multiplying $H$. And $G$
has made $L$ (by) multiplying itself, and has made $P$ (by) multiplying $L$
[Prop.~8.2~corr.]. And since $M$, $N$, $O$, $P$
are the least of those (numbers) having the same ratio as $F$, $G$, and
$A$, $C$, $D$, $B$ are also the least of those (numbers) having the
same ratio as $F$, $G$ [Prop.~8.2], and the multitude
of $M$, $N$, $O$, $P$ is equal to the multitude of $A$, $C$, $D$, $B$, thus
$M$, $N$, $O$, $P$ are  equal to  $A$, $C$, $D$, $B$, respectively. 
Thus, $M$ is equal to $A$, and $P$ to $B$. And since $F$ has made
$H$ (by) multiplying itself, $F$ thus measures $H$ according to the
units in $F$ [Def.~7.15].  And the unit $E$ also
measures $F$ according to the units in it. Thus, the unit $E$
measures the number $F$ as many times as $F$ (measures) $H$. Thus,
as the unit $E$ is to the number $F$, so $F$ (is) to $H$ [Def.~7.20]. Again, since $F$ has made $M$ (by)
multiplying $H$, $H$ thus measures $M$ according to the units in $F$
 [Def.~7.15].  And the unit $E$ also measures
 the number $F$ according to the units in it. Thus, the unit $E$ measures
 the number $F$ as many times as $H$ (measures) $M$. Thus,
 as the unit $E$ is to the number $F$, so $H$ (is) to $M$ [Prop.~7.20]. And it was shown
 that as the unit $E$ (is) to the number $F$, so $F$ (is) to $H$. And thus
 as the unit $E$ (is) to the number $F$, so $F$ (is) to $H$, and $H$ (is)
 to $M$. And $M$ (is) equal to $A$. Thus, as the unit $E$ is to the number
 $F$, so $F$ (is) to $H$, and $H$ to $A$. And so, for the same (reasons),
 as the unit $E$ (is) to the number $G$, so $G$ (is) to $L$, and $L$ to
 $B$.  Thus, as many  (numbers)  as have fallen
 in between $A$ and $B$  in continued proportion,  so many numbers  have also fallen between each of $A$ and $B$ and the unit 
 $E$ in continued proportion. (Which is) the very thing it was required to show. }
\end{Parallel}

%%%%%%
% Prop 8.10
%%%%%%
\pdfbookmark[1]{Proposition 8.10}{pdf8.10}
\begin{Parallel}{}{} 
\ParallelLText{
\begin{center}
{\large \ggn{10}.}
\end{center}\vspace*{-7pt}

\gr{>E'an d'uo >arijm~wn <ekat'erou ka`i mon'adoc metax`u kat`a t`o
suneq`ec >an'alogon >emp'iptwsin >arijmo'i, <'osoi <ekat'erou a\kern -.7pt >ut~wn
ka`i mon'adoc metax`u kat`a t`o suneq`ec >an'alogon >emp'iptousin >arijmo'i,
toso~utoi ka`i e>ic a\kern -.7pt >uto`uc metax`u kat`a t`o suneq`ec >an'alogon
>empeso~untai.}

\gr{D'uo g`ar >arijm~wn t~wn A, B ka`i mon'adoc t~hc G metax'u kat`a
t`o suneq`ec >an'alogon >empipt'etwsan >arijmo`i o<'i te D, E
ka`i o<i Z, H; l'egw, <'oti <'osoi <ekat'erou t~wn A, B ka`i mon'adoc
t~hc G metax`u kat`a t`o suneq`ec >an'alogon >empept'wkasin
>arijmo'i, toso~utoi ka`i e>ic to`uc A, B metax`u kat`a t`o suneq`ec
>an'alogon >empeso~untai.}

\epsfysize=1.8in
\centerline{\epsffile{Book08/fig10g.eps}}

\gr{<O D g`ar t`on Z pollaplasi'asac t`on J poie'itw,
<ek'ateroc d`e t~wn D, Z t`on J pollaplasi'asac <ek'ateron
t~wn K, L poie'itw.}

\gr{Ka`i >epe'i >estin <wc <h G mon`ac pr`oc t`on D >arijm'on,
o<'utwc <o D pr`oc t`on E, >is'akic >'ara <h G mon`ac
t`on D >arijm`on metre~i ka`i <o D t`on E. <h d`e G mon`ac t`on
D >arijm`on metre~i kat`a t`ac >en t~w| D mon'adac; ka`i <o D >'ara
>arijm`oc t`on E metre~i kat`a t`ac >en t~w| D mon'adac;  <o D >'ara
<eaut`on pollaplasi'asac t`on E pepo'ihken. p'alin, >epe'i
>estin <wc <h G [mon`ac] pr`oc t`on D >arijm`on, o<'utwc <o E pr`oc
t`on A, >is'akic >'ara <h G mon`ac t`on D >arijm`on metre~i ka`i
<o E t`on A. <h d`e G mon`ac t`on D >arijm`on metre~i kat`a t`ac >en t~w|
D mon'adac; ka`i <o E >'ara t`on A metre~i kat`a t`ac >en t~w| D mon'adac; <o D >'ara t`on E pollaplasi'asac t`on A pepo'ihken. di`a
t`a a\kern -.7pt >ut`a d`h ka`i <o m`en Z <eaut`on pollaplasi'asac t`on H pepo'ihken,
t`on d`e H pollaplasi'asac t`on B pepo'ihken. ka`i >epe`i <o D
<eaut`on m`en pollaplasi'asac t`on E pepo'ihken, t`on d`e Z pollaplasi'asac t`on  J pepo'ihken, >'estin >'ara <wc <o D pr`oc t`on Z, o<'utwc <o E pr`oc t`on J. di`a t`a a\kern -.7pt >ut`a
d`h ka`i <wc <o D pr`oc t`on Z, o<'utwc <o J pr`oc t`on H. ka`i <wc
>'ara <o E pr`oc t`on J, o<'utwc <o J pr`oc t`on H. p'alin, >epe`i
<o D <ek'ateron t~wn E, J pollaplasi'asac <ek'ateron t~wn A, K pepo'ihken,
>'estin >'ara <wc <o E pr`oc t`on J, o<'utwc <o A pr`oc t`on K. >all>
<wc <o E pr`oc t`on J, o<'utwc <o D pr`oc t`on Z; ka`i <wc >'ara <o D
pr`oc t`on Z, o<'utwc <o A pr`oc t`on K. p'alin, >epe`i <ek'ateroc
t~wn D, Z t`on J pollaplasi'asac <ek'ateron t~wn K, L pepo'ihken,
<'estin >'ara <wc <o D pr`oc t`on Z, o<'utwc <o K pr`oc t`on L.
>all> <wc <o D pr`oc t`on Z, o<'utwc <o A pr`oc t`on K;  ka`i <wc >'ara
<o A pr`oc t`on
 K, o<'utwc <o
K pr`oc t`on L. >'eti >epe`i <o Z <ek'ateron t~wn J, H pollaplasi'asac
<ek'ateron t~wn L, B pepo'ihken,
>'estin >'ara <wc <o J pr`oc t`on H, o<'utwc <o L pr`oc t`on B.
<wc d`e <o
J pr`oc t`on H, o<'utwc <o D pr`oc  t`on Z; ka`i <wc >'ara <o D pr`oc t`on Z,
o<'utwc <o L pr`oc t`on B. >ede'iqjh d`e ka`i <wc <o D pr`oc t`on Z, o<'utwc
<'o te A pr`oc t`on K ka`i <o K pr`oc t`on L; ka`i <wc >'ara <o A pr`oc
t`on K, o<'utwc <o K pr`oc t`on L ka`i <o L pr`oc t`on B. o<i A, K, L, B
>'ara kat`a t`o suneq`ec <ex~hc e>isin >an'alogon.
<'osoi
>'ara <ekat'erou t~wn A, B ka`i t~hc G mon'adoc metax`u kat`a t`o suneq`ec
>an'alogon >emp'iptousin >arijmo'i, toso~utoi ka`i e>ic to`uc A, B
metax`u kat`a t`o suneq`ec >empeso~untai; <'oper >'edei de~ixai.}}

\ParallelRText{
\begin{center}
{\large Proposition 10}
\end{center}

If (some) numbers fall between each of two numbers and a unit in continued proportion then, as many (numbers)
as fall between each of the (two numbers) and the unit in continued proportion, so many (numbers)  will also fall in between the (two numbers) themselves in continued proportion.

For let the numbers $D$, $E$ and $F$, $G$ fall between the numbers $A$ and $B$ (respectively) and the unit $C$ in continued proportion. I say that,
as many numbers as have fallen between each of $A$ and $B$ and the unit $C$ in continued proportion,
so many will also fall in between $A$ and $B$ in continued proportion.\\

\epsfysize=1.8in
\centerline{\epsffile{Book08/fig10e.eps}}

For let $D$ make $H$ (by) multiplying $F$. And let  $D$, $F$
make $K$, $L$, respectively, by multiplying $H$.

As since as the unit $C$  is to the number $D$, so $D$ (is) to $E$, the unit
$C$ thus measures the number $D$ as many times as $D$ (measures) $E$
[Def.~7.20]. And the unit $C$ measures the
number $D$ according to the units in $D$. Thus, the number $D$
also measures $E$ according to the units in $D$. Thus, $D$ has made $E$
(by) multiplying itself. Again, since as the [unit] $C$ is to
the number $D$, so $E$ (is) to $A$, the unit $C$ thus measures the number
$D$ as many times as $E$ (measures) $A$ [Def.~7.20].
  And the unit $C$ measures the number $D$ according to the units in $D$.
  Thus, $E$ also measures $A$ according to the units  in $D$. Thus, $D$
  has made $A$ (by) multiplying $E$. And so, for the same (reasons), 
  $F$ has made $G$ (by) multiplying itself, and has made $B$ (by)
  multiplying $G$. And since $D$ has made $E$ (by) multiplying itself,
  and has made $H$ (by) multiplying $F$, thus as $D$ is to $F$, so
  $E$ (is) to $H$ [Prop~7.17]. And so, for the
  same reasons, as $D$ (is) to $F$, so $H$ (is) to $G$ [Prop.~7.18].  And thus as $E$ (is) to $H$, so $H$
  (is) to $G$. Again, since $D$ has made  $A$, $K$ (by)
  multiplying $E$, $H$, respectively, thus as $E$ is to $H$, so $A$ (is)
  to $K$ [Prop~7.17]. But, as $E$ (is) to $H$,
  so $D$ (is) to $F$. And thus as $D$ (is) to $F$, so $A$ (is) to $K$.
  Again, since  $D$, $F$ have made  $K$, $L$, respectively, (by) multiplying $H$, thus as $D$ is to $F$, so $K$ (is) to $L$ [Prop.~7.18]. But, as $D$ (is) to $F$, so $A$ (is)
  to $K$.  And thus as $A$ (is) to $K$, so $K$ (is) to $L$.
   Further, since $F$ has made  $L$, $B$ (by) multiplying
   $H$, $G$, respectively, thus as $H$ is to $G$, so $L$ (is) to $B$  [Prop~7.17]. And as $H$ (is) to $G$, so $D$ (is) to $F$. And thus as $D$ 
  (is) to $F$, so $L$ (is) to $B$. And it was also
  shown that as $D$ (is) to $F$, so $A$ (is) to $K$, and $K$ to $L$.
  And thus as $A$ (is) to $K$, so $K$ (is) to $L$, and $L$ to $B$.
  Thus, $A$, $K$, $L$, $B$ are  successively in continued proportion.
  Thus, as many numbers as fall between each of $A$ and $B$ and the unit $C$ in continued proportion, so many will also fall in between $A$ and $B$
  in continued proportion. (Which is) the very thing it was required to show.}
\end{Parallel}

%%%%%%
% Prop 8.11
%%%%%%
\pdfbookmark[1]{Proposition 8.11}{pdf8.11}
\begin{Parallel}{}{} 
\ParallelLText{
\begin{center}
{\large \ggn{11}.}
\end{center}\vspace*{-7pt}

\gr{D'uo tetrag'wnwn >arijm~wn e<~ic m'esoc >an'alog'on >estin >arijm'oc,
ka`i <o tetr'agwnoc pr`oc t`on tetr'agwnon diplas'iona l'ogon >'eqei
>'hper <h pleur`a pr`oc t`hn pleur'an.}\\

\epsfysize=1.2in
\centerline{\epsffile{Book08/fig11g.eps}}

\gr{>'Estwsan tetr'agwnoi >arijmo`i o<i A, B, ka`i to~u m`en A pleur`a
>'estw <o G, to~u d`e B <o D; l'egw, <'oti t~wn A, B e<~ic
m'esoc >an'alog'on >estin >arijm'oc, ka`i <o A pr`oc t`on B
diplas'iona l'ogon >'eqei >'hper <o G pr`oc t`on D.}

\gr{<O G g`ar t`on D pollaplasi'asac t`on E poie'itw. ka`i >epe`i tetr'agwn'oc
>estin <o A, pleur`a d`e a\kern -.7pt >uto~u >estin <o G, <o G >'ara <eaut`on
pollaplasi'asac t`on A pepo'ihken. di`a t`a a\kern -.7pt >ut`a
d`h ka`i <o D <eaut`on pollaplasi'asac t`on B pepo'ihken.
>epe`i o>~un
<o G <ek'ateron t~wn G, D pollaplasi'asac <ek'ateron t~wn A, E pepo'ihken,
>'estin >'ara <wc <o G pr`oc t`on D, o<'utwc <o A pr`oc t`on E. di`a
t`a a\kern -.7pt >ut`a d`h ka`i <wc <o G pr`oc t`on D, o<'utwc <o E pr`oc t`on B.
ka`i <wc >'ara <o A pr`oc t`on E, o<'utwc <o E pr`oc t`on B. t~wn A, B
>'ara e<~ic m'esoc >an'alog'on >estin >arijm'oc.}

\gr{L'egw d'h, <'oti ka`i <o A pr`oc t`on B diplas'iona l'ogon >'eqei >'hper
<o G pr`oc t`on D. >epe`i g`ar tre~ic >arijmo`i >an'alog'on e>isin
o<i A, E, B, <o A >'ara pr`oc t`on B diplas'iona l'ogon >'eqei
>'hper <o A pr`oc t`on E. <wc d`e <o A pr`oc t`on E, o<'utwc <o G
pr`oc t`on D. <o A >'ara pr`oc t`on B diplas'iona l'ogon >'eqei
>'hper <h G pleur`a pr`oc t`hn D; <'oper >'edei de~ixai.}}

\ParallelRText{
\begin{center}
{\large Proposition 11}
\end{center}

There exists one number in mean proportion to two (given) square numbers.$^\dag$ And (one)  square (number) has to the
(other) square (number) a squared$^\ddag$ ratio with respect to (that) the side (of the former
has) to the side (of the latter).

\epsfysize=1.2in
\centerline{\epsffile{Book08/fig11e.eps}}

Let $A$ and $B$ be square numbers, and let $C$ be the side of $A$, and
$D$ (the side) of $B$. I say that there exists one number in mean proportion
to $A$ and $B$, and that $A$ has to $B$ a squared ratio with respect to
(that) $C$ (has) to $D$.

For let $C$ make $E$ (by) multiplying $D$. And since $A$ is square, and $C$ is its side, $C$ has thus made $A$ (by) multiplying
itself. And so, for the same (reasons), $D$ has made $B$ (by) multiplying
itself. Therefore, since $C$ has made  $A$, $E$ (by) multiplying
$C$, $D$, respectively, thus as $C$ is to $D$, so $A$ (is) to $E$ [Prop.~7.17]. And so, for the same (reasons), 
as $C$ (is) to $D$, so $E$ (is) to $B$  [Prop.~7.18]. And thus as $A$ (is) to $E$, so
$E$ (is) to $B$. Thus,  one  number (namely, $E$) is in mean proportion to $A$ and $B$.

So I say that $A$ also has to $B$ a squared ratio with respect to (that)
$C$ (has) to $D$. For since $A$, $E$, $B$ are  three (continuously) proportional numbers, $A$ thus has to $B$ a squared ratio with
respect to (that) $A$ (has) to $E$ [Def.~5.9]. And
as $A$ (is) to $E$, so $C$ (is) to $D$. Thus, $A$ has to $B$ a squared ratio
with respect to (that) side $C$ (has) to (side) $D$. (Which is) the very thing it
was required to show.}
\end{Parallel}
{\footnotesize\noindent$^\dag$ In other words, between two given square numbers there
exists a number in continued proportion.\\[0.5ex]
$^\ddag$ Literally, ``double''.}

%%%%%%
% Prop 8.12
%%%%%%
\pdfbookmark[1]{Proposition 8.12}{pdf8.12}
\begin{Parallel}{}{} 
\ParallelLText{
\begin{center}
{\large \ggn{12}.}
\end{center}\vspace*{-7pt}

\gr{D'uo k'ubwn >arijm~wn d'uo m'esoi >an'alog'on e>isin >arijmo'i, ka`i
<o k'uboc pr`oc t`on k'ubon triplas'iona l'ogon >'eqei >'hper <h pleur`a pr`oc t`hn pleur'an.}

\gr{>'Estwsan k'uboi >arijmo`i o<i A, B ka`i to~u m`en A pleur`a >'estw
<o G, to~u d`e B <o D; l'egw, <'oti t~wn A, B d'uo m'esoi >an'alog'on
e>isin >arijmo'i, ka`i <o A pr`oc t`on B triplas'iona l'ogon >'eqei
>'hper <o G pr`oc t`on D. }\\

\epsfysize=1.in
\centerline{\epsffile{Book08/fig12g.eps}}

\gr{<O g`ar G <eaut`on m`en pollaplasi'asac t`on E poie'itw, t`on d`e D pollaplasi'asac t`on Z poie'itw, <o d`e D <eaut`on pollaplasi'asac t`on H poie'itw, <ek'ateroc d`e t~wn G, D t`on Z pollaplasi'asac <ek'ateron
t~wn J, K poie'itw.}

\gr{Ka`i >epe`i k'uboc >est`in <o A, pleur`a d`e a\kern -.7pt >uto~u <o G, ka`i <o G
<eaut`on m`en pollaplasi'asac t`on E pepo'ihken, <o G >'ara <eaut`on
m`en pollaplasi'asac t`on E pepo'ihken, t`on d`e E pollaplasi'asac t`on A
pepo'ihken. di`a t`a a\kern -.7pt >ut`a d`h ka`i <o D <eaut`on m`en pollaplasi'asac
t`on H pepo'ihken, t`on d`e H pollaplasi'asac t`on B pepo'ihken. ka`i
>epe`i <o G <ek'ateron t~wn G, D pollaplasi'asac <ek'ateron t~wn E, Z
pepo'ihken, >'estin >'ara <wc <o G pr`oc t`on D, o<'utwc <o E pr`oc
t`on Z. di`a t`a a\kern -.7pt >ut`a d`h ka`i <wc <o G pr`oc t`on D, o<'utwc 
<o Z pr`oc t`on H. p'alin, >epe`i <o G <ek'ateron t~wn E, Z pollaplasi'asac
<ek'ateron t~wn A, J pepo'ihken, >'estin >'ara <wc <o E pr`oc t`on Z, o<'utwc
<o A pr`oc t`on J. <wc d`e <o E pr`oc t`on Z, o<'utwc <o G pr`oc t`on D;
ka`i <wc >'ara <o G pr`oc t`on D, o<'utwc <o A pr`oc t`on J. p'alin,
>epe`i <ek'ateroc t~wn G, D t`on Z pollaplasi'asac <ek'ateron t~wn
J, K pepo'ihken, >'estin >'ara <wc <o G pr`oc t`on D, o<'utwc <o J
pr`oc t`on K. p'alin, >epe`i <o D <ek'ateron t~wn Z, H pollaplasi'asac
<ek'ateron t~wn K, B pepo'ihken, >'estin >'ara <wc <o Z pr`oc t`on H,
o<'utwc <o K pr`oc t`on B. <wc d`e <o Z pr`oc t`on H, o<'utwc
<o G pr`oc t`on D; ka`i <wc >'ara <o G pr`oc t`on D, o<'utwc
<'o te A pr`oc t`on J ka`i <o J pr`oc t`on K ka`i <o K pr`oc t`on B. t~wn A, B >'ara
d'uo m'esoi >an'alog'on e>isin o<i J, K.}

\gr{L'egw d'h, <'oti ka`i <o A pr`oc t`on B triplas'iona l'ogon >'eqei
>'hper <o G pr`oc t`on D. >epe`i g`ar t'essarec >arijmo`i
>an'alog'on e>isin o<i A, J, K, B, <o A >'ara pr`oc t`on B
triplas'iona l'ogon >'eqei >'hper <o A pr`oc t`on J. <wc d`e <o A
pr`oc t`on J, o<'utwc <o G pr`oc t`on D; ka`i <o A [>'ara] pr`oc
t`on B triplas'iona l'ogon >'eqei >'hper <o G pr`oc t`on D; <'oper
>'edei de~ixai.}}

\ParallelRText{
\begin{center}
{\large Proposition 12}
\end{center}

There exist two numbers in mean proportion to two (given) cube numbers.$^\dag$ And  (one) cube (number)
has to the (other) cube (number) a cubed$^\ddag$ ratio
with respect to (that) the side (of the former has) to the side (of the latter).

Let $A$ and $B$ be cube numbers, and let $C$ be the side of $A$, and
$D$ (the side) of $B$. I say that there exist two numbers in mean proportion
to $A$ and $B$, and that $A$ has to $B$ a cubed ratio with respect to (that)
$C$ (has) to $D$.

\epsfysize=1.in
\centerline{\epsffile{Book08/fig12e.eps}}

For let $C$ make $E$ (by) multiplying itself, and let it make
$F$ (by) multiplying $D$. And let $D$ make $G$ (by) multiplying itself,
and let  $C$, $D$ make  $H$, $K$, respectively, (by) multiplying $F$.

And since $A$ is  cube, and $C$ (is) its side, and $C$ has
made $E$ (by) multiplying itself,  $C$ has thus made $E$ (by) multiplying
itself, and has made $A$ (by) multiplying $E$. And so, for the same (reasons),
$D$ has made $G$ (by) multiplying itself, and has made $B$ (by) multiplying $G$. And since $C$ has made  $E$, $F$ (by)
multiplying $C$, $D$, respectively, thus as $C$ is to $D$, so $E$ (is) to $F$
[Prop.~7.17]. And so, for the same (reasons), 
as $C$ (is) to $D$, so $F$ (is) to $G$ [Prop.~7.18].
Again, since $C$ has made  $A$, $H$ (by) multiplying
 $E$, $F$, respectively, thus as $E$ is to $F$, so $A$ (is) to $H$ [Prop.~7.17].  And as $E$ (is) to $F$, so $C$ (is)
to $D$. And thus as $C$ (is) to $D$, so $A$ (is) to $H$. Again,
since  $C$, $D$ have made  $H$, $K$, respectively,  (by) multiplying
$F$, thus as $C$ is to $D$, so $H$ (is) to $K$  [Prop.~7.18].  Again, since $D$ has made $K$, $B$ (by) multiplying  $F$, $G$, respectively, thus as $F$ is to $G$, so $K$ (is) to $B$ [Prop.~7.17]. And as $F$
(is) to $G$, so $C$ (is) to $D$. And thus as $C$ (is) to $D$, so $A$
(is) to $H$, and $H$ to $K$, and $K$ to $B$. Thus, $H$ and $K$ are two
(numbers) in mean proportion to $A$ and $B$.

So I say that $A$ also has to $B$ a cubed ratio with respect to (that)
$C$ (has) to $D$. For since  $A$, $H$, $K$, $B$ are  four 
(continuously) proportional numbers, $A$ thus has to $B$ a cubed ratio with respect to (that)
$A$ (has) to $H$ [Def.~5.10]. And as $A$ (is) to $H$,
so $C$ (is) to $D$. And [thus] $A$ has to $B$ a cubed ratio
with respect to (that) $C$ (has) to $D$. (Which is) the very thing it was required to show.}
\end{Parallel}
{\footnotesize\noindent$^\dag$ In other words, between two given cube numbers there
exist two numbers in continued proportion.\\[0.5ex]
$^\ddag$ Literally, ``triple".}

%%%%%%
% Prop 8.13
%%%%%%
\pdfbookmark[1]{Proposition 8.13}{pdf8.13}
\begin{Parallel}{}{} 
\ParallelLText{
\begin{center}
{\large \ggn{13}.}
\end{center}\vspace*{-7pt}

\gr{>E`an >~wsin <osoidhpoto~un >arijmo`i <ex~hc >an'alogon, ka`i
pollaplasi'asac <'ekastoc <eaut`on poi~h| tina, o<i gen'omenoi >ex
a\kern -.7pt >ut~wn >an'alogon >'esontai; ka`i >e`an o<i >ex >arq~hc to`uc
genom'enouc pollaplasi'asantec poi~ws'i tinac, ka`i a\kern -.7pt >uto`i >an'alogon
>'esontai [ka`i >ae`i per`i to`uc >'akrouc to~uto sumba'inei].}

\gr{>'Estwsan <oposoio~un >arijmo`i <ex~hc >an'alogon, o<i A, B, G,
<wc <o A pr`oc t`on B, o<'utwc <o B pr`oc t`on G, ka`i o<i A, B, G
<eauto`uc m`en pollaplasi'asantec to`uc D, E, Z poie'itwsan, to`uc d`e
D, E, Z pollaplasi'asantec to`uc H, J, K poie'itwsan; l'egw, <'oti
o<'i te D, E, Z ka`i o<i H, J, K <ex~hc >an'alogon e>isin.}\\~\\

\epsfysize=2.2in
\centerline{\epsffile{Book08/fig13g.eps}}

\gr{<O m`en g`ar A t`on B pollaplasi'asac t`on L poie'itw, <ek'ateroc d`e
t~wn A, B t`on L pollaplasi'asac <ek'ateron t~wn M, N poie'itw. ka`i
p'alin <o m`en B t`on G pollaplasi'asac t`on X poie'itw, <ek'ateroc
d`e t~wn B, G t`on X pollaplasi'asac <ek'ateron t~wn O, P
poie'itw.}

\gr{<Omo'iwc d`h to~ic >ep'anw de~ixomen, <'oti o<i D, L, E
ka`i o<i H, M, N, J <ex~hc e>isin >an'alogon >en t~w| to~u A pr`oc
t`on B l'ogw|, ka`i >'eti o<i E, X, Z ka`i o<i J, O, P, K <ex~hc
e>isin >an'alogon >en t~w| to~u B pr`oc t`on G l'ogw|. ka'i >estin <wc <o A
pr`oc t`on B, o<'utwc <o B pr`oc t`on G; ka`i o<i D, L, E >'ara to~ic E, X, Z
>en t~w| a\kern -.7pt >ut~w| l'ogw| e>is`i ka`i >'eti o<i H, M, N, J to~ic J, O, P,
K. ka'i >estin >'ison t`o m`en t~wn D, L, E pl~hjoc t~w| t~wn E, X, Z
pl'hjei, t`o d`e t~wn H, M, N, J t~w| t~wn J, O, P, K;
di> >'isou >'ara >est`in <wc m`en <o D pr`oc t`on E, o<'utwc <o E pr`oc
t`on Z, <wc d`e <o H pr`oc t`on J, o<'utwc <o J pr`oc t`on K; <'oper
>'edei de~ixai.}}

\ParallelRText{
\begin{center}
{\large Proposition 13}
\end{center}

If there are any multitude whatsoever of continuously proportional numbers, and each makes some (number by) multiplying itself, then the (numbers) created from them will
(also) be (continuously) proportional. And if the original (numbers) make some
(more numbers by) multiplying the created (numbers) then these will also be (continuously) proportional [and this always happens with the extremes].

Let $A$, $B$, $C$ be any multitude whatsoever of continuously proportional numbers, (such that) as $A$ (is) to $B$, so $B$ (is) to $C$. And
let $A$, $B$, $C$ make $D$, $E$, $F$ (by) multiplying themselves,
and let them make $G$, $H$, $K$ (by) multiplying $D$, $E$, $F$. 
I say that $D$, $E$, $F$ and $G$, $H$, $K$ are continuously proportional.

\epsfysize=2.2in
\centerline{\epsffile{Book08/fig13e.eps}}

For let $A$ make $L$ (by) multiplying $B$. And let  $A$, $B$
make  $M$, $N$, respectively, (by) multiplying $L$. And, again, let
$B$ make $O$ (by) multiplying $C$. And let  $B$, $C$
make  $P$, $Q$, respectively, (by) multplying $O$.

So, similarly to the above, we can show that $D$, $L$, $E$ and
$G$, $M$, $N$, $H$ are continuously proportional in the ratio of
$A$ to $B$, and, further, (that) $E$, $O$, $F$ and $H$, $P$, $Q$, $K$
are continuously proportional in the ratio of $B$ to $C$. And as $A$ is to $B$, so $B$ (is) to $C$. And thus $D$, $L$, $E$ are in the same ratio as
$E$, $O$, $F$, and, further, $G$, $M$, $N$, $H$ (are in the same ratio)
as $H$, $P$, $Q$, $K$. And the multitude of $D$, $L$, $E$ is equal
to the multitude of $E$, $O$, $F$, and that of $G$, $M$, $N$, $H$ to that of $H$, $P$, $Q$, $K$. Thus, via equality, as $D$ is to $E$, so $E$ (is) to $F$, and
as $G$ (is) to $H$, so $H$ (is) to $K$ [Prop.~7.14]. (Which is) the very thing it was required to show.}
\end{Parallel}

%%%%%%
% Prop 8.14
%%%%%%
\pdfbookmark[1]{Proposition 8.14}{pdf8.14}
\begin{Parallel}{}{} 
\ParallelLText{
\begin{center}
{\large \ggn{14}.}
\end{center}\vspace*{-7pt}

\gr{>E`an tetr'agwnoc tetr'agwnon metr~h|, ka`i <h pleur`a t`hn pleur`an
metr'hsei; ka`i >e`an
<h pleur`a t`hn pleur`an metr~h|, ka`i <o tetr'agwnoc t`on tetr'agwnon
metr'hsei.}

\gr{>'Estwsan tetr'agwnoi >arijmo`i o<i A, B, pleura`i d`e a\kern -.7pt >ut~wn >'estwsan
o<i G, D, <o d`e A t`on B metre'itw; l'egw, <'oti ka`i <o G
t`on D metre~i.}\\~\\

\epsfysize=1in
\centerline{\epsffile{Book08/fig14g.eps}}

\gr{<O G g`ar t`on D pollaplasi'asac t`on E poie'itw;
o<i A, E, B >'ara <ex~hc >an'alog'on e>isin >en t~w| to~u G pr`oc
t`on D l'ogw|. ka`i >epe`i o<i A, E, B >ex~hc >an'alog'on e>isin, ka`i
metre~i <o A t`on B, metre~i >'ara ka`i <o A t`on E. ka'i >estin <wc <o
A pr`oc t`on E, o<'utwc <o G pr`oc t`on D; metre~i >'ara ka`i <o G
t`on D.}

\gr{P'alin d`h <o G t`on D metre'itw; l'egw, <'oti ka`i <o A t`on B metre~i.}

\gr{T~wn g`ar a\kern -.7pt >ut~wn kataskeuasj'entwn <omo'iwc de'ixomen, <'oti
o<i A, E, B <ex~hc >an'alog'on e>isin >en t~w| to~u G pr`oc t`on D
l'ogw|. ka`i >epe'i >estin <wc <o G pr`oc t`on D, o<'utwc <o A pr`oc
t`on E, metre~i d`e <o G t`on D, metre~i >'ara ka`i <o A t`on E. ka'i
e>isin o<i A, E, B <ex~hc >an'alogon; metre~i >'ara ka`i <o A t`on B.}

\gr{>E`an >'ara tetr'agwnoc tetr'agwnon metr~h|, ka`i <h pleur`a t`hn pleur`an
metr'hsei; ka`i >e`an <h pleur`a t`hn pleur`an metr~h|, ka`i <o
tetr'agwnoc t`on tetr'agwnon metr'hsei; <'oper >'edei de~ixai.}}

\ParallelRText{
\begin{center}
{\large Proposition 14}
\end{center}

If a square (number) measures a(nother) square
(number) then the side (of the former) will also measure the
side (of the latter). And if the side (of a square number) measures the side
(of another square number) then the (former) square (number) will also
measure the (latter) square (number).

Let $A$ and $B$ be square numbers, and let $C$ and $D$ be their sides (respectively). And let $A$ measure $B$. I say that $C$ also measures $D$.

\epsfysize=1in
\centerline{\epsffile{Book08/fig14e.eps}}

For let $C$ make $E$ (by) multiplying $D$. Thus, $A$, $E$, $B$
are continuously proportional in the ratio of $C$ to $D$
[Prop.~8.11]. And since $A$, $E$, $B$ are
continuously proportional, and $A$ measures $B$, $A$ thus also
measures $E$ [Prop.~8.7]. And as $A$ is to $E$,
so $C$ (is) to $D$. Thus, $C$ also measures $D$ [Def.~7.20].

So, again, let $C$ measure $D$. I say that $A$ also measures $B$.

For similarly, with the same construction, we can show that $A$, $E$, $B$
are continuously proportional in the ratio of $C$ to $D$. And since as $C$
is to $D$, so $A$ (is) to $E$, and $C$ measures $D$, $A$ thus
also measures $E$ [Def.~7.20]. And $A$, $E$,
$B$ are continuously proportional. Thus, $A$ also measures $B$.

Thus, if a square (number) measures a(nother) square
(number) then the side (of the former) will also measure the
side (of the latter). And if the side (of a square number) measures the side
(of another square number) then the (former) square (number) will also
measure the (latter) square (number). (Which is) the very thing it was
required to show.}
\end{Parallel}

%%%%%%
% Prop 8.15
%%%%%%
\pdfbookmark[1]{Proposition 8.15}{pdf8.15}
\begin{Parallel}{}{} 
\ParallelLText{
\begin{center}
{\large \ggn{15}.}
\end{center}\vspace*{-7pt}

\gr{>E`an k'uboc >arijm`oc k'ubon >arijm`on metr~h|, ka`i <h pleur`a t`hn
pleur`an metr'hsei; ka`i >e`an <h pleur`a t`hn pleur`an metr~h|, ka`i
<o k'uboc t`on k'ubon metr'hsei.}

\gr{K'uboc g`ar >arijm`oc <o A k'ubon t`on B metre'itw, ka`i to~u m`en
A pleur`a >'estw <o G, to~u d`e B <o D; l'egw, <'oti <o G t`on D metre~i.}

\gr{<O G g`ar <eaut`on pollaplasi'asac t`on E poie'itw, <o d`e D <eaut`on
pollaplasi'asac t`on H poie'itw, ka`i >'eti <o G t`on D pollaplasi'asac t`on Z
[poie'itw], <ek'ateroc d`e t~wn G, D t`on Z pollaplasi'asac <ek'ateron
t~wn J, K poie'itw. faner`on d'h, <'oti o<i E, Z, H ka`i o<i A, J, K, B
<ex~hc >an'alog'on e>isin >en t~w| to~u G pr`oc t`on D l'ogw|. ka`i >epe`i o<i A, J, K, B <ex~hc >an'alog'on e>isin, ka`i metre~i <o A t`on B, metre~i
>'ara ka`i t`on J. ka'i >estin <wc <o A pr`oc t`on J, o<'utwc <o G pr`oc t`on
D; metre~i >'ara ka`i <o G t`on D.}\\~\\

\epsfysize=1.7in
\centerline{\epsffile{Book08/fig15g.eps}}

\gr{>All`a d`h metre'itw <o G t`on D; l'egw, <'oti ka`i <o A t`on B metr'hsei.}

\gr{T~wn g`ar a\kern -.7pt >ut~wn kataskeuasj'entwn <omo'iwc d`h de'ixomen, <'oti
o<i A, J, K, B <ex~hc >an'alog'on e>isin >en t~w| to~u G pr`oc
t`on D l'ogw|. ka`i >epe`i <o G t`on D metre~i, ka'i >estin <wc <o G pr`oc
t`on D, o<'utwc <o A pr`oc t`on J, ka`i <o A >'ara t`on J metre~i; <'wste
ka`i t`on B metre~i <o A; <'oper >'edei de~ixai.}}

\ParallelRText{
\begin{center}
{\large Proposition 15}
\end{center}

If a cube number measures a(nother) cube
number then the side (of the former) will also measure the side (of the latter).
And if the side (of a cube number) measures the side (of another cube
number) then the (former) cube (number) will also measure the (latter)
cube (number).

For let the cube number $A$ measure the cube (number) $B$, and let
$C$ be the side of $A$, and $D$ (the side) of $B$. I say that $C$ measures $D$.

For let $C$ make $E$ (by) multiplying itself. And let $D$ make $G$ (by)
multiplying itself. And, further, [let] $C$ [make] $F$ (by) multiplying
$D$,  and let  $C$, $D$ make  $H$, $K$, respectively, (by) multiplying
$F$. So it is clear that $E$, $F$, $G$ and $A$, $H$, $K$, $B$ are continuously proportional in the ratio of $C$ to $D$ [Prop.~8.12]. And since $A$, $H$, $K$, $B$ are
continuously proportional, and $A$ measures $B$, ($A$) thus also
measures $H$ [Prop.~8.7]. And as $A$ is to
$H$, so $C$ (is) to $D$. Thus, $C$ also measures $D$ [Def.~7.20].\\

\epsfysize=1.7in
\centerline{\epsffile{Book08/fig15e.eps}}

And so let $C$ measure $D$. I say that $A$ will also measure $B$.

For similarly, with the same construction, we can show that $A$, $H$, $K$, $B$ are continuously proportional in the ratio of $C$ to $D$. And since
$C$ measures $D$, and as $C$ is to $D$, so $A$ (is) to $H$, $A$ thus
also measures $H$ [Def.~7.20]. Hence, $A$ also measures $B$. (Which is) the
very thing it was required to show.}
\end{Parallel}

%%%%%%
% Prop 8.16
%%%%%%
\pdfbookmark[1]{Proposition 8.16}{pdf8.16}
\begin{Parallel}{}{} 
\ParallelLText{
\begin{center}
{\large \ggn{16}.}
\end{center}\vspace*{-7pt}

\gr{>E`an tetr'agwnoc >arijm`oc tetr'agwnon >arijm`on m\kern -.7pt `h metr~h|, o>ud`e
<h pleur`a t`hn pleur`an metr'hsei; k>`an <h pleur`a t`hn pleur`an  m\kern -.7pt `h
metr~h|, o>ud`e <o tetr'agwnoc t`on tetr'agwnon metr'hsei.}\\~\\

\epsfysize=0.5in
\centerline{\epsffile{Book08/fig16g.eps}}

\gr{>'Estwsan tetr`agwnoi >arijmo`i o<i A, B, pleura`i d`e a\kern -.7pt >ut~wn >'estwsan
o<i G, D, ka`i m\kern -.7pt `h metre'itw <o A t`on B; l`egw, <'oti o>ud`e <o G t`on D
metre~i.}

\gr{E>i g`ar metre~i <o G t`on D, metr'hsei ka`i <o A t`on B. o>u metre~i
d`e <o A t`on B; o>ud`e >'ara <o G t`on D metr'hsei.}

\gr{m\kern -.7pt `h metre'itw [d`h] p'alin <o G t`on D; l'egw, <'oti o>ud`e <o A t`on B
metr'hsei.}

\gr{E>i g`ar metre~i <o A t`on B, metr'hsei ka`i <o G t`on D. o>u
metre~i d`e <o G t`on D; o>ud> >'ara <o A t`on B
metr'hsei; <'oper >'edei de~ixai.}}

\ParallelRText{
\begin{center}
{\large Proposition 16}
\end{center}

If a square number does not measure a(nother)
square number then  the side (of the former) will not  measure the
side (of the latter) either. And if the side (of a square number) does not measure the
side (of another square number) then the (former)
square (number) will not measure the (latter) square (number) either.

\epsfysize=0.5in
\centerline{\epsffile{Book08/fig16e.eps}}

Let $A$ and $B$ be square numbers, and let $C$ and $D$ be their
sides (respectively). And let $A$ not measure $B$. I say that  $C$ does not measure $D$ either.

For if $C$ measures $D$ then $A$ will also measure $B$
[Prop.~8.14]. And $A$ does not measure $B$. Thus,
$C$ will not measure $D$ either.

\mbox{[}So], again,  let $C$ not measure $D$. I say that $A$ will not measure $B$
either.

For if $A$ measures $B$ then $C$ will also measure $D$ [Prop.~8.14]. And $C$ does not measure $D$.
Thus, $A$ will not measure $B$ either. (Which is) the very thing it was required to show.}
\end{Parallel}

%%%%%%
% Prop 8.17
%%%%%%
\pdfbookmark[1]{Proposition 8.17}{pdf8.17}
\begin{Parallel}{}{} 
\ParallelLText{
\begin{center}
{\large \ggn{17}.}
\end{center}\vspace*{-7pt}

\gr{>E`an k'uboc >arijm`oc k'ubon >arijm`on m\kern -.7pt `h metr~h|, o>ud`e <h
pleur`a t`hn pleur`an metr'hsei; k>`an <h pleur`a t`hn pleur`an
m\kern -.7pt `h metr~h|, o>ud`e <o k'uboc t`on k'ubon metr'hsei.}\\~\\

\epsfysize=0.5in
\centerline{\epsffile{Book08/fig16g.eps}}

\gr{K'uboc g`ar >arijm`oc <o A k'ubon >arijm`on t`on B
m\kern -.7pt `h metre'itw, ka`i to~u m`en A pleur`a >'estw <o G, to~u
d`e B <o D; l'egw, <'oti <o G t`on D o>u metr'hsei.}

\gr{E>i g`ar metre~i <o G t`on D, ka`i <o A t`on B metr'hsei. o>u metre~i
d`e <o A t`on B; o>ud> >'ara <o G  t`on D metre~i.}

\gr{>All`a d`h m\kern -.7pt `h metre'itw <o G t`on D; l'egw, <'oti o>ud`e <o
A t`on B metr'hsei.}

\gr{E>i g`ar <o A t`on B metre~i, ka`i <o G t`on D metr'hsei. o>u metre~i
d`e <o G t`on D; o>ud> >'ara <o A t`on B metr'hsei; <'oper
>'edei de~ixai.}}

\ParallelRText{
\begin{center}
{\large Proposition 17}
\end{center}

If a cube number does not measure a(nother)
cube number then  the side (of the former) will not  measure the
side (of the latter) either. And if the side (of a cube number) does not measure the
side (of another cube number) then the (former)
cube (number) will not measure the (latter) cube (number) either.

\epsfysize=0.5in
\centerline{\epsffile{Book08/fig16e.eps}}

For let the cube number $A$ not measure the cube number $B$.
And let $C$ be the side of $A$, and $D$ (the side) of $B$.
I say that  $C$ will not measure $D$.

For if $C$ measures $D$ then $A$ will also measure $B$
[Prop.~8.15]. And $A$ does not measure $B$. Thus,
$C$ does not measure $D$ either.

And so let $C$ not measure $D$. I say that $A$ will not measure $B$
either.

For if $A$ measures $B$ then $C$ will also measure $D$ [Prop.~8.15]. And $C$ does not measure $D$.
Thus, $A$ will not measure $B$ either. (Which is) the very thing it was required to show.}
\end{Parallel}

%%%%%%
% Prop 8.18
%%%%%%
\pdfbookmark[1]{Proposition 8.18}{pdf8.18}
\begin{Parallel}{}{} 
\ParallelLText{
\begin{center}
{\large \ggn{18}.}
\end{center}\vspace*{-7pt}

\gr{D'uo <omo'iwn >epip'edwn >arijm~wn e<~ic m'esoc
>an'alog'on >estin >arijm'oc; ka`i <o >ep'ipedoc pr`oc t`on
>ep'ipedon diplas'iona l'ogon >'eqei >'hper <h <om'ologoc pleur`a
pr`oc t`hn <om'ologon pleur'an.}\\~\\

\epsfysize=1.in
\centerline{\epsffile{Book08/fig18g.eps}}

\gr{>'Estwsan d'uo <'omoioi >ep'ipedoi >arijmo`i o<i A, B, ka`i to~u
m`en A pleura`i >'estwsan o<i G, D >arijmo'i, to~u d`e B o<i E, Z. ka`i
>epe`i <'omoioi >ep'ipedo'i e>isin o<i >an'alogon >'eqontec t`ac pleur'ac,
>'estin >'ara <wc <o G pr`oc t`on D, o<'utwc <o E pr`oc t`on Z. l'egw
o>~un, <'oti t~wn A, B e<~ic m'esoc >an'alog'on >estin >arijm'oc,
ka`i <o A pr`oc t`on B diplas'iona l'ogon >'eqei >'hper <o G pr`oc
t`on E >`h <o D pr`oc t`on Z, tout'estin >'hper <h <om'ologoc
pleur`a pr`oc t`hn <om'ologon [pleur'an].}

\gr{Ka`i >epe'i >estin <wc <o G pr`oc t`on D, o<'utwc <o E
pr`oc t`on Z, >enall`ax >'ara >est`in <wc <o G pr`oc t`on
E, <o D pr`oc t`on Z. ka`i >epe`i >ep'iped'oc >estin <o A,
pleura`i d`e a\kern -.7pt >uto~u o<i G, D, <o D >'ara t`on G pollaplasi'asac t`on A
pepo'ihken. di`a t`a a\kern -.7pt >ut`a d`h ka`i <o E t`on Z pollaplasi'asac t`on B
pepo'ihken. <o D d`h t`on E pollaplasi'asac t`on H poie'itw. ka`i
>epe`i <o D t`on m`en G pollaplasi'asac t`on A pepo'ihken,
t`on d`e E pollaplasi'asac t`on 
H pepo'ihken, >'estin >'ara <wc <o G pr`oc t`on E, o<'utwc <o
A pr`oc t`on H. >all> <wc <o G pr`oc t`on E, [o<'utwc]
<o D pr`oc t`on Z; ka`i <wc >'ara <o D pr`oc t`on Z, o<'utwc
<o A pr`oc t`on H. p'alin, >epe`i <o E t`on m`en D pollaplasi'asac
t`on H pepo'ihken, t`on d`e Z pollaplasi'asac t`on B pepo'ihken,
>'estin >'ara <wc <o D pr`oc t`on Z, o<'utwc <o H pr`oc t`on B.
>ede'iqjh d`e ka`i <wc <o D pr`oc t`on Z, o<'utwc <o A pr`oc t`on H; ka`i
<wc >'ara <o A pr`oc t`on H, o<'utwc <o H pr`oc t`on B.
o<i A, H, B >'ara <ex~hc >an'alog'on e>isin. t~wn A, B >'ara e<~ic
m'esoc >an'alog'on >estin >arijm'oc.}

\gr{L'egw d'h, <'oti ka`i <o A pr`oc t`on B diplas'iona l'ogon >'eqei >'hper
<h <om'ologoc pleur`a pr`oc t`hn <om'ologon pleur'an, tout'estin
>'hper <o G pr`oc t`on E >`h <o D pr`oc t`on Z. >epe`i g`ar o<i A, H, B
<ex~hc >an'alog'on e>isin, <o A pr`oc t`on B diplas'iona
l'ogon >'eqei >'hper pr`oc t`on H. ka'i >estin <wc <o A pr`oc t`on H,
o<'utwc <'o te G pr`oc t`on E ka`i <o D pr`oc t`on Z.
ka`i <o A >'ara  pr`oc t`on B diplas'iona l'ogon >'eqei >'hper
<o G pr`oc t`on E >`h <o D pr`oc t`on Z; <'oper >'edei de~ixai.}}

\ParallelRText{
\begin{center}
{\large Proposition 18}
\end{center}

There exists one number in mean proportion to
two similar plane numbers. And (one) plane (number)
has to the (other) plane (number) a squared$^\dag$ ratio with respect to (that)
a corresponding side (of the former has) to a corresponding side (of the latter).

\epsfysize=1.in
\centerline{\epsffile{Book08/fig18e.eps}}

Let $A$ and $B$ be two similar plane numbers. And let the numbers $C$, $D$ be the
sides of $A$, and $E$, $F$ (the sides) of $B$. And since similar
numbers are those having proportional sides [Def.~7.21], thus as $C$ is to $D$, so
$E$ (is) to $F$. Therefore, I say that there exists one number in mean proportion
to $A$ and $B$, and that $A$ has to $B$ a squared ratio with
respect to that $C$ (has) to $E$, or $D$ to $F$---that is to say, with respect to
(that) a
corresponding side (has) to a corresponding [side].

For since as $C$ is to $D$, so $E$ (is) to $F$, thus, alternately, 
as $C$ is to $E$, so $D$ (is) to $F$ [Prop.~7.13].
And since $A$ is  plane, and $C$, $D$ its sides, $D$ has
thus made $A$ (by) multiplying $C$. And so, for the same (reasons),
$E$ has made $B$ (by) multiplying $F$.  So let $D$ make
$G$ (by) multiplying $E$. And since $D$ has made $A$ (by) multiplying
$C$, and has made $G$ (by) multiplying $E$, thus as $C$ is to
$E$, so $A$ (is) to $G$ [Prop.~7.17].
But as $C$ (is) to $E$, [so] $D$ (is) to $F$. And thus as $D$ (is) to  $F$,
so $A$ (is) to $G$. Again, since $E$ has made $G$ (by) multiplying
$D$, and has made $B$ (by) multiplying $F$, thus as $D$ is to $F$, so
$G$ (is) to $B$ [Prop.~7.17]. And it was
also shown that as $D$ (is) to $F$, so $A$ (is) to $G$. And thus as 
$A$ (is) to $G$, so $G$ (is) to $B$.  Thus, $A$, $G$, $B$ are continously
proportional. Thus, there exists one number (namely, $G$) in mean proportion to $A$ and $B$.

So I say that $A$ also has to $B$ a squared ratio with respect to (that)
a corresponding side (has) to a corresponding side---that is to say, with respect to (that) $C$ (has) to $E$, or $D$ to $F$. For since $A$, $G$, $B$
are continuously proportional, $A$ has to $B$ a squared ratio
with respect to (that $A$ has) to $G$ [Prop.~5.9].
And  as $A$ is to $G$, so $C$ (is) to $E$, and $D$ to $F$.
And thus $A$ has to $B$ a squared ratio with respect to (that)
$C$ (has) to $E$, or $D$ to $F$. (Which is) the very thing it was required to
show.}
\end{Parallel}
{\footnotesize\noindent$^\dag$  Literally, ``double''.}

%%%%%%
% Prop 8.19
%%%%%%
\pdfbookmark[1]{Proposition 8.19}{pdf8.19}
\begin{Parallel}{}{} 
\ParallelLText{
\begin{center}
{\large \ggn{19}.}
\end{center}\vspace*{-7pt}

\gr{D'uo <omo'iwn stere~wn >arijm~wn d'uo m'esoi >an'alogon >emp'iptousin
>arijmo'i; ka`i <o stere`oc pr`oc t`on <'omoion stere`on triplas'iona l'ogon
>'eqei >'hper <h <om'ologoc pleur`a pr`oc t`hn <om'ologon pleur'an.}

\epsfysize=1.9in
\centerline{\epsffile{Book08/fig19g.eps}}

\gr{>'Estwsan d'uo <'omoioi stereo`i o<i A, B, ka`i to~u m`en A pleura`i >'estwsan o<i G, D, E, to~u d`e B o<i Z, H, J. ka`i >epe`i <'omoioi stereo'i
e>isin o<i >an'alogon >'eqontec t`ac pleur'ac, >'estin >'ara <wc m`en
<o G pr`oc t`on D, o<'utwc <o Z pr`oc t`on H, <wc d`e <o D pr`oc
t`on E, o<'utwc <o H pr`oc t`on J. l'egw, <'oti t~wn A, B d'uo m'esoi
>an'alog'on >emp'iptousin >arijmo'i, ka`i <o A pr`oc t`on B triplas'iona
l'ogon >'eqei >'hper <o G pr`oc t`on Z ka`i <o D pr`oc t`on H ka`i >'eti
<o E pr`oc t`on J.}

\gr{<O G g`ar t`on D pollaplasi'asac t`on K poie'itw, <o d`e Z t`on H
pollaplasi'asac t`on L poie'itw. ka`i  >epe`i o<i G, D to`ic Z, H >en t~w| a\kern -.7pt >ut~w| l'ogw| e>is'in, ka`i >ek m`en t~wn G, D >estin <o K, >ek d`e t~wn
Z, H <o L, o<i K, L [>'ara] <'omoioi >ep'ipedo'i e>isin >arijmo'i; t~wn K,
L >'ara e<~ic m'esoc >an'alog'on >estin >arijm'oc. >'estw <o M.
<o M >'ara >est`in <o >ek t~wn D, Z, <wc >en t~w| pr`o to'utou
jewr'hmati >ede'iqjh. ka`i >epe`i <o D t`on m`en G pollaplasi'asac t`on K
pepo'ihken, t`on d`e Z pollaplasi'asac t`on M pepo'ihken, >'estin
>'ara <wc <o G pr`oc t`on Z, o<'utwc <o K pr`oc t`on M. >all> <wc
<o K pr`oc t`on M, <o M pr`oc t`on L. o<i K, M, L >'ara <ex~hc e>isin >an'alogon >en t~w| to~u G pr`oc t`on Z l'og~w|. ka`i >epe'i >estin <wc
<o G pr`oc t`on D, o<'utwc <o Z pr`oc t`on H, >enall`ax >'ara >est`in <wc <o
G pr`oc t`on Z, o<'utwc <o D pr`oc t`on H.
di`a t`a a\kern -.7pt >ut`a d`h ka`i <wc <o D pr`oc t`on H,
o<'utwc <o E pr`oc t`on J.
o<i K, M, L >'ara <ex~hc e>isin >an'alogon >'en te t~w| to~u G pr`oc
t`on Z l'ogw| ka`i t~w| to~u D pr`oc t`on H ka`i >'eti t~w| to~u E
pr`oc t`on J. <ekateroc d`h t~wn E, J t`on M pollaplasi'asac
 <ek'ateron t~wn N, X poie'itw. ka`i >epe`i stere'oc >estin <o A,
pleura`i d`e a\kern -.7pt >uto~u e>isin o<i G, D, E, <o E >'ara t`on >ek t~wn G, D
pollaplasi'asac t`on A pepo'ihken. <o d`e >ek t~wn G, D 
>estin <o K; <o E >'ara t`on K
pollaplasi'asac t`on A
pepo'ihken. di`a t`a a\kern -.7pt >ut`a d`h ka`i <o J t`on L 
pollaplasi'asac t`on B pepo'ihken. ka`i >epe`i <o
E t`on K pollaplasi'asac t`on A pepo'ihken,
 >all`a m\kern -.7pt `hn ka`i t`on M pollaplasi'asac
t`on N pepo'ihken, >'estin >'ara <wc <o K pr`oc t`on M, o<'utwc
<o A pr`oc t`on N. <wc d`e <o K pr`oc t`on M, o<'utwc
<'o te G pr`oc t`on Z ka`i <o D pr`oc t`on H  ka`i >'eti <o E pr`oc t`on J;
ka`i <wc >'ara <o  G
pr`oc t`on Z ka`i <o D pr`oc t`on H ka`i <o E pr`oc
t`on J, o<'utwc <o A pr`oc t`on N. p'alin, >epe`i <ek'ateroc t~wn E, J
t`on M pollaplasi'asac <ek'ateron t~wn N, X pepo'ihken, >'estin >'ara <wc <o E pr`oc t`on J, o<'utwc <o N pr`oc t`on X. >all> <wc <o E pr`oc t`on J, o<'utwc <'o te G pr`oc t`on Z ka`i <o D pr`oc t`on  H; ka`i <wc
>'ara <o G pr`oc t`on Z ka`i <o D pr`oc t`on H ka`i <o E pr`oc t`on J, o<'utwc
<'o te A pr`oc t`on N ka`i <o N pr`oc t`on X. p'alin, >epe`i <o J t`on M
pollaplasi'asac t`on X pepo'ihken, >all`a m\kern -.7pt `hn ka`i t`on L pollaplasi'asac
t`on B pepo'ihken, >'estin >'ara <wc <o M pr`oc t`on L, o<'utwc
<o X pr`oc t`on B. >all> <wc <o M pr`oc t`on L, o<'utwc <'o te G pr`oc t`on
Z ka`i <o D pr`oc t`on H ka`i <o E pr`oc t`on J. ka`i <wc >'ara <o G pr`oc t`on Z ka`i <o D pr`oc t`on H ka`i <o E pr`oc t`on J, o<'utwc o>u
m'onon <o X pr`oc t`on B, >all`a ka`i <o A pr`oc t`on N ka`i  <o N pr`oc
t`on X. o<i A, N, X, B >'ara <ex~hc e>isin >an'alogon >en to~ic
e>irhm'enoic t~wn pleur~wn l'ogoic.}

\gr{L'egw, <'oti ka`i <o A pr`oc t`on B triplas'iona l'ogon >'eqei
>'hper <h <om'ologoc pleur`a pr`oc t`hn <om'ologon pleur'an, tout'estin
>'hper <o G >arijm`oc pr`oc t`on Z >`h <o D pr`oc t`on H ka`i >'eti
<o E pr`oc t`on J. >epe`i g`ar t'essarec >arijmo`i <ex~hc >an'alog'on e>isin
o<i A, N, X, B, <o A >'ara pr`oc t`on B triplas'iona l'ogon >'eqei >'hper
<o A pr`oc t`on N. >all> <wc <o A pr`oc t`on N, o<'utwc >ede'iqjh
<'o te G pr`oc t`on Z ka`i <o D pr`oc t`on H ka`i >'eti <o E pr`oc
t`on J. ka`i <o A >'ara pr`oc t`on  B triplas'iona l'ogon >'eqei >'hper
<h om'ologoc pleur`a pr`oc t`hn <om'ologon pleur'an, tout'estin
>'hper <o G >arijm`oc pr`oc t`on Z ka`i <o D pr`oc t`on H ka`i >'eti <o E pr`oc t`on J;  <'oper >'edei de~ixai.}}

\ParallelRText{
\begin{center}
{\large Proposition 19}
\end{center}

Two numbers fall (between)
two similar solid numbers in mean proportion. And a solid (number) has to a similar solid (number) a cubed$^\dag$ ratio with respect to
(that) a corresponding side (has) to a corresponding side.

\epsfysize=1.9in
\centerline{\epsffile{Book08/fig19e.eps}}

Let $A$ and $B$ be two similar solid numbers, and let $C$, $D$, $E$ be the sides of $A$, and $F$, $G$, $H$ (the sides) of $B$. And
since similar solid (numbers) are those having proportional sides [Def.~7.21], thus as $C$ is to $D$, so $F$
(is) to $G$, and as $D$ (is) to $E$, so $G$ (is) to $H$. I say that two
numbers  fall  (between) $A$ and $B$ in mean proportion, and (that) $A$
has to $B$ a cubed ratio with respect to (that) $C$ (has) to $F$, and
$D$ to $G$, and, further, $E$ to $H$.

For let $C$ make $K$ (by) multiplying $D$, and let $F$ make $L$
(by) multiplying $G$. And since $C$, $D$ are in the same ratio as $F$, $G$,
and $K$ is the (number created) from (multiplying) $C$, $D$, and $L$ the (number created) from
(multiplying) $F$, $G$, [thus] $K$ and $L$ are  similar plane numbers [Def.~7.21]. Thus, there exits one number
in mean proportion to $K$ and $L$ [Prop.~8.18].
Let it be $M$. Thus, $M$ is the (number created) from (multiplying) $D$, $F$, as shown in the theorem before this (one). And since $D$ has made $K$ (by) multiplying $C$, and has made $M$ (by) multiplying $F$, thus as $C$ is to $F$, so
$K$ (is) to $M$ [Prop.~7.17]. But, as $K$ (is) to $M$, (so) $M$ (is) to $L$. Thus, $K$, $M$, $L$ are continuously
proportional in the ratio of $C$ to $F$. And since as $C$ is to $D$, so $F$
(is) to $G$, thus, alternately, as $C$ is to $F$, so $D$ (is) to $G$ [Prop.~7.13]. And so, for the same (reasons), 
as $D$ (is) to $G$, so $E$ (is) to $H$. Thus, $K$, $M$, $L$ are continuously
proportional in the ratio of $C$ to $F$, and of $D$ to $G$, and, further,  of $E$ to $H$. So let  $E$, $H$ make  $N$, $O$, respectively, (by) multiplying $M$. And since $A$ is  solid, and $C$, $D$, $E$ are its sides, $E$ has thus made $A$ (by) multiplying the (number created) from
(multiplying) $C$, $D$. And  $K$ 
is the (number created) from (multiplying)
$C$, $D$. Thus, $E$ has made $A$ (by) multiplying $K$. And so, for the
same (reasons), $H$ has made $B$ (by) multiplying $L$. And since
$E$ has made $A$ (by) multiplying $K$, but has, in fact, also made $N$ (by)
multiplying $M$, thus as $K$ is to $M$, so $A$ (is) to $N$ [Prop.~7.17].  And as $K$ (is) to $M$, so
$C$ (is) to $F$, and $D$ to $G$, and, further, $E$ to $H$. And thus as $C$
(is) to $F$, and $D$ to $G$, and $E$ to $H$, so $A$ (is) to $N$. Again, since
 $E$, $H$ have made  $N$, $O$, respectively, (by) multiplying $M$,
thus as $E$ is to $H$, so $N$ (is) to $O$ [Prop.~7.18]. But, as $E$ (is) to $H$, so $C$
(is) to $F$, and $D$ to $G$.  And thus as $C$ (is) to $F$, and $D$ to $G$,
and $E$ to $H$, so (is) $A$ to $N$, and $N$ to $O$. Again, since
$H$ has made $O$ (by) multiplying $M$, but  has, in fact, also made
$B$ (by) multiplying $L$, thus as $M$ (is) to $L$, so $O$ (is) to $B$ [Prop.~7.17]. But, as $M$ (is) to $L$, so $C$ (is)
to $F$, and $D$ to $G$, and $E$ to $H$. And thus as $C$ (is) to $F$,
and $D$ to $G$, and $E$ to $H$, so not only (is) $O$ to $B$, but
also $A$ to $N$, and $N$ to $O$. Thus,	 $A$, $N$, $O$, $B$ are continuously proportional in the aforementioned ratios of the sides.

So I say that $A$ also has to $B$ a cubed ratio with respect to (that)
a corresponding side (has) to a corresponding side---that is to say, with respect to (that) the number $C$ (has) to $F$, or $D$ to $G$, and, further,  $E$ to $H$. For
since $A$, $N$, $O$, $B$ are four continuously proportional numbers,
$A$ thus has to $B$	a cubed ratio with respect to (that) $A$ (has)
to $N$ [Def.~5.10]. But, as $A$ (is) to $N$, so
it was shown (is) $C$ to $F$, and $D$ to $G$, and, further, $E$ to $H$. And thus $A$ has to $B$ a cubed ratio with respect to (that) a corresponding
side (has) to a corresponding side---that is to say, with respect to
(that) the number $C$ (has) to $F$, and $D$ to $G$, and, further, $E$ to
$H$. (Which is) the very thing it was required to show.}
\end{Parallel}
{\footnotesize\noindent$^\dag$  Literally, ``triple''.}

%%%%%%
% Prop 8.20
%%%%%%
\pdfbookmark[1]{Proposition 8.20}{pdf8.20}
\begin{Parallel}{}{} 
\ParallelLText{
\begin{center}
{\large \ggn{20}.}
\end{center}\vspace*{-7pt}

\gr{>E`an d'uo >arijm~wn e<~ic m'esoc  >an'alogon >emp'ipt~h| >arijm'oc,
<'omoioi >ep'ipedoi >'esontai o<i >arijmo'i.}

\gr{D'uo g`ar >arijm~wn t~wn A, B e<~ic m'esoc >an'alogon >empipt'etw >arijm`oc <o G; l'egw, <'oti o<i A, B <'omoioi >ep'ipedo'i e>isin >arijmo'i.}\\

\epsfysize=1in
\centerline{\epsffile{Book08/fig20g.eps}}

\gr{E>il'hfjwsan [g`ar] >el'aqistoi >arijmo`i t~wn t`on a\kern -.7pt >ut`on l'ogon
>eq'ontwn to~ic A, G o<i D, E; >is'akic >'ara <o D t`on A metre~i ka`i
<o E t`on G.  <os'akic
d`h <o D t`on A metre~i, tosa\kern -.7pt ~utai mon'adec >'estwsan >en t~w| Z; <o Z >'ara t`on D
pollaplasi'asac t`on A pepo'ihken. <'wste <o A >ep'iped'oc >estin, pleura`i
d`e a\kern -.7pt >uto~u o<i D, Z. p'alin, >epe`i o<i D, E >el'aqisto'i e>isi t~wn t`on
a\kern -.7pt >ut`on l'ogon >eq'ontwn to~ic G, B, >is'akic >'ara <o D t`on G
metre~i ka`i <o E t`on B. <os'akic d`h <o E t`on B metre~i, tosa\kern -.7pt ~utai
mon'adec >'estwsan >en t~w| H. <o E >'ara t`on B metre~i kat`a t`ac
>en t~w| H mon'adac; <o H >'ara t`on E pollaplasi'asac t`on B pepo'ihken.
<o B >'ara  >ep'ipedoc >esti, pleura`i d`e a\kern -.7pt >uto~u e>isin o<i E, H.
o<i A, B >'ara >ep'ipedo'i e>isin >arijmo'i. l'egw d'h, <'oti
ka`i <'omoioi. >epe`i g`ar <o Z t`on m`en D pollaplasi'asac t`on
A pepo'ihken, t`on d`e E pollaplasi'asac t`on G pepo'ihken, >'estin >'ara
<wc <o D pr`oc t`on E, o<'utwc <o A pr`oc t`on G, tout'estin
<o G pr`oc t`on B. p'alin, >epe`i <o E <ek'ateron t~wn Z, H pollaplasi'asac 
to`uc G, B pepo'ihken, >'estin >'ara <wc <o Z pr`oc t`on H, o<'utwc
<o G pr`oc t`on B. <wc d`e <o G pr`oc t`on B, o<'utwc
<o D pr`oc t`on E; ka`i <wc >'ara <o D pr`oc t`on  E, o<'utwc
<o Z pr`oc t`on H; ka`i >enall`ax <wc <o D pr`oc t`on Z, o<'utwc <o E
pr`oc t`on H. o<i A, B >'ara <'omoioi >ep'ipedoi >arijmo'i e>isin; a<i g`ar
pleura`i a\kern -.7pt >ut~wn >an'alog'on e>isin; <'oper >'edei de~ixai.}}

\ParallelRText{
\begin{center}
{\large Proposition 20}
\end{center}

If one number falls between two
numbers in mean proportion then the numbers will be similar plane (numbers).

For let one number $C$ fall  between the two numbers $A$
and $B$ in mean proportion. I say that $A$ and $B$ are similar plane numbers.

\epsfysize=1in
\centerline{\epsffile{Book08/fig20e.eps}}

\mbox{[}For] let the least numbers, $D$ and $E$,  having the same ratio as $A$ and $C$ be taken [Prop.~7.33]. Thus,
$D$ measures $A$ as many  times as  $E$ (measures) $C$
[Prop.~7.20]. So as many times as $D$ measures $A$, so many units let there be in $F$. Thus, $F$ has made $A$ (by)
multiplying $D$ [Def.~7.15]. Hence, $A$ is 
plane, and $D$, $F$ (are) its sides. Again, since $D$ and $E$
are the least of those (numbers) having  the same ratio as $C$ and $B$, 
$D$ thus measures $C$ as many times as $E$ (measures) $B$ [Prop.~7.20]. So as many times as $E$ measures
$B$, so many units let there be in $G$. Thus, $E$ measures $B$ according to
the units in $G$. Thus, $G$ has made $B$ (by) multiplying $E$ [Def.~7.15]. Thus, $B$ is plane, and $E$, $G$ are its sides. Thus, $A$ and $B$ are (both) plane numbers. So I say that
(they are) also similar. For since $F$ has made $A$ (by) multiplying $D$,
and has made $C$ (by) multiplying $E$, thus as $D$ is to $E$, so $A$ (is) to $C$---that is to say, $C$ to $B$ [Prop.~7.17].$^\dag$ Again, since $E$ has made  $C$, $B$ (by)
multiplying  $F$, $G$, respectively, thus as $F$ is to $G$, so $C$ (is) to
$B$ [Prop.~7.17]. And as $C$ (is) to $B$, so
$D$ (is) to $E$. And thus as $D$ (is) to $E$, so $F$ (is) to $G$. And,
alternately, as $D$ (is) to $F$, so $E$ (is) to $G$ [Prop.~7.13]. Thus, $A$ and $B$ are similar plane
numbers. For their sides are proportional [Def.~7.21].
(Which is) the very thing it was required to show.}
\end{Parallel}
{\footnotesize\noindent$^\dag$ This part of the proof is defective, since it is not demonstrated that $F\times E = C$. Furthermore, it is
not necessary to show that $D:E::A:C$, because this
is true by hypothesis.}

%%%%%%
% Prop 8.21
%%%%%%
\pdfbookmark[1]{Proposition 8.21}{pdf8.21}
\begin{Parallel}{}{} 
\ParallelLText{
\begin{center}
{\large\ggn{21}.}
\end{center}\vspace*{-7pt}

\gr{>E`an d'uo >arijm~wn d'uo m'esoi >an'alogon >emp'iptwsin
>arijmo'i, <'omoioi stereo'i e>isin o<i >arijmo'i.}

\gr{D'uo g`ar >arijm~wn t~wn A, B d'uo m'esoi >an'alogon >empipt'etwsan
>arijmo`i o<i G, D; l'egw, <'oti o<i A, B <'omoioi stereo'i e>isin.}

\gr{E>il'hfjwsan g`ar >el'aqistoi >arijmo`i t~wn t`on a\kern -.7pt >ut`on l'ogon >eq'ontwn
to~ic A, G, D tre~ic o<i E, Z, H; o<i >'ara >'akroi a\kern -.7pt >ut~wn o<i E, H
pr~wtoi pr`oc >all'hlouc e>is'in. ka`i >epe`i t~wn E, H e<~ic
m'esoc >an'alogon >emp'eptwken >arijm`oc <o Z, o<i E, H >'ara
>arijmo`i <'omoioi >ep'ipedo'i e>isin. >'estwsan o>~un to~u m`en E pleura`i
o<i J, K, to~u d`e H o<i L, M. faner`on >'ara >est`in >ek to~u pr`o to'utou,
<'oti o<i E, Z, H <ex~hc e>isin >an'alogon >'en te t~w| to~u J pr`oc t`on L
l'ogw| ka`i t~w| to~u K pr`oc t`on M. ka`i >epe`i o<i E, Z, H >el'aqisto'i
e>isi t~wn t`on a\kern -.7pt >ut`on l'ogon >eq'ontwn to~ic A, G, D, ka'i >estin
>'ison t`o pl~hjoc t~wn E, Z, H t~w| pl'hjei t~wn A, G, D, di> >'isou >'ara >est`in <wc <o E pr`oc t`on H, o<'utwc <o A pr`oc t`on D. o<i d`e E, H
pr~wtoi, o<i d`e pr~wtoi ka`i >el'aqistoi, o<i d`e >el'aqistoi metro~usi
to`uc t`on a\kern -.7pt >ut`on l'ogon >'eqontac a\kern -.7pt >uto~ic >is'akic <'o te me'izwn
t`on me'izona ka`i <o >el'asswn t`on >el'assona, tout'estin <'o
te <hgo'umenoc t`on <hgo'umenon ka`i <o <ep'omenoc t`on <ep'omenon;
>is'akic >'ara <o E t`on A metre~i ka`i <o H t`on D. <os'akic d`h <o E
t`on A metre~i, tosa\kern -.7pt ~utai mon'adec >'estwsan >en t~w| N. <o N >'ara
t`on E pollaplasi'asac t`on A pepo'ihken. <o d`e E >estin <o >ek t~wn J, K;
<o N >'ara t`on >ek t~wn J, K pollaplasi'asac t`on A pepo'ihken. stere`oc
>'ara >est`in <o A, pleura`i d`e a\kern -.7pt >uto~u e>isin o<i J, K, N. p'alin, >epe`i
o<i E, Z, H >el'aqisto'i e>isi t~wn t`on a\kern -.7pt >ut`on l'ogon >eq'ontwn
to~ic G, D, B, >is'akic >'ara <o E t`on G metre~i ka`i <o H t`on B. <os'akic
d`h <o E t`on G metre~i, tosa\kern -.7pt ~utai mon'adec >'estwsan >en t~w| X.
<o H >'ara t`on B metre~i kat`a t`ac >en t~w| X mon'adac; <o X
>'ara t`on H pollaplasi'asac t`on B pepo'ihken. <o d`e H >estin <o >ek
t~wn L, M; <o X >'ara t`on >ek t~wn L, M pollaplasi'asac t`on B
pepo'ihken. stere`oc >'ara >est`in <o B, pleura`i d`e a\kern -.7pt >uto~u e>isin
o<i L, M, X; o<i A, B >'ara stereo'i e>isin.}\\~\\~\\~\\

\epsfysize=2in
\centerline{\epsffile{Book08/fig21g.eps}}

\gr{L'egw [d'h], <'oti ka`i <'omoioi. >epe`i g`ar o<i N, X t`on E pollaplasi'asantec
to`uc A, G pepoi'hkasin, >'estin >'ara <wc <o N pr`oc t`on X, <o A
pr`oc t`on G, tout'estin <o E pr`oc  t`on Z. >all> <wc <o E pr`oc t`on Z, <o
J pr`oc t`on L ka`i <o K pr`oc t`on M; ka`i <wc >'ara <o J pr`oc t`on
L, o<'utwc <o K pr`oc t`on M  ka`i <o N pr`oc  t`on X. ka'i e>isin
o<i
 m`en J, K, N pleura`i to~u A, o<i d`e X, L, M pleura`i to~u B. o<i 
A, B >'ara >arijmo`i <'omoioi stereo'i e>isin; <'oper >'edei
de~ixai.}}

\ParallelRText{
\begin{center}
{\large Proposition 21}
\end{center}

If two numbers fall between two numbers
in mean proportion  then the (latter) are similar solid (numbers).

For let the two numbers $C$ and $D$ fall between the two numbers
$A$ and $B$ in mean proportion. I say that $A$ and $B$ are similar
solid (numbers).

For let the three least numbers $E$, $F$, $G$  having the
same ratio as $A$, $C$, $D$ be taken [Prop.~8.2]. Thus,
the outermost of them, $E$ and $G$, are prime to one another
[Prop.~8.3]. And since one number, $F$, has
fallen (between) $E$ and $G$ in mean proportion, $E$ and $G$ are thus
similar plane numbers [Prop.~8.20].
Therefore, let $H$, $K$ be the sides of $E$, and $L$, $M$ (the sides) of $G$. Thus, it is clear from the (proposition) before this (one)
that $E$, $F$, $G$ are continuously proportional in the ratio of $H$ to $L$,
and of $K$ to $M$. And since $E$, $F$, $G$ are the least (numbers)
having the same ratio as $A$, $C$, $D$, and the multitude of $E$, $F$, $G$
is equal to the multitude of $A$, $C$, $D$, thus, via equality, as $E$ is to
$G$, so $A$ (is) to $D$ [Prop.~7.14]. And
$E$ and $G$ (are) prime (to one another), and prime (numbers) are also the least (of
those numbers having the same ratio as them) [Prop.~7.21],
and the least (numbers) measure those (numbers) having the same ratio
as them an equal number of times, the greater (measuring) the greater, and
the lesser the lesser---that is to say, the leading (measuring) the leading, and
the following the following [Prop.~7.20]. Thus,
$E$ measures $A$ the same number of times as $G$ (measures) $D$. 
So as many times as $E$ measures $A$, so many units let there be in $N$.
Thus, $N$ has made $A$ (by) multiplying $E$ [Def.~7.15]. And $E$ is the (number created) from (multiplying) $H$ and $K$. Thus, $N$ has made $A$ (by) multiplying
the (number created) from (multiplying) $H$ and $K$. Thus, $A$ is  solid, and its sides are $H$, $K$, $N$. Again, since $E$, $F$, $G$
are the least (numbers) having the same ratio as $C$, $D$, $B$, thus $E$
measures $C$ the same number of times as $G$ (measures) $B$ [Prop.~7.20]. So as many times as $E$ measures $C$, so many units let there be in $O$. Thus, $G$ measures $B$ according to
the units in $O$. Thus, $O$ has made $B$ (by) multiplying $G$. And
$G$ is the (number created) from (multiplying) $L$ and $M$. Thus, $O$ has made $B$ (by) multiplying the (number created) from (multiplying)
$L$ and $M$. Thus, $B$ is  solid, and its sides are $L$, $M$, $O$.
Thus,  $A$ and $B$ are (both) solid.

\epsfysize=2in
\centerline{\epsffile{Book08/fig21e.eps}}

\mbox{[}So] I say that (they are) also similar. For since $N$, $O$ have made
$A$, $C$ (by) multiplying $E$, thus as $N$ is to $O$, so $A$ (is) to $C$---that is to say, $E$ to $F$ [Prop.~7.18]. But, as $E$
(is) to $F$, so $H$ (is) to $L$, and $K$ to $M$. And thus as $H$ (is) to $L$,
so $K$ (is) to $M$, and $N$ to $O$. And $H$, $K$, $N$
are the sides of $A$, and $L$, $M$, $O$ the sides of $B$. Thus,
$A$ and $B$ are similar solid numbers [Def.~7.21]. 
(Which is) the very thing it was required to show.}
\end{Parallel}
{\footnotesize\noindent$^\dag$ The Greek text has ``$O$, $L$, $M$'', which is obviously a mistake.}

%%%%%%
% Prop 8.22
%%%%%%
\pdfbookmark[1]{Proposition 8.22}{pdf8.22}
\begin{Parallel}{}{} 
\ParallelLText{
\begin{center}
{\large\ggn{22}.}
\end{center}\vspace*{-7pt}

\gr{>E`an tre~ic >arijmo`i <ex~hc >an'alogon >~wsin, <o d`e pr~wtoc
tetr'agwnoc >~h|, ka`i <o tr'itoc tetr'agwnoc >'estai.}

\gr{>'Estwsan tre~ic >arijmo`i <ex~hc >an'alogon o<i A, B, G, <o d`e
pr~wtoc <o A tetr'agwnoc >'estw; l'egw, <'oti ka`i <o tr'itoc
<o G tetr'agwn'oc >estin.}

\epsfysize=0.8in
\centerline{\epsffile{Book08/fig22g.eps}}

\gr{>Epe`i g`ar t~wn A, G e<~ic m'esoc >an'alog'on >estin
>arijm`oc <o B, o<i A, G >'ara <'omoioi >ep'ipedo'i e>isin.
tetr'agwnoc d`e <o A; tetr'agwnoc >'ara ka`i <o G;
<'oper >'edei de~ixai.}}

\ParallelRText{
\begin{center}
{\large Proposition 22}
\end{center}

If three numbers are continuously proportional,
and the first is  square, then the third will also be  square.

Let $A$, $B$, $C$ be three continuously proportional numbers, and let the
first $A$ be  square. I say that the third $C$ is also  square.

\epsfysize=0.8in
\centerline{\epsffile{Book08/fig22e.eps}}

For since one number, $B$, is in mean proportion to $A$ and $C$, $A$
and $C$ are thus similar plane (numbers) [Prop.~8.20]. And $A$ is  square.
Thus, $C$ is also square [Def.~7.21].
(Which is) the very thing it was required to show.}
\end{Parallel}

%%%%%%
% Prop 8.23
%%%%%%
\pdfbookmark[1]{Proposition 8.23}{pdf8.23}
\begin{Parallel}{}{} 
\ParallelLText{
\begin{center}
{\large \ggn{23}.}
\end{center}\vspace*{-7pt}

\gr{>E`an t'essarec >arijmo`i <ex~hc >an'alogon >~wsin, <o d`e pr~wtoc
k'uboc >~h|, ka`i <o t'etartoc k'uboc >'estai.}

\epsfysize=1.2in
\centerline{\epsffile{Book08/fig23g.eps}}

\gr{>'Estwsan t'essarec >arijmo`i <ex~hc >an'alogon o<i A, B, G, D, <o d`e
A k'uboc >'estw; l'egw, <'oti ka`i <o D k'uboc >est'in.}

\gr{>Epe`i g`ar t~wn A, D d'uo m'esoi >an'alog'on e>isin >arijmo`i
o<i B, G, o<i A, D >'ara <'omoio'i e>isi stereo`i >arijmo'i. k'uboc
d`e <o A; k'uboc >'ara ka`i <o D; <'oper >'edei de~ixai.}}

\ParallelRText{
\begin{center}
{\large Proposition 23}
\end{center}

If four numbers are continuously proportional,
and the first is  cube, then the fourth will also be cube.

\epsfysize=1.2in
\centerline{\epsffile{Book08/fig23e.eps}}

Let $A$, $B$, $C$, $D$ be four continuously proportional numbers, and
let $A$ be cube. I say that $D$ is also  cube.

For since two numbers, $B$ and $C$, are in mean proportion to $A$ and
$D$, $A$ and $D$ are thus similar solid numbers [Prop.~8.21]. And $A$ (is)  cube.
Thus, $D$ (is) also  cube   [Def.~7.21].
(Which is) the very thing it was required to show.}
\end{Parallel}

%%%%%%
% Prop 8.24
%%%%%%
\pdfbookmark[1]{Proposition 8.24}{pdf8.24}
\begin{Parallel}{}{} 
\ParallelLText{
\begin{center}
{\large \ggn{24}.}
\end{center}\vspace*{-7pt}

\gr{>E`an d'uo >arijmo`i pr`oc >all'hlouc l'ogon >'eqwsin, <`on tetr'agwnoc
>arijm`oc pr`oc tetr'agwnon >arijm'on, <o d`e pr~wtoc tetr'agwnoc
>~h|, ka`i <o de'uteroc tetr'agwnoc >'estai.}

\epsfysize=0.5in
\centerline{\epsffile{Book08/fig24g.eps}}

\gr{D'uo g`ar >arijmo`i o<i A, B pr`oc >all'hlouc l'ogon >eq'etwsan, <`on
tetr'agwnoc >arijm`oc <o G pr`oc tetr'agwnon >arijm`on t`on D,
<o d`e A tetr'agwnoc >'estw; l'egw, <'oti ka`i <o B tetr'agwn'oc
>estin.}

\gr{>Epe`i g`ar o<i G, D tetr'agwno'i e>isin, o<i G, D >'ara <'omoioi
>ep'ipedo'i e>isin. t~wn G, D >'ara e~<ic m'esoc >an'alogon >emp'iptei
>arijm'oc. ka'i >estin <wc <o G pr`oc t`on D, <o A pr`oc t`on B; ka`i
t~wn A, B >'ara e~<ic m'esoc >an'alogon >emp'iptei >arijm'oc. ka'i
>estin <o A tetr'agwnoc; ka`i <o B >'ara tetr'agwn'oc >estin; <'oper
>'edei de~ixai.}}

\ParallelRText{
\begin{center}
{\large Proposition 24}
\end{center}

If two numbers have to one another the ratio
which a square number (has) to a(nother) square number, and
the first is  square, then the second will also be  square.

\epsfysize=0.5in
\centerline{\epsffile{Book08/fig24e.eps}}

For let two numbers, $A$ and $B$, have to one another the ratio which
the square number $C$ (has) to the square number $D$. And let
$A$ be square. I say that $B$ is also  square.

For since $C$ and $D$ are square, $C$ and $D$ are thus
similar plane (numbers). Thus, one number falls 
(between) $C$ and $D$ in mean proportion [Prop.~8.18]. And
as $C$ is to $D$, (so) $A$ (is) to $B$. Thus, one number also
falls  (between) $A$ and $B$ in mean proportion [Prop.~8.8]. And $A$ is  square. 
Thus, $B$ is also  square [Prop.~8.22].
(Which is) the very thing it was required to show.}
\end{Parallel}

%%%%%%
% Prop 8.25
%%%%%%
\pdfbookmark[1]{Proposition 8.25}{pdf8.25}
\begin{Parallel}{}{} 
\ParallelLText{\
\begin{center}
{\large \ggn{25}.}
\end{center}\vspace*{-7pt}

\gr{>E`an d'uo >arijmo`i pr`oc >all'hlouc l'ogon >'eqwsin, <`on k'uboc
>arijm`oc pr`oc k'ubon >arijm'on, <o d`e pr~wtoc k'uboc >~h|, ka`i
<o de'uteroc k'uboc >'estai.}

\epsfysize=0.9in
\centerline{\epsffile{Book08/fig25g.eps}}

\gr{D'uo g`ar >arijmo`i o<i A, B pr`oc >all'hlouc l'ogon >eq'etwsan,
<`on k'uboc >arijm`oc <o G pr`oc k'ubon >arijm`on t`on D, k'uboc
d`e >'estw <o A; l'egw [d'h], <'oti ka`i <o B k'uboc >est'in.}

\gr{>Epe`i g`ar o<i G, D k'uboi e>is'in, o<i G, D <'omoioi
stereo'i e>isin; t~wn G, D >'ara d'uo m'esoi >an'alogon >emp'iptousin
>arijmo'i. <'osoi d`e e>ic to`uc G, D metax`u kat`a
t`o suneq`ec >an'alogon >emp'iptousin, toso~utoi ka`i e>ic to`uc
t`on a\kern -.7pt >ut`on l'ogon >'eqontac a\kern -.7pt >uto~ic; <'wste ka`i t~wn A, B d'uo
m'esoi >an'alogon >emp'iptousin >arijmo'i. >empipt'etwsan o<i E, Z.
>epe`i o>~un t'essarec >arijmo`i o<i A, E, Z, B <ex~hc >an'alog'on
e>isin, ka'i >esti k'uboc <o A, k'uboc >'ara ka`i <o B;
<'oper >'edei de~ixai.}}

\ParallelRText{
\begin{center}
{\large Proposition 25}
\end{center}

If two numbers have to one another the
ratio which a cube number (has) to a(nother) cube number, and
the first is cube, then the second will also be cube.

\epsfysize=0.9in
\centerline{\epsffile{Book08/fig25e.eps}}

For  let two numbers, $A$ and $B$, have to one another the ratio which
the cube number $C$ (has) to the cube number $D$. And let $A$ be
 cube. [So] I say that $B$ is also  cube.
 
For since $C$ and $D$ are cube (numbers), $C$ and $D$ are (thus)
similar solid (numbers). Thus, two numbers fall
(between) $C$ and $D$  in mean proportion [Prop.~8.19]. And as many (numbers) as fall in between
$C$ and $D$  in continued proportion, so many also (fall)  in (between) those (numbers) having the same ratio as them (in continued proportion)
[Prop.~8.8].  And hence two numbers fall  (between) $A$ and $B$ in mean
proportion. Let $E$ and $F$ (so) fall.
Therefore, since the four numbers $A$, $E$, $F$, $B$ are continuously
proportional, and $A$ is  cube, $B$ (is) thus also 
cube [Prop.~8.23]. (Which is) the
very thing it was required to show.}
\end{Parallel}

%%%%%%
% Prop 8.26
%%%%%%
\pdfbookmark[1]{Proposition 8.26}{pdf8.26}
\begin{Parallel}{}{} 
\ParallelLText{
\begin{center}
{\large \ggn{26}.}
\end{center}\vspace*{-7pt}

\gr{O<i <'omoioi >ep'ipedoi >arijmo`i pr`oc >all'hlouc l'ogon
>'eqousin, <`on tetr'agwnoc >arijm`oc pr`oc tetr'agwnon
>arijm'on.}\\

\epsfysize=0.8in
\centerline{\epsffile{Book08/fig26g.eps}}

\gr{>'Estwsan <'omoioi >ep'ipedoi >arijmo`i o<i A, B; l'egw, <'oti <o A
pr`oc t`on B l'ogon >'eqei, <`on tetr'agwnoc >arijm`oc pr`oc tetr'agwnon
>arijm'on.}

\gr{>Epe`i g`ar o<i A, B <'omoioi >ep'ipedo'i e>isin, t~wn A, B >'ara
e~<ic m'esoc >an'alogon >emp'iptei >arijm'oc. >empipt'etw
ka`i >'estw <o G, ka`i e>il'hfjwsan >el'aqistoi >arijmo`i t~wn
t`on a\kern -.7pt >ut`on l'ogon >eq'ontwn to~ic A, G, B o<i D, E, Z; o<i
>'ara >'akroi a\kern -.7pt >ut~wn o<i D, Z tetr'agwno'i e>isin. ka`i >epe'i
>estin <wc <o D pr`oc t`on Z, o<'utwc
<o A pr`oc t`on B, ka'i e>isin o<i D, Z tetr'agwnoi, <o
A >'ara pr`oc t`on B l'ogon >'eqei, <`on tetr'agwnoc >arijm`oc pr`oc
tetr'agwnon >arijm'on; <'oper >'edei de~ixai.}}

\ParallelRText{
\begin{center}
{\large Proposition 26}
\end{center}

Similar plane numbers have to one another
the ratio which (some) square number (has) to a(nother) square number.

\epsfysize=0.8in
\centerline{\epsffile{Book08/fig26e.eps}}

Let $A$ and $B$ be similar plane numbers. I say that $A$ has to $B$ the
ratio which (some) square number (has) to a(nother) square number.

For since $A$ and $B$ are similar plane numbers, one number thus
falls  (between) $A$ and $B$ in mean proportion [Prop.~8.18]. Let it (so) fall, and let it be $C$. 
And let the least numbers, $D$, $E$, $F$, having the same
ratio as $A$, $C$, $B$ be taken [Prop.~8.2].
The outermost of them, $D$ and $F$, are thus square 
[Prop.~8.2~corr.]. And since as $D$ is to $F$, so
$A$ (is) to $B$, and $D$ and $F$ are square, $A$ thus
has to $B$ the ratio which (some) square number (has) to a(nother)
square number. (Which is) the very thing it was required to show.}
\end{Parallel}

%%%%%%
% Prop 8.27
%%%%%%
\pdfbookmark[1]{Proposition 8.27}{pdf8.27}
\begin{Parallel}{}{} 
\ParallelLText{
\begin{center}
{\large \ggn{27}.}
\end{center}\vspace*{-7pt}

\gr{O<i <'omoioi stereo`i >arijmo`i pr`oc >all'hlouc l'ogon >'eqousin, <`on
k'uboc >arijm`oc pr`oc k'ubon >arijm'on.}

\epsfysize=1in
\centerline{\epsffile{Book08/fig27g.eps}}

\gr{>'Estwsan <'omoioi stereo`i >arijmo`i o<i A, B; l'egw, <'oti <o A pr`oc
t`on B l'ogon >'eqei, <`on k'uboc >arijm`oc pr`oc k'ubon >arijm'on.}

\gr{>Epe`i g`ar o<i A, B <'omoioi stereo'i e>isin, t~wn A, B >'ara d'uo
m'esoi >an'alogon >emp'iptousin >arijmo'i. >empipt'et\-wsan o<i G, D,
ka`i e>il'hfjwsan >el'aqistoi >arijmo`i t~wn t`on a\kern -.7pt >ut`on l'ogon >eq'ontwn
to~ic A, G, D, B >'isoi a\kern -.7pt >uto~ic t`o pl~hjoc o<i E, Z, H, J; o<i >'ara
>'akroi a\kern -.7pt >ut~wn o<i E, J k'uboi e>is'in. ka'i >estin <wc <o E pr`oc
t`on J, o<'utwc <o A pr`oc t`on B; ka`i <o A >'ara pr`oc t`on B l'ogon
>'eqei, <`on k'uboc >arijm`oc pr`oc k'ubon >arijm'on; <'oper >'edei
de~ixai.}}

\ParallelRText{
\begin{center}
{\large Proposition 27}
\end{center}

Similar solid numbers have to one another the
ratio which (some) cube number (has) to a(nother) cube number.

\epsfysize=1in
\centerline{\epsffile{Book08/fig27e.eps}}

Let $A$ and $B$ be similar solid  numbers. I say that $A$ has to $B$
the ratio which (some) cube number (has) to a(nother) cube number.

For since $A$ and $B$ are similar solid (numbers), two numbers thus
fall
  (between) $A$ and $B$ in mean proportion [Prop.~8.19]. Let $C$ and $D$ have (so) fallen.
And let the least numbers, $E$, $F$, $G$, $H$, having the same
ratio as $A$, $C$, $D$, $B$, (and) equal in multitude to them, be taken [Prop.~8.2]. Thus, the outermost of them,
$E$ and $H$, are cube  [Prop.~8.2~corr.].
And as $E$ is to $H$, so $A$ (is) to $B$. And thus $A$ has to $B$ the ratio
which (some) cube number (has) to a(nother) cube number. (Which
is) the very thing it was required to show.}
\end{Parallel}
